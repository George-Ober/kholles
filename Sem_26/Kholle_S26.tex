\documentclass{article}

\date{5 Mai 2024}
\usepackage[nb-sem=26, auteurs={}]{../kholles}

\newcommand{\attrib}[1]{%
\nopagebreak{\raggedleft\footnotesize #1\par}}

\begin{document}
	\maketitle
	
	%Cette semaine avait trois jours fériés : les khôlles furent annulées.
\emph{La semaine décimée}
\begin{verse}
Sous les cieux printaniers de mai,\\
Vient la semaine tant attendue,\\
Où l'Ascension, d'un souffle léger,\\
Réduit les cours, moments suspendus.\\
\end{verse}
\begin{verse}
Les classes prépas, si intensives,\\
Prennent une pause, presque inédite,\\
Les jours se parent de l'harmonie,\\
D'une trêve aux heures décrépites.\\
\end{verse}
\begin{verse}
Trois jours, juste une poignée,\\
Suffisent à briser la cadence,\\
Les esprits se libèrent, apaisés,\\
De l’ordinaire lourde exigence.\\
\end{verse}
\begin{verse}
On appelle cette semaine, décimée,\\
Une parenthèse dans la rigueur,\\
Où les horloges semblent arrêter,\\
Le temps, éphémère douceur.\\
\end{verse}
\begin{verse}
Les étudiants, d'un souffle profond,\\
Respirent la clémence du printemps,\\
Ils laissent de côté leurs crayons,\\
Pour goûter à l’instant présent.\\
\end{verse}
\begin{verse}
Oh, douce pause, lumineuse évasion,\\
Dans la frénésie de l’éducation,\\
Tu offres un repos bien mérité,\\
Avant de replonger dans la densité.\\
\end{verse}
\begin{verse}
Que chaque année revienne encore,\\
Cette semaine au charme éthéré,\\
Bénie par l’Ascension d’alors,\\
Une oasis dans l’immensité.\\
\end{verse}

\end{document}

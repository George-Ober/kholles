\documentclass{article}

\date{19 novembre 2023}
\usepackage[nb-sem=6, auteurs={Kylian Boyet, George Ober, Hugo Vangilluwen (relecture)}]{../kholles}

%\begin{abstract}
%Ce pdf sera vraisemblablement sujet à de nombreuses maj. Je vais essayer à partir de maintenant de faire un pdf contenant toutes les khôlles de la 6-ième semaine à la dernière (j'espère avoir le temps). De plus, il peut arriver que je propose une solution originale à une question, seulement ladite solution n'aura sans doute été vérifiée par personne, si vous constatez une ou plusieurs erreur.s tachez de m'en faire part, d'ailleurs si vous relevez une erreur quelconque dans ce document je vous prie de faire de même. Enfin, les semaines $3$ et $5$ resteront hors de ce document car leur insertion est \textbf{BEAUCOUP TROP} compliquée pour moi donc j'ai laissé tombé... Si vous les voulez, faîtes moi signe. Aussi, si plusieurs démonstrations sont présentes dans le cours et que nous avons la possiblité de choisir celle.s que nous voulons, je prendrai toujours celle.s qui me paraît.ssent la.es plus naturelle.s et "facile.s" car je n'aime pas apprendre des choses inutilement compliquées, tout comme je prendrai parfois la liberté de laisser certaines choses au lecteur car certains passages sont trop évidents pour être traités. Finalement, si vous avez des questions sur ce que j'ai écrit ou si vous voulez des informations sur le code \LaTeX   , (ou Tikz), n'hésitez pas. Bonne lecture.
%\end{abstract}

\begin{document}

\maketitle


\begin{question_kholle}{Montrer que si $f$ est impaire et bijective, alors $f^{-1}$ est aussi impaire. Donnez un/des exemples.}
  Soit $f: I \to F$, avec $I,F$ deux parties non-vides de $\mathbb{R}$, une telle fonction et notons $f^{-1}$ sa bijection réciproque. Si $f$ est impaire sur $I$, alors pour tout $x\in I$, $-x\in I$, ainsi $I$ est centré en $0$ et on a :

  \begin{equation*}
    \forall x \in I, \ f(-x) = -f(x).
  \end{equation*}

  Ainsi, prenons $y\in F$, alors $-y \in F$ par imparité et bijectivité de $f$. On a donc :

  \begin{eqnarray*}
    f^{-1}(-y) & = & f^{-1}(-f(f^{-1}(y))) \\
    & = & f^{-1}(f(-f^{-1}(y))) \\
    & = & -f^{-1}(y).
  \end{eqnarray*}

  \

  D'où l'imparité de $f^{-1}$.

  \

  Pour ce qui est de l'exemple, prenons notre fonction bijective impaire préférée, la fonction $\textstyle \sin |_{\left[ -\frac{\pi}{2}, \frac{\pi}{2}\right] }^{[-1,1]}$ que l'on notera $\widetilde{\sin}$. Sa bijection réciproque est bien entendu $\textstyle \arcsin : [-1,1] \to \left[ -\frac{\pi}{2}, \frac{\pi}{2}\right]$.

  De la même manière que dans la démonstration du cas général, prenons $y\in [-1, 1]$, comme $[-1,1]$ est centré en $0$, $-y\in [-1,1]$, on a dès lors :

  \begin{eqnarray*}
    \arcsin(-y) & = & \arcsin(-\widetilde{\sin}(\arcsin(y))) \\
    & = & \arcsin(\widetilde{\sin}(-\arcsin(y))) \\
    & = & -\arcsin(y).
  \end{eqnarray*}
\end{question_kholle}

\begin{question_kholle}{Limite (et preuve) lorsque $x$ tend vers $+\infty$ de $\frac{(\ln x)^{\alpha}}{x^{\beta}}$ pour $\alpha ,\beta \in \left( \mathbb{R}_+^*\right) ^2$.}

  Premièrement, posons :

  \begin{equation*}
    \forall  (x,\alpha,\beta)\in [1,+\infty[ \times \left( \mathbb{R}_+^*\right) ^2, \quad  f_{\alpha,\beta}(x)=\frac{(\ln x)^{\alpha}}{x^{\beta}}.
  \end{equation*}

  Deuxièmement, montrons que :
  \[ \frac{\ln (x)}{x^2}\xrightarrow[n \to +\infty ]{} 0. \]

  Soit $x \in [1,+\infty[ \ = \mathcal{A}$. Nous savons que la fonction $\ln$ est concave sur $\mathbb{R}_+^*$, donc en particulier sur $\mathcal{A}$. Ainsi, $\ln$ est en dessous de toutes ses tangentes, d'où :
  \[
    \forall x \in \mathcal{A}, \quad 0 \; \leq \; \ln (x) \; \leq \; x-1.
  \]

  \newpage

  Illustration de l'inégalité :

  \

  \begin{center}
    \definecolor{ttttff}{rgb}{0.2,0.2,1}
    \definecolor{ffttww}{rgb}{1,0.2,0.4}
    \definecolor{cqcqcq}{rgb}{0.75,0.75,0.75}
    \begin{tikzpicture}[line cap=round,line join=round,>=triangle 45,x=1.0cm,y=1.0cm]
      \draw [color=cqcqcq,dash pattern=on 1pt off 1pt, xstep=1.0cm,ystep=1.0cm] (-1,-2) grid (3,2);
      \draw[->,color=black] (-1,0) -- (3,0);
      \foreach \x in {-1,1,2}
      \draw[shift={(\x,0)},color=black] (0pt,2pt) -- (0pt,-2pt) node[below] {\footnotesize $\x$};
      \draw[->,color=black] (0,-2) -- (0,2);
      \foreach \y in {-2,-1,1}
      \draw[shift={(0,\y)},color=black] (2pt,0pt) -- (-2pt,0pt) node[left] {\footnotesize $\y$};
      \draw[color=black] (0pt,-10pt) node[right] {\footnotesize $0$};
      \clip(-1,-2) rectangle (3,2);
      \draw[color=ffttww, smooth,samples=100,domain=3.244085366007135E-3:3.0] plot(\x,{ln(\x)});
      \draw[color=ttttff, smooth,samples=100,domain=-1.0:3.0] plot(\x,{\x-1});
      \draw (12.46,5.36) node[anchor=north west] {y = x-1};
      \draw (9.62,5.22) node[anchor=north west] {y = ln(x)};
    \end{tikzpicture}

    \

    \textbf{Figure 1.} $\ln$ en rouge et la première bissectrice en bleu.
  \end{center}

  \

  On peut alors diviser par $x^2$ (car $x \neq 0$):

  \

  \[
    \forall x \in \mathcal{A},\quad 0 \; \leq \; \underset{f_{1,2}(x)}{\underbrace{\frac{\ln (x)}{x^2}}} \; \leq \; \underset{\xrightarrow[x\to+\infty]{} \ 0}{\underbrace{\frac{1}{x}}} - \underset{\xrightarrow[x\to+\infty]{} \ 0}{\underbrace{\frac{1}{x^2}}}.
  \]

  \

  Donc par théorème d'encadrement $f_{1,2}(x)\xrightarrow[x\to+\infty]{} 0$.

  \

  Dernièrement, le cas général. Soit $x\in \mathcal{A}$ et soient $(\alpha,\beta)\in \left( \mathbb{R}_+^* \right)^2$. On fait une preuve directe.

  \begin{eqnarray*}
    \frac{(\ln (x))^\alpha}{x^\beta} & = & \left( \frac{\ln (x)}{x^{\frac{\beta}{\alpha}}} \right)^\alpha \\[1.5ex]
    & = & \underset{\underset{\text{par produit}}{\xrightarrow[x\to+\infty]{} \ 0}}{\underbrace{\underset{c^{\underline{te}} \ \text{(définie!)}}{\underbrace{\left( \frac{2\alpha}{\beta} \right)^\alpha}} \cdot \underset{\underset{\text{par composition des limites}}{\xrightarrow[x\to+\infty]{} \ 0}}{\underbrace{\left[ \underset{\underset{\text{d'après le dernier point}}{\xrightarrow[x\to+\infty]{} \ 0}}{\underbrace{ \frac{\ln \left( x^{\frac{\beta}{2 \alpha}} \right) }{\left( x^{\frac{\beta}{2\alpha}} \right)^2} }}\right]^\alpha}}}}.
  \end{eqnarray*}

\end{question_kholle}

\

%-------------------------------------------------------------

\begin{question_kholle}{Limite en $0$ de $\frac{1-\cos (x)}{x^2}$ et limite en $+\infty$ suivant $n$ de $\frac{\left(q^n \right)^\alpha}{(n!)^\beta}$ pour $q\in \mathbb{R}$ et $(\alpha,\beta)\in \left( \mathbb{R}_+^* \right)^2.$}

  \

  Montrons que $\frac{1-\cos (x)}{x^2} \xrightarrow[x\to 0]{} \frac{1}{2}$.

  \

  On fait toujours une preuve directe.
  \begin{eqnarray*}
    \lim_{x\to0} \ \frac{1-\cos (x)}{x^2} & = & \lim_{x\to0} \ \frac{1-\cos \left( \frac{2x}{2}\right) }{x^2} \\[1ex]
    & = & \lim_{x\to0} \ \frac{1-\left( 1-2\sin ^2 \left( \frac{x}{2}\right) \right) }{x^2} \\[1ex]
    & = &  \lim_{x\to0} \ \frac{2\sin ^2 \left( \frac{x}{2}\right) }{4 \left( \frac{x}{2}\right) ^2} \\[1ex]
    & = & \lim_{x\to0} \ \underset{\underset{\text{par produit}}{\xrightarrow[x\to 0]{} \ \frac{1}{2}}}{\underbrace{\underset{c^{\underline{te}}}{\underbrace{\frac{1}{2}}} \cdot \underset{\underset{\text{par composition}}{\xrightarrow[x\to 0]{} \ 1}}{\underbrace{\left[\underset{\underset{\text{limite usuelle}}{\xrightarrow[x\to 0]{} \ 1}}{\underbrace{\frac{\sin \left( \frac{x}{2}\right) }{\left( \frac{x}{2}\right)} }} \right] ^2}}}} \\[1ex]
    & = & \frac{1}{2}
  \end{eqnarray*}

  \



  \

  Trouvons la limite, sous réserve d'existence, de $\frac{\left(q^n \right)^\alpha}{(n!)^\beta}$ pour $q\in \mathbb{R}$ et $(\alpha,\beta)\in \left( \mathbb{R}_+^* \right)^2$ suivant $n$ en $+\infty$.

  \

  Remarquons que si $q\leq0$, il est \textbf{\textit{nécessaire}} d'avoir $\alpha\in\mathbb{Z}^*$ sinon l'expression n'a tout simplement \textbf{\textit{aucun sens}}. De fait, on supposera $q>0$ tout le long, les cas $q<0$ se font naturellement (convergence pour $q\in \mathbb{R_-}$).

  \

  Soit donc $0<q<1$, ce cas est immédiat, $\left( \left(q^n \right)^\alpha\right)_{n\in\mathbb{N}}=\left( \left(q^\alpha \right)^n\right)_{n\in\mathbb{N}}$, donc il s'agit de la suite géométrique de raison $q^\alpha \in ]0,1[$ et de premier terme $q^{\min_{I}(n)\alpha}$ ($\min_{I}(n)$, avec $I$ une partie non vide de $\mathbb{N}$, car la suite ne démarre pas forcément à $0$), donc elle converge vers $0$.

  \

  Si $q\geq 1$, on montre le cas trivial $\alpha = \beta =1$ :
  \[
    \forall n\in [\![ \lfloor q \rfloor +1,+\infty [\![, \quad 0 \leq \frac{q^n}{n!} = \underset{=\ \lambda \text{ (une constante)}}{\underbrace{\frac{q}{1} \times \frac{q}{2} \times \dots \times \frac{q}{\lfloor q \rfloor} }}\times \underset{\leq 1}{\underbrace{\frac{q}{\lfloor q\rfloor +1}}} \times \dots \times \underset{\leq1}{\underbrace{\frac{q}{n-1}}} \times \frac{q}{n} \leq \underset{\xrightarrow[n\to +\infty]{} \ 0}{\underbrace{\frac{\lambda q}{n}}}
  \]

  \

  Par théorème d'existence de limite par encadrement, $\left( \frac{q^n}{n!} \right)_{n\in \mathbb{N}}$ converge et sa limite est $0$.

  \

  Soient $(\alpha,\beta)\in \mathbb{R}^*_+$, montrons le cas général pour $q\geq 1$.

  \[
    \forall n \in \mathbb{N}, \quad \frac{(q^n)^\alpha}{(n!)\beta} = \left( \frac{\left(q^{\frac{\alpha}{\beta}}\right)^n}{n!} \right)^\beta = \underset{\underset{\text{par composition des limites }(\beta>0)}{\xrightarrow[n\to +\infty]{} 0}}{\underbrace{\left( \underset{\underset{\text{c'est le cas trivial}}{\xrightarrow[n\to +\infty]{} 0}}{\underbrace{\frac{\left(q^{\frac{\alpha}{\beta}}\right)^n}{n!}}} \right)^\beta}}
  \]

\end{question_kholle}

\begin{question_kholle}{Présentation exhaustive de la fonction $\arcsin$.}
  Premièrement, ladite fonction est la bijection réciproque de la fonction $\widetilde{\sin}$ (voir \textbf{1}.). D'où :
  \begin{equation*}
    \arcsin = \left\{
    \begin{array}{c c c}
      [-1,1] & \to     & [-\frac{\pi}{2} , \frac{\pi}{2}]        \\ [1ex]
      x      & \mapsto & \left( \widetilde{\sin} \right)^{-1}(x)
    \end{array}
    \right.
  \end{equation*}

  \

  Ainsi, pour $x\in [-1,1]$, $\arcsin (x)$ est l'unique solution de l'équation d'inconnue $\theta \in \textstyle \left[-\frac{\pi}{2} , \frac{\pi}{2}\right]$, $\sin(\theta) = x$.

  \

  \noindent Il découle alors naturellement des propriétés héréditairement acquises de $\widetilde{\sin}$ :

  \begin{enumerate}
    \item $\arcsin$ est impaire.
    \item $\arcsin$ est strictement croissante sur $[-1,1]$.
    \item $\arcsin \in \mathcal{C}^0\left([-1,1],[-\frac{\pi}{2} , \frac{\pi}{2}] \right)$.
    \item $\arcsin \in \mathcal{D}^1\left(]-1,1[,\left]-\frac{\pi}{2} , \frac{\pi}{2}\right[ \right)$.
    \item $\arcsin'(x) = \frac{1}{\sqrt{1-x^2}}$ pour tout $x\in]-1,1[$.
    \item $\arcsin$ admet deux demi-tangentes verticales en $-1$ et $1$.
  \end{enumerate}

  \

  Graphe de $\arcsin$ :
  \begin{center}
    \definecolor{ffttww}{rgb}{1,0.2,0.4}
    \definecolor{ttzzff}{rgb}{0.2,0.6,1}
    \definecolor{zzffzz}{rgb}{0.6,1,0.6}
    \begin{tikzpicture}[line cap=round,line join=round,>=triangle 45,x=1.910828025477707cm,y=1.910828025477707cm]
      \draw[->,color=black] (-1.57,0) -- (1.57,0);
      \foreach \x in {-1.5,-1,-0.5,0.5,1,1.5}
      \draw[shift={(\x,0)},color=black] (0pt,2pt) -- (0pt,-2pt) node[below] {\footnotesize $\x$};
      \draw[->,color=black] (0,-1.57) -- (0,1.57);
      \foreach \y in {-1.5,-1,-0.5,0.5,1,1.5}
      \draw[shift={(0,\y)},color=black] (2pt,0pt) -- (-2pt,0pt) node[left] {\footnotesize $\y$};
      \draw[color=black] (0pt,-10pt) node[right] {\footnotesize $0$};
      \clip(-1.57,-1.57) rectangle (1.57,1.57);
      \draw[color=zzffzz] plot[raw gnuplot, id=func0] function{set samples 100; set xrange [-1.57:1.57]; plot sin(x)};
      \draw[color=ttzzff] plot[raw gnuplot, id=func1] function{set samples 100; set xrange [-1.57:1.57]; plot asin(x)};
      \draw[color=ffttww] plot[raw gnuplot, id=func2] function{set samples 100; set xrange [-1.57:1.57]; plot x};
      \draw [line width=0.4pt,dash pattern=on 1pt off 1pt,domain=-1.57:1.57] plot(\x,{(--1.57-0*\x)/1});
      \draw [line width=0.4pt,dash pattern=on 1pt off 1pt] (1,-1.57) -- (1,1.57);
      \draw [line width=0.4pt,dash pattern=on 1pt off 1pt] (1.57,-1.57) -- (1.57,1.57);
      \draw [line width=0.4pt,dash pattern=on 1pt off 1pt,domain=-1.57:1.57] plot(\x,{(--1-0*\x)/1});
      \draw [line width=0.4pt,dash pattern=on 1pt off 1pt] (-1.57,-1.57) -- (-1.57,1.57);
      \draw [line width=0.4pt,dash pattern=on 1pt off 1pt] (-1,-1.57) -- (-1,1.57);
      \draw [line width=0.4pt,dash pattern=on 1pt off 1pt,domain=-1.57:1.57] plot(\x,{(-1-0*\x)/1});
      \draw [line width=0.4pt,dash pattern=on 1pt off 1pt,domain=-1.57:1.57] plot(\x,{(-1.57-0*\x)/1});
      \draw [->] (-1.57,-1) -- (-1.18,-1);
      \draw [->] (-1,-1.57) -- (-1,-1.16);
      \draw [->] (1,1.57) -- (1,1.16);
      \draw [->] (1.57,1) -- (1.12,1);
    \end{tikzpicture}

    \

    \textbf{Figure 2.} $\arcsin$ en bleu, $\widetilde{\sin}$ en vert et la première bissectrice en rouge.
  \end{center}

  \

  On a aussi, grâce au taux d'accroissement en 0 d'$\arcsin$ :
  \[
    \lim_{x\to0} \frac{\arcsin(x)}{x} \ = \ 1.
  \]

  \

  Puis finalement (visible sur le graphe) :
  \[
    \forall x \in [0,1], \quad \arcsin(x) \geq x.
  \]
\end{question_kholle}

\begin{question_kholle}{Présentation exhaustive de la fonction $\arccos$.}

  Premièrement, ladite fonction est la bijection réciproque de la fonction $\cos |_{[0,\pi]}^{[-1,1]} := \widetilde{\cos}$. D'où :
  \begin{equation*}
    \arccos = \left\{
    \begin{array}{c c c}
      [-1,1] & \to     & [0 , \pi]                               \\ [1ex]
      x      & \mapsto & \left( \widetilde{\cos} \right)^{-1}(x)
    \end{array}
    \right.
  \end{equation*}

  \

  Ainsi, pour $x\in [-1,1]$, $\arccos (x)$ est l'unique solution de l'équation d'inconnue $\theta \in \textstyle [0 ,\pi]$, $\cos(\theta) = x$.

  \noindent Il découle alors naturellement des propriétés héréditairement acquises de $\widetilde{\cos}$ :

  \begin{enumerate}
    \item $\arccos$ est strictement décroissante sur $[-1,1]$.
    \item $\arccos \in \mathcal{C}^0\left([-1,1],[0 , \pi] \right)$.
    \item $\arccos \in \mathcal{D}^1\left(]-1,1[,]0 ,\pi [ \right)$.
    \item $\arccos'(x) = -\frac{1}{\sqrt{1-x^2}}$ pour tout $x\in]-1,1[$.
    \item $\arccos$ admet deux demi-tangentes verticales en $-1$ et $1$.
  \end{enumerate}

  \

  Graphe de $\arccos$ :
  \begin{center}
    \definecolor{cczzff}{rgb}{0.8,0.6,1.0}
    \definecolor{qqffqq}{rgb}{0.0,1.0,0.0}
    \definecolor{ffqqqq}{rgb}{1.0,0.0,0.0}
    \definecolor{xfqqff}{rgb}{0.4980392156862745,0.0,1.0}
    \begin{tikzpicture}[line cap=round,line join=round,>=triangle 45,x=1.3636363636363635cm,y=1.3636363636363635cm]
      \draw[->,color=black] (-1.2,0.0) -- (3.2,0.0);
      \foreach \x in {-1.0,-0.5,0.5,1.0,1.5,2.0,2.5,3.0}
      \draw[shift={(\x,0)},color=black] (0pt,2pt) -- (0pt,-2pt) node[below] {\footnotesize $\x$};
      \draw[->,color=black] (0.0,-1.2) -- (0.0,3.2);
      \foreach \y in {-1.0,-0.5,0.5,1.0,1.5,2.0,2.5,3.0}
      \draw[shift={(0,\y)},color=black] (2pt,0pt) -- (-2pt,0pt) node[left] {\footnotesize $\y$};
      \draw[color=black] (0pt,-10pt) node[right] {\footnotesize $0$};
      \clip(-1.2,-1.2) rectangle (3.2,3.2);
      \draw[color=xfqqff] plot[raw gnuplot, id=func0] function{set samples 100; set xrange [0:3.14]; plot cos(x)};
      \draw[color=ffqqqq] plot[raw gnuplot, id=func1] function{set samples 100; set xrange [-1.2:3.2]; plot x};
      \draw[color=qqffqq] plot[raw gnuplot, id=func2] function{set samples 100; set xrange [-1:1]; plot acos(x)};
      \draw[dotted,color=cczzff] plot[raw gnuplot, id=func3] function{set samples 100; set xrange [-1.2:3.2]; plot 3.1415926535/2.0-x};
      \draw [dotted] (-1.0,0.0)-- (-1.0,3.141592653589793);
      \draw [dotted] (-1.0,3.141592653589793)-- (0.0,3.141592653589793);
      \draw [dotted] (0.0,1.0)-- (1.0,1.0);
      \draw [dotted] (1.0,0.0)-- (1.0,1.0);
      \draw [dotted] (0.0,-1.0)-- (3.141592653589793,-1.0);
      \draw [dotted] (3.141592653589793,0.0)-- (3.141592653589793,-1.0);
      \draw [->] (-1.0,3.141592653589793) -- (-1.0,2.6982051899765422);
      \draw [->] (0.0,1.0) -- (0.41398424427733616,1.0);
      \draw [->] (1.0,0.0) -- (1.0,0.41498464079523883);
      \draw [->] (3.141592653589793,-1.0) -- (2.697204793458636,-1.0);
    \end{tikzpicture}

    \

    \textbf{Figure 3.} $\arccos$ en vert, $\widetilde{\cos}$ en violet, la première bissectrice en rouge et $y = \frac{\pi}{2} - x$ en rose.
  \end{center}

\end{question_kholle}

\begin{question_kholle}{Présentation exhaustive de la fonction $\arctan$.}

  \

  Premièrement, ladite fonction est la bijection réciproque de la fonction $\tan |_{\left] -\frac{\pi}{2}, \frac{\pi}{2}\right[ }:=\widetilde{\tan}$. D'où :
        \begin{center}

          $\arctan = \left\{
            \begin{array}{c c c}
              \mathbb{R} & \to     & \left] -\frac{\pi}{2}, \frac{\pi}{2}\right[ \\ [1ex]
              x          & \mapsto & \left( \widetilde{\tan} \right)^{-1}(x)
            \end{array}
            \right.
          $
        \end{center}

        \

        Ainsi, pour $x\in \mathbb{R}$, $\arctan (x)$ est l'unique solution de l'équation d'inconnue $\theta \in \textstyle \left] -\frac{\pi}{2}, \frac{\pi}{2}\right[$, $\tan(\theta) = x$.

  \

  \noindent Il découle alors naturellement des propriétés héréditairement acquises de $\widetilde{\tan}$ :

  \begin{enumerate}
    \item $\arctan$ est impaire.
    \item $\arctan \in \mathcal{C}^0\left(\mathbb{R},\left] -\frac{\pi}{2}, \frac{\pi}{2}\right[ \right)$.
    \item $\arctan \in \mathcal{D}^1\left(\mathbb{R},\left] -\frac{\pi}{2}, \frac{\pi}{2}\right[ \right)$.
    \item $\arctan'(x) = \frac{1}{1+x^2}$ pour tout $x\in\mathbb{R}$.
  \end{enumerate}

  \newpage

  Graphe de $\arctan$ :
  \begin{center}
    \definecolor{ffqqqq}{rgb}{1.0,0.0,0.0}
    \definecolor{qqffqq}{rgb}{0.0,1.0,0.0}
    \definecolor{qqffff}{rgb}{0.0,1.0,1.0}
    \begin{tikzpicture}[line cap=round,line join=round,>=triangle 45,x=1.0cm,y=1.0cm]
      \draw[->,color=black] (-3.0,0.0) -- (3.0,0.0);
      \foreach \x in {-3.0,-2.5,-2.0,-1.5,-1.0,-0.5,0.5,1.0,1.5,2.0,2.5}
      \draw[shift={(\x,0)},color=black] (0pt,2pt) -- (0pt,-2pt) node[below] {\footnotesize $\x$};
      \draw[->,color=black] (0.0,-3.0) -- (0.0,3.0);
      \foreach \y in {-3.0,-2.5,-2.0,-1.5,-1.0,-0.5,0.5,1.0,1.5,2.0,2.5}
      \draw[shift={(0,\y)},color=black] (2pt,0pt) -- (-2pt,0pt) node[left] {\footnotesize $\y$};
      \draw[color=black] (0pt,-10pt) node[right] {\footnotesize $0$};
      \clip(-3.0,-3.0) rectangle (3.0,3.0);
      \draw[color=qqffff] plot[raw gnuplot, id=func0] function{set samples 100; set xrange [-1.56:1.56]; plot tan(x)};
      \draw[color=qqffqq] plot[raw gnuplot, id=func1] function{set samples 100; set xrange [-3:3]; plot atan(x)};
      \draw[color=ffqqqq] plot[raw gnuplot, id=func2] function{set samples 100; set xrange [-3:3]; plot x};
      \draw [domain=-3.0:3.0] plot(\x,{(--1.5707963267948966-0.0*\x)/1.0});
      \draw (1.5707963267948966,-3.0) -- (1.5707963267948966,3.0);
      \draw [domain=-3.0:3.0] plot(\x,{(-1.5707963267948966-0.0*\x)/1.0});
      \draw (-1.5707963267948966,-3.0) -- (-1.5707963267948966,3.0);
      \draw [dotted] (0.7853981633974483,1.0)-- (0.7853981633974483,0.0);
      \draw [dotted] (0.0,0.7853981633974483)-- (1.0,0.7853981633974483);
      \draw [dotted] (0.0,1.0)-- (1.0,1.0);
      \draw [dotted] (1.0,0.0)-- (1.0,1.0);
      \draw [dotted] (-1.0,0.0)-- (-1.0,-1.0);
      \draw [dotted] (0.0,-1.0)-- (-1.0,-1.0);
      \draw [dotted] (-0.7853981633974483,0.0)-- (-0.7853981633974483,-1.0);
      \draw [dotted] (0.0,-0.7853981633974483)-- (-1.0,-0.7853981633974483);
      \begin{scriptsize}
        \draw[color=qqffff] (-3.1177254400173355,-0.014744648606977613) node {$f$};
        \draw[color=qqffqq] (-3.1177254400173355,-1.3072326284088318) node {};
        \draw[color=ffqqqq] (-1.8026940652189338,-1.8332451783281911) node {};
        \draw[color=black] (-3.1177254400173355,1.5106917461591645) node {$a$};
        \draw[color=black] (1.4660982092799508,2.322253966034747) node {};
        \draw[color=black] (-3.1177254400173355,-1.4500074633869435) node {$c$};
        \draw[color=black] (-1.4946010002661654,2.322253966034747) node {};
      \end{scriptsize}
    \end{tikzpicture}

    \

    \textbf{Figure 4.} $\arctan$ en vert, $\widetilde{\tan}$ en bleu, la première bissectrice en rouge, et les fonctions $y = \pm \frac{\pi}{2}$ et $x = \pm \frac{\pi}{2}$ en noir.
  \end{center}

  \

  On a aussi (visible sur le graphe) :
  \[
    \forall x \in \mathbb{R}_+, \quad \arctan(x) \leq x.
  \]

  Et enfin :
  \[
    \forall x \in \mathbb{R}^*, \quad \arctan(x) + \arctan \left( \frac{1}{x} \right) =
    \left\{ \begin{array}{cl}
      \frac{\pi}{2}  & \text{si } x \ > \ 0  \\
      -\frac{\pi}{2} & \text{si } x \ < \ 0.
    \end{array} \right.
  \]

\end{question_kholle}

%-------------------------------------------------------------
\begin{question_kholle}{$2$ preuves de $\arcsin(x) + \arccos(x) =\frac{\pi}{2}$ sur $[-1,1]$, dont une basée sur une interprétation géométrique du cercle trigonométrique.}

  L'interprétation géométrique sur $[0,1]$, celle sur $[-1,0]$ est laissée au lecteur car il s'agit du même principe modulo des détails :

  \begin{center}
    \definecolor{eqbqff}{rgb}{0.8784313725490196,0.6901960784313725,1.0}
    \definecolor{xfqqff}{rgb}{0.4980392156862745,0.0,1.0}
    \definecolor{uuuuuu}{rgb}{0.26666666666666666,0.26666666666666666,0.26666666666666666}
    \definecolor{ffqqqq}{rgb}{1.0,0.0,0.0}
    \definecolor{qqffff}{rgb}{0.0,1.0,1.0}
    \definecolor{cqcqcq}{rgb}{0.7529411764705882,0.7529411764705882,0.7529411764705882}
    \begin{tikzpicture}[line cap=round,line join=round,>=triangle 45,x=2.5cm,y=2.5cm]
      \clip(-1.2,-1.2) rectangle (1.2,1.2);
      \fill[fill=black,fill opacity=1.0] (0.08956860342780414,0.20707624336357297) -- (0.08145277478981197,0.18983010750783963) -- (0.0742401101509475,0.2039730262575521) -- cycle;
      \fill[fill=black,fill opacity=1.0] (0.3604343842207928,0.11272973544691409) -- (0.3441578360574464,0.12526320821437803) -- (0.33811585546631434,0.10562838538867093) -- cycle;
      \draw [color=cqcqcq] (0.0,0.0) circle (2.5cm);
      \draw [shift={(0.0,0.0)},line width=1.2000000000000002pt,color=qqffff]  plot[domain=0.0:3.141592653589793,variable=\t]({1.0*1.0*cos(\t r)+-0.0*1.0*sin(\t r)},{0.0*1.0*cos(\t r)+1.0*1.0*sin(\t r)});
      \draw [shift={(3.061616997868383E-17,0.0)},line width=1.2000000000000002pt,color=ffqqqq]  plot[domain=-1.5707963267948966:1.5707963267948963,variable=\t]({1.0*1.0*cos(\t r)+-0.0*1.0*sin(\t r)},{0.0*1.0*cos(\t r)+1.0*1.0*sin(\t r)});
      \draw [shift={(0.0,-0.0)},line width=1.2000000000000002pt,color=xfqqff]  plot[domain=0.0:1.5707963267948966,variable=\t]({1.0*1.0*cos(\t r)+-0.0*1.0*sin(\t r)},{0.0*1.0*cos(\t r)+1.0*1.0*sin(\t r)});
      \draw [line width=1.2000000000000002pt,color=qqffff] (-1.0589461703485412,0.0)-- (-0.9410538296514588,0.0);
      \draw [line width=1.2000000000000002pt,color=qqffff] (1.0589461703485412,0.0)-- (0.9410538296514588,0.0);
      \draw [line width=1.2000000000000002pt,color=ffqqqq] (6.484175189933444E-17,1.0589461703485412)-- (5.762292801540088E-17,0.9410538296514588);
      \draw [line width=1.2000000000000002pt,color=ffqqqq] (-1.945252556980033E-16,-1.0589461703485412)-- (-1.7286878404620264E-16,-0.9410538296514588);
      \draw [line width=1.2000000000000002pt,color=ffqqqq] (6.484175189933444E-17,1.0589461703485412)-- (0.05398708109469719,1.0589461703485412);
      \draw [line width=1.2000000000000002pt,color=ffqqqq] (5.762292801540088E-17,0.9410538296514588)-- (0.05398708109469719,0.9410538296514588);
      \draw [line width=1.2000000000000002pt,color=ffqqqq] (6.484175189933444E-17,-1.0589461703485412)-- (0.05398708109469719,-1.0589461703485412);
      \draw [line width=1.2000000000000002pt,color=ffqqqq] (5.762292801540088E-17,-0.9410538296514588)-- (0.05398708109469719,-0.9410538296514588);
      \draw [line width=1.2000000000000002pt,color=qqffff] (-1.0589461703485412,1.2968350379866888E-16)-- (-1.0589461703485412,0.05398708109469725);
      \draw [line width=1.2000000000000002pt,color=qqffff] (-0.9410538296514588,1.1524585603080177E-16)-- (-0.9410538296514588,0.05398708109469724);
      \draw [line width=1.2000000000000002pt,color=qqffff] (0.9410538296514588,1.1524585603080177E-16)-- (0.9410538296514588,0.05398708109469724);
      \draw [line width=1.2000000000000002pt,color=qqffff] (1.0589461703485412,1.2968350379866888E-16)-- (1.0589461703485412,0.05398708109469725);
      \draw [dash pattern=on 1pt off 1pt on 3pt off 4pt] (6.123233995736766E-17,1.0)-- (0.0,-1.0);
      \draw [dash pattern=on 1pt off 1pt on 3pt off 4pt] (-1.0,0.0)-- (1.0,-0.0);
      \draw[color=eqbqff] plot[raw gnuplot, id=func0] function{set samples 100; set xrange [-1.0999999999999999:1.0999999999999999]; plot x};
      \draw (0.3420201433256688,0.9396926207859083)-- (0.0,-0.0);
      \draw (0.0,-0.0)-- (0.9396926207859084,0.3420201433256687);
      \draw [shift={(0.0,-0.0)}] plot[domain=0.0:0.3490658503988659,variable=\t]({1.0*0.36624511935574344*cos(\t r)+-0.0*0.36624511935574344*sin(\t r)},{0.0*0.36624511935574344*cos(\t r)+1.0*0.36624511935574344*sin(\t r)});
      \draw [shift={(0.0,-0.0)}] plot[domain=0.0:1.2217304763960306,variable=\t]({1.0*0.21706356072793254*cos(\t r)+-0.0*0.21706356072793254*sin(\t r)},{0.0*0.21706356072793254*cos(\t r)+1.0*0.21706356072793254*sin(\t r)});
      \draw (0.08956860342780414,0.20707624336357297)-- (0.08145277478981197,0.18983010750783963);
      \draw (0.08145277478981197,0.18983010750783963)-- (0.0742401101509475,0.2039730262575521);
      \draw (0.0742401101509475,0.2039730262575521)-- (0.08956860342780414,0.20707624336357297);
      \draw (0.3604343842207928,0.11272973544691409)-- (0.3441578360574464,0.12526320821437803);
      \draw (0.3441578360574464,0.12526320821437803)-- (0.33811585546631434,0.10562838538867093);
      \draw (0.33811585546631434,0.10562838538867093)-- (0.3604343842207928,0.11272973544691409);
      \draw [dash pattern=on 3pt off 3pt] (5.938595898438009E-17,0.9396926207859083)-- (0.3420201433256688,0.9396926207859083);
      \draw [dash pattern=on 3pt off 3pt] (4.108751682287631E-17,0.34202014332566877)-- (0.9396926207859084,0.3420201433256687);
      \draw [dash pattern=on 3pt off 3pt] (0.3420201433256688,0.9396926207859083)-- (0.3420201433256688,-0.0);
      \draw [dash pattern=on 3pt off 3pt] (0.9396926207859084,0.3420201433256687)-- (0.9396926207859084,-0.0);
      \draw (0.3106196735830377,0.20690669566269085) node[anchor=north west] {arccos(x)};
      \draw (0.06201603682796799,0.3536235960427321) node[anchor=north west] {arcsin(x)};
      \draw (-0.18658759992710172,0.9853213615679096) node[anchor=north west] {x};
      \draw (0.8241288249131816,0.0031332229126335774) node[anchor=north west] {x};
      \begin{scriptsize}
        \draw [fill=uuuuuu] (0.0,-0.0) circle (1.5pt);
        \draw[color=uuuuuu] (-0.011342413362052578,0.09279355092265877) node {$\Omega$};
        \draw [fill=uuuuuu] (5.938595898438009E-17,0.9396926207859083) circle (1.5pt);
        \draw [fill=uuuuuu] (0.9396926207859084,-0.0) circle (1.5pt);
      \end{scriptsize}
    \end{tikzpicture}

    \

    \textbf{Figure 5.}
  \end{center}

  \

  Preuve formelle :

  \

  Soit $x\in [-1,1]$. Posons  $\varphi \ = \ \arcsin(x) \in \left[-\frac{\pi}{2},\frac{\pi}{2}\right]$. Ainsi :

  \[
    \arcsin(x) + \arccos(x) \ = \ \varphi + \arccos(\sin(\varphi)) \ = \ \varphi + \arccos \left( \cos \left( \frac{\pi}{2}- \varphi \right) \right),
  \]

  \

  or $\varphi \in \left[-\frac{\pi}{2},\frac{\pi}{2}\right]$ donc
  $\frac{\pi}{2}- \varphi \in [0,\pi]$ d'où $\arccos \left( \cos \left( \frac{\pi}{2}- \varphi \right) \right) = \frac{\pi}{2}- \varphi$ si bien que :
  \[
    \arcsin(x) + \arccos(x) \ = \  \varphi +\frac{\pi}{2} - \varphi \ = \ \frac{\pi}{2}.
  \]

\end{question_kholle}

\begin{question_kholle}{Présentation analytique rapide des fonctions \(\cosh\) et \(\sinh \).}
  ~\smallbreak

  \begin{itemize}[label=$\bullet$]
    \item Domaine de définition et symétries.
          \newline
          $\sinh$ et $\cosh$ sont définies sur $\mathbb{R}$.
          \newline
          De plus,
          \newline
          $(i)$ $\forall x \in \mathbb{R}$, $-x\in \mathbb{R}$,
          \newline
          $(ii)$ $\forall x \in \mathbb{R}$,
          $
            \left\{ \begin{array}{c c c c c c c}
              \sinh (-x) & = & \frac{e^{-x} - e^{x}}{2}     & = & - \frac{e^x - e^{-x}}{2} & = & -\sinh(x) \\
              \text{et}  &   &                              &   &                          &   &           \\
              \cosh (-x) & = & \frac{e^{-x} + e^{-(-x)}}{2} & = & \frac{e^x + e^{-x}}{2}   & = & \cosh(x).
            \end{array}
            \right.
          $
          \newline
          Donc $\sinh$ et $\cosh$ sont respectivement impaire et paire.
          \newline
          Nous les étudierons sur $\mathbb{R}_+$ et pour les obtenir les graphes $(\mathcal{C}_{\sinh} \text{ et } \mathcal{C}_{\cosh})$ de ces fonctions sur $\mathbb{R}$ à partir de ceux $(\mathcal{C}_{\sinh}^+ \text{ et } \mathcal{C}_{\cosh}^+)$ obtenus sur $\mathbb{R}_+$, nous le complèterons en traçant les images de ces graphes par la symétrie centrale $s$ de centre $O$ et par la réflexion $r$ d'axe $\left( O, \overrightarrow{\jmath} \right)$ :
          \[
            \mathcal{C}_{\sinh} = \mathcal{C}_{\sinh}^+ \cup s \left( \mathcal{C}_{\sinh}^+ \right) \qquad \text{ et } \qquad \mathcal{C}_{\cosh} = \mathcal{C}_{\cosh}^+ \cup r \left( \mathcal{C}_{\cosh}^+ \right)
          \]

          \

    \item Variations : triviales.

          \

    \item Branches infinies en $+\infty$ et position relative de $\mathcal{C}_{\sinh}$ et $\mathcal{C}_{\cosh}$.

          \[
            \frac{\cosh(x)}{x} = \underset{\xrightarrow[x\to +\infty]{} \ +\infty}{\underbrace{\frac{e^{x}}{x}}} + \underset{\xrightarrow[x\to +\infty]{} \ 0}{\underbrace{\frac{e^{-x}}{x}}} \xrightarrow[x\to +\infty]{} \ +\infty
          \]
          Donc le graphe de $\cosh$ admet une branche parabolique de direction asymptotique $\left( O, \overrightarrow{\jmath}\right)$.
          \newline
          On a :
          \[
            \forall x \in \mathbb{R}, \quad \cosh(x) - \sinh(x) = e^{-x} \xrightarrow[x\to +\infty]{} 0^+
          \]
          Donc les graphes des deux fonctions se rapprochent l'un de l'autre arbitrairement près lorsque $x \to +\infty$, et le graphe de $\cosh$ est au-dessus de celui de $\sinh$.

          \

    \item Tangente au graphe de $\sinh$ à l'origine et position relative.
  \end{itemize}

  Il s'agira d'étudier $g : x\in \mathbb{R}_+ \mapsto \sinh(x) -x$, de remarquer sa dérivabilité d'en étudier les variations puis de conclure, en précisant que cette étude révèle l'inflexion du graphe de $\sinh$ en 0.
\end{question_kholle}

\end{document}
\documentclass{article}

\date{17 avril 2024}
\usepackage[nb-sem=1, auteurs={George Ober}]{../kholles}

\usepackage{pgfplots}
\usetikzlibrary{intersections,angles,quotes}
\usepgfplotslibrary{fillbetween}
%\pgfplotsset{compat=1.17}

\begin{document}

\maketitle


\begin{question_kholle}{Preuve formelle de la somme des entiers et des termes d'une suite géométrique} 
    Soit $n \in \mathbb{N}$ fq. Posons $$S_{n} = \sum_{k=0}^{n}k$$
    En posant la symétrie d'indice $i = n-k$, on a aussi
    $$
    S_{n}= \sum_{i=0}^{n}(n-i)=(n \times \mathrm{card}[ \! [ 0, n ] \!]) - \sum_{i=0}^{n} i
    $$
    Or, puisque $\mathrm{card}[ \! [ 0, n ] \!] = n + 1$ et que $\sum_{i=0}^{n} i = S_{n}$
    $$
    S_{n} = n \times (n+1) + S_{n}
    $$
    Donc
    $$
    S_{n} = \frac{n(n+1)}{2}
    $$
    Soient $q \in \mathbb{R}$ , $k \in \mathbb{N}$ fixés quelconques.
    Si $q = 1$, 
    $$
    \sum_{i=0}^{k}q^{i} = \sum_{i=0}^{k}1 = k+1
    $$
    Avec l'identité algébrique, on a
    $$
    q^{k+1}-1^{k+1} = (q-1) \sum_{i=0}^{k}q^{i}\times 1^{k-i}
    $$
    Ainsi, si $q \neq 1$ on a, par multiplication par $(q-1)^{-1}$
    $$
    \sum_{i=0}^{k}q^{i} = \frac{q^{k+1}-1}{q-1}
    $$
    Nous avons donc établi que
    $$
    \sum_{i=0}^{k} q^{i} = \left\{ \begin{array}{ll}
        \frac{1-q^{k+1}}{1-q}  &\text{ si } q \neq 1 \\
        k + 1 &\text{ sinon}
    \end{array}\right.
    $$
\end{question_kholle}

\begin{question_kholle}{Preuve de la factorisation de $a^n - b^n$ puis de celle de $a^{2m+1} + b^{2m+1} $}
    Calculons 
    
    \begin{align*}
        (a-b)\sum_{k=0}^{m-1}a^{k}b^{m-1-k} 
        &=a \sum_{k=0}^{m-1}a^{k}b^{m-1-k} -b \sum_{k=0}^{m-1}a^{k}b^{m-1-k} \\
        &= \sum_{k=0}^{m-1}a^{k+1}b^{m-1-k} - \sum_{k=0}^{m-1}a^{k}b^{m-k} \\
    \end{align*}
    
    Si bien qu'en posant le changement d'indice $j = k + 1$ on reconnait le téléscopage.
    $$
    \sum_{j=1}^{m}a^{j}b^{m-j} - \sum_{k=0}^{m-1}a^{k}b^{m-k} = a^{m} - b ^{m}
    $$
\end{question_kholle}
\begin{question_kholle}[
    {$$\left( \sum_{k=1}^{n}x_{k} \right)^{2} = \sum_{\substack{1\leqslant k \leqslant n\\ 1 \leqslant j \leqslant n}} x_{k}x_{j} = 2 \sum_{1\leqslant k < j \leqslant n} x_{k}x_{j}+\sum_{k=1}^{n}x_{i}^{2}$$}
    ]{Développement d'une somme}
    $$
    \left( \sum_{k=1}^{n}x_{k} \right)^{2} = \sum_{\substack{1\leqslant k \leqslant n\\ 1 \leqslant j \leqslant n}} x_{k}x_{j} = 2 \sum_{1\leqslant k < j \leqslant n} x_{k}x_{j}+\sum_{k=1}^{n}x_{i}^{2}
    $$
    
    
    \begin{align*}
        \left( \sum_{k=0}^{n}x_{k} \right)^{2}
        &= \left( \sum_{k=1}^{n} x_{k} \right) \times \left( \sum_{j=1}^{n} x_{j} \right) \\
        &= \sum_{k=1}^{n}\left[ x_{k} \times \sum_{j=1}^{n}x_{j} \right] \\
        &= \sum_{k=1}^{n}\left( \sum_{j=1}^{n} x_{k}\times x_{j} \right) \\
        &= \sum_{\substack{1\leqslant k \leqslant n\\ 1 \leqslant j \leqslant n}} x_{k}x_{j}
    \end{align*}
    
    
    On peut aussi séparer cette somme
    
    \begin{align*}
        \sum_{\substack{1\leqslant k \leqslant n\\ 1 \leqslant j \leqslant n}} x_{k}x_{j}
        &=
        \sum_{\substack{1\leqslant k \leqslant n\\ 1 \leqslant j \leqslant n \\ k<j}} x_{k}x_{j} &+
        \sum_{\substack{1\leqslant k \leqslant n\\ 1 \leqslant j \leqslant n\\k=j}} x_{k}x_{j} &+
        \sum_{\substack{1\leqslant k \leqslant n\\ 1 \leqslant j \leqslant n\\k>j}} x_{k}x_{j}\\
        &=
        \sum_{\substack{1\leqslant k \leqslant n\\ 1 \leqslant j \leqslant n \\ k<j}} x_{k}x_{j} & +
        \underbrace{ \sum_{k=1}^{n}x_{k}^{2} }_{ \text{somme sur les indices }(k,j) \text{ tels que } k = j } &+
        \sum_{\substack{1\leqslant k \leqslant n\\ 1 \leqslant j \leqslant n\\k>j}} x_{k}x_{j}
    \end{align*}
    
    On remarque aussi qu'en permutant les indices des deux sommes (les variables sont muettes)
    $$
    \sum_{\substack{1\leqslant k \leqslant n\\ 1 \leqslant j \leqslant n \\ k<j}} x_{k}x_{j} = \sum_{\substack{1\leqslant j \leqslant n\\ 1 \leqslant k \leqslant n\\j<k}} x_{j}x_{k}
    $$
    Qui, par commutativité du produit dans $\mathbb{C}$ nous donne cette égalité
    $$
    \sum_{\substack{1\leqslant k \leqslant n\\ 1 \leqslant j \leqslant n \\ k<j}} x_{k}x_{j} =
    \sum_{\substack{1\leqslant k \leqslant n\\ 1 \leqslant j \leqslant n\\k>j}} x_{k}x_{j}
    $$
    On a donc bien l'identité attendue :
    $$
    \sum_{\substack{1\leqslant k \leqslant n\\ 1 \leqslant j \leqslant n}} x_{k}x_{j} = 2 \sum_{\substack{1\leqslant k \leqslant n\\ 1 \leqslant j \leqslant n \\ k<j}} x_{k}x_{j} +
    \sum_{k=1}^{n}x_{k}^{2}
    $$
\end{question_kholle}

\begin{question_kholle}{Preuve de la formule du binôme de Newton}
    
\end{question_kholle}

\begin{question_kholle}{Montrer que tout entier $n > 2$ admet un diviseur premier}
    Raisonnons par récurrence forte avec la propriété $\mathcal{P}(\cdot)$ définie pour tout $n > 2$ par
    $$
    \mathcal{P}(n) : « \forall k \in [ \! [ 2, n ] \!], k \text{ admet un diviseur premier } »
    $$
    \begin{itemize}
        \item Initialisation: $n \leftarrow 2$
        
        Soit $k \in [ \! [ 2, 2 ] \!]$ fixé quelconque. Nécéssairement, $k = 2$.
        or, $2$ admet $2$ pour diviseur premier.
        
        Donc $\forall k \in [ \! [ 2, 2 ] \!], k \text{ admet un diviseur premier}$, ce qui prouve $\mathcal{P}(2)$.
        
        \item Hérédité: Soit $n \in \mathbb{N} \setminus \{ 1, 0 \}$ fixé quelconque tel que $\mathcal{P}(n)$ est vraie.
        
        Pour montrer $\mathcal{P}(n+1)$, il nous faudra montrer que $\forall k \in [ \! [ 2, n+1 ] \!], k \text{ admet un diviseur premier }$
        
        Soit $k \in [ \! [ 2, n+1 ] \!]$ fixé quelconque.
        \begin{itemize}
            \item    Si $k \in [ \! [ 2, n ] \!]$, alors la véracité de $\mathcal{P}(n)$ nous permet de conclure, et de dire que k admet un diviseur premier.
            
            \item Sinon $k = n + 1$
            \begin{itemize}
                \item Si $n+1$ est premier, alors il admet $k$ comme diviseur premier
                \item Sinon, $\exists d \in [ \! [ 2, n ] \! ]: d \mid n+1$ 
                
                Mais, puisque $d \in [ \! [ 2, n ] \! ]$, la véracité de $\mathcal{P}(n)$ nous permet d'affirmer que $d$ admet un diviseur premier $p$. Donc par transitivité de la relation de divisibilité $$(p \mid d \text{ et } d \mid n) \implies p \mid n$$
            \end{itemize}
        \end{itemize}
    \end{itemize}
\end{question_kholle}

\begin{question_kholle}{Montrer par récurrence qu'une fonction polynomiale à coefficients réels est nulle si et seulement si tous ses coefficients sont nuls}
    
\end{question_kholle}
\begin{question_kholle}{Montrer par analyse/synthèse qu'une fonction réelle d'une variable réelle s'écrit de manière unique comme somme d'une fonction paire et d'une fonction impaire}
    Soit $f \in \mathcal{F}(\mathbb{R}, \mathbb{R})$ fixée quelconque.
    \begin{itemize}
        
        
        \item \underline{Analyse}:
        Supposons que $f : \mathbb{R} \to \mathbb{R}$ se décompose de manière unique en $f = g+h$ avec $g$ paire et $h$ impaire (i.e. $\forall x \in \mathbb{R}, g(-x)= g(x)$ et $h(-x)= -h(x)$).
        Soit $x \in \mathbb{R}$ fixé quelconque
        Calculons $f(-x)$:
        $$f(-x) = g(-x) + h(-x) = g(x) -h(x)$$
        Par demi somme, nous avons donc
        $$
        \left\{ \begin{array}{ll}
            2g(x)  = f(x)+f(-x) \\
            2h(x) = f(x)-f(-x)
        \end{array}\right.
        $$
        Ainsi, si une telle décomposition existe, c'est
        $$
        \left\{ \begin{array}{ll}
            g : x \mapsto \frac{f(x)+f(-x)}{2}  \\
            h : x \mapsto \frac{f(x)-f(-x)}{2}
        \end{array}\right.
        $$
        
        \item \underline{Synthèse}:
        Posons
        
        \begin{align} 
            g\left|\begin{array}{ll} \mathbb{R} &\to \mathbb{R} \\ x &\mapsto \frac{f(x)+f(-x)}{2} \end{array}\right.
            \text{ et } h \left|\begin{array}{ll} \mathbb{R} &\to \mathbb{R} \\ x &\mapsto \frac{f(x)-f(-x)}{2} \end{array}\right.
        \end{align}
        
        
        Remarquons, d'une part que:
        $$
        \forall x \in \mathbb{R}, g(x)+h(x) = \frac{f(x)+f(-x)}{2} + \frac{f(x)-f(-x)}{2} = f(x)
        $$
        Vérifions si les fonctions $g$ et $h$ vérifient les conditions de parité:
        $$
        \forall x \in \mathbb{R}, g(-x) = \frac{f(-x)+f(-(-x))}{2}= \frac{f(x)+f(-x)}{2} = g(x) \text{ ainsi } g \text{ est paire.}
        $$
        $$
        \forall x \in \mathbb{R}, h(-x) = \frac{f(-x)-f(-(-x))}{2} = -\frac{f(x)-f(-x)}{2} = -h(x) \text{ ainsi } h \text{ est impaire}
        $$
    \end{itemize}
\end{question_kholle}
\begin{question_kholle}{Illustration graphique de certaines identités trigonométriques}
    
\end{question_kholle}
\begin{question_kholle}{Technique de résolution des équations trigonométriques du type $A \cos x + B \sin x = C$}
    
\end{question_kholle}
\begin{question_kholle}{Étude complète de la fonction tangente, tracé du graphe et en déduire celui de cotangente.}
    
\end{question_kholle}
\begin{question_kholle}{Expression $\sin \theta$, $\cos \theta$, $\tan \theta$ en fonction de $\tan \frac \theta 2$}

\end{question_kholle}
\begin{question_kholle}{Preuve des formules du type $\cos p + \cos q = \dots$}
    
\end{question_kholle}
\end{document}
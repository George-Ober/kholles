\documentclass{article}

\date{27 Avril 2024}
\usepackage[nb-sem=25, auteurs={George Ober}]{../kholles}

\begin{document}
\maketitle

\begin{question_kholle}
	[Soit $A \in \mathcal{M}_{n}(\mathbb{K})$
	\begin{itemize}
		\item S'il existe $B \in \mathcal{M}_{n}(\mathbb{K}):A\times B = I_{n}$, alors $A\in GL_{n}(\mathbb{K})$ et $A^{-1}=B$
		\item S'il existe $B \in \mathcal{M}_{n}(\mathbb{K}):B \times A = I_{n}$, alors $A\in GL_{n}(\mathbb{K})$ et $A^{-1}=B$
	\end{itemize}
	]
	{S'il existe un inverse à droite (ou à gauche) pour une matrice carrée, alors celle ci est inversible}
	
	
	Supposons $\exists B \in \mathcal{M}_{n}(\mathbb{K}):A\times B = I_{n}$. Notons $(\hat{a}, \hat{b}) \in \mathcal{L}(\mathbb{K}^{n})$ les endomorphismes canoniquement associés à $A$ et à $B$.
	
	
	\begin{align*}
		\Phi_{\mathcal{B}_{\text{can } \mathbb{K}^{n}}}(\hat{a} \circ \hat{b}) &= \mathrm{mat}(\hat{a} \circ  \hat{b}, \mathcal{B}_{\text{can } \mathbb{K}^{n}}, \mathcal{B}_{\text{can } \mathbb{K}^{n}}) \\
		&= \mathrm{mat}(\hat{a}, \mathcal{B}_{\text{can } \mathbb{K}^{n}}, \mathcal{B}_{\text{can } \mathbb{K}^{n}}) \times_{\mathcal{M_{n}(\mathbb{K})}} \mathrm{mat}(\hat{b}, \mathcal{B}_{\text{can } \mathbb{K}^{n}}, \mathcal{B}_{\text{can } \mathbb{K}^{n}}) \\
		&= A \times B \\
		&= I_{n} \\
		&= \mathrm{mat}(\mathrm{Id}_{\mathbb{K}^{n}}, \mathcal{B}_{\text{can } \mathbb{K}^{n}}, \mathcal{B}_{\text{can } \mathbb{K}^{n}}) = \Phi_{\mathcal{B}_{\text{can } \mathbb{K}^{n}}}(\mathrm{Id}_{\mathbb{K}^{n}})
	\end{align*}
	
	
	D'où, par injectivité de $\Phi_{\mathcal{B}_{\text{can } \mathbb{K}^{n}}}$, $\hat{a} \circ \hat{b} = \mathrm{Id}_{\mathbb{K}^{n}}$.
	
	Ainsi, $\hat{a} \circ \hat{b}$ est surjective, donc $\hat{a}$ est surjective, mais par l'accident de la dimension finie, $\hat{a}$ est bijective, donc c'est un automorphisme, donc toutes ses matrices associées sont inversibles. On effectue un même raisonnement pour l'inversibilité à gauche, en utilisant cette fois l'injectivité.
\end{question_kholle}
\end{document}

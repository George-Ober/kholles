\documentclass{article}

\usepackage{../kholles}

\begin{document}
	\maketitle
	
	\begin{question_kholle}
		[\noindent Soient $(A, B) \in \mathcal{P}(\R)^2$ fq
  \begin{align*}
      A \text{ est dense dans } B \iff \left\{\begin{array}{ll} A \subset B \\\mathrm{et}\\ \forall b \in B, \forall \varepsilon \in \R_+^*, \exists a \in A: |b-a|< \varepsilon \end{array}\right.
  \end{align*}
  ]
		{Caractérisation de la densité d’une partie $A$ de \R dans une partie $B$ de \R la contenant avec des $\varepsilon$ (si la densité de A dans B est définie par $A \subset B$ et tout intervalle ouvert de \R qui rencontre B rencontre A).}
		
		\textit{Montrons la caractérisation de la densité}\\
		\emph{Sens Direct} Supposons $A$ dense dans $B$
		\begin{itemize}[label=\textemdash]
            \item Par déf $A \subset B$
            \item Soit $b \in B$ et $\varepsilon \in \R_+^*$ fq

            Appliquons le (ii) de la déf de Densité pour $u \leftarrow b - \varepsilon$ et $v \leftarrow b + \varepsilon$ 
            $$B \cap ]b - \varepsilon, b + \varepsilon[ \neq \emptyset \implies A \cap ]b - \varepsilon,  b + \varepsilon[ \neq \emptyset$$
            Or, $B \cap ]b - \varepsilon, b + \varepsilon[ \neq \emptyset$ est vraie
            Donc $A \cap ]b - \varepsilon,  b + \varepsilon[ \neq \emptyset$

            Ce qui permet de choisir $a \in A \cap ]b - \varepsilon,  b + \varepsilon[$
            Un tel $a$ vérifie $a \in A$ et $a \in ]b - \varepsilon,  b + \varepsilon[ \iff |b-a| < \varepsilon$
		\end{itemize}
		\bigbreak
    	\emph{Sens réciproque} Supposons \left\{\begin{array}{ll} A \subset B \\\mathrm{et}\\ \forall b \in B, \forall \varepsilon \in \R_+^*, \exists a \in A: |b-a|< \varepsilon \end{array}\right.

        \begin{itemize}
            \item On a donc $A \subset B$
            \item Soient $(u, v) \in \R^2$ fq tq $B \cap ]u, v[ \neq \emptyset$

            Appliquons l'hypothèse pour $b\leftarrow b$ et $\varepsilon \leftarrow \min\{v - b, b - u\}$, qui est autorisé $v-b$ et $b-u$ sont positifs

            Donc $\exists a \in A: | b - a| < \varepsilon $
            
            Fixons un tel a, alors:
            $$
                b-\varepsilon < a < b + \varepsilon
            $$

            Donc $$
            \left\{\begin{array}{ll}
            a < b + \varepsilon = b + \min\{v - b, b - u\} \leqslant b + v - b = v \\ \mathrm{et}\\
            a > b - \varepsilon = b - \underbrace{\min\{v - b, b - u\}}_{\leqslant b - u} \geqslant b - (b - u) = u
            \end{array}\right.
            $$

            Donc $a \in ]u, v[$.    
        \end{itemize}
        Donc $A \cap ]u, v[ \neq \emptyset$
		
	\end{question_kholle}

	\begin{question_kholle}[
        Soit u $\in \K ^ \N, (\ell_1, \ell_2) \in \K ^2$
        Si u converge vers $\ell_1$ et $\ell_2$, alors $\ell_1 = \ell_2$
    ]{Preuve de l'unicité de la limite d'une suite convergente}
    Par l'absurde, supponsons que $u$ converge vers $\ell_1$ et $\ell_2$, et $\ell_1 \neq \ell_2$.
    On prendra $\varepsilon_0 = \varepsilon_1 = \varepsilon_2$ assez petit pour que les tubes soient disjoints.\\
    Posons donc $\varepsilon_0 = \frac{|\ell_1 - \ell_2|}{3}$
    \begin{itemize}
        \item Appliquons la définition de la convergence de u vers $\ell_1$, pour $\varepsilon \leftarrow \varepsilon_0$, ce qui est autorisé car $\varepsilon_0 \in \R_+^*$
        \begin{equation}\label{eq:1}
            \exists N_1 \in \N : \forall n \in \N, n \geqslant N_1 \implies |u_n - \ell_1| \leqslant \varepsilon_0
        \end{equation}
        \begin{equation}\label{eq:2}
            \exists N_2 \in \N : \forall n \in \N, n \geqslant N_2 \implies |u_n - \ell_2| \leqslant \varepsilon_0
        \end{equation}
        Fixons de tels $N_1$ et $N_2$.
        \item Posons $n_0 = N_1 + N_2$
        \begin{itemize}
            \item $n_0 \geqslant N_1$, donc (\ref{eq:1}) s'applique: $|u_{n_0} - \ell_1| \leqslant \varepsilon_0$
            \item $n_0 \geqslant N_2$, donc (\ref{eq:2}) s'applique: $|u_{n_0} - \ell_2| \leqslant \varepsilon_0$
        \end{itemize}
        \item \begin{align*}
            |\ell_1 - \ell_2| &= |\ell_1 - u_{n_0} + u_{n_0} - \ell_2|\\
            &\leqslant \underbrace{|\ell_1 - u_{n_0}|}_{\leqslant \varepsilon_0} + \underbrace{|u_{n_0} - \ell_2|}_{\leqslant \varepsilon_0}\\
            &\leqslant 2 \frac{|\ell_1 - \ell_2|}{3}\\
            \implies 1 &\leqslant \frac 2 3
        \end{align*}
        Contradiction
    \end{itemize}
	\end{question_kholle}
\end{document}

\documentclass{article}

\date{3 décembre 2023}
\usepackage{../kholles}

\begin{document}
	\maketitle

	\begin{question_kholle}
		[\noindent Soient $(A, B) \in \mathcal{P}(\R)^2$ fq. \\
		\textit{Définition de la densité}
		\begin{align}
			A \text{ est dense dans } B
			\text{ si } \left\{ \begin{array}{ll}
				A \subset B \\
				\mathrm{et} \\
				\forall (u,v) \in \R^2, B \cap {]}u;v{[} \neq \emptyset \implies A \cap  {]}u;v{[} \neq \emptyset
			\end{array} \right.
		\end{align}
		\textit{Caractérisation de la densité par les $\varepsilon$}
		\begin{align}
			A \text{ est dense dans } B
		 	\iff \left\{ \begin{array}{ll}
				A \subset B \\
				\mathrm{et} \\
				\forall b \in B, \forall \varepsilon \in \R_+^*, \exists a \in A: |b-a|< \varepsilon
			\end{array} \right.
		\end{align}
		]
		{Caractérisation de la densité d’une partie $A$ de \R dans une partie $B$ de \R la contenant avec des $\varepsilon$.}

		\textit{Montrons la caractérisation de la densité}\\
		\emph{Sens Direct} Supposons $A$ dense dans $B$
		\begin{itemize}[label=\textemdash]
            \item Par déf $A \subset B$
            \item Soit $b \in B$ et $\varepsilon \in \R_+^*$ fq

            Appliquons le (ii) de la déf de Densité pour $u \leftarrow b - \varepsilon$ et $v \leftarrow b + \varepsilon$
            $$B \cap ]b - \varepsilon, b + \varepsilon[ \neq \emptyset \implies A \cap ]b - \varepsilon,  b + \varepsilon[ \neq \emptyset$$
            Or, $B \cap ]b - \varepsilon, b + \varepsilon[ \neq \emptyset$ est vraie
            donc $A \cap ]b - \varepsilon,  b + \varepsilon[ \neq \emptyset$

            Ce qui permet de choisir $a \in A \cap ]b - \varepsilon,  b + \varepsilon[$.
            Un tel $a$ vérifie $a \in A$ et $a \in ]b - \varepsilon,  b + \varepsilon[ \iff |b-a| < \varepsilon$
		\end{itemize}
		\bigbreak
    	\noindent \emph{Sens réciproque} Supposons $\left\{\begin{array}{ll} A \subset B \\\mathrm{et}\\ \forall b \in B, \forall \varepsilon \in \R_+^*, \exists a \in A: |b-a|< \varepsilon \end{array}\right.$

        \begin{itemize}
            \item On a donc $A \subset B$
            \item Soient $(u, v) \in \R^2$ fq tq $B \cap ]u, v[ \neq \emptyset$

			Soit $b \in B \cap ]u, v[$ fq.
            Appliquons l'hypothèse pour $b\leftarrow b$ et $\varepsilon \leftarrow \min\{v - b, b - u\}$, qui est autorisé $v-b$ et $b-u$ sont positifs

            Donc $\exists a \in A: | b - a| < \varepsilon $

            Fixons un tel a, alors:
            $$
                b-\varepsilon < a < b + \varepsilon
            $$

            Donc $$
            \left\{\begin{array}{ll}
            a < b + \varepsilon = b + \underbrace{\min\{v - b, b - u\}}_{\leqslant v - b} \leqslant b + v - b = v \\ \mathrm{et}\\
            a > b - \varepsilon = b - \underbrace{\min\{v - b, b - u\}}_{\leqslant b - u} \geqslant b - (b - u) = u
            \end{array}\right.
            $$

            Donc $a \in ]u, v[$.
        \end{itemize}
        Donc $A \cap ]u, v[ \neq \emptyset$

	\end{question_kholle}

	\begin{question_kholle}
		[\begin{equation}
			\forall (a, b) \in \R \times \R^*,
			\exists ! (q, r) \in \Z \times \R :
			\left\{ \begin{matrix}
				a = b q + r \\
				r \in [0;|b|[
			\end{matrix} \right.
		\end{equation}]
		{Théorème de la division pseudo-euclidienne dans \R}

		\textit{Unicité} \;
		Soient deux tels entiers $(a,b) \in \R^2$ et deux couples $((q,r),(q',r')) \in \left(\Z \times \R\right)^2$ tels que
		\begin{equation*}
			\left\{ \begin{matrix}
				a = b q + r \\
				r \in [0;|b|[
			\end{matrix} \right.
			\qquad
			\left\{ \begin{matrix}
				a = b q' + r' \\
				r' \in [0;|b|[
			\end{matrix} \right.
		\end{equation*}
		Directement,
		\[
		b(q-q') = r'-r,
		\]
		mais comme $-|b| < r' - r < |b|$, il vient en divisant par $|b|$ l'inégalité précédente :
		\[
		-1 < q - q' < 1,
		\]
		puisque $q$ et $q'$ sont dans $\Z$ leur différence est obligatoirement $0$, ainsi $q = q'$ ce qui implique $ r= r'$ et donc on a unicité de ladite écriture de $a$.
		\newline
		\\
		\textit{Existence} \; Posons pour $b > 0$, $\Omega = \{ k\in \Z  \ | \ kb \leqslant a \}$
		\begin{itemize}
			\item $\Omega \subset \Z$
			\item non-vide car $-|a| \in \Omega$ ($\Z$ archimédien suffit \ldots)
			\item $\Omega$ est majoré par $|a|$ car supposons, par l'absurde, que $\exists k \in \Omega : k > |a|$, alors $kb > |a|b > a$ ce qui contradiction avec la définition d'$\Omega$.
		\end{itemize}
		Donc $\Omega$ admet un plus grand élément, notons-le $q$. \\
		Posons $r = a - bq$. Par construction, $a = bq + r$ et comme $q = \max \Omega$ et $r \in \R$.
		\\
		Par suite, $q \in \Omega$ donc $bq \leqslant a$ d'où $0 \leqslant r$. Et $q = \max \Omega$ donc $b(q+1) > a$ d'où $b > r$, c'est-à-dire, $r \in [ 0, |b| [$.

		Si $b < 0$, il suffit de prendre $q \leftarrow -q$ dans la preuve précédente.C'est donc l'existence de ladite écriture de $a$.
	\end{question_kholle}

	\begin{question_kholle}
		{\Q est dense dans \R et $\R \setminus \Q$ est aussi dense dans \R}

		Soit $x \in \R$ fq.
		Posons $\forall n \in \N, a_n = \frac{\lfloor2^n x\rfloor}{2^n}$. \\
		Soit $n \in \N$ fq. \\
		\begin{itemize}
			\item $a_n \in \Q$ car $\lfloor2^n x\rfloor \in \Z$ et $2^n \in \N$.
			\item \begin{equation*}
					a_n = \frac{\lfloor2^n x\rfloor}{2^n}
					\implies \frac{2^n x - 1}{2^n} \leqslant a_n \leqslant \frac{2^n x}{2^n}
					\implies x - \frac{1}{2^n} \leqslant a_n \leqslant x
				\end{equation*}
				Or $\nicefrac{1}{2^n} \arrowlim{n}{+\infty} 0$ donc d'après le théorème d'existence de limite par encadrement, \\ $a_n \arrowlim{n}{+\infty} x$.
		\end{itemize}
		Donc d'après la caractérisation séquentielle de la densité, \Q est dense dans \R.
		\bigbreak

		\noindent Soit $x \in \R$ fq. \\
		Alors $x + \sqrt{2} \in \R$.
		D'après la démonstration précédente, $\exists b \in \Q^\N : b_n \arrowlim{n}{+\infty} x + \sqrt{2}$. \\
		Fixons un telle suite $b$.
		Considérons $c = b - \sqrt{2}$. \\
		Soit $n \in \N$ fq.
		\begin{itemize}
			\item $c_n \in \R\setminus\Q$ car $b_n \in \Q$ et $\sqrt{2} \in \R \setminus \Q$.
			\item \begin{equation*}
				\left. \begin{matrix}
					b_n \arrowlim{n}{+\infty} x + \sqrt{2} \\
					c_n = b_n - \sqrt{2}
				\end{matrix} \right\}
				\implies c_n \arrowlim{n}{+\infty} x
			\end{equation*}
		\end{itemize}
		Donc d'après la caractérisation séquentielle de la densité, $\R\setminus \Q$ est dense dans \R.
	\end{question_kholle}

	\begin{question_kholle}[
        Soit u $\in \K ^ \N, (\ell_1, \ell_2) \in \K ^2$
        Si u converge vers $\ell_1$ et $\ell_2$, alors $\ell_1 = \ell_2$
    ]{Preuve de l'unicité de la limite d'une suite convergente}
    Par l'absurde, supponsons que $u$ converge vers $\ell_1$ et $\ell_2$, et $\ell_1 \neq \ell_2$.
    On prendra $\varepsilon_0 = \varepsilon_1 = \varepsilon_2$ assez petit pour que les tubes soient disjoints.\\
    Posons donc $\varepsilon_0 = \frac{|\ell_1 - \ell_2|}{3}$
    \begin{itemize}
        \item Appliquons la définition de la convergence de u vers $\ell_1$, pour $\varepsilon \leftarrow \varepsilon_0$, ce qui est autorisé car $\varepsilon_0 \in \R_+^*$
        \begin{equation}\label{eq:1}
            \exists N_1 \in \N : \forall n \in \N, n \geqslant N_1 \implies |u_n - \ell_1| \leqslant \varepsilon_0
        \end{equation}
        \begin{equation}\label{eq:2}
            \exists N_2 \in \N : \forall n \in \N, n \geqslant N_2 \implies |u_n - \ell_2| \leqslant \varepsilon_0
        \end{equation}
        Fixons de tels $N_1$ et $N_2$.
        \item Posons $n_0 = N_1 + N_2$
        \begin{itemize}
            \item $n_0 \geqslant N_1$, donc (\ref{eq:1}) s'applique: $|u_{n_0} - \ell_1| \leqslant \varepsilon_0$
            \item $n_0 \geqslant N_2$, donc (\ref{eq:2}) s'applique: $|u_{n_0} - \ell_2| \leqslant \varepsilon_0$
        \end{itemize}
        \item \begin{align*}
            |\ell_1 - \ell_2| &= |\ell_1 - u_{n_0} + u_{n_0} - \ell_2|\\
            &\leqslant \underbrace{|\ell_1 - u_{n_0}|}_{\leqslant \varepsilon_0} + \underbrace{|u_{n_0} - \ell_2|}_{\leqslant \varepsilon_0}\\
            &\leqslant 2 \frac{|\ell_1 - \ell_2|}{3}\\
            \implies 1 &\leqslant \frac 2 3
        \end{align*}
        Contradiction
    \end{itemize}
	\end{question_kholle}

	\begin{question_kholle}{Une suite convergente est bornée}

        Soit $u \in \mathbb{K}^{\mathbb{N}}$ convergente.
Posons $\ell = \lim u$
Appliquons la définition de la convergence pour $\varepsilon \leftarrow 1$
$$
\exists N_{1}\in \mathbb{N}: \forall n \in \mathbb{N}, n \geqslant N_{1} \implies |u_{n}-\ell| \leqslant 1
$$
Fixons un tel $N_{1}$
Posons alors $M = \max\left\{ |u_{0}|, |u_{1}|, |u_{2}| \dots |u_{N_{1}}|, |\ell|+1 \right\}$, qui est bien défini, car toute partie finie, non vide d'un ensemble totalement ordonné (ici $(\mathbb{R}, \leqslant)$) admet un pgE.

Soit $n \in \mathbb{N}$ fq.
\begin{itemize}
    \item Si $n \in [[0, N_{1}]], |u_{n}| \in \left\{ |u_{0}|, |u_{1}|, |u_{2}| \dots |u_{N_{1}}|, |\ell|+1 \right\}$ donc $|u_{n}| \leqslant M$
	\item Sinon,
\end{itemize}

\begin{align*}
n> N_{1} &\implies |u_{n} - \ell| \leqslant 1 \\
&\implies |u_{n}| - |\ell| \leqslant 1 \\
 & \implies |u_{n}| \leqslant 1+ |\ell| \leqslant M
\end{align*}

Ainsi, $\forall n \in \mathbb{N}, |u_{n}| \leqslant M$.
    \end{question_kholle}
\end{document}

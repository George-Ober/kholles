\documentclass{article}

\date{18 avril 2024}
\usepackage[nb-sem=5, auteurs={Kylian Boyet, George Ober}]{../kholles}


\begin{document}

\maketitle

\begin{question_kholle}{Montrer que l'ensemble des similitudes directes du plan complexe est un groupe pour la composition} 
	Montrons donc que $(S, \circ)$ est un sous groupe de $(\mathcal{S}(\mathbb{C}), \circ)$
	\begin{itemize}[label=$\lozenge$]
		\item D'une part, $S \subset \mathcal{S}(\mathbb{C})$. Or l'ensemble des permutations $(\mathcal{S}(\mathbb{C}), \circ)$ est un groupe. En effet, les similitudes sont des bijections de $\mathbb{C} \to \mathbb{C}$.
		\item De plus, $S$ est non vide, par exemple l'application $\mathrm{Id(\mathbb{C}})$ est une similitude pour $a \leftarrow 1$ et $b \leftarrow 1$.
		\item Prenons finalement $a$ et $c$ dans $\mathbb{C}^*$ puis $b$ et $d$ dans $\mathbb{C}$. et posons les deux applications suivantes:
		\begin{align*}
			s \left| \begin{array}{ll}
				\mathbb{C} &\to \mathbb{C} \\
				z &\mapsto a z + b
			\end{array}\right.
			\text{ et }
			s' \left| \begin{array}{ll}
				\mathbb{C} &\to \mathbb{C} \\
				z &\mapsto a z + b
			\end{array}\right.
		\end{align*}
		Ainsi, comme toute similitude directe est une bijection, en particulier $s'$ en est une, et
		\begin{align*}
			s'^{-1} \left| \begin{array}{ll}
				\mathbb{C} &\to \mathbb{C} \\
				z &\mapsto \frac{z}{c} - \frac{d}{c}
			\end{array}\right.
		\end{align*}
		Soit $z \in \mathbb{C}$ fixé quelconque:
		\begin{align*}
			(s \circ s'^{-1})(z) &= s(s'^{-1}(z)) \\
			&= s\left(\frac{z}{c} - \frac{d}{c}\right)\\
			&= a\left(\frac{z}{c} - \frac{d}{c}\right) + b\\
			&= \frac{a}{c}z + \left( b - \frac{ad}{c} \right)
		\end{align*}
		Qui est une similitude directe, puisque $\frac{a}{c} \neq  0$ donc $s \circ s'^{-1} \in S$. Donc $(S, \circ)$ est bien un sous-groupe de $(\mathcal{S}(\mathbb{C}), \circ)$.
	\end{itemize}
\end{question_kholle}

\begin{question_kholle}{Classifier et interpréter une similitude directe donnée sous la forme $z \mapsto a z + b$ sur un exemple, donner l'expression complexe d'une similitude dont on connaît les éléments caractéristiques.}
	Soient $(a, b) \in \mathbb{C}^*$ fixés quelconques. Posons la similitude
	\begin{align*}
		s \left| \begin{array}{ll}
			\mathbb{C} &\to \mathbb{C} \\
			z &\mapsto a z + b
		\end{array}\right.
	\end{align*}
	\begin{itemize}[label=$\lozenge$]
		\item Si $a = 1$, c'est la translation de vecteur d'affixe $b$
		\item Si $a \neq 1$, $s$ admet un unique point fixe appelé "centre de la similitude" $\omega = \frac{b}{1-b}$
		\begin{itemize}[label=$\star$]
			\item Si $a \in \mathbb{R}^*$, $s$ est l'homotétie de centre $\omega$ et de rapport $a$.
			\item Si $a \in (\mathbb{C}^* \setminus \mathbb{R})$, $s$ est la composée de
			\begin{itemize}
				\item La rotation de centre $\omega$ et d'angle $\alpha$, où $\alpha$ est un argument de $a$.
				\item L'homotétie de centre $\omega$ et de rapport $|a|$. 
			\end{itemize}
			On nommera alors $|a|$ le rapport de $s$ et $\alpha$ une mesure de l'angle de $s$.
		\end{itemize}
	\end{itemize}
	\textbf{Exemple} : Prenons la similitude $s: z \mapsto (1 - i) z - 1$.
	\begin{align*}
		s(z) = z &\iff (1 - i)z - 1 = z\\
		&\iff -iz = 1\\
		&\iff z = i
	\end{align*}
	De plus,
	\begin{align*}
		(1 - i)z - 1 = \sqrt{2}e^{-i \frac{\pi}{4}}z - 1
	\end{align*}
	On en déduit donc que $s$ est la similitude directe de centre d'affixe $i$, de rapport $\sqrt{2}$, et d'angle $-\frac{\pi}{4}$.
\end{question_kholle}
\begin{question_kholle}{Montrer qu'une combinaison linéaire de deux fonctions bornées (respectivement lipschitziennes) est bornée (resp. lipschitzienne)}
	Soit $I$ un intervalle réel.
	
	Soient $f$ et $g$ deux fonctions de $I$ dans $\mathbb{R}$. Soient $(\lambda, \mu) \in \mathbb{R}^2$
	\begin{itemize}[label=$\lozenge$]
		\item Si $f$ et $g$ sont respectivement bornées par $A$ et par $B$.
		Soit $x \in I$.
		\begin{align*}
			\Big| (\lambda.f + \mu.g)(x) \Big| &= \Big| \lambda.f(x) + \mu.g(x) \Big| \\
			&\leqslant \big| \lambda \big|  \big|f(x)\big| + \big| \mu \big|  \big| g(x) \big| \\
			&\leqslant \big| \lambda \big| A + \big| \mu \big| B
		\end{align*}
		Donc $\lambda.f + \mu.g$ est bornée.
		\item Si $f$ et $g$ sont respectivement $K$ et $L$ lipschitziennes.
		Soient $(x, y) \in I^2$.
		\begin{align*}
			\Big| (\lambda.f + \mu.g)(x) - (\lambda.f + \mu.g)(y)\Big| &= \Big| \lambda.f(x) + \mu.g(x) - \lambda.f(y) - \mu.g(y) \Big| \\
			&= \Big| \lambda(f(x) - f(y)) + \mu(g(x) - g(y)) \Big| \\
			&\leqslant \Big| \lambda  \Big|  \Big| f(x) - f(y) \Big| + \Big| \mu \Big|  \Big|g(x) - g(y) \Big|  \\
			&\leqslant \Big| \lambda  \Big|  K \Big| x-y \Big| + \Big| \mu  \Big| L  \Big| x - y \Big|  \\
			& \leqslant (|\lambda| K + |\mu|L ) |x - y|
		\end{align*}
	\end{itemize}
\end{question_kholle}
\end{document}

\documentclass{article}

\date{24 Mars 2024}
\usepackage[nb-sem=22, auteurs={Hugo Vangilluwen}]{../kholles}

\begin{document}
	\maketitle

	Pour cette semaine, \K désigne un corps commutatif, $E$ et $F$ des \K\!\!-espaces vectoriels, $E'$ et $F'$ des sous-espaces vectoriels respectivement de $E$ et de $F$, $I$ un ensemble quelconque non vide.

	\begin{question_kholle}
		[Une famille est liée si et seulement si l'un de ses vecteurs est une combinaison linéaires d'autres vecteurs de la famille.
		\begin{equation}
			(x_i)_{i \in I} \text{ est liée}
			\iff \exists i_0 \in I : \exists (\lambda_i)_{i \in I\setminus\{i_0\}} \in \K^{\left( I \setminus \{i_0\} \right)} :
			x_{i_0} = \sum_{\substack{i \in I \\ i \neq i_0}} \lambda_i \ldotp x_i
		\end{equation}]
		{Caractérisation d'une famille liée}

		Supposons que $(x_i)_{i \in I}$ est liée. \\
		Par définition, $\displaystyle \exists (\mu_i) \K^{(I)} :
		\left\{ \begin{array}{ccc}
			\sum_{i \in I} \mu_i x_i &= O_E \\
			(_mu_i)_{i \in I} &\neq (0_\K)_{i \in I}
		\end{array} \right.$ \\
		Donc $\exists i_0 \in I : \mu_{i_0} \neq 0_\K$. Fixons un tel $i_0$. \\
		$\displaystyle \mu_{i_0} x_{i_0} + \sum_{i \in I \setminus \{i_0\}} \mu_i x_i = 0_E$ \\
		Or $\mu_{i_0} \neq 0$, donc $\displaystyle x_{i_0} = \sum_{i \in I \setminus \{i_0\}} \left( \mu_{i_0}^{-1} \times (-\mu_i) \right) \ldotp x_i$. \\
		En posant $\lambda_i = \mu_{i_0}^{-1} \times (-\mu_i)$, on obtient $\displaystyle \sum_{i \in I \setminus \{i_0\}} \lambda_i \ldotp x_i$.

		Supposons $\exists i_0 \in I : \exists (\lambda_i)_{i \in I\setminus\{i_0\}} \in \K^{\left( I \setminus \{i_0\} \right)} :
		x_{i_0} = \sum_{\substack{i \in I \\ i \neq i_0}} \lambda_i \ldotp x_i$. \\
		Alors $\displaystyle -x_{i_0} + \sum_{\substack{i \in I \\ i \neq i_0}} \lambda_i \ldotp x_i = 0_E$. \\
		Posons $\mu_{i_0} = - 1_\K$ et $\forall i \in I \!\setminus\! \{i_0\}, \mu_i = \lambda_i$. \\
		Ainsi, $(\mu_i)_{i \in I} \in \K^{(I)}$ et $\sum_{i \in I} \mu_i \ldotp x_i$. Or $\mu_{i_0} \neq 0_\K$ donc $(\mu_i)_{i \in I} \neq (0_\K)_{i \in I}$. \\
		Donc $(\mu_i)_{i \in I}$ est liée.
		
		\begin{equation*}
			\begin{matrix}
				(i) &\!\!\!\implies\!\!\!& (ii) \\
				\Uparrow & & \Downarrow \\
				(iv) &\!\!\!\impliedby\!\!\!& (iii) \\
			\end{matrix}
		\end{equation*}
	\end{question_kholle}

	\begin{question_kholle}
		[Soit $\mathcal{F}$ une famille de vecteurs de $E$. Les propositions suivantes sont équivalentes :
		\begin{propositions}
			\item $\mathcal{F}$ est une base.
			\item Tout vecteur de $E$ se décompose de manière unique dans $\mathcal{F}$ .
			\item $\mathcal{F}$ est génératrice minimale (au sens de l'inclusion)
			\item $\mathcal{F}$ est libre maximale (au sens de l'inclusion)
		\end{propositions}]
		{Caractérisations d'une base}

		Notons $(e_i)_{i \in I}$ la famille $\mathcal{F}$.


		$(i) \implies (ii)$ Supposons que $\mathcal{F}$ est une base de $E$. \\
		Soit $x \in E$ \fq. Montrons que $x$ s'écrit de manière unique comme une combinaison linéaire des vecteurs de $\mathcal{F}$. \\
		$\mathcal{F}$ est une base donc elle est une famille génératrice et libre de $E$. La propriété génératrice donne, par définition, l'existence d'une telle écriture tandis que la propriété libre donne l'unicité d'une telle écriture.

		$(ii) \implies (iii)$ Supposons que tout vecteur de $E$ s'écrit de manière unique comme une combinaison linéaire de vecteurs de $\mathcal{F}$. \\
		L'existence d'un telle décomposition permet d'affirmer que $\mathcal{F}$ est génératrice. \\
		Supposons que $\mathcal{F}$ ne soit pas génératrice minimale c'est-à-dire qu'il existe une famille $\mathcal{F}'$ de vecteurs de $E$ telle que $\mathcal{F}' \subsetneq \mathcal{F}$ et $\mathcal{F}'$ engendre $E$. \\
		Alors $\exists i_0 \in I : e_{i_0} \notin \mathcal{F}'$. Comme $\mathcal{F}'$ est génératrice, $\exists (\lambda_i)_{i \in I \setminus \{i_0\}} \in \K^{(I \setminus \{i_0\})} : e_{i_0} = \sum_{\substack{i \in I \\ i \neq i_0}} \lambda_i \ldotp e_i$.
		Donc \begin{equation*}
			\begin{aligned}
				e_{i_0} &= 0_\K \ldotp e_{i_0} + \sum_{\substack{i \in I \\ i \neq i_0}} \lambda_i \ldotp e_i \\
				e_{i_0} &= 1_\K \ldotp e_{i_0} + \sum_{\substack{i \in I \\ i \neq i_0}} 0_\K \ldotp e_i
			\end{aligned}
		\end{equation*}
		$e_{i_0}$ peut donc s'écrire de deux manières différentes au moins comme combinaison linéaire de vecteurs de $\mathcal{F}$ ce qui contredit le caractère libre de $\mathcal{F}$. \\
		Par conséquent, $\mathcal{F}$ est génératrice et minimale parmi les familles génératrices.

		$(iii) \implies (iv)$ Supposons que $\mathcal{F}$ est une famille génératrice minimale.
		Par l'absurde, supposons que $\mathcal{F}$ est liée. Alors il existe un $i_0 \in I$ tel que $e_{i_0}$ s'écrit comme une combinaison linéaire d'autres vecteurs de $\mathcal{F}$ donc $(e_i)_{i \in I \setminus \{i_0\}}$ est génératrice de $E$ . Or cette famille est strictement incluse dans $\mathcal{F}$ ce qui contredit la propriété de génératrice minimale. \\
		Donc $\mathcal{F}$ est libre. \\
		Par l'absurde, supposons que $\mathcal{F}$ n'est pas libre maximale c'est-à-dire qu'il existe une famille $\mathcal{F}'$ de vecteurs de $E$ telle que $\mathcal{F} \subsetneq \mathcal{F}'$ et $\mathcal{F}'$ est libre. \\
		Alors $\exists x \in \mathcal{F}' : x \notin \mathcal{F}$. Or $\mathcal{F}$ est génératrice d'où :
		\begin{equation*}
			\exists (\lambda_i)_{i \in I} \in \K^{(I)} :
			x = \sum_{i \in I} \lambda_i \ldotp x_i
			= 0_\K \ldotp x + \sum_{i \in I} \lambda_i \ldotp x_i + \sum_{\substack{y \in \mathcal{F}' \\ y \notin \mathcal{F} \\ y \neq x}} 0_\K \ldotp y
		\end{equation*}
		Puisque $x \in \mathcal{F}'$,
		\begin{equation*}
			\exists (\lambda_i)_{i \in I} \in \K^{(I)} :
			x = 1_\K \ldotp x + \sum_{i \in I} 0_\K \ldotp x_i + \sum_{\substack{y \in \mathcal{F}' \\ y \notin \mathcal{F} \\ y \neq x}} 0_\K \ldotp y
		\end{equation*}
		Donc $x$ s'écrit de deux manières différentes au moins comme combinaison linéaire de vecteurs $\mathcal{F}'$, ce qui contredit la liberté de $\mathcal{F}'$. \\
		Par conséquent, $\mathcal{F}$ est libre maximale.

		$(iv) \implies (i)$ Supposons que $\mathcal{F}$ est une famille libre maximale. \\
		Par hypothèse même, $\mathcal{F}$ est libre.
		Par l'absurde, supposons que $\mathcal{F}$ n'est pas génératrice. Alors il existe $x \in E$ tel que $x \notin \Vect \mathcal{F}$. Donc $\mathcal{F} \wedge \{x\}$ est libre et contient strictement $\mathcal{F}$, ce qui contredit la propriété de liberté maximale. \\
		Par conséquent, $\mathcal{F}$ est aussi génératrice, donc une base.
	\end{question_kholle}
	
	\begin{question_kholle}
		[Soit $f \in \mathcal{L}_\K(E, F)$.
		\begin{equation}
			\begin{aligned}
				\Ker{f} &= \left\{ x \in E \;|\; f(x) = 0_F \right\} = f^{-1} (\{0_F\}) \\
				\Im f &= \left\{ y \in F \;|\; \exists x \in E : f(x) = y \right\}
			\end{aligned}
		\end{equation}
		Nous démontrerons le résultat plus général suivant :
		\begin{propositions}
			\item $f(E')$ est un \sev de $F$.
			\item $f^{-1}(F')$ est un \sev de $E$.
		\end{propositions}
		]
		{Le noyau et l'image d'une application linéaire sont des \sevs}
		
		$(i)$ $0_E \in E'$ et $f(0_E) = 0_F$ donc $0_F \in f(E')$ d'où $f(E') \neq \emptyset$ \\
		Soit $(\alpha, \beta, y, y') \in \K^2 \times f(E')^2$ \fqs. \\
		Par définition, $\exists (x, x') \in E'^2 : f(x) = y \wedge f(x') = y$.
		\begin{equation*}
			\begin{aligned}
				\alpha y + \beta y'
				&= \alpha f(x) + \beta f(x') \\
				&= f( \alpha x + \beta x' ) \quad \text{ car } f \in \mathcal{L}_\K(E, F) \\
				&\in f(E') \quad \text{ car } \alpha x + \beta x' \in E' \text{ puisque } E' \text{ est un \sev}
			\end{aligned}
		\end{equation*}
		Donc $f(E')$ est un \sev.
		
		$(ii)$ $0_F \in F'$ et $f(0_E) = 0_F$ donc $0_E \in f^{-1}(F')$ d'où $f(F') \neq \emptyset$ \\
		Soit $(\alpha, \beta, x, x') \in \K^2 \times f^{-1}(F')^2$ \fqs. \\
		Par définition, $\exists (y, y') \in F'^2 : f(x) = y \wedge f(x') = y$. \\
		Or $F'$ est \sev donc $\alpha y + \beta y' \in F'$. $f \in \mathcal{L}_\K(E, F)$ d'où $f(\alpha x + \beta x') = \alpha y + \beta y'$. Donc $\alpha x + \beta x' \in f^{-1}(F')$. \\
		Ainsi, $f^{-1}(F')$ est un \sev.
		
		En appliquant pour $E' = E$ et $F' = \{0_F\}$, nous obtenons que \Ker{f} et \Im $f$ sont des \sevs.
	\end{question_kholle}

\end{document}

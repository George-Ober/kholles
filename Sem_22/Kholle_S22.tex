\documentclass{article}

\date{25 Mars 2024}
\usepackage[nb-sem=22, auteurs={Hugo Vangilluwen, George Ober, Kylian Boyet}]{../kholles}

\begin{document}
\maketitle

Pour cette semaine, \K désigne un corps commutatif, $E$ et $F$ des \K\!\!-espaces vectoriels, $E'$ et $F'$ des sous-espaces vectoriels respectivement de $E$ et de $F$, $I$ un ensemble quelconque non vide.

\begin{question_kholle}
	[Une famille est liée si et seulement si l'un de ses vecteurs est une combinaison linéaires d'autres vecteurs de la famille.
		\begin{equation}
			(x_i)_{i \in I} \text{ est liée}
			\iff \exists i_0 \in I : \exists (\lambda_i)_{i \in I\setminus\{i_0\}} \in \K^{\left( I \setminus \{i_0\} \right)} :
			x_{i_0} = \sum_{\substack{i \in I \\ i \neq i_0}} \lambda_i \ldotp x_i
		\end{equation}]
	{Caractérisation d'une famille liée}

	Supposons que $(x_i)_{i \in I}$ est liée. \\
	Par définition, $\displaystyle \exists (\mu_i) \K^{(I)} :
		\left\{ \begin{array}{ccc}
			\sum_{i \in I} \mu_i x_i & = 0_E                 \\
			(\mu_i)_{i \in I}        & \neq (0_\K)_{i \in I}
		\end{array} \right.$ \\
	Donc $\exists i_0 \in I : \mu_{i_0} \neq 0_\K$. Fixons un tel $i_0$. \\
	$\displaystyle \mu_{i_0} x_{i_0} + \sum_{i \in I \setminus \{i_0\}} \mu_i x_i = 0_E$ \\
	Or $\mu_{i_0} \neq 0$, donc $\displaystyle x_{i_0} = \sum_{i \in I \setminus \{i_0\}} \left( \mu_{i_0}^{-1} \times (-\mu_i) \right) \ldotp x_i$. \\
	En posant $\lambda_i = \mu_{i_0}^{-1} \times (-\mu_i)$, on obtient $x_{i_0} = \displaystyle \sum_{i \in I \setminus \{i_0\}} \lambda_i \ldotp x_i$.

	Supposons maintenant que $\exists i_0 \in I : \exists (\lambda_i)_{i \in I\setminus\{i_0\}} \in \K^{\left( I \setminus \{i_0\} \right)} :
		x_{i_0} = \sum_{\substack{i \in I \\ i \neq i_0}} \lambda_i \ldotp x_i$. \\
	Alors $-x_{i_0} + \sum_{\substack{i \in I \\ i \neq i_0}} \lambda_i \ldotp x_i = 0_E$.
	Posons $\mu_{i_0} = - 1_\K$ et $\forall i \in I \!\setminus\! \{i_0\}, \mu_i = \lambda_i$.
	Ainsi, $(\mu_i)_{i \in I} \in \K^{(I)}$ et $\sum_{i \in I} \mu_i \ldotp x_i = 0_\K$. Or $\mu_{i_0} \neq 0_\K$ donc $(\mu_i)_{i \in I} \neq (0_\K)_{i \in I}$. \\
	Donc $(\mu_i)_{i \in I}$ est liée.
\end{question_kholle}

\begin{question_kholle}
	[Soit $\mathcal{F}$ une famille de vecteurs de $E$. Les propositions suivantes sont équivalentes :
		\begin{propositions}
			\item $\mathcal{F}$ est une base.
			\item Tout vecteur de $E$ se décompose de manière unique dans $\mathcal{F}$.
			\item $\mathcal{F}$ est génératrice minimale (au sens de l'inclusion)
			\item $\mathcal{F}$ est libre maximale (au sens de l'inclusion)
		\end{propositions}]
	{Caractérisations d'une base}

	Notons $(e_i)_{i \in I}$ la famille $\mathcal{F}$.


	$(i) \implies (ii)$ Supposons que $\mathcal{F}$ est une base de $E$. \\
	Soit $x \in E$ \fq. Montrons que $x$ s'écrit de manière unique comme une combinaison linéaire des vecteurs de $\mathcal{F}$. \\
	$\mathcal{F}$ est une base donc elle est une famille génératrice et libre de $E$. La propriété génératrice donne, par définition, l'existence d'une telle écriture tandis que la propriété libre donne l'unicité d'une telle écriture.

	$(ii) \implies (iii)$ Supposons que tout vecteur de $E$ s'écrit de manière unique comme une combinaison linéaire de vecteurs de $\mathcal{F}$. \\
	L'existence d'un telle décomposition permet d'affirmer que $\mathcal{F}$ est génératrice. \\
	Supposons que $\mathcal{F}$ ne soit pas génératrice minimale c'est-à-dire qu'il existe une famille $\mathcal{F}'$ de vecteurs de $E$ telle que $\mathcal{F}' \subsetneq \mathcal{F}$ et $\mathcal{F}'$ engendre $E$. \\
	Alors $\exists i_0 \in I : e_{i_0} \notin \mathcal{F}'$. Comme $\mathcal{F}'$ est génératrice, $\exists (\lambda_i)_{i \in I \setminus \{i_0\}} \in \K^{(I \setminus \{i_0\})} : e_{i_0} = \sum_{\substack{i \in I \\ i \neq i_0}} \lambda_i \ldotp e_i$.
	Donc \begin{equation*}
		\begin{aligned}
			e_{i_0} & = 0_\K \ldotp e_{i_0} + \sum_{\substack{i \in I \\ i \neq i_0}} \lambda_i \ldotp e_i \\
			e_{i_0} & = 1_\K \ldotp e_{i_0} + \sum_{\substack{i \in I \\ i \neq i_0}} 0_\K \ldotp e_i
		\end{aligned}
	\end{equation*}
	$e_{i_0}$ peut donc s'écrire de deux manières différentes au moins comme combinaison linéaire de vecteurs de $\mathcal{F}$ ce qui contredit le caractère libre de $\mathcal{F}$. \\
	Par conséquent, $\mathcal{F}$ est génératrice et minimale parmi les familles génératrices.

	$(iii) \implies (iv)$ Supposons que $\mathcal{F}$ est une famille génératrice minimale.
	Par l'absurde, supposons que $\mathcal{F}$ est liée. Alors il existe un $i_0 \in I$ tel que $e_{i_0}$ s'écrit comme une combinaison linéaire d'autres vecteurs de $\mathcal{F}$ donc $(e_i)_{i \in I \setminus \{i_0\}}$ est génératrice de $E$ . Or cette famille est strictement incluse dans $\mathcal{F}$ ce qui contredit la propriété de génératrice minimale. \\
	Donc $\mathcal{F}$ est libre. \\
	Par l'absurde, supposons que $\mathcal{F}$ n'est pas libre maximale c'est-à-dire qu'il existe une famille $\mathcal{F}'$ de vecteurs de $E$ telle que $\mathcal{F} \subsetneq \mathcal{F}'$ et $\mathcal{F}'$ est libre. \\
	Alors $\exists x \in \mathcal{F}' : x \notin \mathcal{F}$. Or $\mathcal{F}$ est génératrice d'où :
	\begin{equation*}
		\exists \bdak :
		x = \sum_{i \in I} \lambda_i \ldotp x_i
		= 0_\K \ldotp x + \sum_{i \in I} \lambda_i \ldotp x_i + \sum_{\substack{y \in \mathcal{F}' \\ y \notin \mathcal{F} \\ y \neq x}} 0_\K \ldotp y
	\end{equation*}
	Puisque $x \in \mathcal{F}'$,
	\begin{equation*}
		\exists \bdak :
		x = 1_\K \ldotp x + \sum_{i \in I} 0_\K \ldotp x_i + \sum_{\substack{y \in \mathcal{F}' \\ y \notin \mathcal{F} \\ y \neq x}} 0_\K \ldotp y
	\end{equation*}
	Donc $x$ s'écrit de deux manières différentes au moins comme combinaison linéaire de vecteurs $\mathcal{F}'$, ce qui contredit la liberté de $\mathcal{F}'$. \\
	Par conséquent, $\mathcal{F}$ est libre maximale.

	$(iv) \implies (i)$ Supposons que $\mathcal{F}$ est une famille libre maximale. \\
	Par hypothèse même, $\mathcal{F}$ est libre.
	Par l'absurde, supposons que $\mathcal{F}$ n'est pas génératrice. Alors il existe $x \in E$ tel que $x \notin \Vect \mathcal{F}$. Donc $\mathcal{F} \wedge \{x\}$ est libre et contient strictement $\mathcal{F}$, ce qui contredit la propriété de liberté maximale. \\
	Par conséquent, $\mathcal{F}$ est aussi génératrice, donc une base.

	\begin{equation*}
		\begin{matrix}
			(i)      & \!\!\!\implies\!\!\!   & (ii)       \\
			\Uparrow &                        & \Downarrow \\
			(iv)     & \!\!\!\impliedby\!\!\! & (iii)      \\
		\end{matrix}
	\end{equation*}
\end{question_kholle}

\begin{question_kholle}
	[Soit $f \in \mathcal{L}_\K(E, F)$.
		\begin{equation}
			\begin{aligned}
				\ker f & = \left\{ x \in E \;|\; f(x) = 0_F \right\} = f^{-1} (\{0_F\}) \\
				\Im f  & = \left\{ y \in F \;|\; \exists x \in E : f(x) = y \right\}
			\end{aligned}
		\end{equation}
		Nous démontrerons le résultat plus général suivant :
		\begin{propositions}
			\item $f(E')$ est un \sev de $F$.
			\item $f^{-1}(F')$ est un \sev de $E$.
		\end{propositions}
	]
	{Le noyau et l'image d'une application linéaire sont des \sevs}

	$(i)$ $0_E \in E'$ et $f(0_E) = 0_F$ donc $0_F \in f(E')$ d'où $f(E') \neq \emptyset$ \\
	Soit $(\alpha, \beta, y, y') \in \K^2 \times f(E')^2$ \fqs. \\
	Par définition, $\exists (x, x') \in E'^2 : f(x) = y \wedge f(x') = y$.
	\begin{equation*}
		\begin{aligned}
			\alpha y + \beta y'
			 & = \alpha f(x) + \beta f(x')                                                                     \\
			 & = f( \alpha x + \beta x' ) \quad \text{ car } f \in \mathcal{L}_\K(E, F)                        \\
			 & \in f(E') \quad \text{ car } \alpha x + \beta x' \in E' \text{ puisque } E' \text{ est un \sev}
		\end{aligned}
	\end{equation*}
	Donc $f(E')$ est un \sev.

	$(ii)$ $0_F \in F'$ et $f(0_E) = 0_F$ donc $0_E \in f^{-1}(F')$ d'où $f(F') \neq \emptyset$ \\
	Soit $(\alpha, \beta, x, x') \in \K^2 \times f^{-1}(F')^2$ \fqs. \\
	Par définition, $\exists (y, y') \in F'^2 : f(x) = y \wedge f(x') = y$. \\
	Or $F'$ est \sev donc $\alpha y + \beta y' \in F'$. $f \in \mathcal{L}_\K(E, F)$ d'où $f(\alpha x + \beta x') = \alpha y + \beta y'$. Donc $\alpha x + \beta x' \in f^{-1}(F')$. \\
	Ainsi, $f^{-1}(F')$ est un \sev.

	En appliquant pour $E' = E$ et $F' = \{0_F\}$, nous obtenons que $\ker f$ et $\Im f$ sont des \sevs.
\end{question_kholle}


\begin{question_kholle}
	[{Soient $(E,F)$ deux $\mathbb{K}$-espaces vectoriels $f \in \mathcal{L}_\K(E, F)$, $\mathcal{F}=(x_{i})_{i\in I}$ une base de $E$.

				Alors \begin{equation}
					\text{Vect} \!\!\!\! \underbrace{ f(\mathcal{F}) }_{ \left\{ f(x_{i}) \mid i \in I \right\}  } = f(\Vect\mathcal{ F})
				\end{equation}
			}]
	{L’image par une application linéaire d’une partie génératrice engendre l’image de l’application linéaire}
	Soit $y \in \text{Vect}f(\mathcal{F})$
	Alors $\exists (\lambda_{i})_{i \in I} \in \mathbb{K}^{(I)}$ tel que $y = \sum_{i \in I} \lambda_{i}f(x_{i})$
	Mais
	\begin{align*}
		y & =  \sum_{i \in I} \lambda_{i}f(x_{i})                                                           \\
		  & = f\left( \sum _{i \in I} \lambda _{i} x_{i} \right) \implies y \in f(\text{Vect}{\mathcal{F}})
	\end{align*}


	Réciproquement soit $y \in f(\text{Vect}{\mathcal{F}})$ fq.
	$$\exists x \in \text{Vect} \mathcal{F} : f(x) = y \implies \exists (x_{i})_{i \in I} : x = \sum_{i \in I} \lambda_{i}x_{i}$$
	Donc:

	\begin{align*}
		y = f(x) & = f\left( \sum_{i \in I} \lambda_{i} x_{i} \right)                     \\
		         & = \sum _{i \in I} \lambda_{i} f(x_{i}) \in \text{Vect} f ( \mathcal F)
	\end{align*}
\end{question_kholle}

\begin{question_kholle}
	[Nous donnerons les caractérisations au fur et à mesure de la démonstration.]
	{Caractérisation inj/surj/bij d'une application linéaire par l'image d'une base de l'espace de départ.}
	Soient donc pour la suite, $f \in \mathcal{L}_\K(E,F)$, $\mathcal{B} = (e_i)_{i\in I}$ une base de $E$, $\mathcal{B}' = (e'_i)\ii$ une base de $F$,$\mathcal{F} = (x_i)_{i\in I}$ une famille libre de $E$ et $\mathcal{G} = (y_i)_{i \in I}$ une famille génératrice de $E$, ces objets servent ici de notation et seront utilisés indépendamment lors de la preuve. \\ \\
	Montrons que l'image d'une base $\mathcal{B}$ par une application injective est une famille libre $\mathcal{F}$. \\
	Supposons $f$ injective, donc pour $\bdak$,
	\[
		0_F = \sum_{i\in  I}\bda_i f(e_i) = f \left( \sum_{i\in  I} \bda_i e_i \right) \ \overset{f \text{ inj}}{\implies} \ \sum_{i\in  I} \bda_i e_i = 0_E \ \overset{\mathcal{B} \text{ base donc libre}}{\implies} \ (\bda _i)_{i\in  I} = \widetilde{0_\K},
	\]
	donc $f(\mathcal{B})= \mathcal{F}$ libre. \\
	Supposons qu'il existe $\mathcal{B}$ telle que $f(\mathcal{B})$ soit libre, montrons qu'alors $f$ est injective. \\
	Soit $x \in \ker f$ :
	\[
		\exists \ \bdak \ : \ 0_F = f(x) = f\left( \sum\ii \bda_i e_i \right) =\sum \ii \bda _i f(e_i) \  \overset{f(\mathcal{B}) \text{ libre}}{\implies} \ (\bda _i)_{i\in  I} = \widetilde{0_\K},
	\]
	donc $x = 0_E$ donc $\ker f = \{0_E\}$ et $f$ injective. \\ \\
	Montrons que l'image d'une base $\mathcal{B}$ par une application surjective est une famille génératrice $\mathcal{G}$. \\
	Supposons $f$ surjective. Ainsi, $\Im f = F$, or $\mathcal{B}$ est une base donc est génératrice donc :
	\[
		\Vect f(\mathcal{B}) = f\left( \Vect \mathcal{B} \right) = f( E) = \Im f = F,
	\]
	donc $f(\mathcal{B}) = \mathcal{G}$ est génératrice. \\
	Supposons qu'il existe $\mathcal{B}$ telle que $f(\mathcal{B})$ soit génératrice, montrons que $f$ est surjective. \\
	On a ainsi,
	\[
		F = \Vect f(\mathcal{B}) = f \left( \Vect \mathcal{B} \right) = f(E) = \Im f,
	\]
	donc $f$ surjective. \\ \\
	Montrons que l'image d'une base $\mathcal{B}$ par un isomorphisme est une base $\mathcal{B}'$. \\
	Supposons que $f$ soit un isomorphisme. $f$ est injective et $\mathcal{B}$ est une base donc $f(\mathcal{B})$ est libre. $f$ est surjective et $\mathcal{B}$ est une base donc $f(\mathcal{B})$ est génératrice. Ainsi, $f(\mathcal{B})=\mathcal{B}'$ est une base. \\
	Réciproquement, supposons qu'il existe $\mathcal{B}$ telle que $f(\mathcal{B})=\mathcal{B}'$ soit une base, montrons que $f$ est un isomorphisme.\\
	$\mathcal{B}'$ est une base donc est libre donc $f$ est injective. $\mathcal{B}'$ est une base donc est génératrice donc $f$ est surjective.
	Ainsi, $f$ est un isomorphisme.
\end{question_kholle}

\begin{question_kholle}
	[Il existe une unique application linéaire de $E$ dans $F$ qui envoie une base donnée de $E$ sur une famille de $F$ imposée. \\
		Soient $(e_i)\ii$ une base de $E$ et $(y_i)\ii$ une famille de $F$.
		\begin{equation}
			\exists ! f \in \mathcal{L}_\K(E, F) : \forall i \in I, f(e_i) = y_i
		\end{equation}
		Nous pouvons expliciter une telle application :
		\begin{equation}
			f \left| \;\; \begin{matrix}
				E                                          & \rightarrow & F                                          \\
				\displaystyle \sum\ii \lambda_i \ldotp e_i & \mapsto     & \displaystyle \sum\ii \lambda_i \ldotp y_i
			\end{matrix} \right.
		\end{equation}]
	{Caractérisation d'une application linéaire par l'image d'une base}
	~\newline
	\underline{Analyse} Supposons qu'il existe $f \in \mathcal{L}_\K(E, F)$ \tq $\forall i \in I, f(e_i) = y_i$. \\
	Tout vecteur de $E$ peut se décomposer de manière unique dans la base $(e_i)\ii$, ce qui détermine son image. Ainsi, $f$ est unique.
	\newline

	\noindent \underline{Synthèse} Posons une telle application $f$. \\
	\begin{itemize}
		\item $(e_i)\ii$ est une base donc $(\lambda_i)\ii$ est presque nulle et unique donc $\sum\ii \lambda_i \ldotp y_i$ existe et unique.
		      Ainsi, $f$ est bien définie.
		\item Soient $(\alpha, \beta, x, x') \in \K^2 \times E^2$ \fqs. Notons $(\lambda_i)\ii$ et $(\lambda'_i)\ii$ les coordonnées de $x$ et $x'$ dans $(e_i)\ii$.
		      \begin{equation*}
			      \begin{aligned}
				      f( \alpha x + \beta x' )
				       & = f\left( \alpha \sum\ii \lambda_i \ldotp e_i + \beta \sum\ii \lambda'_i \ldotp e_i \right)               \\
				       & = f\left( \sum\ii \left( \alpha \lambda_i + \beta \lambda'_i \right) \ldotp e_i \right)                   \\
				       & = \sum\ii \left( \alpha \lambda_i + \beta \lambda'_i \right) \ldotp y_i \quad \text{ par définiton de } f \\
				       & = \alpha \sum\ii \lambda_i y_i + \beta \sum\ii \lambda'_i y_i                                             \\
				       & = \alpha f(x) + \beta f(x')
			      \end{aligned}
		      \end{equation*}
		      Donc $f$ est linéaire.
		\item Soit $j \in I$ \fq.
		      \begin{equation*}
			      \begin{aligned}
				      f(e_j)
				       & = f \left(\sum\ii \delta_{i,j} \ldotp e_i \right) \\
				       & = \sum\ii \delta_{i,j} \ldotp y_i                 \\
				       & = y_j
			      \end{aligned}
		      \end{equation*}
	\end{itemize}
\end{question_kholle}

\end{document}

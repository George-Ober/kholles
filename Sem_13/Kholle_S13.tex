\documentclass{article}

\date{28 décembre 2023}
\usepackage[nb-sem=13, auteurs={Hugo Vangilluwen, Ober George}]{../kholles}

\begin{document}
\maketitle

\begin{question_kholle}
	[Soient $g$ une fonction définie sur $\mathcal{D}_g \subset \R$ et $f$ une fonction définie sur $\mathcal{D}_f \subset \R$ telle que $f(\mathcal{D}_f) \subset \mathcal{D}_g$.
		Si $\left. \begin{array}{cc}
				g \text{ admet une limite } \ell \in \overline{\R} \text{ en } b \in \overline{\mathcal{D}_g} \\
				f \text{ admet } b \text{ comme limite en } a \in \overline{\mathcal{D}_f}
			\end{array} \right\}$
		alors $g \circ f$ admet $\ell$ comme limite en $a$.]
	{Théorème de composition des limites}

	Traitons le cas où $\ell \in \R$, $a \in \R$ et $b \in \R$. \\
	Soit $\varepsilon \in \R_+^*$ fq. \\
	Appliquons la définition de $g(y) \arrowlim{y}{b} \ell$ pour cet $\varepsilon$ :
	\begin{equation*}
		\exists \eta_g \in \R_+^* : \forall y \in \mathcal{D}_g, | y - b | \leqslant \eta_g \implies | g(y) - \ell | \leqslant \varepsilon
	\end{equation*}
	Appliquons la définition de $f(x) \arrowlim{x}{a} b$ pour cet $\eta_g$ :
	\begin{equation*}
		\exists \eta_f \in \R_+^* : \forall x \in \mathcal{D}_f, | x - a | \leqslant \eta_f \implies | f(x) - b | \leqslant \eta_g
	\end{equation*}
	Posons $\eta = \eta_f$.

	Soit $x \in \mathcal{D}_{g \circ f}$ fq tq $ | x - a | \leqslant \eta $. Or $f(\mathcal{D}_f) \subset \mathcal{D}_g$ donc $\mathcal{D}_{g \circ f} = \mathcal{D}_f$. \\
	Ainsi, $x \in \mathcal{D}_f$ et $ | x - a | \leqslant \eta_f $ d'où $ | f(x) - b | \leqslant \eta_g $ d'où $ | g(f(x)) - \ell | \leqslant \varepsilon $. Donc
	\begin{equation*}
		g \circ f \arrowlim{x}{a} \ell
	\end{equation*}
\end{question_kholle}

\begin{question_kholle}
	[{ Soit une fonction continue $f : [a;b] \rightarrow \R$ avec $(a,b) \in \R^2$ et $a < b$. \\
	Si $f(a)f(b) \leqslant 0$ alors $\exists c \in [a;b] : f(c) = 0$. \\
	On rencontre aussi : \textit{Si $\mathit{f(a)f(b) < 0}$ alors $\mathit{\exists c \in ]a;b[ : f(c) = 0}$.} }]
	{Théorème des valeurs intermédiaires}

	La démonstration repose sur la technique de la dichotomie.

	\begin{figure}[!h]
		%					\centering
		\tikzmath{ \labTVI = 12; } % la longueur du segment [a;b] dans la démonstration du TVI
		\begin{tikzpicture}
			\draw (0,0) node[anchor=north] {a} -- (\labTVI,0) node[anchor=north] {b};
			\foreach \x in {0,...,\labTVI} {
					\draw (\x,0.1) -- (\x,-0.1);
				};
			\draw[red] (0,0) to[out angle=80, in angle=240, curve through={(\labTVI/5,2) (\labTVI/3,-1) (\labTVI*3/5,0.5)}] (\labTVI,0);

			\draw[green] (\labTVI/2,0.1) -- (\labTVI/2,-0.1);
			\draw (\labTVI/2,0) node[green, anchor=north] {$b_1$};
			\draw[green] (\labTVI/4,0.1) -- (\labTVI/4,-0.1);
			\draw (\labTVI/4,0) node[green, anchor=north] {$a_2$};
			\draw[green] (\labTVI*3/8,0.1) -- (\labTVI*3/8,-0.1);
			\draw (\labTVI*3/8,0) node[green, anchor=north] {$b_3$};
			\draw[green] (\labTVI*5/16,0.1) -- (\labTVI*5/16,-0.1);
			\draw (\labTVI*5/16,0) node[green, anchor=north] {$b_4$};
		\end{tikzpicture}
	\end{figure}

	\noindent Soient $a,b,f$ de tels objets. Procédons à la construction des suites $(a_n)_{n\in\N}, (b_n)_{n\in\N}, (c_n)_{n\in\N}$.

	Posons $a_0 = a$, $b_0 = b$ et $c_0 = \frac{a+b}{2}$ (le milieu du segment $[a;b]$). Nous avons, par hypothèse $f(a_0)f(b_0) \leqslant 0$.

	Soit $n \in \N$ fq. Supposons les trois suites construites au rang $n$ telles que $f(a_n)f(b_n) \leqslant 0$ et $c_n = \frac{a_n+b_n}{2}$ (milieu de $[a_n;b_n]$).
	\begin{itemize}
		\item Si $f(a_n)f(b_n) \leqslant 0$, posons $\left| \begin{array}{lcl}
				      a_{n+1} & = & a_n                       \\
				      b_{n+1} & = & c_n                       \\
				      c_{n+1} & = & \frac{a_{n+1}+b_{n+1}}{2}
			      \end{array} \right.$
		\item Sinon $f(a_n)f(b_n) > 0$. Or $f(a_n)f(b_n) \leqslant 0$, donc $f(a_n)^2 f(b_n) f(c_n) \leqslant 0$. Donc $f(b_n)f(c_n) \leqslant 0$. Posons $\left| \begin{array}{lcl}
				      a_{n+1} & = & c_n                       \\
				      b_{n+1} & = & b_n                       \\
				      c_{n+1} & = & \frac{a_{n+1}+b_{n+1}}{2}
			      \end{array} \right.$
	\end{itemize}
	Ainsi, nous avons bien construits $a_{n+1}, b_{n+1}, c_{n+1}$ telles que $f(a_{n+1})f(b_{n+1}) \leqslant 0$ et ${ c_{n+1} = \frac{a_{n+1}+b_{n+1}}{2} }$ (milieu de $[a_{n+1};b_{n+1}]$).

	Par récurrence immédiate, $(a_n)_{n\in\N}$ est croissante, $(b_n)_{n\in\N}$ est décroissante et ${ \forall n \in \N, b_n - a_n = \frac{b-a}{2^n} }$ d'où $b_n - a_n \arrowlim{n}{+\infty} 0$.
	Donc les suites $a$ et $b$ sont adjacentes.
	D'après le théorème des suites adjacentes, elles convergent vers la même limite. Notons la $c$.

	D'après le bonus de ce même théorème, $\forall n \in \N, a_n \leqslant c \leqslant b_n$ donc pour $n = 0$, $a \leqslant c \leqslant b$. Ainsi,
	\begin{equation*}
		c \in [a;b]
	\end{equation*}

	Par ailleurs, $\forall n \in \N, f(a_n)f(b_n) \leqslant 0$. Par continuité de $f$ sur $[a;b]$ donc en $c$, $f(a_n) \arrowlim{n}{+\infty} f(c)$ et $f(b_n) \arrowlim{n}{+\infty} f(c)$. Par passage à limite dans l'inégalité,
	\begin{equation*}
		f(c) \times f(c) \leqslant 0
	\end{equation*}
	Or $f(c)^2 \geqslant 0$, d'où $f(c)^2 = 0$. Ainsi,
	\begin{equation*}
		f(c) = 0
	\end{equation*}
	Donc $c$ est un point fixe.

\end{question_kholle}

\begin{question_kholle}
	[{L'image d'un segment par une fonction continue sur ce segment est un segment : soient $(a, b) \in \mathbb{R}^2$ tels que $a < b$ et $f: [a, b] \to \mathbb{R}$. Si $f \in \mathcal{C}^0([a, b], \mathbb{R})$ alors $\exists (x_{1}, x_{2}) \in \mathbb{R}^2 : f([a, b]) = [f(x_{1}), f(x_{2})]$}]
	{Théorème de Weierstraß}
	\begin{itemize}

		\item \emph{Étape 1} Montrons que $f([a, b])$ est majoré.

		      Par l'absurde, supposons que $f([a, b])$ n'est pas majoré

		      Alors \begin{equation}\label{eq:1}
			      \forall A \in \mathbb{R}, \exists x \in [a, b] : f(x) > A
		      \end{equation}

		      Soit $n \in \mathbb{N}$ fq.
		      Appliquons (\ref{eq:1}) pour $A \leftarrow n$:
		      $\exists x \in [a, b] : f(x) > n$, et fixons un tel $x$ que l'on note $x_{n}$
		      Nous venons de créer la suite $(x_{n})_{n \in \mathbb{N}} \in [a, b]^{\mathbb{N}}$ qui vérifie:


		      $$
			      \left.
			      \begin{array}{ll}
				      \forall n \in \mathbb{N}, f(x_{n}) \geqslant n \\
				      \lim_{ n \to \infty } n =  +\infty
			      \end{array}
			      \right\} \underbrace{ \implies }_{ \text{théorème de divergence par minoration} } f(x_{n}) \xrightarrow[n \to +\infty]{} + \infty
		      $$


		      $(x_{n})_{n \in \mathbb{N}}$ est bornée (à valeurs dans $[a, b]$) donc, selon le théorème de Bolzanno-Weierstraß:
		      $$
			      \exists \ell \in \mathbb{R} : \exists \varphi : \mathbb{N} \to \mathbb{N} : \text{strict. croissante tel que } (x_{\varphi(n)})_{n \in \mathbb{N}} \text{ tend vers } \ell
		      $$
		      Donc, en passant à la limite : $\forall n \in \mathbb{N}, a \leqslant x_{\varphi(n)} \leqslant b \implies a \leqslant \ell \leqslant b \implies \ell \in [a, b]$

		      Par continuité de $f$ sur $[a, b]$, donc en $\ell$, $(f(x_{\varphi(n)}))_{n \in \mathbb{N}}$ converge vers $f(\ell)$.

		      Or $$
			      \left\{ \begin{array}{ll}
				      (f(x_{\varphi(n)}))_{n \in \mathbb{N}} \text{ est une sous suite de } (f(x_{n}))_{n \in \mathbb{N}} \\
				      f(x_{n}) \xrightarrow[n \to + \infty]{} + \infty
			      \end{array}\right.$$

		      donc $(f(x_{\varphi(n)}))_{n \in \mathbb{N}}$, tend vers $+ \infty$, ce qui est absurde, donc $f$ est majorée.

		      On fait de même pour la minoration.

		\item  \emph{Étape 2:} Montrons que $f([a, b])$ admet un pge et un ppe.

		      Montrons donc que $f([a, b])$ admet une borne sup, qui, puisque c'est une valeur atteinte, deviendra un max.

		      $$
			      f([a, b]) \text{ est } \left\{ \begin{array}{ll}
				      \text{ une partie de } \mathbb{R}   \\
				      \text{ non vide car contient } f(a) \\
				      \text{majorée d'après l'étape 1}
			      \end{array}\right.
		      $$

		      $f([a, b])$ admet donc une borne supérieure $\sigma$.

		      Appliquons la caractérisation séquentielle de la borne supérieure:
		      $$
			      \exists (y_{n})_{n \in \mathbb{N}}, \in f([a, b])^{\mathbb{N}} : (y_{n}) \text{ converge vers } \sigma
		      $$
		      $$
			      \forall n \in \mathbb{N}, y_{n} \in f([a, b]) \implies \exists x_{n} \in [a, b] : f(x_{n} ) = y_{n}
		      $$
		      Fixons un tel $x_{n}$ pour tout $y_{n}$.
		      On a donc construit $(x_{n})_{n \in \mathbb{N}} \in [a, b]^{\mathbb{N}} : f(x_{n}) \xrightarrow[n \to +\infty]{} \sigma$

		      De plus, $(x_{n})$ est bornée (à valeurs dans $[a, b]$) donc, selon le théorème de Bolzanno-Weierstraß:
		      $$
			      \exists \ell \in \mathbb{R} : \exists \varphi : \mathbb{N} \to \mathbb{N} : \text{strict. croissante tel que } (x_{\varphi(n)})_{n \in \mathbb{N}} \text{ tend vers } \ell
		      $$
		      Donc, en passant à la limite : $\forall n \in \mathbb{N}, a \leqslant x_{\varphi(n)} \leqslant b \implies a \leqslant \ell \leqslant b \implies \ell \in [a, b]$


		      Par continuité de $f$ sur $[a, b]$, donc en $\ell$, $(f(x_{\varphi(n)}))_{n \in \mathbb{N}}$ converge vers $f(\ell)$.

		      Or,
		      $$
			      \left\{ \begin{array}{ll}
				      (f(x_{\varphi(n)}))_{n \in \mathbb{N}} \text{ est une sous suite de } (f(x_{n}))_{n \in \mathbb{N}} \\
				      f(x_{n}) \xrightarrow[n \to + \infty]{} \sigma
			      \end{array}\right.$$

		      Par unicité de la limite, $\sigma = f(\ell)$.

		      On montre de même qu'il existe $\ell' \in [a, b]: f(\ell') = \inf f([a, b])$

		      Ainsi, $f(\ell) = \max f([a, b])$ et $f(\ell') = \min f([a, b])$



		\item \emph{Étape 3:}
		      Montrons que $f([a, b]) = [f(\ell'), f(\ell)]$.

		      Par la construction précédente, $\forall y \in f([a, b]), y \in [f(\ell'), f(\ell)]$.

		      Ainsi, $f([a, b]) \subset [f(\ell'), f(\ell)]$.

		      Réciproquement, l'image par la fonction continue $f$ du segment $[a, b]$ qui est un intervalle est un intervalle:

		      $$
			      \left.
			      \begin{array}{ll}
				      f([a,b]) \text{ est un intevalle} \\
				      f(\ell) \in f([a, b])             \\
				      f(\ell') \in f([a, b])
			      \end{array}
			      \right\}
			      \implies
			      [f(\ell'), f(\ell)] \subset f([a,b])
		      $$

		      D'où $[f(\ell'), f(\ell)] = f([a,b])$

	\end{itemize}
\end{question_kholle}
\end{document}
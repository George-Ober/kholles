\documentclass{article}

\usepackage{kholles}

\begin{document}
\maketitle

\begin{question_kholle}
	[\noindent Soit $(E, \leq)$ un ensemble ordonné, et $A$ une partie non-vide de $E$. \\
    S'il existe $M\in A$ tel que $M = \max{A}$ alors $\sup{A}$ existe et $\max{A} = \sup{A}.$ \\
    S'il existe $S\in A$ tel que $S = \sup{A}$ alors $\max{A}$ existe et $\max{A} = \sup{A}.$
    ]
	{Si $A$ admet un plus grand élément c'est aussi sa borne supérieure. Si $A$ admet une borne supérieure dans $A$ c'est sont plus grand élément.}

Soient un tel ensemble $E$ et une telle partie $A$ et notons $M$ son plus grand élément. \\
Posons $M(A) = \{ m\in E \ | \ \forall a \in A, \ a \leq m\}$. \\
Par définition : 
\[
\forall m \in M(A), \ M \leq m,
\]
car $M\in A$, mais comme $M\in M(A)$, on a directement que $M = \min{M(A)} = \sup{A}.$
D'où, si : 
\[
\exists A \subset E  \ : \ \exists M \in A \ : \ M = \max{A} \ \implies \ \exists \sup{A} \in E \ \wedge \ \sup{A} = \max{A} \in A.
\] 
Pseudo-réciproquement, soit $A$ une partie de $E$ admettant une borne supérieure dans elle même, notons cette borne $S$. \\
Il n'y a rien à prouver, si $S$ est dans $A$, par définition, $S$ est plus grand que tous les éléments de $A$ mais est dans $A$, donc de tous les éléments de $A$, $S$ est le plus grand, si : 
\[
\exists A \subset E \ : \ \exists S \in A \ : \ S = \sup{A} \ \implies \ \exists \max{A} \in A \ \wedge \ \sup{A} = \max{A}.
\]
\end{question_kholle}

\begin{question_kholle}
	[\noindent Pour tout couple d'entiers relatifs $a$ et $b$, $b$ non nul, il existe un unique couple d'entiers relatifs $q$ et $r$ tel que $a = bq +r$ et $0 \leq r \leq |b| -1$]
	{ Théorème de la division Euclidienne dans $\Z $ }

Soient deux tels couples $((q,r),(q',r'))$ et deux tels entiers $(a,b)$. \\
Directement, 
\[
b(q-q') = r'-r,
\]
mais comme $-(|b|-1) \leq r' - r \leq |b| -1$, il vient en divisant par $|b|$ l'inégalité suivante : 
\[
-1 < q - q' < 1,
\]
puisque $q$ et $q'$ sont dans $\Z$ leur différence est obligatoirement $0$, ainsi $q = q'$ ce qui implique $ r= r'$ et donc on a unicité de ladite écriture de $a$.
\newline
\\
Posons pour $b \geq 1$, $\Omega = \{ k\in \Z  \ | \ kb \leq a \}$, non-vide car $-|a|\in \Omega$ ($\Z$ archimédien suffit...), ainsi $\Omega \subset \Z$. Supposons qu'il existe un $k$ dans $\Omega$ tel que $k > |a|$, si tel est le cas alors $k\notin \Omega$ (multiplier par b). De fait, $\Omega$ est majoré par $|a|$, il admet donc un plus grand élément noté $q$. \\
Posons $r = a - bq$. Par construction, $a = bq + r$ et comme $q = \max \Omega$ et $\Omega \subset \Z$, $q\in \Z$ donc $r \in \Z$.
\\
Par suite, $q\in \Omega$ donc $bq \leq a$ d'où $0\leq r$ et $q=\max \Omega$ donc $b(q+1) > a$ d'où $b > r$, c'est-à-dire, $r\in [\![ 0, |b| -1 ]\!]$. Si $b< 1$, il suffit de prendre $q \leftarrow -q$ dans la preuve précédente.C'est donc l'existence de ladite écriture de $a$.
\end{question_kholle}
\end{document}
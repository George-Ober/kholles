\documentclass{article}

\date{29 novembre 2023}
\usepackage{../kholles}

\begin{document}
	
	\maketitle
	
	\begin{question_kholle}
		[\noindent Soit \Rel une relation d'équivalence sur $E$. \\
		Soit $x \in E$. \\
		La classe de $x$, notée $\bar{x}$, est l'ensemble des éléments de $E$ en relation avec x.
		\begin{equation}
			\bar{x} = \left\{ y \in E \;|\; x \Rel y \right\}
		\end{equation}]
		{Deux classes d'équivalence sont disjointes ou confondues. Les classes d'équivalence constituent une partition de l'ensemble sur lequel on considère la relation d'équivalence.}
		
		\textit{Montrons que deux classes d'équivalence sont disjointes ou confondues.}
		
		Soit $(x, y) \in E^2$ fq.
		\begin{itemize}[label=\textemdash]
			\item Si $\bar{x} \cap \bar{y} = \emptyset$, rien à démontrer.
			\item Sinon $\bar{x} \cap \bar{y} \neq \emptyset$ donc $\exists z \in \bar{x} \cap \bar{y}$. Fixons un tel $z$.

			Soit $x' \in \bar{x}$ fq.
			\begin{equation*}
				\left.
				\begin{matrix}
					\left. \begin{matrix}
						x' \in \bar{x} \implies x \Rel x' \underset{sym\acute{e}trie}{\implies} x' \Rel x \\
						z \in \bar{x} \implies x \Rel z
					\end{matrix}
					\right\} \underset{transitivit\acute{e}}{\implies} x' \Rel z \\
					z \in \bar{y} \implies y \Rel z \underset{sym\acute{e}trie}{\implies} z \Rel y
				\end{matrix}
				\right\} \underset{transitivit\acute{e}}{\implies} x' \Rel y
				\underset{sym\acute{e}trie}{\implies} y \Rel x'
			\end{equation*}
			
			Donc $x' \in \bar{y}$ donc $\bar{x} \subset \bar{y}$.
			
			En échangeant les rôles de $x$ et $y$, on montre la deuxième inclusion $\bar{y} \subset \bar{x}$.
		\end{itemize}
		\bigbreak
	
		\textit{Montrons que les classes d'équivalence de E constituent une partition de E.}
		
		Soit $\Sol$ un système de représentant des classes fixé quelconque.
		
		\begin{itemize}[label=\textemdash]
			\item Soit $s\in \Sol$ fq. $\bar{s} \neq \emptyset$ car $s \Rel s$ par réflexivité.
			\item Soit $(s, s') \in \Sol^2$ fq. D'après la démonstration ci-dessus ci-dessus, $\bar{s} \cap \bar{s'} = \emptyset$ ou $\bar{s} = \bar{s'}$. Si $\bar{s} = \bar{s'}$ alors $s$ et $s'$ représente la même classe ce qui est impossible car un système de représentants des classes contient un unique représentant de chaque classe. Par conséquent, $\bar{s}$ et $\bar{s'}$ sont disjoints.
			\item $\underset{s \in \Sol}{\bigcup} \bar{s} \subset E$ car $\forall s \in \Sol, \bar{s} \in E$ par définition d'une classe d'équivalence. \\
			Réciproquement, soit $x \in E$ fq. \\
			Par réflexivité de \Rel, $x \in \bar{x}$. \\
			Par définition d'un système de classe $\exists ! s_x \in \Sol : s_x \in \bar{x}$ donc $\bar{s_x} = \bar{x}$. Donc $x \in \bar{s_x} \subset \underset{s \in \Sol}{\bigcup} \bar{s}$. Donc $E \subset \underset{s \in \Sol}{\bigcup} \bar{s}$. \\
			Par double inclusion, $E = \underset{s \in \Sol}{\bigcup} \bar{s}$.			
		\end{itemize}
		
		Ainsi,
		\begin{equation}
			E = \coprod_{s \in \Sol} \bar{s}
		\end{equation}
		
	\end{question_kholle}

	\begin{question_kholle}
		[\noindent Soit $(E, \leq)$ un ensemble ordonné, et $A$ une partie non-vide de $E$. \\
		Si $A$ admet un plus grand élément alors $A$ admet une borne supérieure et $\sup{A} = \max{A}$. \\
		Si $A$ admet une borne supérieure appartenant à elle-même alors $A$ admet un plus grand élément et $\max{A} = \sup{A}$.]
		{Si $A$ admet un plus grand élément c'est aussi sa borne supérieure. Si $A$ admet une borne supérieure dans $A$ c'est sont plus grand élément.}
		
		Soient un tel ensemble $E$ et une telle partie $A$ et notons $M$ son plus grand élément. \\
		Posons l'ensemble des majorants de $A$, $M(A) = \{ m\in E \ | \ \forall a \in A, \ a \leq m\}$. \\
		Par définition : 
		\[
		\forall m \in M(A), \ M \leq m,
		\]
		car $M\in A$, mais comme $M\in M(A)$, on a directement que $M = \min{M(A)} = \sup{A}$. \\
		
		Pseudo-réciproquement, soit $A$ une partie de $E$ admettant une borne supérieure dans elle même, notons cette borne $S$. \\
		Comme $S \in M(A)$, par définition, $S$ est plus grand que tous les éléments de $A$ mais appartient à $A$, donc de tous les éléments de $A$, $S$ est le plus grand.
	\end{question_kholle}

	\begin{question_kholle}
		[Pour tout couple d'entiers relatifs $a$ et $b$, $b$ non nul, il existe un unique couple d'entiers relatifs $q$ et $r$ tel que $a = bq +r$ et $0 \leq r \leq |b| -1$
		\begin{equation}
			\forall (a, b) \in \Z^2,
			\exists ! (q, r) \in \Z \times \N :
			\left\{ \begin{matrix}
				a = b q + r \\
				r \in {[\![} 0 ; |b|-1 {]\!]}
			\end{matrix} \right.
		\end{equation}]
		{Théorème de la division Euclidienne dans \Z}
		
		\textit{Existence} \;
		Soient deux tels entiers $(a,b)$ et deux couples $((q,r),(q',r'))$ tels que
		\begin{equation*}
			\left\{ \begin{matrix}
				a = b q + r \\
				0 \leq r \leq |b| - 1
			\end{matrix} \right.
			\qquad
			\left\{ \begin{matrix}
				a = b q' + r' \\
				0 \leq r' \leq |b| - 1
			\end{matrix} \right.
		\end{equation*}
		Directement, 
		\[
		b(q-q') = r'-r,
		\]
		mais comme $-(|b|-1) \leq r' - r \leq |b| -1$, il vient en divisant par $|b|$ l'inégalité précédente : 
		\[
		-1 < q - q' < 1,
		\]
		puisque $q$ et $q'$ sont dans $\Z$ leur différence est obligatoirement $0$, ainsi $q = q'$ ce qui implique $ r= r'$ et donc on a unicité de ladite écriture de $a$.
		\newline
		\\
		\textit{Unicité} \; Posons pour $b \geq 1$, $\Omega = \{ k\in \Z  \ | \ kb \leq a \}$, non-vide car $-|a|\in \Omega$ ($\Z$ archimédien suffit...), ainsi $\Omega \subset \Z$. Supposons qu'il existe un $k$ dans $\Omega$ tel que $k > |a|$, si tel est le cas alors $k\notin \Omega$ (multiplier par b). De fait, $\Omega$ est majoré par $|a|$, il admet donc un plus grand élément noté $q$. \\
		Posons $r = a - bq$. Par construction, $a = bq + r$ et comme $q = \max \Omega$ et $\Omega \subset \Z$, $q\in \Z$ donc $r \in \Z$.
		\\
		Par suite, $q\in \Omega$ donc $bq \leq a$ d'où $0\leq r$ et $q=\max \Omega$ donc $b(q+1) > a$ d'où $b > r$, c'est-à-dire, $r\in [\![ 0, |b| -1 ]\!]$.
		
		Si $b< 1$, il suffit de prendre $q \leftarrow -q$ dans la preuve précédente.C'est donc l'existence de ladite écriture de $a$.
	\end{question_kholle}

	\begin{question_kholle}{Une suite décroissante et minorée de nombres entiers relatifs est stationnaire}
		Soit $u \in \Z^\N$ une suite décroissante et minorée fixée quelconque. \\
		Considérons $A = \{ u_n \;|\; n \in \N \}$ c'est-à-dire l'ensemble des valeurs prises par la suite $u$. \\
		$A$ est : \begin{itemize}[label=\textemdash]
			\item une partie de \Z car $u$ est à valeur dans \Z
			\item non vide car $u_0 \in A$
			\item minoré car $u$ est minorée
		\end{itemize}
		Donc $A$ admet un plus petit élément. Donc $\exists n_0 \in \N: u_{n_0} = min A$. Fixons un tel $n_0$. \\
		Soit $n \in \N$ fq tq $n \geqslant n_0$.
		\begin{equation*}
			\left. \begin{matrix}
				u_n \in A \implies u_n \geqslant min A = u_{n_0} \\
				 u \text{ est décroissante et } n \geqslant n_0 \text{ donc } u_n \leqslant u_{n_0}
			\end{matrix}
			\right\} \implies u_n = u_{n_0}
		\end{equation*}
  		Ainsi, $u$ est stationnaire.
	\end{question_kholle}
\end{document}

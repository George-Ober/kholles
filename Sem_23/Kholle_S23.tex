\documentclass{article}

\date{31 Mars 2024}
\usepackage[nb-sem=23, auteurs={Hugo Vangilluwen}]{../kholles}

\begin{document}
\maketitle

Pour cette semaine, \K désigne un corps commutatif, $E$ et $F$ des \K\!\!-espaces vectoriels, $E'$ et $F'$ des sous-espaces vectoriels respectivement de $E$ et de $F$, $I$ un ensemble quelconque non vide.

\begin{question_kholle}
	{L'ensemble des automorphisme d'un espace vectoriel muni de la loi de composition forme un groupe}

	Montrons que $(\mathcal{GL}_\K(E), \circ)$ est un sous-groupe de $(\mathcal{S}(E), \circ)$.
	\begin{itemize}
		\item $\mathcal{GL}_\K(E) \subset \mathcal{S}(E)$ et $(\mathcal{S}(E), \circ)$ est bien un groupe.
		\item $\mathcal{GL}_\K(E) \neq \emptyset$ puisque $Id_E \in \mathcal{GL}_\K$.
		\item Soit $(f, g) \in \mathcal{GL}(E)$. Montrons que $f \circ g^{-1} \in \mathcal{GL}(E)$. \\
		      Soit $(\alpha, \beta, x, y) \in \K^2 \times E^2$ \fqs. \\
		      \begin{equation*}
			      \begin{aligned}
				      \left(f \circ g^{-1}\right) \left(\alpha x + \beta y\right)
				       & = f \left( g^{-1} \left(\alpha x + \beta y\right) \right)                                                                             \\
				       & = f \left( g^{-1} \left(\alpha g^{-1}(g(x)) + \beta g^{-1}(g(y))\right) \right)                                                       \\
				       & = f \left( g^{-1} \left( \alpha g\left(g^{-1}(x)\right) + \beta g\left(g^{-1}(y)\right) \right) \right)                               \\
				       & = f \left( g^{-1} \left( g \left( \alpha g^{-1}(x) + \beta g^{-1}(y) \right) \right) \right) \quad \text{car } g \text{ est linéaire} \\
				       & = f \left( \alpha g^{-1}(x) + \beta g^{-1}(y) \right)                                                                                 \\
				       & = \alpha f \left( g^{-1}(x) \right) + \beta f \left( g^{-1}(y) \right)                                                                \\
				       & = \alpha \left(f \circ g^{-1}\right) (x) + \beta \left(f \circ g^{-1}\right) (y)
			      \end{aligned}
		      \end{equation*}
	\end{itemize}
\end{question_kholle}

\begin{question_kholle}
	[Soit $\ffinie{E}{p}{E}$ $p$ \sev de E avec $p \in \N^*$ \fq. \\
		Par définition, cette famille est en somme directe si tout vecteur de $E_1 + E_2 + \ldots + E_p$ peut s'écrire comme une somme unique d'élément de $E_1 \times E_2 \times \ldots \times E_p$. Formellement :
		\begin{equation}
			\forall x \in \sum_{i=1}^{p} E_i,
			\exists ! x \in \! \overset{p}{\underset{i=1}{\mathlarger{\mathlarger{\times}}}} E_i :
			x = \sum_{i=1}^{p} x_i
		\end{equation}
		Nous allons démontrer que $E_1, E_2, \ldots$ et $E_p$ sont en somme directe \ssi
		{\begin{equation}
				\forall x \in \! \overset{p}{\underset{i=1}{\mathlarger{\mathlarger{\times}}}} E_i,
				\left( \sum_{i=1}^{p} x_i = 0_E \implies \forall i \in [\![1;p]\!], x_i = 0_E \right)
			\end{equation}}]
	{Caractérisation de la somme directe de $p$ \sevs}

	Supposons que $E_1, E_2, \ldots E_p$ sont en somme directe. \\
	Soient $x \in \!\! \overset{p}{\underset{i=1}{\mathlarger{\mathlarger{\times}}}} E_i$ \fqs \tqs $x_1 + x_2 + \ldots + x_p = 0_E$. \\
	Or $0_E = \underbrace{0_E}_{\in E_1} + \underbrace{0_E}_{\in E_2} + \ldots + \underbrace{0_E}_{\in E_p}$.
	Par unicité de l'écriture de x comme somme d'éléments de $\overset{p}{\underset{i=1}{\mathlarger{\mathlarger{\times}}}} E_i$, $\forall i \in [\![]1;p\!], x_i = 0_E$.

	Supposons maintenant l'équation de la caractérisation. \\
	Soit $x \in \overset{p}{\underset{i=1}{\mathlarger{\mathlarger{\times}}}} E_i$ \tq $x$ puisse s'écrire comme somme de $x' \!\! \in \!\! \overset{p}{\underset{i=1}{\mathlarger{\mathlarger{\times}}}} E_i$ et somme de $x'' \!\! \in \!\! \overset{p}{\underset{i=1}{\mathlarger{\mathlarger{\times}}}} E_i$. Montrons que $x' = x''$.
	\begin{equation*}
		\sum_{i=1}^{p} x'_i = x = \sum_{i=1}^{p} x''_i
	\end{equation*}
	Donc
	\begin{equation*}
		\sum_{i=1}^{p} \left( x''_i -x''_i \right) = 0_E
	\end{equation*}
	D'après l'équation de la caractérisation, $\forall i \in [\![1;p]\!], x'_i - x''_i = 0_E$. \\
	Donc $\forall i \in [\![1;p]\!], x'_i = x''_i$
\end{question_kholle}
\end{document}

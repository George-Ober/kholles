\documentclass{article}

\date{12 Octobre 2024}
\usepackage[nb-sem=5, auteurs={Kylian Boyet, George Ober, Felix Rondeau}]{../kholles}


\begin{document}

\maketitle

\begin{question_kholle}{Montrer qu'une combinaison linéaire de deux fonctions bornées (respectivement lipschitziennes) est bornée (resp. lipschitzienne)}
	Soit $I$ un intervalle réel.
	Soient $f$ et $g$ deux fonctions de $I$ dans $\mathbb{R}$ et $(\lambda, \mu) \in \mathbb{R}^2$.
	\begin{itemize}[label=$\lozenge$]
		\item Supposons que $f$ et $g$ sont respectivement bornées par $A$ et par $B$.
		      Soit $x \in I$ fixé quelconque.
		      \begin{align*}
			      \Big| (\lambda.f + \mu.g)(x) \Big| & = \Big| \lambda.f(x) + \mu.g(x) \Big|                                             \\
			                                         & \leqslant \big| \lambda \big|  \big|f(x)\big| + \big| \mu \big|  \big| g(x) \big| \\
			                                         & \leqslant \big| \lambda \big| A + \big| \mu \big| B
		      \end{align*}
		      Donc $\lambda.f + \mu.g$ est bornée.
		\item Supposons que $f$ et $g$ sont respectivement $K$ et $L$ lipschitziennes.\\
		      Soient $(x, y) \in I^2$ fixés quelconques.
		      \begin{align*}
			      \Big| (\lambda.f + \mu.g)(x) - (\lambda.f + \mu.g)(y)\Big| & = \Big| \lambda.f(x) + \mu.g(x) - \lambda.f(y) - \mu.g(y) \Big|                                   \\
			                                                                 & = \Big| \lambda(f(x) - f(y)) + \mu(g(x) - g(y)) \Big|                                             \\
			                                                                 & \leqslant \Big| \lambda  \Big|  \Big| f(x) - f(y) \Big| + \Big| \mu \Big|  \Big|g(x) - g(y) \Big| \\
			                                                                 & \leqslant \Big| \lambda  \Big|  K \Big| x-y \Big| + \Big| \mu  \Big| L  \Big| x - y \Big|         \\
			                                                                 & \leqslant (|\lambda| K + |\mu|L ) |x - y|
		      \end{align*}
	\end{itemize}
\end{question_kholle}

\begin{question_kholle}{Montrer que si $f$ est impaire et bijective, alors $f^{-1}$ est aussi impaire. Donnez un/des exemples.}
	Soient $I$ et $J$ deux parties non-vides de $\R$ et $f$ une application bijective impaire de $I$ dans $J$. Notons $f^{-1}$ sa bijection réciproque.\\
	L'imparité de $f$ impose la symétrie de $I$ par rapport à l'origine. De plus, pour tout $y\in J$,
	\[
		\exists{x}\in I: f(x)=y
	\]
	donc par imparité de la fonction $f$, le domaine $I$ étant centré en $0$,
	\[
		f(-x)=-f(x)=-y
	\]
	Ainsi, $J$ est centré en $0$. On a alors, pour tout $y\in J$,
	\begin{align*}
		f^{-1}(-y) & = f^{-1}(-f(f^{-1}(y))) \\
		           & = f^{-1}(f(-f^{-1}(y))) \\
		           & = -f^{-1}(y).
	\end{align*}
	D'où l'imparité de $f^{-1}$.

	\noindent\textbf{$\vartriangleright$ Exemple :} Prenons notre fonction bijective impaire préférée, la fonction $\bigl.\sin\bigr|_{\left[ -\frac{\pi}{2}, \frac{\pi}{2}\right]}^{[-1,1]}$ que l'on notera $\widetilde{\sin}$. Sa bijection réciproque est $\arcsin : [-1,1] \to \left[ -\frac{\pi}{2}, \frac{\pi}{2}\right]$.\\
	Comme dans la démonstration, prenons $y\in [-1, 1]$. Comme $[-1,1]$ est centré en $0$, $-y\in [-1,1]$, et dès lors,
	\begin{align*}
		\arcsin(-y) & = \arcsin(-\widetilde{\sin}(\arcsin(y))) \\
		            & = \arcsin(\widetilde{\sin}(-\arcsin(y))) \\
		            & = -\arcsin(y).
	\end{align*}
	Ce qui prouve l'imparité de la fonction $\arcsin$.
\end{question_kholle}

\begin{question_kholle}{Montrer que les graphes d'une fonction et de sa bijection réciproque sont symétriques par rapport à la première bissectrice.}
	Calculons les coordonnées $(x',y')$ du point $M'$, image de $M$ de coordonnées $(x,y)$ par la réflexion $r$ d'axe la première bissectrice.
	\begin{align*}
		\begin{cases}
			\vv{MM'}\perp (\vv{i}+\vv{j}) \\
			\text{le milieu de $[MM']$ appartient à $\Delta$.}
		\end{cases}
		 & \iff
		\begin{cases}
			\vv{MM'}\cdot (\vv{i}+\vv{j})=0 \\
			\text{le milieu de $[MM']$ vérifie l'équation $y=x$.}
		\end{cases} \\
		 & \iff
		\begin{cases}
			\begin{pmatrix}x'-x\\y'-y\end{pmatrix}\cdot \begin{pmatrix}1\\1\end{pmatrix}=0 \\[12pt]
			\left(\frac{x+xy}{2},\frac{y+y'}{2}\right) \text{ vérifie l'équation $y=x$.}
		\end{cases}           \\
		 & \iff
		\begin{cases}
			x'-x+y'-y=0 \\
			\frac{x+x'}{2}=\frac{y+y'}{2}
		\end{cases}                                                          \\
		 & \iff
		\begin{cases}
			x'+y'=x+y \\
			x'-y'=-x+y
		\end{cases}
		\iff
		\begin{cases}
			x'=y \\
			y'=x
		\end{cases}
	\end{align*}
	L'expression de la réflexion $r$ d'axe la première bissectrice est ainsi
	\[
		r:\left|
		\begin{array}{rcl}
			\R^2  & \longrightarrow & \R^2  \\
			(x,y) & \longmapsto     & (y,x) \\
		\end{array}
		\right.
	\]
	Les graphes de $f$ et $f^{-1}$ étant respectivement
	\[
		G_f = \{(x,f(x))\in\R^2 \mid x\in I\} \quad\text{ et }\quad G_{f^{-1}}=\{(y,f^{-1}(y)) \mid y\in J\}
	\]
	on a bien
	\begin{align*}
		r(G_f) & =\{(f(x),x) \mid x\in I\}                                                      \\
		       & =\{(k,f^{-1}(y)) \mid y\in J\}\quad\text{ en posant $y=f(x) \iff x=f^{-1}(y)$} \\
		       & = G_{f^{-1}}
	\end{align*}
\end{question_kholle}

\begin{question_kholle}{Limite (et preuve) lorsque $x$ tend vers $+\infty$ de $\displaystyle\frac{(\ln x)^{\alpha}}{x^{\beta}}$ pour $\alpha ,\beta \in \left( \mathbb{R}_+^*\right) ^2$.}
	La fonction $\ln$ est concave sur $\R_+^*$, donc au dessous de toutes ses tangentes sur cet intervalle, et en particulier à celle au point d'abscisse $1$ qui a pour équation $y=x-1$. On a donc
	\[
		\forall x \in [1,+\infty[, \quad 0 \; \leq \; \ln (x) \; \leq \; x-1.
	\]
	Ce qui permet d'affirmer, en divisant par $x^2$, que
	\[
		\forall{x}\in [1,+\infty[, \quad 0 \; \leq \; \frac{\ln x}{x^2} \; \leq \; \underbrace{\frac{1}{x}}_{\arrowlim{x}{+\infty}0} - \underbrace{\frac{1}{x^2}}_{\arrowlim{x}{+\infty}0}
	\]
	Ainsi, le théorème d'existence de limite par encadrement permet de conclure que $\frac{\ln x}{x^2}$ admet une limite en $+\infty$ et que cette limite est nulle :
	\[
		\lim_{x\to+\infty}\frac{\ln x}{x} = 0
	\]
	On en déduit alors le cas général :
	\\
	\begin{minipage}{0.7\textwidth}
		\begin{align*}
			\frac{(\ln x)^\alpha}{x^\beta} & = \left(\frac{\ln x}{x^{\frac{\beta}{\alpha}}}\right)^{\alpha}                                                                                                                                              \\
			                               & = \left(\frac{\frac{2\alpha}{\beta}\ln\left(x^{\frac{\beta}{2\alpha}}\right)}{x^{\frac{\beta}{\alpha}}}\right)^{\alpha}                                                                                     \\
			                               & = \left(\frac{2\alpha}{\beta}\right)^{\alpha} \underbrace{\left[\underbrace{\frac{\ln\left(x^{\frac{\beta}{2\alpha}}\right)}{\left(x^{\frac{\beta}{2\alpha}}\right)^{2}}}_{\substack{\arrowlim{x}{+\infty}0 \\\text{\scriptsize v. ci-dessus}}}\right]^{\alpha}}_{\substack{\arrowlim{x}{+\infty}0\\\text{par composition des limites}}} \arrowlim{x}{+\infty} 0
		\end{align*}
	\end{minipage}
	\begin{minipage}{0.3\textwidth}
		\begin{figure}[H]
			\centering
			\begin{tikzpicture}
				\draw [->] (-0.5,0) -- (3,0);
				\draw [->] (0,-2) -- (0,2);
				\draw (1,-0.1) -- (1,0.1) node[below, yshift=-2mm] {\footnotesize $1$};
				\draw (-0.1,1) -- (0.1,1) node[left, xshift=-2mm] {\footnotesize $1$};
				\draw[domain=0.15:3, smooth, variable=\x, teal] plot ({\x}, {ln(\x)});
				\draw[domain=-0.5:3, smooth, variable=\x, purple] plot ({\x}, {\x-1});
			\end{tikzpicture}
			\caption{$\ln$ en bleu et $y=x-1$ en violet.}
		\end{figure}
	\end{minipage}
\end{question_kholle}

\begin{question_kholle}{Limite en $0$ de $\displaystyle\frac{(1+x)^{\alpha}-1}{x}$ et de $\displaystyle\frac{1-\cos x}{x^2}$.}
	\;\\
	\begin{itemize}[label=$*$]
		\item Le taux d'accroissement en $x_0$ d'une fonction $f$ dérivable en $x_0$ est
		      \[\tag{$\star$}
			      \lim_{x\to x_0}\frac{f(x)-f(x_0)}{x-x_0}=f'(x_0)
		      \]
		      \begin{itemize}
			      \item En appliquant $(\star)$ pour $f\leftarrow \bigl(x\mapsto (1+x)^{\alpha}\bigr)$ et $x_0\leftarrow 0$, on a
			            \[
				            \lim_{x\to 0}\frac{f(x)-f(0)}{x-0}=\lim_{x\to x_0}\frac{(1+x)^{\alpha}-(1+0)^{\alpha}}{x-0} = \lim_{x\to 0}\frac{(1+x)^{\alpha}-1}{x}=f'(0)=\alpha
			            \]
			            car $f': x\mapsto 1\cdot \alpha\cdot (1+x)^{\alpha - 1}$ vaut $\alpha$ en $0$.
			      \item De même, en appliquant $(\star)$ pour $f\leftarrow \sin$ et $x_0\leftarrow 0$, on a
			            \[
				            \lim_{x\to 0}\frac{f(x)-f(0)}{x-0}=\lim_{x\to x_0}\frac{\sin x-\sin 0}{x-0} = \lim_{x\to 0}\frac{\sin x}{x}=\sin' 0=\cos 0=1
			            \]
		      \end{itemize}
		\item Soit $x\in\R^*$ fixé quelconque.
		      \[
			      \frac{\cos x - 1}{x^2} = \frac{(\cos x-1)(\cos x+1)}{x^2(\cos x+1)} = \frac{\cos^2 x-1}{x^2(\cos x+1)}=-\biggl(\!\!\!\smash{\underbrace{\frac{\sin x}{x}}_{\arrowlim{x}{0}1}}\!\!\!\biggr)^2\cdot\underbrace{\left(\frac{1}{\cos x+1}\right)}_{\arrowlim{x}{0}\frac{1}{2}} \arrowlim{x}{0}\frac{-1}{2}
		      \]
	\end{itemize}
\end{question_kholle}

\begin{question_kholle}{Présentation exhaustive de la fonction $\arcsin$.}
	Premièrement, ladite fonction est la bijection réciproque de la fonction $\bigl.\sin\bigr|_{\left[ -\frac{\pi}{2}, \frac{\pi}{2}\right]}^{[-1,1]}$ (voir \textbf{1}.). D'où :
	\begin{equation*}
		\arcsin = \left\{
		\begin{array}{c c c}
			[-1,1] & \to     & [-\frac{\pi}{2} , \frac{\pi}{2}]                                                             \\ [1ex]
			x      & \mapsto & \left(\bigl.\sin\bigr|_{\left[ -\frac{\pi}{2}, \frac{\pi}{2}\right]}^{[-1,1]}\right)^{-1}(x)
		\end{array}
		\right.
	\end{equation*}
	Ainsi, pour $x\in [-1,1]$, $\arcsin (x)$ est l'unique solution de l'équation d'inconnue $\theta \in \textstyle \left[-\frac{\pi}{2} , \frac{\pi}{2}\right]$ :
	\[
		\sin(\theta) = x
	\]
	.
	\noindent Il découle alors naturellement des propriétés héréditairement acquises de $\bigl.\sin\bigr|_{\left[ -\frac{\pi}{2}, \frac{\pi}{2}\right]}^{[-1,1]}$ :

	\begin{enumerate}
		\item $\arcsin$ est impaire.
		\item $\arcsin$ est strictement croissante sur $[-1,1]$.
		\item $\arcsin \in \mathcal{C}^0\left([-1,1],[-\frac{\pi}{2} , \frac{\pi}{2}] \right)$.
		\item $\arcsin \in \mathcal{D}^1\left(]-1,1[,\left]-\frac{\pi}{2} , \frac{\pi}{2}\right[ \right)$.
		\item $\arcsin'(x) = \frac{1}{\sqrt{1-x^2}}$ pour tout $x\in]-1,1[$.
		\item $\arcsin$ admet deux demi-tangentes verticales en $-1$ et $1$.
	\end{enumerate}

	\

	Graphe de $\arcsin$ :
	\begin{figure}[H]
		\centering
		\begin{tikzpicture}[scale=2]
			\draw [->] (-1.5,0) -- (1.5,0);
			\draw [->] (0,-pi/2-0.5) -- (0,pi/2+0.5);
			\draw (1,-0.05) -- (1,0.05) node[below, yshift=-3mm] {\footnotesize $1$};
			\draw (-0.05,pi/2) -- (0.05,pi/2) node[left, xshift=-3mm] {\footnotesize $\frac{\pi}{2}$};
			\draw[thick, domain=-1.5:1.5, smooth, variable=\x, lime] plot ({\x}, {sin(\x*180/pi)});
			\draw[thick, domain=-1.5:1.5, smooth, variable=\x, teal] plot ({\x}, {\x});
			\draw[thick, domain=-1:1, samples=100, variable=\x, purple] plot ({\x}, {pi*asin(\x)/180});
			\draw [->] (-1,-pi/2) -- (-1,-pi/2+0.5);
			\draw [->] (1,pi/2) -- (1,pi/2-0.5);
			\draw [->] (pi/2,1) -- (pi/2-0.5,1);
			\draw [->] (-pi/2,-1) -- (-pi/2+0.5,-1);
		\end{tikzpicture}
		\caption{$\arcsin$ en violet, $\sin$ en vert et la première bissectrice en bleu.}
	\end{figure}
	On a aussi, grâce au taux d'accroissement en 0 d'$\arcsin$ :
	\[
		\lim_{x\to0} \frac{\arcsin(x)}{x} \ = \ 1.
	\]

	\

	Puis finalement (visible sur le graphe) :
	\[
		\forall x \in [0,1], \quad \arcsin(x) \geq x.
	\]
\end{question_kholle}

\begin{question_kholle}{Étude et tracé de $\sin\circ\arcsin$ (avec réduction du domaine d'étude à $[0,\pi/4]$).}
\end{question_kholle}
\end{document}

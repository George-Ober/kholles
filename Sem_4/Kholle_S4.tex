\documentclass{article}

\date{17 avril 2024}
\usepackage[nb-sem=4, auteurs={George Ober}]{../kholles}



\begin{document}

\maketitle
\begin{question_kholle}[{Considérons l'équation algébrique de degré 2:
    $$az^{2}+bz+c=0$$
    Où $z\in L$ est l'inconnue et $(a,b,c)\in\mathbb{C}^*\times\mathbb{C}^2$ sont des paramètres. Posons $\Delta = b^{2}-4ac$ que l'on appelle le discriminant de l'équation.
    \begin{itemize}
        \item Si $\Delta=0$, l'équation admet une unique solution dite double qui est $-\frac b{2a}$ et la forme factorisée du trinôme est
        $$
        az^2+bz+c = a \left( z + \frac b {2a} \right)^2
        $$
        \item Si $\Delta\neq0$, notons $\delta$ une racine carrée de $\Delta$, l'équation admet deux solutions distinctes $\frac{-b-\delta}{2a}$ et $\frac{-b+\delta}{2a}$ dites simples et la forme factorisée du trinôme est
        $$
        az^2+bz+c = a(z - \frac{-b-\delta}{2a})(z - \frac{-b+\delta}{2a})
        $$
    \end{itemize}
    }]{Résolution des équations algébriques de degré $2$ dans $\C$ et algorithme de recherche d'une racine carrée sous forme cartésienne (sur un exemple explicite).}
    La preuve est immédiate à partir de la forme canonique du trinôme du second degré :
    
    La preuve est immédiate à partir de la forme canonique du trinôme du second degré :
    
    \begin{align*}
        az^{2}+bz+c = a\left[ \underbrace{ z^{2}+\frac{b}{a}z }_{ \text{But: Absorber ces termes dans un carré} }+\frac{c}{a} \right] &= a\left[ z^{2}+2\frac{b}{2a}z +\frac{b^{2}}{4a^{2}}-\frac{b^{2}}{4a^{2}}+\frac{c}{a} \right]\\
        &=a \left[ \left( z+\frac{b}{2a} \right)^{2} - \frac{b^{2}}{4a^{2}}+\frac{c}{a} \right]  \\
        &=a \left[ \left( z+\frac{b}{2a} \right)^{2} - \frac{b^{2}-4ac}{4a^{2}}\right]  \\
        &=a \left[ \left( z+\frac{b}{2a} \right)^{2} - \frac{\Delta}{4a^{2}}\right] 
    \end{align*}
    \begin{itemize}
        \item Si $\Delta = 0$
        $$
        az^{2}+bz+c = a\left( z-\frac{-b}{2a} \right)^{2}
        $$
        de sorte que 
        $$
        az^{2}+bz+c = 0 \iff a\left( z-\frac{-b}{2a} \right)^{2} = 0 \iff z = -\frac{b}{2a}
        $$
        
        \item Sinon
        \begin{align*}
            az^{2}+bz+c = a \left[ \left( z+\frac{b}{2a} \right)^{2}-\left( \frac{\delta}{2a} \right)^{2} \right] &= a \left( z + \frac{b}{2a}-\frac{\delta}{2a} \right)\left( z+\frac{b}{2a}+\frac{\delta}{2a} \right)\\ &= a \left( z - \frac{-z+\delta}{2a} \right)\left( z- \frac{-z - \delta}{2a} \right)
        \end{align*}
        de sorte que
        
        \begin{align*}
            az^{2}+bz+c =0 &\iff a \left( z - \frac{-z+\delta}{2a} \right)\left( z- \frac{-z - \delta}{2a} \right) = 0\\ \\
            &\iff \left\{ \begin{array}{ll}
                z -  \frac{-z-\delta}{2a}  = 0 \\
                \text{ou} \\
                z- \frac{-z+\delta}{2a} =0
            \end{array}\right. \\
            &\iff \left\{ \begin{array}{ll}
                z = \frac{-z-\delta}{2a}  \\ \text{ou} \\
                z = \frac{-z+\delta}{2a} 
            \end{array}\right.
        \end{align*}
        
    \end{itemize}
\end{question_kholle}
\begin{question_kholle}
    [{L'exponentielle complexe a pour image $\mathbb{C}^{*}$ et, pour tout $z_{0} \in \mathbb{C}^{*}$,
    $$
    \exp_{\mathbb{C}}^{-1}(\{ z_{0} \}) = \{ \ln \lvert z_{0} \rvert +i\theta_{0} +2ik\pi \mid k \in \mathbb{Z}\}
    $$
    où $\theta_{0} \in \arg(z_{0})$}]{Résolution de $e^z = z_0$ où $z_0 \in \C^*$}
    
    La propriété, pour tout $z \in \mathbb{C}$, $\lvert e^{z} \rvert = \lvert e^{\mathrm{Re}(z)} \rvert>0$ montre que $0 \not\in \exp_{\mathbb{C}}(\mathbb{C})$.
    
    $z_{0}\neq 0$ donc $\exists \theta_{0} \in \arg(z_{0}): z_{0} = \lvert z_{0} \rvert e^{i\theta_{0}}$.
    Résolvons l'équation d'inconnue $z \in \mathbb{C}$
    
    \begin{align*}
        \exp_{\mathbb{C}}(z) = z_{0}  & \iff e^{\mathrm{Re}(z)}e^{i\mathrm{Im}(z)} = \lvert z_{0} \rvert e^{i\theta_{0}} \\
        & \iff \left\{ \begin{array}{ll}
            e^{\mathrm{Re}(z)} = \lvert z_{0} \rvert  \\
            \text{et} \\
            \mathrm{Im}(z) \equiv \theta_{0} [2\pi]
        \end{array}\right. \\
        & \iff \left\{ \begin{array}{ll}
            \mathrm{Re}(z)  = \ln \lvert  z_{0} \rvert  \\
            \text{et} \\
            \mathrm{Im}(z) \equiv \theta_{0} [2\pi]
        \end{array}\right. \\
        &\iff z \in \left\{ \ln \lvert z_{0} \rvert +i \theta_{0} +2ik\pi \mid k \in \mathbb{Z} \right\} 
    \end{align*}
    
\end{question_kholle}
\begin{question_kholle}{Montrer l'unicité de l'élément neutre et du symétrique d'un élément sous des hypothèses sur la loi à préciser.}
    \begin{itemize}[label=$\lozenge$]
        \item Unicité de l'élément neutre bilatère
        
        Soient $(e_{1}, e_{2}) \in E^{2}$ fixés quelconques tels que $\left\{ \begin{array}{ll} \forall x \in E, x * e_{1} = e_{1} * x = x \\ \forall x \in E, x*e_{2}=e_{2}*x = x\end{array}\right.$
        Particularisons la première relation pour $x \leftarrow e_{2}$:
        $$
        e_{2}*e_{1} = e_{1}*e_{2} = e_{2}
        $$
        En particularisant, de même la deuxième relation pour $x \leftarrow e_{1}$
        $$
        e_{1}*e_{2} = e_{2}*e_{1} = e_{2}
        $$
        D'où, par transitivité de l'égalité : $e_{1} = e_{2}$
        
        \item Unicité du symétrique sous réserve d'existence (LCI associative d'unité $e$)
        Soit $a \in E$ symétrisable
        $$
        \exists z \in E : a * z = z*a = e
        $$
        Fixons un tel $z$ pour la suite de la preuve
        \begin{itemize}
            \item L'ensemble $\{ y \in E \mid a * y = y * a = e \}$ n'est pas vide puisqu'il contient $z$.
            
            \item Soit $b \in \{ y \in E \mid a * y = y * a = e \}$ fixé quelconque.
            Alors
            \begin{align*}
                a * b = e &\implies z * ( a * b ) = z * e\\
                &\implies \underbrace{ z*a }_{ e } * b = z * e \text{ par associativité}\\
                &\implies b = z
            \end{align*}
        \end{itemize}
        Donc l'ensemble $\{ y \in E \mid a * y = y * a = e \}$ contient au plus un élément neutre, qui est $z$.
    \end{itemize}
\end{question_kholle}
\end{document}
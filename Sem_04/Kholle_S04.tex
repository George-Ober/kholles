\documentclass{article}

\date{05 octobre 2024}
\usepackage[nb-sem=4, auteurs={George Ober, Félix Rondeau}]{../kholles}



\begin{document}

\maketitle
\begin{question_kholle}{Montrer que l'ensemble des similitudes directes du plan complexe est un groupe pour la composition (la preuve de la bijectivité des similitudes fait partie de la question).}
  Soient $s$ et $s'$ deux similitudes directes. Alors, il existe $(a,a',b,b')\in(\C^*)^2\times \C^2$ tels que
  \[
    s:z\longmapsto az+b \quad\text{ et }\quad s':z\longmapsto a'z+b'
  \]
  \begin{itemize}[label=$\star$]
    \item $s\circ s': z\longmapsto a(a'z+b')+b = \underbrace{aa'}_{\in\C^*}z+\underbrace{ab'+b}_{\in\C}$ donc $s\circ s'$ est une similitude directe.\\
          Ainsi, la composition est une LCI sur l'ensemble des similitudes directes.
    \item La composition est associative
    \item La composition admet $id_{\C}:z\longmapsto 1z+0$ (qui est une similitude) comme neutre.
    \item Les similitudes directes sont des bijection du plan complexe car si $f$ est une similitude directe ($f:z\longmapsto az+b$ avec $(a,b)\in\C*\times\C$), pour tout $u\in\C$, l'équation d'inconnue $z\in\C$
          \[
            f(z)=u \iff az+b=u \iff z=\frac{1}{a}u-\frac{b}{a}
          \]
          admet une unique solution. De plus, la bijection réciproque $f^{-1}$ d'une similitude directe $f$ (vérifiant $f\circ f^{-1} = f^{-1}\circ f = id_{\C}$) est une similitude directe.\\
          Ainsi, toute similitude directe est symétrisable pour la loi de composition.
  \end{itemize}
  L'ensemble des similitudes directes du plan complexe muni de la loi de composition est donc bien un groupe.
\end{question_kholle}
\begin{question_kholle}{Classifier et interpréter une similitude directe donnée sous la forme $z \mapsto a z + b$ sur un exemple, donner l'expression complexe d'une similitude dont on connaît les éléments caractéristiques.}
  Soient $(a, b) \in \mathbb{C}^*$ fixés quelconques. Posons la similitude
  \begin{align*}
    s \left| \begin{array}{ll}
               \mathbb{C} & \to \mathbb{C}  \\
               z          & \mapsto a z + b
             \end{array}\right.
  \end{align*}
  \begin{itemize}[label=$\lozenge$]
    \item Si $a = 1$, c'est la translation de vecteur d'affixe $b$.
    \item Si $a \neq 1$, $s$ admet un unique point fixe appelé «~centre de la similitude~» $\omega = \frac{b}{1-a}$
          \begin{itemize}[label=$\star$]
            \item Si $a \in \mathbb{R}^*$, $s$ est l'homothétie de centre $\omega$ et de rapport $a$.
            \item Si $a \in (\mathbb{C}^* \setminus \mathbb{R})$, $s$ est la composée commutative de :
                  \begin{itemize}
                    \item La rotation de centre $\omega$ et d'angle $\alpha$, où $\alpha$ est un argument de $a$.
                    \item L'homothétie de centre $\omega$ et de rapport $|a|$.
                  \end{itemize}
                  On nommera alors $|a|$ le rapport de $s$ et $\alpha$ une mesure de l'angle de $s$.
          \end{itemize}
  \end{itemize}
  \textbf{Exemple} : Prenons la similitude $s: z \mapsto (1 - i) z - 1$.
  \begin{align*}
    s(z) = z & \iff (1 - i)z - 1 = z \\
             & \iff -iz = 1          \\
             & \iff z = i
  \end{align*}
  De plus,
  \begin{align*}
    (1 - i)z - 1 = \sqrt{2}e^{-i \frac{\pi}{4}}z - 1
  \end{align*}
  On en déduit que $s$ est la similitude directe de centre d'affixe $i$, de rapport $\sqrt{2}$, et d'angle $-\frac{\pi}{4}$.
\end{question_kholle}
\begin{question_kholle}
  [{L'exponentielle complexe a pour image $\mathbb{C}^{*}$ et, pour tout $z_{0} \in \mathbb{C}^{*}$,
        $$
          \exp_{\mathbb{C}}^{-1}(\{ z_{0} \}) = \{ \ln \lvert z_{0} \rvert +i\theta_{0} +2ik\pi \mid k \in \mathbb{Z}\}
        $$
        où $\theta_{0} \in \arg(z_{0})$}]{Résolution de $e^z = z_0$ où $z_0 \in \C^*$}

  La propriété: $\forall{z} \in \mathbb{C}$, $\lvert e^{z} \rvert = \lvert e^{\mathrm{Re}(z)} \rvert>0$ montre que $0 \not\in \exp_{\mathbb{C}}(\mathbb{C})$.\\
  $z_{0}\neq 0$ donc $\exists \theta_{0} \in \arg(z_{0}): z_{0} = \lvert z_{0} \rvert e^{i\theta_{0}}$.
  Résolvons l'équation d'inconnue $z \in \mathbb{C}$
  \begin{align*}
    \exp_{\mathbb{C}}(z) = z_{0} & \iff e^{\mathrm{Re}(z)}e^{i\mathrm{Im}(z)} = \lvert z_{0} \rvert e^{i\theta_{0}}                \\
                                 & \iff \left\{ \begin{array}{ll}
                                                  e^{\mathrm{Re}(z)} = \lvert z_{0} \rvert \\
                                                  \text{et}                                \\
                                                  \mathrm{Im}(z) \equiv \theta_{0} [2\pi]
                                                \end{array}\right.                                    \\
                                 & \iff \left\{ \begin{array}{ll}
                                                  \mathrm{Re}(z)  = \ln \lvert  z_{0} \rvert \\
                                                  \text{et}                                  \\
                                                  \mathrm{Im}(z) \equiv \theta_{0} [2\pi]
                                                \end{array}\right.                                         \\
                                 & \iff z \in \left\{ \ln \lvert z_{0} \rvert +i \theta_{0} +2ik\pi \mid k \in \mathbb{Z} \right\}
  \end{align*}

\end{question_kholle}
\begin{question_kholle}{Montrer l'unicité de l'élément neutre et du symétrique d'un élément sous des hypothèses sur la loi à préciser.}
  \;\\
  \begin{itemize}[label=$\lozenge$]
    \item Unicité de l'élément neutre bilatère

          Soient $(e_{1}, e_{2}) \in E^{2}$ fixés quelconques tels que $\left\{ \begin{array}{ll} \forall x \in E, x * e_{1} = e_{1} * x = x \\ \forall x \in E, x*e_{2}=e_{2}*x = x\end{array}\right.$.\\
          Particularisons la première relation pour $x \leftarrow e_{2}$:
          $$
            e_{2}*e_{1} = e_{1}*e_{2} = e_{2}
          $$
          En particularisant de même la deuxième relation pour $x \leftarrow e_{1}$ :
          $$
            e_{1}*e_{2} = e_{2}*e_{1} = e_{1}
          $$
          D'où, par transitivité de l'égalité : $e_{1} = e_{2}$

    \item Unicité du symétrique sous réserve d'existence (LCI associative d'unité $e$).\\
          Soit $a \in E$ symétrisable
          $$
            \exists z \in E : a * z = z*a = e
          $$
          Fixons un tel $z$ pour la suite de la preuve
          \begin{itemize}
            \item L'ensemble $\{ y \in E \mid a * y = y * a = e \}$ n'est pas vide puisqu'il contient $z$.

            \item Soit $b \in \{ y \in E \mid a * y = y * a = e \}$ fixé quelconque.
                  Alors
                  \begin{align*}
                    a * b = e & \implies z * ( a * b ) = z * e                                          \\
                              & \implies \underbrace{ z*a }_{ e } * b = z * e \text{ par associativité} \\
                              & \implies b = z
                  \end{align*}
          \end{itemize}
          Donc l'ensemble $\{ y \in E \mid a * y = y * a = e \}$ contient au plus un élément qui est $z$.
  \end{itemize}
\end{question_kholle}
\begin{question_kholle}[{Soit $(G, *)$ un groupe, et $H$ une partie de $G$ $$
          H\text{ est un sous-groupe de }G \iff \left\{ \begin{array}{ll}
            H \neq \emptyset \\
            \forall (x, y) \in H^{2}, x * y^{-1} \in H
          \end{array}\right.
        $$}]{Preuve de la caractérisation d'un sous-groupe, application au fait que $(\mathbb U _n, \times)$ est un sous-groupe de $(\mathbb U, \times)$.}
  \;\\
  \begin{itemize}[label=$\star$]
    \item Supposons que $H$ est un sous-groupe de $G$.
          Par définition d'un sous-groupe, $H \neq \emptyset$.

          Soient $(x, y) \in H^{2}$ fixés quelconques.\\
          $H$ est un sous-groupe donc $y$ est symétrisable dans $H$: $y^{-1} \in H$.\\
          De plus, c'est un groupe, donc stable pour la loi $*$, donc $x*y^{-1} \in H$

    \item Supposons que $\left\{ \begin{array}{ll}H \neq \emptyset &(1) \\\forall (x, y) \in H^{2}, x * y^{-1} \in H & (2)\end{array}\right.$
          \begin{itemize}[label=$\lozenge$]
            \item $H$ est non vide par hypothèse
            \item Puisque $H \neq \emptyset$, $\exists h \in H$.\\
                  Ainsi, en appliquant (2) pour $(x,y)\leftarrow (h,h)$, on obtient $h * h^{-1} \in H$ donc $H$ possède un élément neutre $e$.
            \item Soit $h \in H$ fixé quelconque.
                  $h \in H$ permet d'appliquer (2) pour $(x,y) \leftarrow (h,e)$:
                  $$
                    e * h^{-1} \in H
                  $$
                  Donc $h^{-1} \in H$. Ainsi, tout élément est symétrisable dans $H$
            \item Soient $(x, y) \in H^{2}$ fixés quelconques. On a montré que $y$ est symétrisable dans H, donc en appliquant (2) pour $x \leftarrow x$ et $y \leftarrow y^{-1}$:
                  $$
                    x * (y^{-1})^{-1} \in H \implies x* y \in H
                  $$
                  Donc $H$ est stable pour la loi $H$.
          \end{itemize}
          Donc $H$ est un sous-groupe de $G$.
  \end{itemize}
\end{question_kholle}
\textbf{Application aux racines n-ièmes de l'unité}

Soit $n \in \mathbb{N}^{*}$ fixé quelconque
\begin{itemize}[label=$\star$]
  \item $\forall z \in \mathbb{U}_{n}, z^{n} = 1$ donc $1 = \lvert  z^{n} \rvert = \lvert z \rvert^{n}$. Or $\lvert z \rvert \geqslant 0$ donc $\lvert z \rvert = 1$, si bien que $\mathbb{U}_{n} \subset \mathbb{U}$

  \item $\mathbb{U}_{n} \neq \emptyset$ car $1 \in \mathbb{U}_{n}$

  \item Soient $(z_{1}, z_{2}) \in \mathbb{U}_{n}$ fixés quelconques.
        Calculons
        $$
          (z_{1}z_{2}^{-1})^{n} = \left( \frac{z_{1}}{z_{2}} \right)^{n} = \frac{z_{1}^{n}}{z_{2}^{n}} = \frac{1}{1} = 1
        $$
\end{itemize}
Donc $z_{1}z_{2}^{-1} \in \mathbb{U}_{n}$. On a donc montré que $(\mathbb{U}_{n, \times})$ est un sous-groupe de $(\mathbb{U}, \times)$.

\begin{question_kholle}{Si $\varphi$ est un morphisme de groupes de $G_1$ de neutre $e_1$ dans $G_2$ de neutre $e_2$, calculer $\varphi(e_1)$ et $\varphi(x^{-1})$}
  \;\\
  \begin{itemize}[label=$\star$,nosep]
    \item Soit $f$ un morphisme de groupe de $(G_{1}, *_{1})$ dans $(G_{2}, *_{2})$

          D'une part $f(e_{1} *_{1} e_{1}) = f(e_{1})$.\\
          D'autre part, par propriété de morphisme, $f(e_{1} *_{1} e_{1}) =f(e_{1})*_{2} f(e_{1})$, donc
          $$
            f(e_{1}) *_{2} f(e_{1}) = f(e_{1})
          $$
          Si l'on compose à gauche par $f(e_{1})^{-1}$,
          \begin{align*}
            f(e_{1})^{-1} *_{2} f(e_{1}) *_{2} f(e_{1}) = f(e_{1})^{-1} *_{2}f(e_{1}) \implies f(e_{1}) = e_{2}
          \end{align*}


    \item Soit $x \in G_{1}$ fixé quelconque.
          $$
            f(x)*_{2}f(x^{-1}) = f(x*_{1}x^{-1})=f(e_{1}) = e_{2}
          $$
          Composons les deux membres à gauche par $f(x)^{-1}$ :
          $$
            f(x)^{-1} *_{2} f(x)*_{2}f(x^{-1}) = f(x)^{-1} *_{2} * e_{2}
          $$
          Donc
          $$
            f(x^{-1}) = f(x)^{-1}
          $$

  \end{itemize}
\end{question_kholle}
\begin{question_kholle}{Montrer que l'image directe d'un sous-groupe par un morphisme de groupes est un sous-groupe du groupe d'arrivée}
  Soit $f$ un morphisme de groupe de $(G_{1},*_{1})$ dans $(G_{2}, *_{2})$.\\
  Notons $e_{1}$ et $e_{2}$ les neutres respectifs de $G_{1}$ et $G_{2}$.

  Soit $H_{1}$ un sous-groupe de $G_{1}$ fixé quelconque
  \begin{itemize}[label=$\star$]
    \item $f(H_{1})$ est par définition une partie de $G_{2}$.
    \item $f(H_{1}) \neq \emptyset$ car $H_{1}$ est un groupe qui contient $e_{1}$ et $f(e_{1}) = e_{2}$ donc $e_{2} \in f(H_{1})$.
    \item Soient $(g_{2},h_{2}) \in f(H_{1})^{2}$ fixés quelconques, alors $\exists (g_{1}, h_{1}) \in H_{1} : f(g_{1})= g_{2} \text{ et } f(h_{1})=h_{2}$.
          Par conséquent,
          $$
            g_{2}*_{2}h_{2}^{-1} = f(g_{1}) *_{2} f(h_{1}^{-1}) = f(\underbrace{ g_{1} *_{1} h_{1}^{-1} }_{ \in H_{1} \text{ car sous-groupe de }G_{1} })
          $$
          Ainsi, $g_{2} *_{2} h_{2}^{-1} \in f(H_{1})$
          d'où $f(H_{1})$ est un sous-groupe de $G_{2}$
  \end{itemize}
\end{question_kholle}

\begin{question_kholle}{Montrer que l'image réciproque par un morphisme de groupes d'un sous-groupe est toujours un sous-groupe du groupe de départ, }
  Soit $f$ un morphisme de groupe de $(G_{1},*_{1})$ dans $(G_{2}, *_{2})$.\\
  Notons $e_{1}$ et $e_{2}$ les neutres respectifs de $G_{1}$ et $G_{2}$.

  Soit $H_{2}$ un sous-groupe de $G_{2}$ fixé quelconque.
  \begin{itemize}[label=$\star$]
    \item $f^{-1}(H_{2})$ est par définition une partie de $G_{1}$.
    \item $f(H_{2}) \neq \emptyset$ car $H_{2}$ est un groupe qui contient $e_{2}$ et $f(e_{1}) = e_{2}$ donc $e_{1} \in f^{-1}(H_{2})$.
    \item Soient $(g_{1},h_{1}) \in f^{-1}(H_{2})^{2}$ fixés quelconques, alors $f(g_{1}) \in H_{2}$ et $f(h_{1}) \in H_{2}$, donc
          $$
            f(g_{1} *_{1} h_{1}^{-1}) = \underbrace{ f(g_{1}) }_{ \in H_{2} }*_{2}\underbrace{ f(h_{1})^{-1} }_{ \in H_{2} } \in H_{2} \quad\text{ car c'est un sous-groupe}
          $$
          Ainsi, $f(g_{1} *_{1} h_{1}^{-1}) \in H_{2}$ d'où $g_{1} *_{1} h_{1}^{-1} \in f^{-1}(H_{2})$
  \end{itemize}
  $f^{-1}(H_{2})$ est donc un sous-groupe de $G_{1}$.
\end{question_kholle}

\begin{itemize}[label=$\vartriangleright$]
  \item \textbf{Application 1:} Le noyau d'un morphisme est un sous-groupe.\\
        Le noyau, noté $\ker f$ est par définition égal à $f^{-1}(\{e_2\})$ c'est donc un sous-groupe de $G_1$.

  \item \textbf{Application 2:} Pour tout $n \in \mathbb{N}$, l'application
        \[
          \phi _{n}\left|\begin{array}{ll} (\mathbb{C}^{*},\times) &\to (\mathbb{C}^{*},\times) \\ z &\mapsto z^{n} \end{array}\right.
        \]
        est un morphisme de groupes. Son noyau est $\ker \phi _n = \{ z \in \mathbb{C}^{*} | z^{n} = 1 \} = \mathbb{U}_{n}$.\\
        D'après l'application 1, $\ker \phi_{n}$ est un sous-groupe de $(\mathbb{C}^{*}, \times)$, donc $(\mathbb{U}_{n}, \times)$ est un sous-groupe de $(\mathbb{C}^{*},\times)$.
\end{itemize}

\begin{question_kholle}{\textit{[non demandée]} Montrer que l'ensemble des similitudes directes du plan complexe est un groupe pour la composition (démonstration alternative)}
  Montrons donc que $(S, \circ)$ est un sous-groupe de $(\mathcal{S}(\mathbb{C}), \circ)$
  \begin{itemize}[label=$\lozenge$]
    \item D'une part, $S \subset \mathcal{S}(\mathbb{C})$. Or l'ensemble des permutations $(\mathcal{S}(\mathbb{C}), \circ)$ est un groupe. En effet, les similitudes sont des bijections de $\mathbb{C} \to \mathbb{C}$.
    \item De plus, $S$ est non vide, par exemple l'application $\mathrm{Id(\mathbb{C}})$ est une similitude pour $a \leftarrow 1$ et $b \leftarrow 1$.
    \item Prenons finalement $a$ et $c$ dans $\mathbb{C}^*$ puis $b$ et $d$ dans $\mathbb{C}$, et posons les deux applications suivantes:
          \begin{align*}
            s \left| \begin{array}{ll}
                       \mathbb{C} & \to \mathbb{C}  \\
                       z          & \mapsto a z + b
                     \end{array}\right.
            \text{ et }
            s' \left| \begin{array}{ll}
                        \mathbb{C} & \to \mathbb{C}  \\
                        z          & \mapsto a z + b
                      \end{array}\right.
          \end{align*}
          Ainsi, comme toute similitude directe est une bijection, en particulier $s'$ en est une, et
          \begin{align*}
            s'^{-1} \left| \begin{array}{ll}
                             \mathbb{C} & \to \mathbb{C}                    \\
                             z          & \mapsto \frac{z}{c} - \frac{d}{c}
                           \end{array}\right.
          \end{align*}
          Soit $z \in \mathbb{C}$ fixé quelconque:
          \begin{align*}
            (s \circ s'^{-1})(z) & = s(s'^{-1}(z))                                  \\
                                 & = s\left(\frac{z}{c} - \frac{d}{c}\right)        \\
                                 & = a\left(\frac{z}{c} - \frac{d}{c}\right) + b    \\
                                 & = \frac{a}{c}z + \left( b - \frac{ad}{c} \right)
          \end{align*}
          Qui est une similitude directe, puisque $\frac{a}{c} \neq  0$ donc $s \circ s'^{-1} \in S$. Donc $(S, \circ)$ est bien un sous-groupe de $(\mathcal{S}(\mathbb{C}), \circ)$.
  \end{itemize}
\end{question_kholle}
\end{document}

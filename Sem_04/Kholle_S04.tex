\documentclass{article}

\date{17 avril 2024}
\usepackage[nb-sem=4, auteurs={George Ober}]{../kholles}



\begin{document}

\maketitle
\begin{question_kholle}[{Considérons l'équation algébrique de degré 2:
				$$az^{2}+bz+c=0$$
				Où $z\in L$ est l'inconnue et $(a,b,c)\in\mathbb{C}^*\times\mathbb{C}^2$ sont des paramètres. Posons $\Delta = b^{2}-4ac$ que l'on appelle le discriminant de l'équation.
				\begin{itemize}
					\item Si $\Delta=0$, l'équation admet une unique solution dite double qui est $-\frac b{2a}$ et la forme factorisée du trinôme est
					      $$
						      az^2+bz+c = a \left( z + \frac b {2a} \right)^2
					      $$
					\item Si $\Delta\neq0$, notons $\delta$ une racine carrée de $\Delta$, l'équation admet deux solutions distinctes $\frac{-b-\delta}{2a}$ et $\frac{-b+\delta}{2a}$ dites simples et la forme factorisée du trinôme est
					      $$
						      az^2+bz+c = a\left(z - \frac{-b-\delta}{2a}\right)\left(z - \frac{-b+\delta}{2a}\right)
					      $$
				\end{itemize}
			}]{Résolution des équations algébriques de degré $2$ dans $\C$ et algorithme de recherche d'une racine carrée sous forme cartésienne (sur un exemple explicite).}
	La preuve est immédiate à partir de la forme canonique du trinôme du second degré :

	La preuve est immédiate à partir de la forme canonique du trinôme du second degré :

	\begin{align*}
		az^{2}+bz+c = a\left[ \underbrace{ z^{2}+\frac{b}{a}z }_{ \text{But: Absorber ces termes dans un carré} }+\frac{c}{a} \right] & = a\left[ z^{2}+2\frac{b}{2a}z +\frac{b^{2}}{4a^{2}}-\frac{b^{2}}{4a^{2}}+\frac{c}{a} \right] \\
		                                                                                                                              & =a \left[ \left( z+\frac{b}{2a} \right)^{2} - \frac{b^{2}}{4a^{2}}+\frac{c}{a} \right]        \\
		                                                                                                                              & =a \left[ \left( z+\frac{b}{2a} \right)^{2} - \frac{b^{2}-4ac}{4a^{2}}\right]                 \\
		                                                                                                                              & =a \left[ \left( z+\frac{b}{2a} \right)^{2} - \frac{\Delta}{4a^{2}}\right]
	\end{align*}
	\begin{itemize}
		\item Si $\Delta = 0$
		      $$
			      az^{2}+bz+c = a\left( z-\frac{-b}{2a} \right)^{2}
		      $$
		      de sorte que
		      $$
			      az^{2}+bz+c = 0 \iff a\left( z-\frac{-b}{2a} \right)^{2} = 0 \iff z = -\frac{b}{2a}
		      $$

		\item Sinon
		      \begin{align*}
			      az^{2}+bz+c = a \left[ \left( z+\frac{b}{2a} \right)^{2}-\left( \frac{\delta}{2a} \right)^{2} \right] & = a \left( z + \frac{b}{2a}-\frac{\delta}{2a} \right)\left( z+\frac{b}{2a}+\frac{\delta}{2a} \right) \\ &= a \left( z - \frac{-z+\delta}{2a} \right)\left( z- \frac{-z - \delta}{2a} \right)
		      \end{align*}
		      de sorte que

		      \begin{align*}
			      az^{2}+bz+c =0 & \iff a \left( z - \frac{-z+\delta}{2a} \right)\left( z- \frac{-z - \delta}{2a} \right) = 0 \\ \\
			                     & \iff \left\{ \begin{array}{ll}
				                                    z -  \frac{-z-\delta}{2a}  = 0 \\
				                                    \text{ou}                      \\
				                                    z- \frac{-z+\delta}{2a} =0
			                                    \end{array}\right.                                                \\
			                     & \iff \left\{ \begin{array}{ll}
				                                    z = \frac{-z-\delta}{2a} \\ \text{ou} \\
				                                    z = \frac{-z+\delta}{2a}
			                                    \end{array}\right.
		      \end{align*}

	\end{itemize}
\end{question_kholle}
\begin{question_kholle}
	[{L'exponentielle complexe a pour image $\mathbb{C}^{*}$ et, pour tout $z_{0} \in \mathbb{C}^{*}$,
				$$
					\exp_{\mathbb{C}}^{-1}(\{ z_{0} \}) = \{ \ln \lvert z_{0} \rvert +i\theta_{0} +2ik\pi \mid k \in \mathbb{Z}\}
				$$
				où $\theta_{0} \in \arg(z_{0})$}]{Résolution de $e^z = z_0$ où $z_0 \in \C^*$}

	La propriété, pour tout $z \in \mathbb{C}$, $\lvert e^{z} \rvert = \lvert e^{\mathrm{Re}(z)} \rvert>0$ montre que $0 \not\in \exp_{\mathbb{C}}(\mathbb{C})$.

	$z_{0}\neq 0$ donc $\exists \theta_{0} \in \arg(z_{0}): z_{0} = \lvert z_{0} \rvert e^{i\theta_{0}}$.
	Résolvons l'équation d'inconnue $z \in \mathbb{C}$

	\begin{align*}
		\exp_{\mathbb{C}}(z) = z_{0} & \iff e^{\mathrm{Re}(z)}e^{i\mathrm{Im}(z)} = \lvert z_{0} \rvert e^{i\theta_{0}}                \\
		                             & \iff \left\{ \begin{array}{ll}
			                                            e^{\mathrm{Re}(z)} = \lvert z_{0} \rvert \\
			                                            \text{et}                                \\
			                                            \mathrm{Im}(z) \equiv \theta_{0} [2\pi]
		                                            \end{array}\right.                                   \\
		                             & \iff \left\{ \begin{array}{ll}
			                                            \mathrm{Re}(z)  = \ln \lvert  z_{0} \rvert \\
			                                            \text{et}                                  \\
			                                            \mathrm{Im}(z) \equiv \theta_{0} [2\pi]
		                                            \end{array}\right.                                         \\
		                             & \iff z \in \left\{ \ln \lvert z_{0} \rvert +i \theta_{0} +2ik\pi \mid k \in \mathbb{Z} \right\}
	\end{align*}

\end{question_kholle}
\begin{question_kholle}{Montrer l'unicité de l'élément neutre et du symétrique d'un élément sous des hypothèses sur la loi à préciser.}
	\begin{itemize}[label=$\lozenge$]
		\item Unicité de l'élément neutre bilatère

		      Soient $(e_{1}, e_{2}) \in E^{2}$ fixés quelconques tels que $\left\{ \begin{array}{ll} \forall x \in E, x * e_{1} = e_{1} * x = x \\ \forall x \in E, x*e_{2}=e_{2}*x = x\end{array}\right.$
		      Particularisons la première relation pour $x \leftarrow e_{2}$:
		      $$
			      e_{2}*e_{1} = e_{1}*e_{2} = e_{2}
		      $$
		      En particularisant, de même la deuxième relation pour $x \leftarrow e_{1}$
		      $$
			      e_{1}*e_{2} = e_{2}*e_{1} = e_{2}
		      $$
		      D'où, par transitivité de l'égalité : $e_{1} = e_{2}$

		\item Unicité du symétrique sous réserve d'existence (LCI associative d'unité $e$)
		      Soit $a \in E$ symétrisable
		      $$
			      \exists z \in E : a * z = z*a = e
		      $$
		      Fixons un tel $z$ pour la suite de la preuve
		      \begin{itemize}
			      \item L'ensemble $\{ y \in E \mid a * y = y * a = e \}$ n'est pas vide puisqu'il contient $z$.

			      \item Soit $b \in \{ y \in E \mid a * y = y * a = e \}$ fixé quelconque.
			            Alors
			            \begin{align*}
				            a * b = e & \implies z * ( a * b ) = z * e                                          \\
				                      & \implies \underbrace{ z*a }_{ e } * b = z * e \text{ par associativité} \\
				                      & \implies b = z
			            \end{align*}
		      \end{itemize}
		      Donc l'ensemble $\{ y \in E \mid a * y = y * a = e \}$ contient au plus un élément neutre, qui est $z$.
	\end{itemize}
\end{question_kholle}
\begin{question_kholle}[{Soit $(G, *)$ un groupe, et $H$ une partie de $G$ $$
					H\text{ est un sous groupe de }G \iff \left\{ \begin{array}{ll}
						H \neq \emptyset \\
						\forall (x, y) \in H^{2}, x * y^{-1} \in H
					\end{array}\right.
				$$}]{Preuve de la caractérisation d'un sous-groupe, application au fait que $(\mathbb U _n, \times)$ est un sous-groupe de $(\mathbb U, \times)$.}
	\begin{itemize}[label=$\star$]
		\item Supposons que $H$ est un sous groupe de $G$
		      Par définition d'un sous groupe, $H \neq \emptyset$.

		      Soient $(x, y) \in H^{2}$ fixés quelconques
		      $H$ est un sous groupe, donc $y$ est symétrisable dans $H$ : $y^{-1} \in H$
		      De plus, c'est un groupe, donc stable pour la loi $*$, donc $x*y^{-1} \in H$

		\item Supposons que $\left\{ \begin{array}{ll}H \neq \emptyset &(1) \\\forall (x, y) \in H^{2}, x * y^{-1} \in H & (2)\end{array}\right.$
		      \begin{itemize}[label=$\lozenge$]
			      \item $H$ est non vide par hypothèse
			      \item Puisque $H \neq \emptyset$, $\exists x \in H$
			            Ainsi, $x * x^{-1} \in H$ donc $e \in H$
			      \item Soient $(x, y) \in H^{2}$ fixés quelconques.
			            $y \in H$ permet d'appliquer (2) pour $x \leftarrow y$ et $y \leftarrow e$:
			            $$
				            e * y^{-1} \in H
			            $$
			            Donc $y^{-1} \in H$. Tout élément est symétrisable dans $H$
			      \item Soient $(x, y) \in H^{2}$ fixés quelconques. On a montré que $y$ est symétrisable dans H, donc en appliquant (2) pour $x \leftarrow x$ et $y \leftarrow y^{-1}$:
			            $$
				            x * (y^{-1})^{-1} \in H \implies x* y \in H
			            $$
			            Donc $H$ est stable pour la loi $H$.
		      \end{itemize}
		      Donc $H$ est un sous groupe de $G$.
	\end{itemize}
\end{question_kholle}
\textbf{Application aux racines n-ièmes de l'unité}

Soit $n \in \mathbb{N}^{*}$ fixé quelconque
\begin{itemize}[label=$\star$]
	\item $\forall z \in \mathbb{U}_{n}, z^{n} = 1$ donc $1 = \lvert  z^{n} \rvert = \lvert z \rvert^{n}$. Or $\lvert z \rvert \geqslant 0$ donc $\lvert z \rvert = 1$, si bien que $\mathbb{U}_{n} \subset \mathbb{U}$

	\item $\mathbb{U}_{n} \neq \emptyset$ car $1 \in \mathbb{U}_{n}$

	\item Soient $(z_{1}, z_{2}) \in \mathbb{U}_{n}$ fixés quelconques.
	      Calculons
	      $$
		      (z_{1}z_{2}^{-1})^{n} = \left( \frac{z_{1}}{z_{2}} \right)^{n} = \frac{z_{1}^{n}}{z_{2}^{n}} = \frac{1}{1} = 1
	      $$
\end{itemize}
Donc $z_{1}z_{2}^{-1} \in \mathbb{U}_{n}$. On a donc montré que $(\mathbb{U}_{n, \times})$ est un sous groupe de $(\mathbb{U}, \times)$.

\begin{question_kholle}{Si $\varphi$ est un morphisme de groupes de $G_1$ de neutre $e_1$ dans $G_2$ de neutre $e_2$, calculer $\varphi(e_1)$ et $\varphi(x^{-1})$}
	\begin{itemize}[label=$\star$]
		\item Soit $f$ un morphisme de groupe de $(G_{1}, *_{1})$ dans $(G_{2}, *_{2})$

		      D'une part $f(e_{1} *_{1} e_{1}) = f(e_{1})$
		      D'autre part, par propriété de morphisme, $f(e_{1} *_{1} e_{1}) =f(e_{1})*_{2} f(e_{1})$
		      Donc
		      $$
			      f(e_{1}) *_{2} f(e_{1}) = f(e_{1})
		      $$
		      Si l'on compose à gauche par $f(e_{1})^{-1}$

		      \begin{align*}
			      f(e_{1})^{-1} *_{2} f(e_{1}) *_{2} f(e_{1}) & = f(e_{1})^{-1} *_{2}f(e_{1}) \\
			      \implies f(e_{1})                           & = e_{2}
		      \end{align*}


		\item Soit $x \in G_{1}$ fixé quelconque
		      $$
			      f(x)*_{2}f(x^{-1}) = f(x*_{1}x^{-1})=f(e_{1}) = e_{2}
		      $$
		      Composons les deux membres à gauche par $f(x)^{-1}$
		      $$
			      f(x)^{-1} *_{2} f(x)*_{2}f(x^{-1}) = f(x)^{-1} *_{2} * e_{2}
		      $$
		      Donc
		      $$
			      f(x^{-1}) = f(x)^{-1}
		      $$

	\end{itemize}
\end{question_kholle}
\begin{question_kholle}{Montrer que l'image d'un sous-groupe par un morphisme de groupes est un sous-groupe du groupe d'arrivée}
	Soit $f$ un morphisme de groupe de $(G_{1},*_{1})$ dans $(G_{2}, *_{2})$. Notons $e_{1}$ et $e_{2}$ les neutres respectifs de $G_{1}$ et $G_{2}$.

	Soit $H_{1}$ un sous groupe de $G_{1}$ fixé quelconque
	\begin{itemize}[label=$\star$]
		\item $f(H_{1})$ est par définition une partie de $G_{2}$
		\item $f(H_{1}) \neq \emptyset$ car $H_{1}$ est un groupe, qui contient $e_{1}$ $f(e_{1}) = e_{2}$ donc $e_{2} \in f(H_{1})$
		\item Soient $(g_{2},h_{2}) \in f(H_{1})^{2}$ fixés quelconques
		      alors $\exists (g_{1}, h_{1}) \in H_{1} : f(g_{1})= g_{2}, f(h_{1})=h_{2}$
		      donc
		      $$
			      g_{2}*_{2}h_{2}^{-1} = f(g_{1}) *_{2} f(h_{1}^{-1}) = f(\underbrace{ g_{1} *_{1} h_{1}^{-1} }_{ \in H_{1} \text{ car sous groupe de }G_{1} })
		      $$
		      Donc $g_{2} *_{2} h_{2}^{-1} \in f(H_{1})$
		      Donc $f(H_{1})$ est un sous groupe de $G_{2}$
	\end{itemize}
\end{question_kholle}

\begin{question_kholle}{Montrer que l'image réciproque par un morphisme de groupes d'un sous-groupe est toujours un sous-groupe du groupe de départ, }
	Soit $f$ un morphisme de groupe de $(G_{1},*_{1})$ dans $(G_{2}, *_{2})$. Notons $e_{1}$ et $e_{2}$ les neutres respectifs de $G_{1}$ et $G_{2}$.

	Soit $H_{2}$ un sous groupe de $G_{2}$ fixé quelconque
	\begin{itemize}[label=$\star$]
		\item $f^{-1}(H_{2})$ est par définition une partie de $G_{1}$
		\item $f(H_{2}) \neq \emptyset$ car $H_{2}$ est un groupe, qui contient $e_{2}$ $f(e_{1}) = e_{2}$ donc $e_{1} \in f^{-1}(H_{2})$
		\item Soient $(g_{1},h_{1}) \in f^{-1}(H_{2})^{2}$ fixés quelconques
		      alors $f(g_{1}) \in H_{2}$ et $f(h_{1}) \in H_{2}$
		      donc
		      $$
			      f(g_{1} *_{1} h_{1}^{-1}) = \underbrace{ f(g_{1}) }_{ \in H_{2} }*_{2}\underbrace{ f(h_{1})^{-1} }_{ \in H_{2} } \in H_{2} \text{ car c'est un sous groupe}$$
		      Donc $f(g_{1} *_{1} h_{1}^{-1}) \in H_{2}$ donc $g_{1} *_{1} h_{1}^{-1} \in f^{-1}(H_{2})$
	\end{itemize}
	Donc $f^{-1}(H_{2})$ est un sous groupe de $G_{1}$
\end{question_kholle}

\textbf{Application 1:} Le noyau d'un morphisme est un sous-groupe

Le noyau, noté $\ker f$ est par définition égal à $f^{-1}(\{e_2\})$ c'est donc un sous groupe de $G_1$

\textbf{Application 2:} Pour tout $n \in \mathbb{N}$, $\phi _{n}\left|\begin{array}{ll} (\mathbb{C}^{*},\times) &\to (\mathbb{C}^{*},\times) \\ z &\mapsto z^{n} \end{array}\right.$ est un morphisme de groupes. Son noyau, $\ker \phi _n = \{ z \in \mathbb{C}^{*} | z^{n} = 1 \} = \mathbb{U}_{n}$. D'après l'application 1, $\ker \phi_{n}$ est un sous groupe de $(\mathbb{C}^{*}, \times)$. Donc $(\mathbb{U}_{n}, \times)$ est un sous groupe de $(\mathbb{C}^{*},\times)$.
\end{document}
\documentclass{article}

\usepackage[french]{babel}
\usepackage{amsmath}
%\usepackage{stmaryrd}
\usepackage{amssymb}
\usepackage[T1]{fontenc}
\usepackage[a4paper,top=2cm,bottom=2cm,left=3cm,right=3cm,marginparwidth=1.75cm]{geometry}
\usepackage{graphicx}
\usepackage[colorlinks=true, allcolors=blue]{hyperref}
\usepackage{amsthm}
\usepackage{amsfonts}
\usepackage{hyperref}
\usepackage{tikz}
\usepackage{pgfplots}
\pgfplotsset{compat=1.15}
\usepackage{mathrsfs}
\usetikzlibrary{shapes.geometric}
\usetikzlibrary{arrows.meta,arrows}
\hypersetup{
    colorlinks=true,
    linkcolor=blue,
    filecolor=magenta,      
    urlcolor=cyan,
    pdftitle={Khôlles Maths 6-16},
    pdfpagemode=FullScreen,
    }

\title{Khôlles : Maths}
\author{Kylian Boyet, George Ober, Hugo Vangilluwen, Jérémie Menard}

\begin{document}
\maketitle

\flushleft
\textbf{Question2.} Existence et unicité d'une solution au problème de Cauchy pour les EDL d'ordre 2 à coefficients constants et second membre continu sur $I$ (cas complexe puis cas réel).
\newline

\textbf{Réponse. } Considérons le problème de Cauchy suivant :
$$\left\{ \begin{array}{l}
    a_{2}y''+a_{1}y'+a_{0}y = b \text{ sur } J  \\
    y(t_{0}) = \alpha_{0} \\
    y'(t_{0}) = \alpha_{1}
\end{array} \right. \text{ où } (\alpha_{0}, \alpha_{1}) \in \mathbb{K}^{2}, t_{0} \in J, (a_{0}, a_{1}, a_{2}) \in \mathbb{K}^{2} \times \mathbb{K}^{*}, b \in \mathcal{F}(J, \mathbb{K})$$
Si $b$ est continu sur $J$, alors ce problème de Cauchy admet une unique solution définie sur $J$.
\newline\newline

\textbf{Démonstration. } \\
\textbf{Cas 1. } $\mathbb{K} = \mathbb{C}$ \\
Nous savons que sous l'hyphothèse de continuité de $b$ sur $J$, les solutions de (EDL2) définies sur $J$ constituent le plan affine $S$ :
$$S = \left\{ \lambda f_{1} + \mu f_{2} + s | (\lambda, \mu) \in \mathbb{C}^{2} \right\}$$
où $s$ est une solution particulière de (EDL2), $(f_{1}, f_{2})$ sont deux solutions de (EDLH2) qui engendrent $S_{h}$. On a : \\

$$\begin{array}{ccl}
    f : J \to \mathbb{C} \text{ est sol. du pb de Cauchy } 
    &\iff &\left\{ \begin{array}{l}
      f \text{ sol de (EDL2) sur } J    \\
      f(t_{0}) = \alpha_{0}    \\
      f'(t_{0}) = \alpha_{1}
    \end{array}  \right. \\\\
    &\iff &\left\{ \begin{array}{l}
      f \in S    \\
      f(t_{0}) = \alpha_{0}    \\
      f'(t_{0}) = \alpha_{1}
    \end{array}\right. \\\\
    &\iff &\exists (\lambda, \mu) \in \mathbb{C}^{2}: \left\{ \begin{array}{l}
      f = \lambda f_{1} + \mu f_{2} + s \\
      \lambda f_{1}(t_{0}) + \mu f_{2}(t_{0}) + s(t_{0}) = \alpha_{0} \\
      \lambda f'_{1}(t_{0}) + \mu f'_{2}(t_{0}) + s'(t_{0}) = \alpha_{1} \\
    \end{array} \right. \\\\
    &\iff &\exists (\lambda, \mu) \in \mathbb{C}^{2}: \left\{ \begin{array}{l}
      f = \lambda f_{1} + \mu f_{2} + s \\
      \lambda f_{1}(t_{0}) + \mu f_{2}(t_{0}) = \alpha_{0} - s(t_{0}) \\
      \lambda f'_{1}(t_{0}) + \mu f'_{2}(t_{0}) = \alpha_{1} - s'(t_{0}) \\
    \end{array} \right. \\\\
\end{array} $$
On en déduit donc que $(\lambda, \mu)$ doit être solution d'un système linéaire $(2,2)$. On a une unique solution si et seulement si les déterminant de ce système est nul. \\
Explicitons alors le déterminant de ce système, que l'on notera $D$.
$$D = \left| 
\begin{array}{cc}
f_{1}(t_{0}) &f_{2}(t_{0}) \\
f'_{1}(t_{0}) &f'_{2}(t_{0}) \\
\end{array}
\right| = f_{1}(t_{0}) \cdot f'_{2}(t_{0}) - f_{2}(t_{0}) \cdot f'_{1}(t_{0}) $$
Notons $\Delta$ le discriminant de l'équation caractéristique de (EDL2) ($a_{2}r^{2} + a_{1}r^{1} + a_{0} = 0$). On distingue alors deux cas selon la nullité ou non de $\Delta$. Traitons d'abord le cas $\Delta \neq 0$. On peut choisir : 
$$ f_{1}(t_{0}) = e^{r_{1}t_{0}} \text{ et } f_{2}(t_{0}) = e^{r_{2}t_{0}}$$
$$ f'_{1}(t_{0}) = r_{1}e^{r_{1}t_{0}} \text{ et } f'_{2}(t_{0}) = r_{2}e^{r_{2}t_{0}}$$
Donc (en sachant que $\Delta \neq 0 \Rightarrow r_{1} \neq r_{2}$):
$$ D = e^{r_{1}t_{0}} \cdot r_{2}e^{r_{2}t_{0}} - r_{1}e^{r_{1}t_{0}} \cdot e^{r_{2}t_{0}} = (r_{2} - r_{1}) \cdot e^{r_{1}t_{0} + r_{2}t_{0}} \neq 0$$

Dans le deuxième cas, on a $\Delta = 0$ ; on peut alors prendre :
$$ f_{1}(t_{0}) = e^{r_{0}t_{0}} \text{ et } f_{2}(t_{0}) = t_{0}e^{r_{0}t_{0}}$$
Ainsi : 
$$ D = e^{r_{0}t_{0}} \left(r_{0}t_{0}e^{r_{0}t_{0}} + e^{r_{0}t_{0}} \right) - r_{0}e^{r_{0}t_{0}} \times t_{0}e^{r_{0}t_{0}} = e^{2r_{0}t_{0}} \neq 0$$
On remarque alors que, dans les deux cas, $D \neq 0$, donc le système $(2, 2)$ étudié admet une unique solution, donc il existe un unique couple $(\lambda, \mu)$ le vérifiant d'où l'unicité et existence d'une solution au problème de Cauchy. 
\newline\newline

\textbf{Cas 2. } $\mathbb{K} = \mathbb{R}$ \\
$(a_{0}, a_{1}, a_{2}) \in \mathbb{R}^{2} \times \mathbb{R}^{*},(\alpha_{0}, \alpha_{1}) \in \mathbb{R}^{2}, b \in C^{0}(J, \mathbb{R})$ 
\newline
\textbf{Existence :} Puisque $\mathbb{R} \subset \mathbb{C}$, le problème de Cauchy admet, dans $\mathbb{R}$, une solution à valeurs complexes $g$. Posons $f = \mathfrak{Re}(g)$ et montrons que $f$ est une solution réelle du problème de Cauchy. \\
\begin{itemize}
    \item[$\star$] $g \in \mathcal{D}^{2}(J, \mathbb{C}) \text{ donc } f \in \mathcal{D}^{2}(J, \mathbb{R})$
    \item[$\star$] $g$ vérifie $a_{2}g'' + a_{1}g' + a_{0}g = b$ sur $J$ donc en prenant $\mathfrak{Re}(\cdot)$ : 
    $$\begin{array}{ccl}
      \mathfrak{Re}(a_{2}g'' + a_{1}g' + a_{0}g = b) = \mathfrak{Re}(b)   
      &\iff &a_{2}\mathfrak{Re}(g'') + a_{1}\mathfrak{Re}(g') + a_{0}\mathfrak{Re}(g) = b  \\\\
      &\iff & a_{2}f'' + a_{1}f' + a_{0}f = b \text{ sur } J
    \end{array}$$
    \item[$\star$] $f(t_{0} = \mathfrak{Re}(g(t_{0})) = \mathfrak{Re}(\alpha_{0}) = \alpha_{0}$
    \item[$\star$] $f'(t_{0} = \mathfrak{Re}(g(t_{0}))' = \mathfrak{Re}(g'(t_{0})) = \mathfrak{Re}(\alpha_{1}) = \alpha_{1}$
\end{itemize}
Donc $f$ est une solution réelle définie sur $J$ au problème de Cauchy. 
\newline

\textbf{Unicité : }Soient $f_{1}$ et $f_{2}$ deux fonctions à valeurs réelles solutions du problème de Cauchy ci-dessus fixées quelconques : puisque $\mathbb{R} \subset \mathbb{C}$, $f_{1}$ et $f_{2}$ sont des fonctions à valeurs dans $\mathbb{C}$ solutions du même problème de Cauchy; or il y a unicité de la solution au problème de Cauchy dans les fonctions à valeurs complexes, donc $f_{1} = f_{2}$ dans $\mathcal{F}(J, \mathbb{C})$, donc $f_{1} = f_{2}$ dans $\mathcal{F}(J, \mathbb{R})$.
\end{document}
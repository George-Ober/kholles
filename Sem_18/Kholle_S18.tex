\documentclass{article}

\date{08 Février 2024}
\usepackage[nb-sem=18, auteurs={Hugo Vangilluwen, Kylian Boyet, Felix Rondeau}]{../kholles}
\begin{document}
\maketitle

\begin{question_kholle}[{Soit $f$ définie sur un intervalle contenant $x_{0}$. Alors $f$ est dérivable en $x_{0}$ si et seulement si $f$ admet un $DL_{1}(x_{0})$.}]{Interprétation d’un DL en terme de dérivabilité}
	\hfill\\
	\begin{itemize}[label=$\vartriangleright$]
		\item Soit $f$ une fonction définie sur un intervalle contenant $x_{0}$.
		      \begin{description}
			      \item[$(\implies)$] Supposons que $f$ admet le $DL_{1}(x_{0})$
			            \[
				            f(x)\underset{x\to x_{0}}{=} a_{0} + a_{1}(x-x_{0})+o(x-x_{0})
			            \]
			            Alors, son existence implique celle du développement limité à l’ordre 0
			            \[
				            f(x)\underset{x \rightarrow a}{=} a_{0}+o(1)
			            \]
			            ce qui permet d’affirmer que $f(x_{0})=a_{0}$.\\
			            Par ailleurs, l’existence du $DL_{1}(x_{0})$ garantit qu’il existe une fonction $\varepsilon$ définie sur $\mathcal{D}_{f}$ à valeurs dans $\C$ telle que $\lim_{x\to x_{0}}\varepsilon(x)=0$ et
			            \[
				            \forall x\in \mathcal{D}_{f} ,\; f(x)=\underbrace{a_{0}}_{=f(x_{0})} + a_{1}(x-x_{0}) + (x-x_{0})\varepsilon (x)
			            \]
			            donc
			            \[
				            \forall x\in \mathcal{D}_{f}\setminus \{x_{0}\},\; \frac{f(x)-f(x_{0})}{x-x_{0}}=a_{1}+\underbrace{\varepsilon(x)}_{\arrowlim{x}{x_{0}}0}
			            \]
			            d’où
			            \[
				            \lim_{x\to x_{0}}\frac{f(x)-f(x_{0})}{x-x_{0}}=a_{1}
			            \]
			            ce qui montre que $f$ est dérivable en $x_{0}$. On a de plus $f'(x_{0})=a_{1}$.

			      \item[$(\impliedby)$] Réciproquement, si $f$ est dérivable en $x_{0}$, alors $f$ admet comme $DL_{1}(x_{0})$ son approximation au premier ordre, à savoir
			            \[
				            f(x)\underset{x \rightarrow a}{=} f(x_{0})+f'(x_{0})(x-x_{0})+o(x-x_{0})
			            \]
			            La preuve consiste à poser l’application
			            \[
				            \varepsilon \left|\applic{I}{\R}{x}{\begin{cases}
						            \frac{f(x)-f(x_{0}) - f'(x_{0})(x-x_{0})}{x-x_{0}} & \text{si $x\in I\setminus\{x_{0}\}$} \\
						            0                                                  & \text{si $x=x_{0}$}
					            \end{cases}}\right.
			            \]
			            puis à montrer dans un premier temps que cette application convient, et dans un second temps à écrire $|\varepsilon(x)|$ comme la distance entre le taux d’accroissement en $x_{0}$ de $f$ et $f'(x_{0})$, pour pouvoir conclure en utilisant la définition de la dérivabilité de $f$ en $x_{0}$ (limite de son taux d’accroissement) quand à la convergence vers 0 de $\varepsilon$.
		      \end{description}
		\item \textit{Exemple de fonction admettant un $DL_{2}(0)$ mais pas de dérivée seconde.}\\
		      Considérons la fonction $f$ définie aur $\R$ par
		      \[
			      f(x)=\begin{cases}
				      \displaystyle x^{3}\sin \left(\frac{1}{x^{2}}\right) & \text{si $s\in\R$} \\
				      0                                                    & \text{si $x=0$}
			      \end{cases}
		      \]
		      Alors
		      \[
			      \forall x\in\R,\;|f(x)|\leq |x|^{3} \quad\text{donc}\quad f(x)\underset{x \rightarrow 0}{=} o(x^{2})
		      \]
		      ainsi $f$ admet un $DL_{1}(0)$ et un $DL_{2}(0)$. Elle est donc dérivable en $x_{0}$ et les théorèmes usuels montrent que $f$ est dérivable sur $\R^{*}$. Cependant, pour tout $x\in\R^{*}$,
		      \[
			      f'(x)=3x^{2}\sin \left(\frac{1}{x^{2}}\right) + x^{3}\cos \left(\frac{1}{x^{2}}\right)\frac{-2}{x^{3}} = 3x^{2}\sin \left(\frac{1}{x^{2}}\right) -\underbrace{2\cos \left(\frac{1}{x^{2}}\right)}_{\text{n’as pas de limite en $0^{+}$}}
		      \]
		      donc $f'(x)$ n’a pas de limite à droite en 0 donc $f'(x)$ n’a pas de limte en 0 donc $f'$ n’est pas continue en 0 donc n’est pas dérivable.
	\end{itemize}
\end{question_kholle}

\begin{question_kholle}{Calcul du $DL_{n}(0)$ de $(1+x)^{\alpha}$}
	La fonction $f$ définie pour tout $x\in\R$ par $f_{\alpha}(x)=(1+x)^{\alpha}$ est de classe $\mathcal{C}^{\infty}$ sur $\left]-1, 1\right[$ donc la formule de Taylor-Young permet d’affirmer qu’elle admet en 0 des $DL$ à tout ordre donnés par
	\[
		f_{\alpha}(x) \underset{x\to 0}{=} \sum_{k=0}^{n}\frac{f_{\alpha}^{(k)}(0)}{k!}x^{k} + o(x^{n})
	\]
	or $f_{\alpha}^{(0)}(x)=(1+x)^{\alpha}$, $f_{\alpha}^{(1)}(x)=\alpha(1+x)^{\alpha-1}$, $f_{\alpha}^{(2)}(x)=\alpha(\alpha-1)(1+x)^{\alpha-2}$ et on obtient facilement par récurrence
	\[
		\forall k\in\N^{*}, f_{\alpha}^{(k)}(x)=\alpha(\alpha-1)\times\cdots\times (\alpha-k+1)(1+x)^{\alpha-k}
	\]
	si bien que $f_{\alpha}^{(0)}(0)=1$, $f_{\alpha}^{(1)}(0)=\alpha$, $f_{\alpha}^{(k)}(0)=\alpha(\alpha-1)$ et
	\[
		\forall k\in\N^{*}, f_{\alpha}^{(k)}(0) = \alpha(\alpha-1)\times\cdots\times (\alpha-k+1)
	\]
	Rappelons que, pour $(m,k)\in\N^{*}$ tels que $1\leq k\leq m$, le coefficient binomial $\binom{m}{k}$ vaut 1 si $k=0$ et sinon,
	\[
		\forall k\in\iset{1,m}, \binom{m}{k}=\frac{m!}{k!(m-k)!}=\frac{m(m-1)\times\cdots\times (m-k+1)}{k!}
	\]
	si bien qu’il semble assez naturel de généraliser cette notation en posant pour tout $\alpha\in\R^{*}$,
	\[
		\underbrace{\binom{\alpha}{0}=1}_{=f_{\alpha}^{(0)}(0)} \quad\text{et}\quad \forall k\in\N^{*}, \underbrace{\binom{\alpha}{k}=\frac{\alpha(\alpha-1)\times \cdots \times (\alpha-k+1)}{k!}}_{=\frac{f_{\alpha}^{(k)}(0)}{k!}}
	\]
	de sorte que la formule de Taylor Young s’écrive
	\[
		f_{\alpha}(x) \underset{x\to 0}{=}\sum_{k=0}^{n}\binom{\alpha}{k}x^{k}+o(x^{n})
	\]
\end{question_kholle}

\begin{question_kholle}[{Soient $f$ une fonction définie et continue sur un intervalle $I$, $F$ une primitive de $f$ définie sur $I$, $x_{0}\in I$ et $n\in\N$. S’il existe $(a_{0}, \dots, a_{n})\in\C^{n+1}$ tels que $f$ admet en $x_{0}$ le $DL_{n}(x_{0})$:
	\[
		f(x)\underset{x\to x_{0}}{=} \sum_{k=0}^{n}a_{k}(x-x_{0})^{k} + o((x-x_{0})^{n})
	\]
	alors $F$ admet un $DL_{\bm{n+1}}(x_{0})$ qui est
	\[
		F(x)\underset{x\to x_{0}}{=} F(x_{0}) + \sum_{k=0}^{n}\frac{a_{0}}{k+1}(x-x_{0})^{k+1} + o((x-x_{0})^{n+1})
	\]}]{Théorème d’intégragion d’un développement limité}
	Posons
	\[
		g\left|\applic{I}{\C}{x}{F(x)-F(x_{0}) - \sum_{k=0}^{n}\frac{a_{k}(x-x_{0})^{k+1}}{k+1}}\right.
	\]
	\textit{On souhaite montrer que $g(x)=o((x-x_{0})^{n+1})$ pour avoir $\frac{g(x)}{(x-x_{0})^{n+1}}\arrowlim{x}{x_{0}}0$, i.e.
	\[
		\forall \varepsilon\in\R_{+}^{*}, \exists \eta>0: \forall x\in I, |x-x_{0}|\leq \eta \implies  \left|\frac{g(x)}{(x-x_{0})^{n+1}}\right| \leq \varepsilon
	\]Nous allons pour cela utiliser l’inégalité des accroissements finis généralisée à $\C$.}
	La fonction $g$ est de classe $\mathcal C^{1}$ sur $I$, et
	\[
		\forall x\in I, g'(x)=F'(x)-0-\sum_{k=0}^{n}\frac{a_{k}}{k+1}(k+1)(x-x_{0})^{k+1-1} = f(x)-\sum_{k=0}^{n}a_{k}(x-x_{0})^{k}
	\]
	L’hypothèse de l’existence du $DL_{n}(x_{1})$ de $f$ donne l’existence de $\delta:I\longrightarrow \C$ telle que
	\[
		\forall x\in I, f(x)=\sum_{k=0}^{n}a_{k}(x-x_{0})^{k}+(x-x_{0})^{n}\delta(x) \quad\text{et}\quad \lim_{x\to x_{0}}\delta(x)=0
	\]
	Par conséquent,
	\[
		\forall t\in I, |g'(t)| = |t-x|^{n}\delta(t)
	\]
	Appliquons l’inégalité des accroissements finis sur $[x_{0}, x]\cup [x, x_{0}]$ à $g$ continue sur cet intervalle et dérivable sur son intérieur~:
	\begin{equation}\label{S18:2:1}
		|g(x)-g(x_{0})| \leq \|g'\|_{\infty, [x, x_{0}]\cup[x_{0}, x]}|x-x_{0}|
	\end{equation}
	Soit $\varepsilon\in\R_{+}^{*}$ fixé quelconque. Appliquons la définition de la limite en $0$ de $\delta$~:
	\begin{equation*}\label{S18:2:2}
		\exists \eta\in\R_{+}^{*}: \forall t\in I, |t-x_{0}|\leq \eta \implies  |\delta(t)| \leq \varepsilon
	\end{equation*}
	Fixons un tel $\eta$. Soit $x\in I$ fixé quelconque tel que $|x-x_{0}|\leq \eta$. Majorons $\|g'\|_{\infty, [x, x_{0}]\cup[x_{0}, x]}$~: soit $t\in I$ tel que $t\in [x_{0}, x]\cup[x, x_{0}]$. Alors,
	\[
		|g'(t)| = \underbrace{|t-x_{0}|^{n}}_{\leq |x-x_{0}|} \times \underbrace{|\delta(t)|}_{\leq \varepsilon \text{ car $|t-x_{0}|\leq|x-x_{0}|\leq \eta$ donc (2)}} \leq |x-x_{0}|^{n} \varepsilon
	\]
	ce dernier terme ne dépendant pas de $t$, il majore $\left\{|g'(t)| \mid t\in[x_{0}, x]\cup [x, x_{0}]\right\}$, donc
	\[
		\|g'\|_{\infty, [x, x_{0}]\cup[x_{0}, x]}\leq |x-x_{0}|^{n}\times \varepsilon
	\]
	donc par $\eqref{S18:2:1}$,
	\[
		|g(x)| \leq |x-x_{0}|^{n} \times \varepsilon \times |x-x_{0}|
	\]
	donc $|g(x)|\leq \varepsilon|x-x_{0}|^{n+1}$ ce qui établit $g(x)\underset{x\to x_{0}}{=}o((x-x_{0})^{n+1})$.
\end{question_kholle}

\begin{question_kholle}{Formule explicite du $DL_{2n+1}(0)$ de $\arcsin$.}
	La fonction $\arcsin$ est dérivable sur $\left]-1, 1\right[$, et
			\[
			\forall u\in\left]-1,1\right[, \arcsin'(u) = \frac{1}{\sqrt{1-u^{2}}} \quad\text{or}\quad \frac{1}{\sqrt{1+t^{2}}}\underset{x\to 0}{=} \sum_{k=0}^{n}\binom{-\frac{1}{2}}{k}t^{k} +o(t^{n})
		\]
		donc pour $t\leftarrow -u^{2}$,
		\[
			\arcsin '(u) \underset{x\to 0}{=}\sum_{k=0}^{n}\binom{-\frac{1}{2}}{k}(-u^{2})^{k}+o(u^{2n})
		\]
		La fonction $\arcsin'$ est continue sur $\left]-1, 1\right[$ donc au voisinage de 0 si bien que par intégration du $DL_{2n}(0)$ ci-dessus,
	\[
		\arcsin(x) \underset{x\to 0}{=} \underbrace{\arcsin(0)}_{=0} + \sum_{k=0}^{n}(-1)^{k}\binom{-\frac{1}{2}}{k}\frac{u^{2k+1}}{2k+1} + o(x^{2n+1})
	\]
	Or
	\begin{align*}
		\binom{-\frac{1}{2}}{k} & =\frac{\left(-\frac{1}{2}\right)\times \left(-\frac{1}{2}-1\right)\times \left(-\frac{1}{2}-2\right)\times \cdots \times \left(-\frac{1}{2}-k+1\right)}{k!} \\
		                        & =\frac{\frac{(-1)^{k}}{2}(1)\times (3)\times (5)\times \cdots \times (2k-1)}{k!}                                                                            \\
		                        & = \frac{(1)^{k} (1)\times (2)\times (3)\times \cdots \times (2k-2) \times (2k11)\times (2k)}{2^{k}k! \times 2 \times 4 \times 6 \times \cdots \times 2k}    \\
		                        & = \frac{(1)^{k}(2k)!}{2^{k}k!2^{k}k!}                                                                                                                       \\
		                        & = \frac{(1)^{k}(2k)!}{2^{2k}k!^{2}}
	\end{align*}
	si bien que
	\[
		\forall n\in\N, \arcsin(x) \underset{x\to 0}{=}\sum_{k=0}^{n}\frac{(2k)!}{(2k+1)2^{2k}k!^{2}}x^{2k+1}o(x^{2n+1})
	\]
\end{question_kholle}

\begin{question_kholle}{$DL_{10}(0)$ de $f(x)=\displaystyle\int_{x}^{x^{2}}\frac{\dd t}{\sqrt{1+t^{4}}}$}
	Posons pour tout $x\in\R$, $F(x)=\int_{0}^{x}\frac{\dd t}{\sqrt{1+t^{4}}}$. C’est la primitive qui s’annule en $0$ d’une fonction de classe $\mathcal{C}^{\infty}$ donc $F\in \mathcal{C}^{\infty}(\R, \R)$. La formule de Taylor-Young permet alors d’affirmer que $F$ admet des DL est tout point à tout ordre. De plus,
	\[
		F'(x) = \frac{1}{\sqrt{1+x^{4}}} \underset{x\to 0}{=} 1-\frac{x^{4}}{2} + \frac{3}{8}x^{8}+o(x^{11})
	\]
	si bien que d’après le théorème d’intégration des DL ($F'$ est continue au voisinage de 0 sur \R),
	\begin{align*}
		F(x) & \underset{x\to 0}{=} F(0)+\int_{0}^{x}\left(1-\frac{t^{4}}{2}+\frac{3}{8}\right)\dd t + o(x^{12}) \\
		     & \underset{x\to 0}{=} 0+\left[t-\frac{t^{5}}{10} + \frac{1}{24}t^{9}\right]_{0}^{x} + o(x^{12})    \\
		     & \underset{x\to 0}{=} x-\frac{1}{10}x^{5} + \frac{1}{24}x^{9} + o(x^{12})
	\end{align*}
	On en déduit alors
	\begin{align*}
		f(x) & = F(x^{2}) - F(x)                                                                                              \\
		     & \underset{x\to 0}{=} x^{2}-\frac{1}{10}x^{10} + o(x^{18}) - x+\frac{1}{10}x^{5} -\frac{1}{24}x^{9} + o(x^{12}) \\
		     & \underset{x\to 0}{=} -x+x^{2}+\frac{1}{10}x^{5}-\frac{1}{24}x^{9}-\frac{1}{10}x^{10}+o(x^{18})
	\end{align*}
	Ainsi,
	\[
		f(x)\underset{x\to 0}{=} -x+x^{2}+\frac{1}{10}x^{5} - \frac{1}{24}x^{9} - \frac{1}{10}x^{10} + o(x^{10})
	\]
\end{question_kholle}

\begin{question_kholle}
	{Deux fonctions équivalentes au voisinage de $a$ ont le même signe sur un voisinage de $a$}

	Soient $f : \mathcal{D} \rightarrow \R$ et $g : \mathcal{D} \rightarrow \R$ telles que $f(x) \underset{x \rightarrow a}{\sim} g(x)$ avec $a \in \mathcal{D}$. \\
	Appliquons la définition de l'équivalence pour $\varepsilon \leftarrow \frac{1}{2}$, il existe un voisinage $V$ de $a$ tel que :
	\begin{equation*}
		\forall x \in V \cap \mathcal{D},
		| f(x) - g(x) | \leqslant \frac{1}{2} | g(x) |
	\end{equation*}

	Fixons un tel voisinage $V$.
	Nous obtenons :
	\begin{equation*}
		\forall x \in V \cap \mathcal{D},
		\underbrace{g(x) - \frac{1}{2} | g(x) |}_{\text{du signe de }g(x)}
		\leqslant f(x) \leqslant
		\underbrace{g(x) + \frac{1}{2} | g(x) |}_{\text{du signe de }g(x)}
	\end{equation*}

	Ainsi $f(x)$ et $g(x)$ ont le même signe sur $V \cap \mathcal{D}$.
\end{question_kholle}

\begin{question_kholle}
	[ Soient $f \in \Cont{\infty}{\mathcal{D}}{}$ et $a \in \overset{\circ}{\mathcal{D}}$. Supposons que $E_0 = \left\{ p \in \N^* \setminus \{1\} \;|\; f^{(p)}(a) \neq 0 \right\}$ est non vide. \\
		Posons $p_0 = \min E_0$. \\
		$f$ admet un extremum local en $a$ si et seulement si $f'(a) = 0$ et $p_0$ est pair. \\
		$f$ admet un point d'inflexion en $a$ si et seulement si $p_0$ est impair. ]
	{Condition nécessaire et suffisante pour qu'une fonction \Cont{\infty}{}{} admette un extremum local ou un point d'inflexion}

	Soient de tels objets. Traitons le cas de l'extremum local.

	\noindent $f \in \Cont{\infty}{}{}$ donc, la formule Taylor-Young donne un $DL_{p_0}(a)$ de $f$ :
	\begin{equation*}
		f(x) \underset{x \rightarrow a}{=}
		\sum_{k=0}^{p_0} \frac{f^{(k)}(a)}{k!} (x-a)^k + o \left( (x-a)^{p_0} \right)
	\end{equation*}

	En développant :
	\begin{equation*}
		f(x) \underset{x \rightarrow a}{=}
		f(a) + \underbrace{f'(a)(x-a)}_{= 0} + \underbrace{\ldots + \frac{f^{(p_0-1)}(a)}{(p_0-1)!} (x-a)^{p_0-1}}_{= 0 \text{ par défintion de }p_0} + \frac{f^{(p_0)}(a)}{p_0!} (x-a)^{p_0} + o \left( (x-a)^{p_0} \right)
	\end{equation*}

	Ainsi (car $f^{(p_0)}(a) \neq 0$)
	\begin{equation}
		f(x) - f(a) \underset{x \rightarrow a}{\sim} \frac{f^{(p_0)}(a)}{p_0!} (x-a)^{p_0}
	\end{equation}
	Au voisinage de $a$, $f(x) - f(a)$ et $\frac{f^{(p_0)}(a)}{p_0!} (x-a)^{p_0}$ ont le même signe.
	\\

	Supposons que $f$ admette un extremum local en $a$.
	Or $a \in \overset{\circ}{\mathcal{D}}$ et $f$ est dérivable en 0, donc $f'(a) = 0$.
	Comme $f$ admette un extremum local en $a$, $f(x) - f(a)$ est de signe constant au voisinage de $a$.
	Donc $\frac{f^{(p_0)}(a)}{p_0!} (x-a)^{p_0}$ est de signe constant au voisinage de $a$.
	Par conséquent, $p_0$ est pair.
	\\

	Réciproquement, supposons que $f'(a) = 0$ et que $p_0$ est pair. $\frac{f^{(p_0)}(a)}{p_0!} (x-a)^{p_0}$ est de signe constant au voisinage de $a$. Donc $f(x) - f(a)$ est de signe constant au voisinage de $a$. Ainsi, $a$ est un extremum local de $f$.
	\\

	Traitons le cas du point d'inflexion. La formule de Taylor-Young donne :
	\begin{equation}
		f(x) - \underbrace{\left( f(a) + (x-a)f'(a) \right)}_{\text{tangente en } (a,f(a))}
		\underset{x \rightarrow a}{\sim} \frac{f^{(p_0)}(a)}{p_0!} (x-a)^{p_0}
	\end{equation}
	Le signe de l'écart courbe/tangente en $a$ est donc celui de $\frac{f^{(p_0)}(a)}{p_0!} (x-a)^{p_0}$. Ce qui conclut de la même manière que l'extremum local.
\end{question_kholle}

\end{document}

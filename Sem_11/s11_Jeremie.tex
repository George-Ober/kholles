\documentclass{article}
\usepackage{graphicx} % Required for inserting images
\usepackage{amsmath}
\usepackage{amssymb}
\usepackage{amsfonts}
\usepackage{geometry}
\usepackage{tikz}
\usepackage{pgfplots}
\pgfplotsset{compat=1.15}
\usepackage{mathrsfs}
\usetikzlibrary{arrows}
\geometry{hmargin=2.5cm,vmargin=3cm}

\NeedsTeXFormat{LaTeX2e}
\ProvidesPackage{kholles}[2023/11/25]

% paquets de traduction
\RequirePackage[utf8]{inputenc}
\RequirePackage[T1]{fontenc}
\RequirePackage[french]{babel}

% paquets mathématiques
\RequirePackage{amsmath}
\RequirePackage{amsfonts}
\RequirePackage{amssymb}
\RequirePackage{amsthm}
\RequirePackage{stmaryrd}
\RequirePackage{mathrsfs}

% changement de puces dans les listes
\RequirePackage{enumitem}

% paquets graphiques
\RequirePackage{geometry}
	\geometry{
		a4paper,
		top=2cm,
		bottom=2cm,
		right=3cm,
		left=3cm
	}
\RequirePackage{hyperref}
	\hypersetup{
		colorlinks=true,
		linkcolor=blue,
		filecolor=magenta,
		urlcolor=cyan,
		pdftitle={Khôlles de Mathématiques},
		pdfauthor={Kylian Boyet, George Ober, Hugo Vangilluwen, Jérémie Menard}
	}
\RequirePackage{graphicx}
\RequirePackage{tikz}
	\usetikzlibrary{shapes.geometric}
	\usetikzlibrary{arrows.meta, arrows}
\RequirePackage{pgfplots}
	\pgfplotsset{compat=1.15}

% Symbols
\newcommand{\Sol}{\ensuremath{\mathcal S} }

\renewcommand{\Re}{\ensuremath{\text{Re}}}
\renewcommand{\Im}{\ensuremath{\text{Im}}}

\newcommand{\N}{\ensuremath{\mathbb N} }
\newcommand{\Z}{\ensuremath{\mathbb Z} }
\newcommand{\R}{\ensuremath{\mathbb R} }
\newcommand{\C}{\ensuremath{\mathbb C} }
\newcommand{\K}{\ensuremath{\mathbb K} }

\newcommand{\Rel}{\ensuremath{\mathcal R} }

\begin{document}


\flushleft
\textbf{Question 2.} Caractérisation séquentielle de la densité.
\newline

\textbf{Réponse. } $\text{Soient } (A,B) \in (\mathcal{P}(\R) \setminus \{\varnothing\})^{2}$. Montrons que :
\\
$$A \text{ est dense dans } B \iff \left\{ \begin{array}{l}
    A \subset B \\
    \forall b \in B, \exists(a_{n}) \in A^{\N} : (a_{n}) \text{ converge vers }b
\end{array}\right. $$
\\
Sens indirect : supposons $A \subset B$ et $\forall b \in B, \exists(a_{n}) \in A^{\N} : (a_{n}) \text{ converge vers }b$ :\\
\begin{itemize}
    \item[$\star$] $A \subset B$ par hypothèse.
    \item[$\star$] Montrons que $\forall b \in B, \forall \varepsilon \in \R^{*}_{+}, \exists a \in A : |b - a| < \varepsilon$ (on utilise la caractérisation de la densité avec les $\varepsilon$) \\
    Soient $b \in B$ et $\varepsilon \in \R^{*}_{+}$ fixés quelconques : \\
    Par hypothèse appliquée pour $b \leftarrow b$ : $\exists(a_{n}) \in A^{\N} : a_{n} \underset{n \to +\infty}{\longrightarrow}b$ \\
    Appliquons la définition de la convergence de $(a_{n})$ vers $b$ pour $\varepsilon \leftarrow \frac{\varepsilon}{2}$ : \\
    $$\exists N \in \N : \forall n \in \N, n \geqslant N \Rightarrow |a_{n} - b_{n}| \leqslant \frac{\varepsilon}{2}$$
    Fixons un tel N : \\
    En particulier, $a_{N} \in A$ et $|a_{N} - b| \leqslant \frac{\varepsilon}{2} \leqslant \varepsilon$ \\
    Donc $A$ est dense dans $B$.
\end{itemize}

Sens direct : supposons $A$ dense dans $B$ : \\
\begin{itemize}
    \item[$\star$] Par définition, $A \subset B$
    \item[$\star$] Soit $b \in B$ fixé quelconque. \\
    Soit $n \in \N$ fixé quelconque :  \\
    Appliquons la caractérisation de la densité par les $\varepsilon$ pour $\varepsilon \leftarrow \frac{1}{2^{n}}$ (autorisé car $\frac{1}{2^{n}} > 0$), et $b \leftarrow b$ : 
    $$\exists a \in A : |a - b| \leqslant \frac{1}{2^{n}}$$
    Notons $a_{n}$ un tel élément. Nous venons de construire $(a_{n})_{n \in \N} \in A^{\N}$ vérifiant : \\
    $\forall n \in \N, |a - b| \leqslant \frac{1}{2^{n}}$ \\
    Or : $\underset{n \to +\infty}{\lim} \frac{1}{2^{n}} = 0$ \\
    Ainsi, d'après le théorème sans nom, $(a_{n})_{n \in \N}$ converge vers $b$.
\end{itemize}

\text{ }
\newline\newline
\textbf{Question 5.} Théorème de passage à la limite dans une inégalité.
\newline

\textbf{Réponse. } Soient $(u,v) \in \R^{\N}$ : \\
\begin{itemize}
    \item[$(i)$] Si $\begin{array}{|l}
       \exists N \in \N : \forall n \in \N, n \geqslant N \Rightarrow u_{n} \geqslant 0   \\
       u \text{ converge}
    \end{array}$ \\
    Alors $\lim u \geqslant 0$
    \item[$(ii)$] Si $\begin{array}{|l}
       \exists N \in \N : \forall n \in \N, n \geqslant N \Rightarrow u_{n} \leqslant v_{n}   \\
       u \text{ et } v \text{ convergent}
    \end{array}$ \\
    Alors $\lim u \leqslant \lim v$
\end{itemize}

Démonstration : 
\begin{itemize}
    \item[$(i)$] l'hypothèse $\exists N \in \N : \forall n \in \N, n \geqslant N \Rightarrow u_{n} \geqslant 0$ permet d'affirmer que $u$ et $|u|$ coïncident à partir d'un certain rang. \\
    Par ailleurs, la convergence de $u$ et la continuité de $|\cdot|$ sur $\R$ donc en $\lim u$ donnent $|u|$ converge vers $|\lim u|$. \\
    Le caractère asymptotique de la limite permet de conclure que $u$ et $|u|$ ont la même limite. \\
    Donc $\lim u = |\lim u| \geqslant 0$
    \item[$(ii)$] $\exists N \in \N : \forall n \in \N, n \geqslant N \Rightarrow u_{n} \leqslant v_{n} \Rightarrow v_{n} - u_{n} \geqslant 0$ \\
    $u$ et $v$ convergent $\Rightarrow v-u$ converge vers $\lim v - \lim u$. \\
    On applique $(i)$ pour $u \leftarrow v - u$, autorisé car $u \text{ et }v$ convergent. \\
    On obtient $\lim v - \lim u \geqslant 0$ d'où $\lim u \leqslant \lim v$.
\end{itemize}


\end{document}
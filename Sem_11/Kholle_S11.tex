\documentclass{article}

\date{08 décembre 2024}
\usepackage[nb-sem=11, auteurs={Félix Rondeau}]{../kholles}

\begin{document}
\maketitle

\begin{question_kholle}{Preuve du théorème de Bolzano-Weierstrass
    pour les suites complexes à partir du cas réel.}
  \hfill\\
  \textbf{Résultat préliminaire: existence d’une sous-suite
    convergente commune.}\\
  \begin{itemize}
    \item La suite $a$ étant bornée, on peut lui appliquer le
          théorème de Bolsano-Weierstrass :
          \[
            \exists a_{\infty}\in\R : \exists \phi: \N\longrightarrow \N
            \text{ strictement croissante telle que
              $\left(b_{\phi(n)}\right)_{n\in\N}$ converge vers $b_{\infty}$.}
          \]
    \item La suite $\left(b_{\phi(n)}\right)_{n\in\N}$ étant bornée
          (en tant que sous-suite d’une suite bornée), on peut lui
          appliquer le théorème de Bolzano-Weierstrass:
          \[
            \exists b_{\infty}\in\R: \exists \psi : \N\longrightarrow \N
            \text{ strictement croissante telle que } \left(b_{\phi\circ
              \psi}\right)_{n\in\N} \text{ converge vers $b_{\infty}$}
          \]
  \end{itemize}
  Observons alors que $\left(a_{\phi\circ \psi(n)}\right)_{n\in\N}$
  est une sous-suite de $\left(a_{\phi(n)}\right)_{n\in\N}$ donc elle
  converge vers $a_{\infty}$.\\
  Ainsi, l’extractrice $\chi = \phi\circ \psi: \N\longrightarrow \N$
  strictement croissante est telle que les deux sous-suites
  $\left(a_{\chi(n)}\right)_{n\in\N}$ et
  $\left(b_{\chi(n)}\right)_{n\in\N}$ convergent.\\
  \\
  \textbf{Preuve du théorème.}\\
  Soit $u\in\C^{\N}$ une suite bornée. Par définition,
  \[
    \exists M\in\R : \forall n\in\N, |u_{n}|\leq M
  \]
  Posons $x=\big(\Re(u_{n})\big)_{n\in\N}$ et
  $y=\big(\Im(u_{n})\big)_{n\in\N}$. Alors
  \[
    \forall n\in\N, |x_{n}|\leq |u_{n}|\leq M \quad \text{et} \quad
    |y_{n}|\leq |u_{n}|\leq M
  \]
  Par conséquent, $x$ et $y$ sont deux suites réelles bornées, donc
  le résultat précedemment prouvé permet de construire une
  extractrice $\phi:\N\longrightarrow \N$ strictement croissante
  telle que $\left(x_{\phi(n)}\right)_{n\in\N}$ et
  $\left(y_{\phi(n)}\right)_{n\in\N}$ sont deux suites qui convergent
  dans \R vers leur limites respectives $x_{\infty}$ et $y_{\infty}$.
  Ainsi, la suite $\left(u_{\phi(n)}\right)_{n\in\N}$, extraite de
  $u$, converge vers $x_{\infty}+iy_{\infty}$.

\end{question_kholle}

\begin{question_kholle}{Illustrer par des exemples que la convergence
    de la suite complexe $(u_{n})=(e^{i \theta_{n}})$ n’implique pas la
    convergence de $(\theta_{n})$ même si on impose à $(\theta_{n})$
    d’être dans l’intervalle $[0,2 \pi]$ pour la rendre unique et bornée.}
  Considérons la suite $(\theta_{n})$ définie pour tout $n\in\N$ par
  \[
    \theta_{n}=\frac{\pi}{2}+2n \pi
  \]
  Alors, la suite $(u_{n})=(e^{i \theta_{n}}) = (i)_{n\in\N}$
  converge mais $(\theta_{n})$ diverge vers $+\infty$.\\
  Considérons à présent une seconde définition de la suite $(\theta_{n})$ :
  \[
    \forall n\in\N, \theta_{n}=
    \begin{cases}
      \frac{1}{n}           & \text{si } n\equiv 0[2] \\
      2 \pi - \frac{1}{n+1} & \text{si } n\equiv 1[2]
    \end{cases}
  \]
  Cette définition impose à la suite $(\theta)$ d’être à valeurs dans
  l’intervalle $[0,2 \pi]$, et selon elle, la suite $(u_{n})=(e^{i
        \theta_{n}})$ converge vers 1. Cependant, $(\theta_{n})$ diverge
  car elle a deux valeurs d’adhérence qui sont $0$ et $2 \pi$.

\end{question_kholle}

\begin{question_kholle}{Calculer la limite de $\displaystyle
      \left(1+\frac{z}{n}\right)^{n}$ en fonction de $z\in\C$.}
  \hfill\\
  \begin{itemize}
    \item \textbf{Si $\bm{z\neq 0}$},
          \begin{align*}
            \left(1+\frac{z}{n}\right)^{n} = e^{n\ln
                \left(1+\frac{z}{n}\right)} = e^{n\cdot \frac{\ln \left(1+
                  \frac{z}{n}\right)}{\frac{z}{n}}\cdot \frac{z}{n}} =
            e^{\frac{\ln \left(1+\frac{z}{n}\right)}{\frac{z}{n}}\cdot z}
          \end{align*}
          donc
          \[
            \lim_{n\to +\infty} \left(1+\frac{z}{n}\right)^{n} = e^{z}
          \]
    \item \textbf{Si $\bm{z=0}$},
          \[
            \lim_{n\to+\infty}\left(1+\frac{z}{n}\right)^{n} =
            \lim_{n\to+\infty}1=1=e^{0}
          \]
  \end{itemize}
  Ainsi, pour tout $z\in\C$,
  \[
    \lim_{n\to+\infty} \left(1+\frac{z}{n}\right)^{n}=e^{z}
  \]

\end{question_kholle}

\begin{question_kholle}{Résolution explicite (sur un exemple) d’une
    relation de récurrence linéaire d’ordre 1 ou 2 à coefficients
    constants avec un second membre produit d’un polynôme et d’une
    suite géométrique.}
  \hfill\\
  \begin{itemize}
    \item \textbf{Résolution d’une relation d’ordre 1.}\\
          Considérons une équation de récurrence linéaire d’ordre 1 de la forme
          \[
            \forall n\in\N, u_{n+1} - au_{n}=v_{n}
          \]
          L’ensemble des suites la vérifiant est la droite affine passant
          par une solution particulière et dirigée par la droite
          vectorielle des solution de l’équation homogène associée. Cette
          droite vectorielle vaut, \textbf{dans le cas où $a$ est nul}
          \[
            \left\{\lambda\cdot \gamma^{0} \mid \lambda\in\K\right\}
          \]
          et \textbf{dans le cas où $a$ est non nul}
          \[
            \left\{(\lambda\cdot a^{n})_{n\in\N} \mid \lambda\in\K\right\}
          \]
          \vspace{5pt}
          Si le second membre est de la forme $v_{n}=P(n)c^{n}$ avec $P$
          un polynôme, on cherche une solution particulière de la forme
          $Q(n)c^{n}$ avec, \textbf{si $\bm{c\neq a}$}, $Q$ un polynôme
          de \K tel que
          \[
            \deg Q=\deg P
          \]
          et \textbf{si $\bm{c=a}$}, $Q$ un polynôme tel que
          \[
            \deg Q=\deg P + 1
          \]

    \item \textbf{Résolution d’une relation d’ordre 2.}
          Considérons une équation de récurrence linéaire d’ordre 2 de la forme
          \[
            \forall n\in\N, u_{n+2}+a_{1}u_{n+1}+a_{0}u_{n} = v_{n}
          \]
          avec $(a_{0}\alpha_{1})\in\K^{2}$. L’ensemble des suites la
          vérifiant est le plan affine passant par une solution
          particulière $w$ et dirigé par le plan vectoriel des solutions
          de l’équation homogène.
          \begin{itemize}
            \item Si \K désigne le corps des complexes, on distingue en
                  fonction du discrimiant $\Delta$ de l’équation
                  caractéristique deux cas :
                  \begin{itemize}
                    \item Lorsque $\Delta=0$, l’équation caractéristique
                          possède une racine double $r_{0}\in\C$ et dans le cas
                          où $a_{0}$ et $a_{1}$ ne sont pas tous deux nuls, le
                          plan vectoriel des solution de la relation de
                          récurrence homogène est
                          \[
                            \left\{((\lambda+\mu n)r_{0}^{n})_{n\in\N} \mid
                            (\lambda,\mu)\in\C^{2}\right\}
                          \]
                          et sinon, il vaut
                          \[
                            \left\{\lambda \gamma^{0} + \mu \gamma^{1} \mid
                            (\lambda, \mu)\in\C^{2}\right\}
                          \]
                    \item Lorsque $\Delta\neq 0$, l’équation caractéristique
                          possède deux racines distinctes $r_{1}$ et $r_{2}$ et
                          l’ensemble des solutions de l’équation de récurrence homogène est
                          \[
                            \left\{(\lambda r_{1}^{n}+ \mu r_{2}^{n})_{n\in\N}
                            \mid (\lambda, \mu)\in\C^{2}\right\}
                          \]
                  \end{itemize}
            \item Si \K désigne le corps des réels, on distingue en
                  fonction du discrimiant $\Delta$ de l’équation
                  caractéristique trois cas :
                  \begin{itemize}
                    \item Lorsque $\Delta=0$, l’ensemble des solutions de
                          l’équation de récurrence homogène est similaire à celui
                          du cas complexe.
                    \item De même lorsque $\Delta>0$, l’ensemble des
                          solutions de l’équation de récurrence homogène est
                          similaire du cas $\Delta\neq 0$ dans le cas complexe.
                    \item Enfin, lorsque $\Delta<0$, l’ensemble des solutions
                          de l’équation de récurrence homogène est
                          \[
                            \Bigl\{\bigl(\rho^{n}(\lambda\cos(n \theta)+\mu\sin(n
                            \theta))\bigr)_{n\in\N} \mid (\lambda, \mu)\in\R^{2}\Bigr\}
                          \]
                  \end{itemize}
          \end{itemize}
          On cherche une solution particulière de la forme $Q(n)a^{n}$
          avec, $Q$ un polynôme de degré égal à celui de $P$ si $a$ n’est
          pas racine de l’équation caractéristique et du degré de $P$
          augmenté d’un nombre égal à la multiplicité de la racine $a$ sinon.
  \end{itemize}

\end{question_kholle}

\begin{question_kholle}[
    \[
      \forall (a,b)\in\Z^{2}, \exists (u,v)\in\Z^{2}: au+bv=a\land b
    \]]{Existence d’une relation de Bezout}
  \hfill\\
  \begin{itemize}
    \item \textbf{Démonstration pour $(a,b)\in\Z\times\N$.}\\
          Considérons la propriété de récurrence définie pour tout $b\in\N$ par
          \[
            \prop(b) : {\forall a\in\Z, \forall c\in\iset{0,b}, \exists
            (u,v)\in\Z^{2}:au+cv=a\land c}
          \]
          \begin{itemize}
            \item Supposons que $b=0$.\\
                  Soit $a\in\Z$ fixé quelconque.\\
                  Si $a=0, a\land 0 = 0\land 0=0$ donc $a\land 0=0\times
                    a+232\times b$.\\
                  Sinon, $a\neq 0, u=\frac{a}{|a|}\in\{-1,1\}\subset \Z$ et
                  \[
                    ua+232\times 0=\frac{a^{2}}{|a|}=\frac{{|a|}^{2}}{|a|}=|a|=a\land 0
                  \]
                  Ainsi, $\prop(0)$ est vraie
            \item Soit $b\in\N$ fixé quelconque tel que $\prop(b)$ est vraie.\\
                  Soient $a\in\Z$ et $c\in\iset{0,b+1}$ fixés quelconques.\\
                  \begin{itemize}
                    \item Si $c\in\iset{0,b}$, la véracité de $\prop(b)$
                          permet d’affirmer que $\exists (u,v)\in\Z^{2}: au+cv=a\land c$.
                    \item Sinon, $c=b+1$. Effectuons la division euclidienne
                          de $a$ par $b+1$:
                          \[
                            \exists !(q,r)\in\N\times\iset{0,b}: a=(b+1)q+r
                          \]
                          Or, d’après le lemme d’Euclide,
                          \[
                            a\land (b+1)=r\land (b+1)
                          \]
                          Or $r\in\iset{0,b}$ donc $\prop(b)$ s’applique pour
                          $a\leftarrow b+1$ et $c\leftarrow r$:
                          \[
                            \exists (u_{0},v_{0})\in\Z^{2}: (b+1)u_{0}+rv_{0}=(b+1)\land r
                          \]
                          si bien que
                          \[
                            a\land (b+1)=(b+1)u_{0}+rv_{0}=(b+1)u_{0} +
                            (a-q(b+1))v_{0} = av_{0}+(b+1)(u_{0}-qv_{0})
                          \]
                          d’où le résultat attendu en posant $u=v_{0}$ et $v=u_{0}-qv_{0}$.
                  \end{itemize}
                  Par conséquent, $\prop(b+1)$ est vraie.

                  \hspace{4pt}
            \item \textbf{Démonstration du cas général :
                    $(a,b)\in\Z\times(\Z\setminus\N)$}.\\
                  Appliquons le résultat prouvé dans le cas précédent à
                  $(a,|b|)\in\Z\times\N$ :
                  \[
                    \exists (u_{1},v_{1})\in\Z^{2}: au_{1}+bv_{1}=a\land |b|
                  \]
                  Posons $u=u_{1}$ et $v=-v_{1}$. On a $(u_{1},v_{1})\in\Z^{2}$ et
                  \[
                    au+bv=au_{1}+|b|v_{1} = a\land |b| = a\land b
                  \]
          \end{itemize}
  \end{itemize}

\end{question_kholle}

\begin{question_kholle}[{
        Soient $(a,b,c)\in\Z^{3}$.
        \[
          \left.
          \begin{array}{l}
            a\mid b c \\
            a\land b = 1
          \end{array}
          \right\} \implies a\mid c
        \]
      }]{Théorème de Gauss}
  Soient $(a,b,c)\in\Z^{3}$ fixés quelconques.\\
  $a$ divise $b c$ donc
  \begin{equation}
    \exists k\in\Z: ka=b c
  \end{equation}
  $a$ et $b$ sont premiers entre eux donc
  \begin{equation}
    \exists (u,v)\in\Z^{2}: au+bv=1
  \end{equation}
  En multipliant la première relation par $c$, nous obtenons
  \[
    auc+bvc=c
  \]
  donc, en utilisant la deuxième relation,
  \[
    auc+akv = c \iff a(\underbrace{uc+kv}_{\in\Z})=c
  \]
  ce qui montre que $a$ divise $c$.
\end{question_kholle}

\begin{question_kholle}{Si $a\land c=b\land c=1$ alors $ab\land c=1$
    et sa généralisation au cas de $n$ entiers premiers avec un même entier.}
  Soient $p\in\N^{*}$ et $(a,b_{1},\dots,b_{p})\in\Z^{p+1}$ $n+1$
  entiers fixés quelconques premiers entre eux deux à deux. Le théorème
  d’existence d’une relation de bezout assure donc que
  \[
    \forall i\in\iset{1,p}, \exists (u_{i},v_{i})\in\Z^{2}: au_{i}+b_{i}v_{i}=1
  \]
  donc que
  \[
    \forall i\in\iset{1,p}, \exists (u_{i},v_{i})\in\Z^{2}: b_{i}v_{i}=1-au_{i}
  \]
  si bien qu’en effectuant le produit membre à membre de ces $p$ égalités,
  \[
    \prod_{i=1}^{p}(b_{i}v_{i})=\prod_{i=1}^{p}(1-au_{i})
  \]
  En développant le membre de droite, on obtient que
  \[
    \exists U\in\Z:\prod_{i=1}^{p}(b_{i}v_{i}) = 1-aU
  \]
  si bien qu’en posant $\displaystyle V=\prod_{i=1}^{p}v_{i}$,
  \[
    \left(\prod_{i=1}^{p}b_{i}\right)V = 1-aU
  \]
  ainsi, il existe deux entiers relatifs $U$ et $V$ tels que
  \[
    aU+\left(\prod_{i=1}^{p}b_{i}\right)V=1
  \]
  Le théorème de caractérisation de la propriété <<deux entiers sont
  premiers entre eux>> par une relation de Bezout permet donc de conclure que
  \[
    a\land \left(\prod_{i=1}^{p}b_{i}\right)=1
  \]

\end{question_kholle}

\begin{question_kholle}{$(a\land b)(a\lor b) = |ab|$}
  Soient $(a,b)\if\Z^{2}$ fixés quelconques. Nous savons que
  \[
    \exists (a', b')\in\Z^{2}:
    \begin{cases}
      a=da' \\
      b=db' \\
      a'\land b'=1
    \end{cases} \quad \text{ où } d=a\land b
  \]
  Observons alors que
  \begin{align}
    (a\land b)(a\lor b) & =(da'\land db')(da'\lor db')
    \nonumber
    \\
                        & = d(\underbrace{a'\land b'}_{=1})\times d(a'\lor b') \nonumber \\
                        & = d^{2}(a'\lor b') \label{S11:8}\tag{$\star$}
  \end{align}
  Calculons $a'\lor b'$:
  \begin{itemize}
    \item $a' b'$ est un multiple commun à $a'$ et $b'$ donc $a'\lor b'
            \vert a' b'$.
    \item $a'\lor b'$ est un multiple commun à $a'$ et $b'$ doonc
          \[
            \left.
            \begin{array}{l}
              a'\land b'=1       \\
              a' \vert a'\lor b' \\
              b' \vert a'\lor b'
            \end{array}\right\} \implies  a' b' \vert a'\lor b'
          \]
  \end{itemize}
  Ainsi, $a'\lor b'$ et $a' b'$ se divisent l’un l’autre donc ils sont
  associés (égaux ou opposés), or $a'\lor b'\geq 0$ donc $a'\lor b'=|a' b'|$.\\
  Par conséquent, en reprenant l’égalité \eqref{S11:8},
  \[
    (a\land b)(a\lor b) = d^{2}a' \lor b' = d^{2} |a' b'| = |da' \times
    db'| = |ab|
  \]

\end{question_kholle}

\end{document}

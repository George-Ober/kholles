\documentclass{article}

\date{9 décembre 2023}
\usepackage[nb-sem=11]{../kholles}

\begin{document}
\maketitle

	\begin{question_kholle}
		[Soient $(A,B) \in (\mathcal{P}(\R) \setminus \{\varnothing\})^{2}$. Montrons que :
		\\
		$$A \text{ est dense dans } B \iff \left\{ \begin{array}{l}
			A \subset B \\
			\forall b \in B, \exists(a_{n}) \in A^{\N} : (a_{n}) \text{ converge vers }b
		\end{array}\right. $$
		]
		{Caractérisation séquentielle de la densité.}
		
		Sens indirect : supposons $A \subset B$ et $\forall b \in B, \exists(a_{n}) \in A^{\N} : (a_{n}) \text{ converge vers }b$ :\\
		\begin{itemize}
			\item[$\star$] $A \subset B$ par hypothèse.
			\item[$\star$] Montrons que $\forall b \in B, \forall \varepsilon \in \R^{*}_{+}, \exists a \in A : |b - a| < \varepsilon$ (on utilise la caractérisation de la densité avec les $\varepsilon$) \\
			Soient $b \in B$ et $\varepsilon \in \R^{*}_{+}$ fixés quelconques : \\
			Par hypothèse appliquée pour $b \leftarrow b$ : $\exists(a_{n}) \in A^{\N} : a_{n} \underset{n \to +\infty}{\longrightarrow}b$ \\
			Appliquons la définition de la convergence de $(a_{n})$ vers $b$ pour $\varepsilon \leftarrow \frac{\varepsilon}{2}$ : \\
			$$\exists N \in \N : \forall n \in \N, n \geqslant N \Rightarrow |a_{n} - b| \leqslant \frac{\varepsilon}{2}$$
			Fixons un tel N : \\
			En particulier, $a_{N} \in A$ et $|a_{N} - b| \leqslant \frac{\varepsilon}{2} \leqslant \varepsilon$ \\
			Donc $A$ est dense dans $B$.
		\end{itemize}
		
		Sens direct : supposons $A$ dense dans $B$ : \\
		\begin{itemize}
			\item[$\star$] Par définition, $A \subset B$
			\item[$\star$] Soit $b \in B$ fixé quelconque. \\
			Soit $n \in \N$ fixé quelconque :  \\
			Appliquons la caractérisation de la densité par les $\varepsilon$ pour $\varepsilon \leftarrow \frac{1}{2^{n}}$ (autorisé car $\frac{1}{2^{n}} > 0$), et $b \leftarrow b$ : 
			$$\exists a \in A : |a - b| \leqslant \frac{1}{2^{n}}$$
			Notons $a_{n}$ un tel élément. Nous venons de construire $(a_{n})_{n \in \N} \in A^{\N}$ vérifiant : \\
			$\forall n \in \N, |a_n - b| \leqslant \frac{1}{2^{n}}$ \\
			Or : $\underset{n \to +\infty}{\lim} \frac{1}{2^{n}} = 0$ \\
			Ainsi, d'après le théorème sans nom, $(a_{n})_{n \in \N}$ converge vers $b$.
		\end{itemize}
	\end{question_kholle}
	
	\begin{question_kholle}
	    [Soit $u\in \R ^{\N}$ qui converge vers $\ell \in \R$. \\
	    Alors la moyenne arithmérique des $n\in \N$ premiers termes (appelée moyenne de Césarò) converge vers $\ell$.]
	    {Théorème de Césarò}
	
	    Soient $u$ une telle suite, $\varepsilon \in \R ^*_+$ et $\ell \in \R$ ladite limite de $u$. Appliquons la définition de la convergence de $u$ pour $\varepsilon \gets \frac{\varepsilon}{2}$ : 
	    \[
	    \exists N \in \N \ : \ \forall n \in \N , \ n\geq N \ \implies \ |u_n - \ell | \leq \frac{\varepsilon}{2}.
	    \]
	    Fixons un tel $N$. Posons $\omega = \sum_{k=0}^{N-1} |u_k - \ell | \in \R$. Soit $n\in \N$ tel que $n \geq N$. Calculons : 
	    \[
	    \left| \frac{1}{n} \sum_{k=0}^{n-1}u_k - \ell \right| = \left| \frac{1}{n} \left( \sum_{k=0}^{n-1}u_k - n\ell \right) \right|  = \left| \frac{1}{n} \sum_{k=0}^{n-1}(u_k - \ell)  \right| \leq \frac{1}{n} \underset{= \ \omega \in \R}{\underbrace{\sum_{k=0}^{N-1}|u_k - \ell|}} + \frac{1}{n} \underset{\leq \ \frac{\varepsilon}{2}}{\underbrace{\sum_{k=N}^{n}|u_k - \ell|}} \leq \frac{\omega}{n} + \underset{\leq \ \frac{\varepsilon}{2}}{\underbrace{\frac{\varepsilon}{2n}}}.
	    \]
	    Ces majorations sont issues de l'inégalité triangulaire et de la convergence de $u$. De plus, comme la suite $(v_n) _{n\in \N} = \left( \frac{\omega}{n} \right) _{n\in \N}$ converge vers $0$, on écrit sa définition pour $\varepsilon \gets \frac{\varepsilon}{2}$ : 
	    \[
	    \exists N' \in \N \ : \ \forall n \in \N , \ n\geq N' \ \implies \ |v_n| \leq \frac{\varepsilon}{2}.
	    \]
	    On fixe un tel $N'$ et on pose $\Lambda = \max{(N, N')}$ qui a bien un sens car $\{N, \ N'\}$ est une partie finie de $\N$.
	    De la même manière qu'auparavant, pour $n\in \N$ tel que $n \geq \Lambda$, on a : 
	    \[
	     \left| \frac{1}{n} \sum_{k=0}^{n-1}u_k - \ell \right| \leq \underset{\leq \ \frac{\varepsilon}{2}}{\underbrace{\frac{\omega}{n}}} + \frac{\varepsilon}{2} \leq \varepsilon.
	    \] 
	    C'est le théorème souhaité.
	\end{question_kholle}

	\begin{question_kholle}
		[Soient $(u,v) \in \R^{\N}$ : \\
		{\begin{enumerate}[label=($\roman*$)]
			\item Si $\begin{array}{|l}
				\exists N \in \N : \forall n \in \N, n \geqslant N \Rightarrow u_{n} \geqslant 0   \\
				u \text{ converge}
			\end{array}$ \\
			Alors $\lim u \geqslant 0$
			\item Si $\begin{array}{|l}
				\exists N \in \N : \forall n \in \N, n \geqslant N \Rightarrow u_{n} \leqslant v_{n}   \\
				u \text{ et } v \text{ convergent}
			\end{array}$ \\
			Alors $\lim u \leqslant \lim v$
		\end{enumerate}}
		]
		{Théorème de passage à la limite dans une inégalité.}
		~\smallbreak
		\begin{enumerate}[label=($\roman*$)]
			\item L'hypothèse $\exists N \in \N : \forall n \in \N, n \geqslant N \Rightarrow u_{n} \geqslant 0$ permet d'affirmer que $u$ et $|u|$ coïncident à partir d'un certain rang. \\
			Par ailleurs, la convergence de $u$ et la continuité de $|\cdot|$ sur $\R$ donc en $\lim u$ donnent $|u|$ converge vers $|\lim u|$. \\
			Le caractère asymptotique de la limite permet de conclure que $u$ et $|u|$ ont la même limite. \\
			Donc $\lim u = |\lim u| \geqslant 0$
			\item $\exists N \in \N : \forall n \in \N, n \geqslant N \Rightarrow u_{n} \leqslant v_{n} \Rightarrow v_{n} - u_{n} \geqslant 0$ \\
			$u$ et $v$ convergent $\Rightarrow v-u$ converge vers $\lim v - \lim u$. \\
			On applique $(i)$ pour $u \leftarrow v - u$, autorisé car $u \text{ et }v$ convergent. \\
			On obtient $\lim v - \lim u \geqslant 0$ d'où $\lim u \leqslant \lim v$.
		\end{enumerate}
	\end{question_kholle}

	\begin{question_kholle}
	    [Soient $u$ et $v$ deux suites réelles adjacentes. Alors $u$ et $v$ convergent et ont la même limite.]
	    {Théorème des suites adjacentes}
	
	    Soient $u$ et $v$ de telles suites. Quitte à inverser les rôles desdites suites, prenons $u$ croissante et $v$ décroissante. \\
	    On a donc : 
	    \[
	    \forall n \in \N, \ (u_n \leq v_n \leq \underset{\in \R}{\underbrace{v_0}}) \wedge (\underset{\in \R}{\underbrace{u_0}}\leq  u_n \leq v_n),
	    \]
	    car la monotonie des suites induit ces inégalités. D'après le théorème de limite monotone, $u$ étant croissante et majorée elle converge, $v$ étant décroissante et minorée elle converge. \\
	    Il s'en suit que par définition des suites adjacentes : 
	    \[
	    0 \ = \lim_{n \to +\infty} (u_n - v_n) \ \underset{u,v \ \text{ convergent}}{\underbrace{=}} \ \lim_{n \to +\infty} u_n - \lim_{n \to +\infty} v_n.
	    \]
	    Ainsi, $\lim u = \lim v$.
	\end{question_kholle}
	
	\begin{question_kholle}
	    [Soit $u$ une suite bornée. $u$ converge si et seulement si il existe $\ell \in \mathbb{K}$ tel que $L(u)$ est le singleton $\ell$ ]
	    {*Facultative* Caractérisation de la convergence par l'unicité d'une valeur d'adhérence pour une suite bornée.}
	    Traitons le cas réel, celui sur $\C$ est à adapter sans peine.\\
	    Supposons que $u$ converge et posons $\lim u =\ell \in \R  $. Toutes les sous-suites de $u$ convergent vers $\ell$ donc $L(u)=\{\ell \}$. \\
	    Supposons maintenant qu'il existe un unique $\ell \in \R$ tel que $L(u) = \{ \ell \}$. Par l'absurde, supposons que $u$ ne converge pas vers $\ell$, c'est-à-dire : 
	    \[
	    \exists \varepsilon \in \R ^* _+ \ : \ \forall N \in \N, \ \exists n \in \N \ : \ n\geq N \text{ et } |u_n - \ell | > \varepsilon.
	    \]
	    Fixons un tel $\varepsilon$. \\
	    %\textbf{Etape 1} : \textit{Construction d'une sous-suite de $u$ dont les termes sont $\varepsilon$-éloignés de $\ell$.} \\
	    Posons $\varphi (0) = \min{ \{ k\in \N \ | \ |u_k - \ell| > \varepsilon \} }$, ce qui a du sens car c'est une partie non-vide de $\N$. Posons ensuite $\varphi (1) = \min{ \{ k\in \N \ | \ |u_k - \ell| > \varepsilon, \ \varphi(0) < k \} } $, ce qui a du sens pour les mêmes raisons. On construit en itérant ce procédé $\varphi (n)$ tel que : 
	    \[
	    \forall n \in \N, \ \varphi(n+1) = \min{ \{ k\in \N \ | \ |u_k - \ell| > \varepsilon, \ \varphi(n) < k \} }.
	    \]
	    De cette manière, nous venons de construire une extractrice telle que : 
	    \[
	    \forall n \in \N, \ |u_{\varphi(n)} - \ell| > \varepsilon.
	    \]
	    Par hypothèse $u$ est bornée, donc il existe $M\in \R _+$ tel que : 
	    \[
	    \forall n \in \N, \ |u_n| \leq M,
	    \]
	    donc pour tout $n$ dans $\N$, $|u_{\varphi(n)}| \leq M$, donc $(u_{\varphi(n)})_{n\in \N}$ est bornée. \\
	    Par le théorème de Bolzano-Weierstrass, il existe $\psi$ une extractrice et $\ell ' \in \R$, avec $\varphi \circ \psi$ qui est aussi une extractrice par composition d'applications strictement croissantes, donc$(u_{\varphi \circ \psi (n)})_{n\in \N}$ est une sous-suite de $u$ et $\ell ' \in L(u) = \{ \ell \}$.\\
	    Par ailleurs, pour tout $n$ dans $\N$ :
	    \[
	    \underset{\xrightarrow[n\to +\infty]{}|\ell' -\ell|}{\underbrace{|u_{\varphi \circ \psi (n)} - \ell|}} > \varepsilon,
	    \]
	    donc en passant à la limite dans l'inégalité on a pour tout $n$ dans $\N$, $|\ell ' - \ell | \geq \varepsilon > 0$, ce qui n'est pas possible car $\ell$ est la seule valeur d'adhérence possible et ici la différence n'est pas nulle.
	\end{question_kholle}

\end{document}

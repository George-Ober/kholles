\documentclass{article}

\date{13 décembre 2023}
\usepackage[nb-sem=11, auteurs={Kylian Boyet, Hugo Vangilluwen, Jérémie Menard, George Ober}]{../kholles}

\begin{document}
\maketitle

\begin{question_kholle}
  [Soient $(A,B) \in (\mathcal{P}(\R) \setminus \{\varnothing\})^{2}$. Montrons que :
    \\
    $$A \text{ est dense dans } B \iff \left\{ \begin{array}{l}
        A \subset B \\
        \forall b \in B, \exists(a_{n}) \in A^{\N} : (a_{n}) \text{ converge vers }b
      \end{array}\right. $$
  ]
  {Caractérisation séquentielle de la densité.}

  Sens indirect : supposons $A \subset B$ et $\forall b \in B, \exists(a_{n}) \in A^{\N} : (a_{n}) \text{ converge vers }b$ :\\
  \begin{itemize}
    \item[$\star$] $A \subset B$ par hypothèse.
    \item[$\star$] Montrons que $\forall b \in B, \forall \varepsilon \in \R^{*}_{+}, \exists a \in A : |b - a| < \varepsilon$ (on utilise la caractérisation de la densité avec les $\varepsilon$) \\
          Soient $b \in B$ et $\varepsilon \in \R^{*}_{+}$ fixés quelconques : \\
          Par hypothèse appliquée pour $b \leftarrow b$ : $\exists(a_{n}) \in A^{\N} : a_{n} \underset{n \to +\infty}{\longrightarrow}b$ \\
          Appliquons la définition de la convergence de $(a_{n})$ vers $b$ pour $\varepsilon \leftarrow \frac{\varepsilon}{2}$ : \\
          $$\exists N \in \N : \forall n \in \N, n \geqslant N \Rightarrow |a_{n} - b| \leqslant \frac{\varepsilon}{2}$$
          Fixons un tel N : \\
          En particulier, $a_{N} \in A$ et $|a_{N} - b| \leqslant \frac{\varepsilon}{2} \leqslant \varepsilon$ \\
          Donc $A$ est dense dans $B$.
  \end{itemize}

  Sens direct : supposons $A$ dense dans $B$ : \\
  \begin{itemize}
    \item[$\star$] Par définition, $A \subset B$
    \item[$\star$] Soit $b \in B$ fixé quelconque. \\
          Soit $n \in \N$ fixé quelconque :  \\
          Appliquons la caractérisation de la densité par les $\varepsilon$ pour $\varepsilon \leftarrow \frac{1}{2^{n}}$ (autorisé car $\frac{1}{2^{n}} > 0$), et $b \leftarrow b$ :
          $$\exists a \in A : |a - b| \leqslant \frac{1}{2^{n}}$$
          Notons $a_{n}$ un tel élément. Nous venons de construire $(a_{n})_{n \in \N} \in A^{\N}$ vérifiant : \\
          $\forall n \in \N, |a_n - b| \leqslant \frac{1}{2^{n}}$ \\
          Or : $\underset{n \to +\infty}{\lim} \frac{1}{2^{n}} = 0$ \\
          Ainsi, d'après le théorème sans nom, $(a_{n})_{n \in \N}$ converge vers $b$.
  \end{itemize}
\end{question_kholle}





\end{document}

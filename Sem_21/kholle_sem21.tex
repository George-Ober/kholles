\documentclass{article}

\date{14 avril 2024}
\usepackage[nb-sem=21, auteurs={Kylian Boyet}]{../kholles}

\begin{document}
	\maketitle
    \begin{question_kholle}
        [Tous les polynômes de degré 1 sont irréductibles, les polynômes irréductibles de degré $2$ ou $3$ sont les polynômes sans racine.s dans le corps de base.]
        {Caractériser les polynômes irréductibles de degré 1, 2 et 3 dans $\cx{K}$.}
        Un polynôme de degré $1$ ne peut s'écrire comme produit de 2 polynômes de \\ degré $\geq 1$ donc il est irréductible. \\
        Soit $P \in \cx{K}$ un polynôme irréductible de degré $2$ ou $3$. \\
        Par définition, $P$ n'a pas de racine.s dans $\K$, donc la première inclusion. \\
        Soit $P\in \K [X]$ tel que $\deg P = 2$ et $P$ n'ayant pas de racines dans $\K$. \\
        Montrons que $P$ est irréductible. Montrons la contraposée. Supposons $P$ non-irréductible. \\
        \[
            \exists \ A, \ B \in \K [X] \ : \ P = AB \ \text{et} \ \deg A, \ \deg B \geq 1,
        \]
        On a alors, $P=AB \ \implies \ 2 = \deg A + \deg B \ \implies \ \deg A, \ \deg B = 1$ donc :
        \[
            \exists \ \alpha, \ \gamma \in \K^* \times \K \ : \ A = \alpha X + \gamma, 
        \]
        ainsi, $P=( \alpha X + \gamma )B= \alpha \left( X + \frac{\gamma}{\alpha} \right)B $, donc $P$ admet $-\frac{\gamma}{\alpha}\in \K$ comme racine, ce qui montre la contraposée.\\
         Soit $P\in \K [X]$ tel que $\deg P = 3$ et $P$ n'ayant pas de racines dans $\K$.\\
         Montrons la contraposée. Supposons $P$ non-irréductible. De même, on a : 
         \[
            \exists \ A, \ B \in \K [X] \ : \ P = AB \ \text{et} \ \deg A, \ \deg B \geq 1,
         \]
         Puis encore, $P=AB \ \implies \ 3 = \deg A + \deg B \ \implies \ \deg A, \ \deg B \in \{2,1\}$ (l'un n'étant pas l'autre). Donc l'un des deux est de degré 1 donc $P$ admet une racine dans $\K$, donc encore une fois cela montre la contraposée, ce qui démontre l'inclusion réciproque.
    \end{question_kholle}

    \begin{question_kholle}
        [Les polynômes irréductibles de $\cx{\C}$ sont les polynômes de degré $1$ et ceux de $\cx{R}$ sont les polynômes de degré $1$ et les polynômes de degré $2$ de discriminant strictement négatif. ]
        {Décrire les polynômes irréductibles de $\C [X]$ et de $\R [X]$.}
        Le premier point est immédiat, les polynômes irréductibles d'un corps contiennent les polynômes de degré $1$ et par le théorème de D'Alembert-Gauss, tout polynôme de $\C [X]$ ($\deg \geq 2$) est scindé dans $\C [X]$, donc non-irréductible. \\ \\
        Pour le second point, le cas du degré $1$ est réglé. Soit $P$ un polynôme irréductible de $\R [X]$. \\
        Supposons que $P$ soit de degré supérieur ou égal à $3$. Si son degré est impair, le TVI conclut quant à l'existence d'une racine, donc non-irréductible. Si son degré est pair, par D'Alembert-Gauss, on obtient $\deg P$ couples de racines possiblement égaux. \\
        Or, $P\in \cx{R}$ donc $\forall z\in \C, \ P(z) = 0 \ \implies P(\overline{z}) = 0 $ donc les racines se rassemblent 2 à 2 pour former un polynôme scindé dans $\R$, donc non-irréductible. Ainsi, $\deg P = 2$, immédiatement, si le discriminant de $P$ est positif ou nul, $P$ admet une ou deux racines dans $\R$, donc non irréductible. Enfin, son discriminant est alors négatif, de cette manière $P$ n'admet pas de racine dans $\R$ et est donc irréductible. Ce qui achève la preuve.
    \end{question_kholle}

    \begin{question_kholle}
        [Il s'agit donc de montrer que racine cubique de $2$ n'est pas un rationnel.]
        {Montrer que $X^3 -2$ est irréductible dans $\cx{Q}$.}
        Supposons, par l'absurde, qu'il existe $r\in \Q$ tel que $r^3 - 2 = 0 $. Prenons $p,q \in \Z \times \N^*$ \textit{le} représentant irréductible de $r$ dans $\Q$. On a alors, $p^3 = 2q^3$ donc $2\ | \ p^3$ or $2\in \PRIME$ donc $2\ | \ p$, ainsi, il existe $k\in \Z$ tel que $p = 2k$. Par conséquent, $2(2k^3) = q^3$ donc $2 \ | \ q^3$ or $2 \in \PRIME$ donc $2\ |\ q $ donc ceci contredit $p$ et $q$ premiers entre eux, par définition d'un représentant irréductible. Ainsi, $P = X^3 -2$ n'admet pas de racine.s dans $\Q$, c'est donc un polynôme irréductible.
    \end{question_kholle}

    \begin{question_kholle}
        [C'est une conséquence de la définition du pgcd de deux polynômes $P \wedge Q = \prod_{i\in I}P_i^{\min \{ m_i, p_i\} }$, où les $P_i$ sont les facteurs irréductibles de $P$ et $Q$ dans leur décomposition. ]
        {Pour $P = \prod_{k=1}^p(X-z_k)^{m_k}\in \cx{C}$ avec $m_k \in \N^*$ pour tout $k\in \lient 1,p \rient$, montrer que $P \wedge P' = \prod_{k=1}^p(X-z_k)^{m_k - 1}$.}
        Soit $P$ un tel polynôme et $p$ un entier naturel non nul. Naturellement, $P'$ hérite de $P$, $\deg P - p$ racines, lesquelles sont les $z_k$ pour $k\in \lient 1 , p \rient $, de multiplicité $m_k -1$. Ainsi, 
        \[
            \exists \ B \in \C [X] \ : \ \left[ P' = \left( \prod_{k=1}^p(X-z_k)^{m_k -1} \right) B \right] \ \wedge \ \left[  \deg B = p  \right],
        \]
        de cette manière on peut écrire : 
        \[
            P' = \left( \left( \prod_{k=1}^p(X-z_k)^{m_k -1} \right) B \right) P^0 \ \text{et} \ P = \left( \prod_{k=1}^p(X-z_k)^{m_k} \right) (P')^0,
        \]
        de façon à faire apparaître dans les deux décompositions les mêmes facteurs, possiblement avec une puissance $0$, histoire de coller à la définition de manière explicite. Ceci fait, il ne reste plus qu'à appliquer la définition du pgcd et de remarquer que seuls les $(X-z_k)^{m_k -1}$ subsistent, \\ notons $\mathfrak{I}$ l'ensemble des facteurs de leur décomposition, on a alors :  
        \[
            P \wedge P' = \prod_{D\in \mathfrak{I}}D ^{\min \{ \nu_D (P), \nu _D(P') \}} = \prod_{k=1}^p (X-z_k)^{m_k -1},
        \]
        où $\nu _D (\cdot ) $ est la valuation $D$-adique au sens des polynômes irréductibles. Ce qui conclut.
    \end{question_kholle}

    \begin{question_kholle}
        [Il s'agit là de vérifier que la définition que l'on souhaiterai le plus, c'est-à-dire la même que pour la dérivée d'une fraction de fonctions, s'applique effectivement aux fractions rationnelles, c'est-à-dire que cette définition ne dépend pas du représentant choisi.]
        {Justifier la bonne définition de la dérivée d'une fraction rationnelle.}
        Montrons que pour $A, \ B \in \K [X] \times \K [X] \backslash \{ 0 _{ \K[X]} \}$, on a : 
        \[
            \left( \frac{A}{B} \right) ' = \frac{A'B - B'A}{B^2}.
        \]
        Soient $A$ et $B$ de tels polynômes et $C, \ D \in \K [X] \times \K [X] \backslash \{ 0 _{ \K[X]} \}$ tels que $AD = BC$, en dérivant on obtient $A'D + D'A = B'C + C'B$. Calculons : 
        \begin{center}
        $
        \begin{array}{rcl}
            (A'B-B'A)D^2 & = & D(A'BD- (AD)B')  \\
             & = & D(A'BD- BCB') \\
             & = & BD(A'D- CB') \\
             & = & BD(C'B - D'A) \\
             & = & B(C'BD - (AD)D') \\
             & = & B(C'BD - BCD') \\
             & = & B^2(C'D - D'C),
        \end{array}
        $
        \end{center}
        ce qui prouve que le résultat ne dépend pas du représentant, par définition de $\K(X)$ comme structure quotient.
    \end{question_kholle}

    \begin{question_kholle}
        [Montrer que les racines du polynôme dérivée sont dans l'enveloppe convexe des racines du polynôme.]
        {Théorème de Gauss-Lucas et interprétation graphique.}
        Pour ce qui est de l'interprétation graphique, elle n'est pas prévue à l'heure qu'il est dans ce pdf, pour la faire soi-même dessiner des points et les "clôturer" dans un polygone convexe, ou même faire ceci avec un cas concret. \\
        Soit $P\in \C [X]$ de degré au moins $2$ et notons $z_1, \dots, z_n$ ses racines répétées avec multiplicité. \\
        Soit $u$ une racine de $P'$. 
        Notre but est : 
        \[
            \exists \ (c_1, \dots, c_n)\in \R ^*_+ \ : \ \sum_{k=1}^n c_k z_k = u \ \text{et} \ \sum_{k=1}^n c_k = 1.
        \]
        $\rightarrow$ Si $u$ est une racine de $P$ alors noter son indice et utiliser le symbole de Kronecker. \\
        $\rightarrow$ Sinon, $u$ n'appartient pas aux racines de $P$, donc $u$ n'est pas pôle de $\frac{P'}{P}$ ce qui permet de prendre l'image par le morphisme d'évaluation en $u$ de cette même fraction rationnelle : 
        \[
            0_\K = \frac{P'(u)}{P(u)} = \sum_{k = 1}^n \frac{1}{u - z_k} = \sum_{k = 1}^n \frac{\overline{u}- \overline{z_k}}{|u-z_k|^2} = \sum_{k = 1}^n \frac{\overline{u}}{|u-z_k|^2}- \sum_{k = 1}^n \frac{\overline{z_k}}{|u-z_k|^2}.
        \]
        Donc en passant la seconde somme à gauche et en prenant le conjugué : 
        \[
            \sum_{k = 1}^n\frac{u}{|u-z_k|^2} = \sum_{k = 1}^n \frac{z_k}{|u-z_k|^2} \ \implies \ u = \frac{\sum_{k = 1}^n \frac{z_k}{|u-z_k|^2}}{\sum_{k = 1}^n\frac{1}{|u-z_k|^2}}= \sum_{k = 1}^n\underset{= \ c_k}{\underbrace{\frac{\frac{1}{|u-z_k|^2}}{\sum_{i = 1}^n\frac{1}{|u-z_i|^2}}}} z_k = \sum_{k=1}^n c_k z_k ,
        \]
        ce qui démontre la première partie du résultat, il est immédiat de vérifier que $\sum_{k=1}^nc_k = 1$, vérification laissée aux lecteurs. Ce qui achève la preuve.
    \end{question_kholle}

    \begin{question_kholle}
        []
        {Donner deux expressions du coefficient associé à un pôle simple dans une décomposition en éléments simples.}
        Soient $(P, Q) \in \K [X] \times \left( \K [X] \backslash \{ 0_{\K[X] }\} \right) $ tels que la fraction rationnelle $\frac{P}{Q}$ soit \\ irréductible et en prenant $\deg P < \deg Q$. En appliquant le théorème de décomposition en éléments simples on obtient un expression de la forme : 
        \[
            \exists \ R \in \K (X) \ : \ \frac{P}{Q} = \sum_{k=1}^n \frac{a_k}{X - z_k} + R, 
        \]
        où les $z_k$ pour $k\in \lient 1 ,n \rient$ sont racines de $Q$. Ainsi, en prenant $k_0 \in \lient 1, n \rient $ tel que $z_{k_0}$ soit racine simple, 
        \[
            \frac{P(X-z_{k_0})}{Q} = a _ {k_0} + \sum_{k=0 \ \text{et} \ k \neq k_0} \frac{a_k(X-z_{k_0})}{X - z_{k_0}} + R(X - z_{k_0}),
        \] 
        une première expression se trouvera en notant $\widetilde{Q} = \displaystyle \prod_{\substack{k=1 \\ k\neq k_0}}^n(X-z_k)^{\nu_{(X-z_k)}(Q)}$, on a alors : 
        \[
            \frac{P(z_{k_0})}{\widetilde{Q}(z_{k_0})} = a_{k_0}.
        \]
        Une autre expression est possible en explicitant $\widetilde{Q}$. Pour ce faire, remarquons plutôt : 
        \[
            Q' = \sum_{k=1}^n \nu_{(X-z_k)}(Q) (X- z_k)^{\nu_{(X-z_k)}(Q) -1}\prod _{\substack{i = 1 \\ i \neq k}}^n(X-z_i)^{\nu_{(X-z_i)}(Q)},
        \]
        donc en prenant l'image par le morphisme d'évaluation en $z_{k_0}$ on obtient : 
        \[
            Q'(z_{k_0}) = \prod _{\substack{i = 1 \\ i \neq k_0}}^n(z_{k_0}-z_i)^{\nu_{(X-z_i)}(Q)},
        \]
        il s'agit exactement de $\widetilde{Q}(z_{k_0})$. Ainsi, 
        \[
             \frac{P(z_{k_0})}{Q'(z_{k_0})} =a_{k_0},
        \]
        ce qui suffit.
    \end{question_kholle}

    \begin{question_kholle}
        []
        {Donner des expressions des deux coefficients associés à un pôle double dans une décomposition en éléments simples.}
        Soient $(P, Q) \in \K [X] \times \left( \K [X] \backslash \{ 0_{\K[X] }\} \right) $ tels que la fraction rationnelle $\frac{P}{Q}$ soit \\ irréductible et en prenant $\deg P < \deg Q$. En appliquant le théorème de décomposition en éléments simples on obtient un expression de la forme suivante en considérant $z_{k_0}$, une racine double de $Q$ : 
        \[
            \exists \ R \in \K (X) \ : \ \frac{P}{Q} = \frac{a_1}{X - z_{k_0}}  + \frac{a_2}{(X - z_{k_0})^2} + R, 
        \]
        puis de même, 
        \[
            \frac{P(X - z_{k_0})^2}{Q} = a_2  + \left( \frac{a_1}{X - z_{k_0}} + R \right)(X - z_{k_0})^2,
        \]
        donc en notant $\widetilde{Q} = \displaystyle \prod_{\substack{k=1 \\ k\neq k_0}}^n(X-z_k)^{\nu_{(X-z_k)}(Q)}$, on a : 
        \[
            \frac{P(z_{k_0})}{\widetilde{Q}(z_{k_0})} = a_2, 
        \]
        c'est une première expression. Pour la suivante, encore une fois, explicitons $\widetilde{Q}$. Remarquons que :
        \[
            \exists \ A \in \K [X] \ : \ \left[ Q'' = 2\prod_{\substack{k=1 \\ k= k_0}}^n (X-z_k)^{\nu_{(X-z_k)}(Q)} + A \right] \ \wedge \ \left[ A(z_{k_0} ) = 0 \right],
        \]
        donc, en remarquant que : 
        \[
            2\widetilde{Q}(z_{k_0}) = Q''(z_{k_0}),
        \]
        on a finalement : 
        \[
            \frac{2P(z_{k_0})}{Q''(z_{k_0})} = a_2,
        \]
        ce qui termine la question. 
    \end{question_kholle}
    
\end{document}
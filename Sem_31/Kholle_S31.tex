\documentclass{article}

\date{8 juin 2024}
\usepackage[nb-sem=30, auteurs={Hugo Vangilluwen}]{../kholles}

\begin{document}
	\maketitle
	
	Pour cette semaine, $E$ est un ensemble fini de cardianl $n \in \N^*$ et $(\Omega, \proba)$ désigne un espace probabilisé fini.
	
	\begin{question_kholle}
		[Soit $p \in \N^*$. Un $p$-partage de $E$ est un $p$-liste $(A_1, \ldots, A_p) \in \mathcal{P}(E)^p$ de parties de $E$ (éventuellement vide), deux à deux disjointes qui recouvrent $E$ c'est-à-dire \tq+* t:
		\begin{equation}
			\forall (i, j) \in \lient 1 ; p \rient,
			i \neq j \implies A_i \cap A_j = \emptyset
			\qquad \text{et} \qquad
			\bigcup_{i=1}^{p} A_i = E
		\end{equation}
		
		Soient $(n_1, \ldots n_p) \in \N^p$ \tqs $n = n_1 + \ldots + n_p$ est un $p$-partage de $E$ \tq
		\begin{equation*}
			\forall (i, j) \in \lient 1 ; p \rient, \
			\left|A_i\right| = n_i
		\end{equation*}
		Le nombre de $p$-partage de type $(n_1, \ldots, n_p)$ est :
		\begin{equation}
			\frac{n!}{\displaystyle \prod_{i=1}^{p} n_i !}
		\end{equation}
		]
		{$p$-partage d'un ensemble $E$ et leur dénombrement}
		
		Considérons les $p$-partages de type $(n_1, \ldots, n_p)$ et appliquons le principe des choix successifs :
		\begin{equation*}
			\left(
				\underbrace{A_1}_{\binom{n}{n_1} \text{ choix}},
				\underbrace{A_2}_{\binom{n}{n_2} \text{ choix}},
				\underbrace{A_3}_{\binom{n}{n_3} \text{ choix}},
				\ldots,
				\underbrace{A_p}_{\binom{n}{n_p} \text{ choix}}
			\right)
		\end{equation*}
		donc il y a
		\begin{equation*}
			\frac{ n! }{n_1! \cancel{(n-n_1)!} }
			\frac{ \bcancel{(n-n_1)!} }{n_2! \cancel{(n-n_1-n_2)!} }
			\frac{ \bcancel{(n-n_1-n_2)!} }{n_2! \cancel{(n-n_1-n_2-n_3)!} }
			\ldots
			\frac{ \bcancel{(n-(n_1+\ldots+n_{p-1})!} }{n_p! \underbrace{(n_1+\ldots+n_p)!}_{=0!} }
		\end{equation*}
		Donc, au total, il y a $\frac{n!}{n_1! n_2! \ldots n_p!}$ $p$-partages.
	\end{question_kholle}
	
	\begin{question_kholle}
		[Soit $B$ un évènement de probabilité non nulle.
		L'application $\proba_B$
		{\begin{equation}
			\proba_B \left| \begin{array}{ccc}
				\mathcal{P}(\Omega) & \mapsto & [0;1] \\
				A & \rightarrow & \displaystyle \frac{\proba( A \cap B )}{\proba(B)}
			\end{array} \right.
		\end{equation}}
		est une probabilité sur sur $\Omega$. ]
		{Une probabilité conditionnelle est une probabilité}
		
		~\\
		\begin{liste}
			\item Soit $A \in \mathcal{P}(\Omega)$ \fq. \\
			On a $\emptyset \subset A \cap B  \subset B$ donc par croissance de la probabilité, $0 = \proba(\emptyset) \leqslant \proba(A \cap B) \leqslant \proba(B)$.
			En divisant par $\proba(B) \neq 0$, $0 \leqslant \proba_B(A) \leqslant 1$. Donc $\proba_B$ est \textit{bien définie}.
			\item $\proba_B(\Omega)
			= \frac{\proba(\Omega \cup B)}{\proba(B)}
			= \frac{\proba(B)}{\proba(B)}
			= 1$
			\item Soient $(A, A') \in \mathcal{P}(\Omega)^2$ \fq* \tq* $A$ et $A'$ sont incompatibles.
			\begin{equation}
				\begin{aligned}
					\proba_B(A \amalg A')
					&= \frac{ \proba(B \cap (A \amalg A') ) }{ \proba(B) } \\
					&= \frac{ \proba( (B \cap A) \amalg (B \cap A') ) }{ \proba(B) } \text{ car } (B \cap A) \cap (B \cap A') \subset A \cap A' = \emptyset \\
					&= \frac{ \proba(B \cap A) + \proba(B \cap A') }{ \proba(B) } \\
					&= \proba_B(A) + \proba_B(A')
				\end{aligned}
			\end{equation}
		\end{liste}
		Ainsi, $\proba_B$ est bien une probabilité sur $\Omega$.
	\end{question_kholle}
\end{document}

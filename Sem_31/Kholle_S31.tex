\documentclass{article}

\date{30 juin 2024}
\usepackage[nb-sem=31, auteurs={Hugo Vangilluwen, George Ober}]{../kholles}

\begin{document}
\maketitle

Pour cette semaine, $E$ est un ensemble fini de cardianl $n \in \N^*$ et $(\Omega, \proba)$ désigne un espace probabilisé fini.

\begin{question_kholle}
	[Soit $p \in \N^*$. Un $p$-partage de $E$ est un $p$-liste $(A_1, \ldots, A_p) \in \mathcal{P}(E)^p$ de parties de $E$ (éventuellement vide), deux à deux disjointes qui recouvrent $E$ c'est-à-dire \tq+* t:
		\begin{equation}
			\forall (i, j) \in \lient 1 ; p \rient,
			i \neq j \implies A_i \cap A_j = \emptyset
			\qquad \text{et} \qquad
			\bigcup_{i=1}^{p} A_i = E
		\end{equation}

		Soient $(n_1, \ldots n_p) \in \N^p$ \tqs $n = n_1 + \ldots + n_p$ est un $p$-partage de $E$ \tq
		\begin{equation*}
			\forall (i, j) \in \lient 1 ; p \rient, \
			\left|A_i\right| = n_i
		\end{equation*}
		Le nombre de $p$-partage de type $(n_1, \ldots, n_p)$ est :
		\begin{equation}
			\frac{n!}{\displaystyle \prod_{i=1}^{p} n_i !}
		\end{equation}
	]
	{$p$-partage d'un ensemble $E$ et leur dénombrement}

	Considérons les $p$-partages de type $(n_1, \ldots, n_p)$ et appliquons le principe des choix successifs :
	\begin{equation*}
		\left(
		\underbrace{A_1}_{\binom{n}{n_1} \text{ choix}},
		\underbrace{A_2}_{\binom{n}{n_2} \text{ choix}},
		\underbrace{A_3}_{\binom{n}{n_3} \text{ choix}},
		\ldots,
		\underbrace{A_p}_{\binom{n}{n_p} \text{ choix}}
		\right)
	\end{equation*}
	donc il y a
	\begin{equation*}
		\frac{ n! }{n_1! \cancel{(n-n_1)!} }
		\frac{ \bcancel{(n-n_1)!} }{n_2! \cancel{(n-n_1-n_2)!} }
		\frac{ \bcancel{(n-n_1-n_2)!} }{n_2! \cancel{(n-n_1-n_2-n_3)!} }
		\ldots
		\frac{ \bcancel{(n-(n_1+\ldots+n_{p-1})!} }{n_p! \underbrace{(n_1+\ldots+n_p)!}_{=0!} }
	\end{equation*}
	Donc, au total, il y a $\frac{n!}{n_1! n_2! \ldots n_p!}$ $p$-partages.
\end{question_kholle}

\begin{question_kholle}
	[Soit $B$ un évènement de probabilité non nulle.
		L'application $\proba_B$
		{\begin{equation}
					\proba_B \left| \begin{array}{ccc}
						\mathcal{P}(\Omega) & \mapsto     & [0;1]                                              \\
						A                   & \rightarrow & \displaystyle \frac{\proba( A \cap B )}{\proba(B)}
					\end{array} \right.
				\end{equation}}
		est une probabilité sur sur $\Omega$. ]
	{Une probabilité conditionnelle est une probabilité}

	~\\
	\begin{liste}
		\item Soit $A \in \mathcal{P}(\Omega)$ \fq. \\
		On a $\emptyset \subset A \cap B  \subset B$ donc par croissance de la probabilité, $0 = \proba(\emptyset) \leqslant \proba(A \cap B) \leqslant \proba(B)$.
		En divisant par $\proba(B) \neq 0$, $0 \leqslant \proba_B(A) \leqslant 1$. Donc $\proba_B$ est \textit{bien définie}.
		\item $\proba_B(\Omega)
			= \frac{\proba(\Omega \cup B)}{\proba(B)}
			= \frac{\proba(B)}{\proba(B)}
			= 1$
		\item Soient $(A, A') \in \mathcal{P}(\Omega)^2$ \fq* \tq* $A$ et $A'$ sont incompatibles.
		\begin{equation}
			\begin{aligned}
				\proba_B(A \sqcup A')
				 & = \frac{ \proba(B \cap (A \sqcup A') ) }{ \proba(B) }                                                                                  \\
				 & = \frac{ \proba( (B \cap A) \sqcup (B \cap A') ) }{ \proba(B) } \text{ car } (B \cap A) \cap (B \cap A') \subset A \cap A' = \emptyset \\
				 & = \frac{ \proba(B \cap A) + \proba(B \cap A') }{ \proba(B) }                                                                           \\
				 & = \proba_B(A) + \proba_B(A')
			\end{aligned}
		\end{equation}
	\end{liste}
	Ainsi, $\proba_B$ est bien une probabilité sur $\Omega$.
\end{question_kholle}

\begin{question_kholle}{Si $A$ et $B$ sont des événements indépendants, alors $A$ et $\overline{B}$ aussi}
	Supposons donc que $0 \leqslant \proba(A) \leqslant 1$ et $0 \leqslant \proba(B) \leqslant 1$.
	D'une part, $\{ B, \bar{B} \}$ constitue un système complet donc

	\begin{align*}
		\proba(A)                                 & = \proba(A \cap B) + \proba(A \cap \bar{B})    \\
		\iff \proba(A)                            & = \proba(A) \proba(B) + \proba(A \cap \bar{B}) \\
		\iff \proba(A) - \proba(A) \proba(B)      & = \proba(A \cap \bar{B})                       \\
		\iff \proba(A)(1-\proba(B))               & = \proba(A \cap \bar{B})                       \\
		\iff \proba(A) \proba\left(\bar{B}\right) & = \proba(A \cap \bar{B})
	\end{align*}

	Donc $A$ et $\bar B$ sont indépendants
\end{question_kholle}

\begin{question_kholle}[
		\begin{equation}
			\forall n \in \N^*, \
			\forall A \in \mathcal{P}(\Omega)^n, \
			\proba\left( \bigcap_{i=1}^{k}A_{i} \right) =
			\proba(A_{1}) \proba_{A_{1}}(A_{2}) \proba_{A_{1} \cap A_{2}}(A_{3}) \dots \proba_{A_{1} \cap \dots \cap A_{n-1}}(A_{n})
		\end{equation}
	]{Formule des probabilités composées}
	Procédons par récurrence.
	Soient $(A_{1}, \dots, A_{n})$, $n$ événements tels que $\proba\left( \bigcap_{i=1}^{n} A_{i} \right) \neq 0$.
	Pour tout $k \in \lient 2, n \rient$, posons
	$$\mathcal{H}_{k} : " \proba\left( \bigcap_{i=1}^{k}A_{i} \right) = \proba(A_{1})\proba_{A_{1}}(A_{2})\proba_{A_{1} \cap A_{2}}(A_{3})\proba_{A_{1} \cap A_{2} \cap A_{3}}(A_{4})\dots \proba_{A_{1} \cap\dots \cap A_{k-1}}(A_{k})"$$

	\begin{itemize}[label=$\star$]

		\item Initialisation, $k \leftarrow 2$
		      d'une part, $\bigcap_{i=1}^{n}A_{i} \subset A_{1}$, donc par croissance de $\proba$, $$0<\proba\left( \bigcap_{i=1}^{n}A_{i} \right) \leqslant \proba(A_{1})$$
		      Si bien que $\proba(A_{1}) \neq 0$ donc la probabilité conditionnelle $\proba_{A_{1}}$ a un sens.
		      D'où, par définition d'une probabilité conditionnelle:
		      $$\proba(A_{1} \cap A_{2}) = \proba(A_{1}) \proba_{A_{1}}(A_{2})$$
		      Donc $\mathcal{H}_{2}$ est vérifiée.

		\item Hérédité, Soit $k \in [ \! [ 2, n - 1] \!]$ fixé quelconque tel que $\mathcal{H}_{k}$ est vérifiée.

		      D'abord, remarquons que $\bigcap_{i=1}^{n}A_{i} \subset \bigcap_{i=1}^{k}A_{i}$ donc par croissance de $\proba$,
		      $$0 < \proba\left( \bigcap_{i=1}^{n}A_{i} \right)\leqslant \proba\left( \bigcap_{i=1}^{k}A_{i} \right)$$
		      Si bien que $\proba\left( \bigcap_{i=1}^{k} A_{i} \right) \neq 0$ donc la probabilité conditionnelle $\proba_{A_{1}\cap\dots \cap A_{k}}$  a un sens.


		      \begin{align*}
			      \proba\left( \bigcap_{i=1}^{k+1}A_{i} \right)
			       & = \proba\left( \left( \bigcap_{i=1}^{k}A_{i} \right) \cap A_{k+1} \right)                                                                                            \\
			       & = \proba\left( \bigcap_{i=1}^{k}A_{i} \right)\proba_{\bigcap _{i=1}^{k}A_{i}}(A_{k+1})                                                                               \\
			       & = \proba(A_{1})\proba_{A_{1}}(A_{2})\proba_{A_{1}\cap A_{2}}(A_{3})\dots \proba_{A_{1} \cap \dots \cap A_{k-1}}(A_{k}) \proba_{A_{1} \cap \dots \cap A_{k}}(A_{k+1})
		      \end{align*}
		      Donc $\mathcal{H}_{k+1}$ est aussi vérifiée
	\end{itemize}
\end{question_kholle}
\begin{question_kholle}{Formule des probabilités totales et formule de Bayes}

	\underline{Formule des probabilités totales}

	Soit $(A_{1}, \dots A_{n})$ un système complet d'événements.
	Comme ils sont incompatibles
	$$\proba\left( \bigsqcup_{k=1}^{n} A_{k} \right)= \sum_{k=1}^{n}\proba(A_{k})$$

	Le système est de plus complet donc $\bigsqcup _{k=1}^{n} A_{k} = \Omega$. Donc $\sum_{k=1}^{n}\proba(A_{k}) = 1$.

	$(A_{1}, \dots , A_{n})$ sont aussi deux à deux incompatibles, donc $(B \cap A_{1}, \dots B \cap A_{n})$ aussi.
	De plus $B = B \cap \Omega = B \cap \left( \bigsqcup_{k=1}^{n}A_{k} \right) = \bigsqcup_{k=1}^{n}(B \cap A_{k})$.
	Donc
	$$\proba(B)= \proba\left( \bigsqcup_{k=1}^{n}(B \cap A_{k}) \right) = \sum_{k=1}^{n}\proba(B \cap A_{k})$$
	De plus, en passant aux probabilités conditionnelles $\left(\proba_{A_{i}}\right)_{1\leqslant i\leqslant n}$ on a

	\begin{equation}
		\proba(B) = \sum_{k=1}^{n}\proba(A_{k})\proba_{A_{k}}(B)
	\end{equation}

	\underline{Formule de Bayes}

	Soient $A$ et $B$ deux événements de probabilité non nulle, on a alors :
	$$\proba(A)\proba_{A}(B) = \proba(A \cap B) = \proba(B)\proba_{B}(A)$$
	donc

	\begin{equation}
		\proba_{A}(B) = \frac{\proba(B)\proba_{B}(A)}{\proba(A)}
	\end{equation}
\end{question_kholle}

\begin{question_kholle}[{
				Soit $X$ une variable alétoire sur $\Omega$ et $g$ une fonction définie sur $X(\Omega)$.
				La loi de probabilité $Y = g(X)$ est donnée par $Y(\Omega) = g(X(\Omega))$ et
				\begin{equation}
					\forall y \in Y(\Omega), \proba_{Y}(\{ y \})= \proba(Y=y) = \sum_{x \in g^{-1}(\{ y \})}\proba(X=x)= \sum_{\substack{x \in X(\Omega)\\ g(x)=y}}\proba(X=x)
				\end{equation}
			}]{Loi d'une fonction de $X$}

	Utilisons le système complet $(X = x)_{x \in X(\Omega)}$ associé à la variable aléatoire $X$ et la formule des probabilités totales

	\begin{align*}
		\proba_{Y}(\{ y \}) = \proba(Y=y) & = \sum_{x \in X(\Omega)}\proba((Y=y) \cap (X=x)) \\
		                                  & = \sum_{\substack{x \in X(\Omega)                \\ g(x)=y}}\proba((g(X)=y) \cap (X=x)) + \sum_{\substack{x \in X(\Omega)\\ g(x)\neq y}}\proba((g(X)=y) \cap (X=x))
	\end{align*}


	Remarquons ainsi que
	\begin{itemize}[label=$\star$]
		\item Si $g(x) = y$
		      $$
			      \omega \in(X=x) \implies X(\omega)=x \implies g(X(\omega)) = g(x) \implies \omega \in (g(X) = y)
		      $$
		      D'où $(X = x) \subset (g(X)=y)$ donc $(g(X) = y) \cap(X=x)=(X=x)$

		\item Sinon, si $g(x) \neq y$
		      $$
			      \omega \in (X=x) \implies X(\omega) = x \implies g(X(\omega)) = g(x) \neq y \implies \omega \not\in (g(X) = y)
		      $$
		      Dans ce cas, $(g(X) = y) \cap (X = x) = \emptyset$

	\end{itemize}
	Ainsi,

	\begin{align*}
		\proba_{y}(\{ y \}) & = \sum_{\substack{x \in X(\Omega)         \\ g(x)=y}}\proba(\underbrace{ (g(X)=y) \cap (X=x) }_{ = (X=x) }) + \underbrace{ \sum_{\substack{x \in X(\Omega)\\ g(x)\neq y}}\proba((g(X)=y) \cap (X=x)) }_{ =0 }\\
		                    & = \sum_{\substack{x \in X(\Omega)         \\ g(x)=y}}\proba(X=x) \\
		                    & = \sum_{x \in g^{-1}(\{ y \})}\proba(X=x)
	\end{align*}


\end{question_kholle}
\begin{question_kholle}{Si $X \geqslant 0$ presque sûrement, $\esp(X) = 0 \iff X = 0$ presque sûrement}
	Soit $X\geqslant 0$ presque sûrement
	\begin{itemize}
		\item Supposons que $\esp(X) = 0$
		      Par hypothèse, l'évènement $(X<0)$ est négligeable donc

		      \begin{align*}
			      \esp (X) & = \sum_{\omega \in \Omega}X(\omega)\proba(\{ \omega \})                                                                                                                                                                   \\
			               & = \sum_{\omega \in (X = 0)} \underbrace{ X(\omega) }_{ = 0 } \proba(\{ \omega  \}) + \sum_{\omega \in (X<0)}X(\omega) \underbrace{ \proba(\{ \omega \}) }_{ =0 } + \sum_{\omega \in (X >0)} X(\omega)\proba(\{ \omega \}) \\
			               & = \sum_{\omega \in (X >0)} X(\omega)\proba(\{ \omega \})
		      \end{align*}

		      Soit $\omega_{0} \in (X>0)$ fixé quelconque
		      La nullité de l'espérance donne
		      $$0 \leqslant X(\omega_{0})\proba(\{ \omega_{0} \}) \leqslant \sum_{\omega \in(X>0)}X(\omega) P(\{ \omega \})= \esp(X) = 0$$
		      donc $X(\omega_{0})\proba(\{ \omega_{0} \}) = 0$, or $X(\omega_{0})>0$ donc $\proba(\{ \omega_{0} \})=0$
		      donc
		      $$\proba(X>0) = \sum_{\omega_{0} \in (X>0)} \proba(\{ \omega_{0} \}) = 0$$
		      Donc  $(X>0)$ est négligeable, mais $(X<0)$ est négligeable aussi, donc
		      $$0\leqslant \proba((X>0) \cup (X<0))\leqslant \proba(X>0) + \proba(X<0)=0$$
		      Ainsi l'évènement contraire de $(X>0) \cup (X<0)$, qui est $(X=0)$ est certain.

		\item Supposons $X=0$ presque sûrement.

		      \begin{align*}
			      \esp(X) & = \sum_{\omega \in \Omega}X(\omega)\proba(\{ \omega \})                                                                                                              \\
			              & = \sum_{\omega \in (X = 0)} \underbrace{ X(\omega) }_{ =0 } \proba(\{ \omega \}) + \sum_{\omega \in (X \neq 0)} X(\omega) \underbrace{ \proba(\{ \omega \}) }_{ =0 } \\
			              & = 0
		      \end{align*}

	\end{itemize}
\end{question_kholle}
\begin{question_kholle}{Calcul de l'espérance et la variance d'une variable aléatoire suivant une loi binomiale}
	Soit $n \in \mathbb{N}^*$ et $p \in [0, 1]$.
	Supposons que $X \hookrightarrow \mathcal{B}(n, p)$.

	\begin{align*}
		\esp(X) & = \sum_{\omega \in X(\Omega)}\omega \proba(X = \omega)                   \\
		        & = \sum_{k=0}^{n}k\proba(X=k)                                             \\
		        & = \sum_{k=0}^{n}k \binom{n}{k} p^{k}(1-p)^{n-k}                          \\
		        & = \sum_{k=0}^{n}k \frac{n!}{k! (n-k)!} p^{k}(1-p)^{n-k}                  \\
		        & = n \sum_{k=1}^{n} \frac{(n-1)!}{(k-1)! ((n-1)-(k-1))!} p^{k}(1-p)^{n-k} \\
		        & = n \sum_{k=1}^{n} \binom{n-1}{k-1} p^{k}(1-p)^{n-k}                     \\ \\
		        & = n \sum_{j=0}^{n-1} \binom{n-1}{j} p^{j+1}(1-p)^{n-1-j}                 \\
		        & = n p \sum_{j=0}^{n-1} \binom{n-1}{j} p^{j}(1-p)^{n-1-j}                 \\ \\
		        & = np (p + (1-p))^{n-1} = np
	\end{align*}
	Pour la variance, calculons d'abord $\esp(X^{2})$

	\begin{align*}
		\esp(X^{2}) & = \sum _{k=0}^{n}k^{2} \proba(X=k^{2})                                                                                                                                                                         \\
		            & =  \sum_{k=1}^{n} k \underbrace{ k \binom{n}{k} }_{ n \binom{n-1}{k-1} }p^{k}(1-p)^{n-k}                                                                                                                       \\
		            & = n \sum_{k=1}^{n} \underbrace{ k }_{ (k-1)+1 } \binom{n-1}{k-1} p^{k}(1-p)^{n-k}                                                                                                                              \\
		            & = n \sum_{k=1}^{n}(k-1)\binom{n-1}{k-1}p^{k}(1-p)^{n-k}+ n \sum_{k=1}^{n}\binom{n-1}{k-1}p^{k}(1-p)^{n-k}                                                                                                      \\
		            & = n \sum_{k=2}^{n}\underbrace{ (k-1)\binom{n-1}{k-1} }_{ (n-1)\binom{n-2}{k-2} } p^{k}(1-p)^{n-k} + n \sum_{k=1}^{n}\binom{n-1}{k-1}p^{k}(1-p)^{n-k}                                                           \\
		            & = n(n-1) \underbrace{ \sum_{i=0}^{n-2}\binom{n-2}{i}p^{i+2}(1-p)^{(n-2)-i}  }_{ \text{ en posant } i = k-2 }+ n \underbrace{ \sum_{i=0}^{n-1}\binom{n-1}{i}p^{i+1}(i-p)^{(n-1)-i} }_{ \text{en posant }i=k-1 } \\
		            & = n(n-1)p^{2}(p+(1-p))^{n-2} + np(p+(1-p))^{n-1}                                                                                                                                                               \\
		            & = n(n-1)p^{2}+np                                                                                                                                                                                               \\
		            & = np((n-1)p+1)
	\end{align*}

	D'où,
	$$
		\variance(X) = \esp(X^{2}) - \esp(X)^{2}=np((n-1)p+1)- n^{2} p ^{2}=np(1-p)
	$$

	\noindent \textbf{Calcul alternatif de $\mathbf{E^2}$} En utilisant la formule de transfert pour $f \leftarrow \left( \begin{matrix}
				X(\Omega) = \lient 0; n \rient & \to     & \lient 0; n(n-1) \rient \\
				x                              & \mapsto & x(x-1)
			\end{matrix} \right)$


	\begin{align*}
		\esp(X(X - 1)) & = \sum_{k=0}^{n}k(k-1)\proba(X=k)                               \\
		               & = \sum_{k=0}^{n}k(k-1)\binom{n}{k}p^{k}(1-p)^{n-k}              \\
		               & = n(n-1) \sum_{k=2}^{n}\binom{n-2}{k-2}p^{k}(1-p)^{n-k}         \\
		               & = n(n-1)p^{2} \sum_{j=0}^{n-2} \binom{n-2}{j}p^{j}(1-p)^{n-2-j} \\
		               & = n(n-1) p^{2} (p + (1-p))^{n-2}                                \\
		               & = n(n-1)p^{2}
	\end{align*}

	Donc en remarquant que $$\esp(X^{2}) = \esp(X(X-1)+X) = \esp(X(X-1))+\esp(X)= n(n-1)p^{2} + np$$
	Donc
	$$\variance(X) = \esp(X^{2})- \esp(X)^{2} = n(n-1)p^{2} + np - n^{2}p^{2} = np(1-p)$$


\end{question_kholle}
\end{document}

\documentclass{article}

\date{29 novembre 2023}
\usepackage[nb-sem=9, auteurs={Kylian Boyet, George Ober, Hugo Vangilluwen}]{../kholles}

\begin{document}

\maketitle




\begin{question_kholle}[]{Dans un ensemble totalement ordonné, toute partie finie non vide possède un plus grand élément et un plus petit élément.}
  Soit $(E, \preccurlyeq)$ un ensemble totalement ordonné, considérons pour tour $n \in \mathbb{N}^{*}$ la propriété.
  $$
    \mathcal{H}_{n} : \text{toute partie de }E \text{ de cardinal }n \text{ admet un plus petit et un plus grand élément}
  $$
  \begin{itemize}[label=$\star$]
    \item Initialisation $n \leftarrow 1$

          Soit $A \in \mathcal{P}(E)$ fixée telle que $\lvert A \rvert = 1$
          $A$ est non vide, donc $\exists a \in A : A = \{ a \}$

          $a$ est le plus petit et le plus grand élément, donc $\mathcal{H}_{1}$ est vraie.

    \item Hérédité
          Soit $n \in \mathbb{N}^{*}$ fixé quelconque tel que $\mathcal{H}_{n}$ est vraie.
          Soit $A \in \mathcal{P}(E)$ fixée quelconque tel que $\lvert A \rvert = n+1$
          $$
            A \neq \emptyset \implies \exists a \in A : A = (A \setminus \{ a \}) \cup \{ a \}
          $$
          Or, $\lvert A \setminus \{ a \} \rvert = n$ donc $\mathcal{H}_{n}$ s'applique et $A \setminus \{ a \}$ possède un plus grand et plus petit élément
          $$
            \left\{ \begin{array}{ll}
              m & = \min A \setminus \{ a \} \\
              M & = \max A \setminus \{ a \}
            \end{array}\right.
          $$
          \begin{itemize}[label=$\lozenge$]
            \item Construisons le plus grand élément de $A$
                  \begin{itemize}[label=$\bullet$]
                    \item Supposons $M \preccurlyeq a$
                          D'une part $a \in A$
                          D'autre part
                          $$
                            \forall x \in A, \left. \begin{array}{ll}
                              \text{si }x = a, x \preccurlyeq a \text{ (réflexivité)} \\
                              \text{sinon } x \in A \setminus \{ a \} \implies x \preccurlyeq M \preccurlyeq a \implies x \preccurlyeq a
                            \end{array}\right\} \implies \forall x \in A, x \preccurlyeq a
                          $$

                          Donc $A$ admet un plus grand élément, et c'est $a$.

                    \item Sinon, si $M \succ a$, mais $M \in A$ et
                          $$
                            \forall x \in A, \left. \begin{array}{ll}
                              \text{si }x = a, x \preccurlyeq M \\
                              \text{sinon } x \in A \setminus \{ a \} \implies x \preccurlyeq \max(A\setminus \{ a \}) =  M
                            \end{array}\right\} \implies \forall x \in A, x \preccurlyeq a
                          $$
                          Donc $A$ admet un plus grand élément, et c'est $M$
                  \end{itemize}
            \item On procède de même pour construire le le plus petit élément de $A$ avec $m$.
          \end{itemize}
          Donc $\mathcal{H}_{n+1}$ est vraie.
          Donc toute partie finie non vide d'un ensemble totalement ordonné possède un plus petit et un plus grand élément.
  \end{itemize}
  Étudions l'importance des hypothèses :
  \begin{itemize}[label=$\star$]
    \item Importance de la finitude de la partie :

          On sait qu'une partie infinie d'un ensemble totalement ordonné n'admet pas de plus grand élément : $[0, 1[$ dans $(\mathbb{R}, \leqslant)$, $\mathbb{N}$ dans $(\mathbb{R}, \leqslant)$.
    \item Importance du caractère total de l'ordre : on connait des ensembles finis partiellement ordonnés qui n'ont pas de plus grand élément :
          \begin{itemize}
            \item $\{ 3, 12 \}$ dans $(\mathbb{R}, =)$ n'admet pas de plus grand élément
            \item $\{ [1, 2], [3, 4] \}$ dans $(\mathcal{P}(\mathbb{R}), \subset)$ n'admet pas de plus grand élément
            \item $\{ 2, 3 \}$ dans $(\mathbb{N}, |)$ non plus.
          \end{itemize}
  \end{itemize}
\end{question_kholle}

\begin{question_kholle}
  [\noindent Soit $(E, \leq)$ un ensemble ordonné, et $A$ une partie non-vide de $E$. \\
    Si $A$ admet un plus grand élément alors $A$ admet une borne supérieure et $\sup{A} = \max{A}$. \\
    Si $A$ admet une borne supérieure appartenant à elle-même alors $A$ admet un plus grand élément et $\max{A} = \sup{A}$.]
  {Si $A$ admet un plus grand élément c'est aussi sa borne supérieure. Si $A$ admet une borne supérieure dans $A$ c'est sont plus grand élément.}

  Soient un tel ensemble $E$ et une telle partie $A$ et notons $M$ son plus grand élément. \\
  Posons l'ensemble des majorants de $A$, $M(A) = \{ m\in E \ | \ \forall a \in A, \ a \leq m\}$. \\
  Par définition :
  \[
    \forall m \in M(A), \ M \leq m,
  \]
  car $M\in A$, mais comme $M\in M(A)$, on a directement que $M = \min{M(A)} = \sup{A}$. \\

  Pseudo-réciproquement, soit $A$ une partie de $E$ admettant une borne supérieure dans elle même, notons cette borne $S$. \\
  Comme $S \in M(A)$, par définition, $S$ est plus grand que tous les éléments de $A$ mais appartient à $A$, donc de tous les éléments de $A$, $S$ est le plus grand.
\end{question_kholle}



\begin{question_kholle}[{
        Soit $A \in \mathcal{P}(\mathbb{R})$ une partie non vide et majorée.
        Soit $\sigma \in \mathbb{R}$
        $$
          \sigma = \sup A \iff \left\{ \begin{array}{ll}
            \forall a \in A, a \leqslant \sigma \\
            \forall \varepsilon \in \mathbb{R}_{+}^{*}, \exists a \in A : \sigma - \varepsilon < a \leqslant \sigma
          \end{array}\right.
        $$
      }]{Caractérisation par les $\varepsilon$ de la borne supérieure}

  \begin{itemize}[label = $\star$]
    \item Supposons $\sigma = \sup A$
          \begin{itemize}[label = $\bullet$]
            \item Par définition $\sup A = \min M(A)$ donc $\sigma \in M(A)$ donc $\forall a \in A, a \leqslant \sigma$

            \item Soit $\varepsilon >0$ fixé quelconque

                  \begin{align*}
                    \sigma = \min M(A) & \iff \sigma - \varepsilon \not\in M(A) (\text{ sinon } \sigma - \varepsilon \geqslant \min M(A)= \sigma \implies \varepsilon \leqslant 0) \\
                                       & \iff \exists a \in A: \sigma - \varepsilon < a \leqslant \sigma
                  \end{align*}

          \end{itemize}

    \item Réciproquement, supposons
          $$
            \left\{ \begin{array}{ll}
              \forall a \in A, a \leqslant \sigma \\
              \forall \varepsilon \in \mathbb{R}_{+}^{*}, \exists a \in A : \sigma - \varepsilon < a \leqslant \sigma
            \end{array}\right.
          $$
          \begin{itemize}[label = $\bullet$]
            \item D'après la première propriété, $\sigma \in M(A)$
            \item Montrons que $\sigma$ est le plus petit des minorants par l'absurde en supposant qu'il existe $M \in M(A)$ tel que $M < \sigma$.
                  On a $\sigma - M >0$ donc on peut appliquer la deuxième propriété pour $\varepsilon \leftarrow \sigma - M$
                  $$
                    \exists a \in A : \sigma - (\sigma - M )<a
                  $$
                  Fixons un tel $a$.
                  On a donc trouvé un $a \in A$ tel que $M < a$ ce qui contredit le fait que $M$ soit un majorant de $A$.
                  Donc il n'existe pas de majorant plus petit que $\sigma$.
                  Donc $A$ admet une borne supérieure qui est $\sigma$.
          \end{itemize}
  \end{itemize}
\end{question_kholle}
\begin{question_kholle}[]{Montrer que si $A$ et $B$ sont deux parties non vides majorées de $\R$, alors $\sup(A+B) = \sup A + \sup B$}
  Soient $A$ et $B$ deux parties non vides et majorées de $\mathbb{R}$. On note $A+B$ l'ensemble
  $$
    A+B = \{ a+b \mid (a, b) \in A\times B \}
  $$
  C'est aussi une partie non vide de $\mathbb{R}$.

  Soit $x \in (A+B)$ fixé quelconque. Par définition de $A+B$, $\exists(a, b) \in A\times B : x=a+b$

  $$
    \left. \begin{array}{ll}
      a \leqslant \sup A \\
      b \leqslant \sup B
    \end{array}\right\} \implies x = a+b \leqslant \sup A + \sup B
  $$
  On a donc montré que $\sup A+\sup B$ est un majorant de $A+B$, donc $A+B$ admet un majorant, donc $A+B$ est une partie non vide majorée de $\mathbb{R}$, donc $A+B$ admet une borne supérieure.

  Par définition de la borne supérieure, car $\sup(A+B)$ est le plus petit élément de l'ensemble des majorants :
  $$\sup(A+B) \leqslant \sup A + \sup B$$

  De plus $\sup(A+B)$ est un majorant de $A+B$ donc, pour $(a, b) \in A\times B$ fixés, on a
  $$
    a+b \leqslant \sup (A+B) \iff a \leqslant \sup(A+B) -b
  $$
  en relâchant le caractère fixé de $a$, on a
  $$
    \forall a \in A, a\leqslant \sup(A+B) - b
  $$
  donc $\sup(A+B) - b$ est un majorant de $A$, donc plus petit que $\sup A$, d'où

  $$
    \sup A \leqslant \sup(A+B) - b \iff b \leqslant \sup(A+B) - \sup A
  $$
  Donc en relâchant le caractère fixé de $b$ on a
  $$
    \forall b \in B, b\leqslant \sup(A+B) - \sup A
  $$
  donc $\sup(A+B) - \sup A$ est un majorant de $B$ donc plus petit que $\sup B$
  d'où

  $$
    \sup B \leqslant \sup(A+B) - \sup A \iff \sup A + \sup B \leqslant \sup (A+B)
  $$
  Donc par double inégalité
  $$
    \sup A + \sup B = \sup (A+B)
  $$

\end{question_kholle}
\end{document}

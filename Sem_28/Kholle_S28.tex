\documentclass{article}
\usepackage{braket}

\date{22 juin 2024}
\usepackage[nb-sem=28, auteurs={George Ober}]{../kholles}

\begin{document}
\maketitle

\begin{question_kholle}{Condition nécéssaire de convergence de $\sum_{n \geqslant n_0} u_n$}
    Soit $u \in \K^{[ \! [ n_{0}, + \infty [ \![}$
    Si la série $\sum_{n\geqslant n_{0}}u_{n}$ converge, alors la suite $u$ converge vers $0$.
    
    Supposons que la série converge. Notons $(S_{n})_{n\geqslant n_{0}}$ la suite des sommes partielles.
$$\forall n \in [ \! [ n_{0}+1, +\infty [ \![, u_{n} = S_{n} - S_{n-1}$$
    Puisque $S$ converge, on en déduit que $u$ converge vers $0$.
\end{question_kholle}
\begin{question_kholle}{Condition nécéssaire et suffisante de convergence de $\sum_{n\geqslant 0}q^{n}$ pour $q \in \C^{*}$ et calcul de la somme et du reste lorsqu'ils existent.}
    \begin{itemize}[label=$\star$]
        \item Supposons $\lvert q \rvert<1$
        $$\forall n \in \N, S_{n} = \sum_{k=0}^{n}q^{k} = \frac{1-q^{n+1}}{1-q}$$
        De plus, $\lvert q^{n+1} \rvert=\lvert q \rvert^{n+1}= \left\{ \begin{array}{ll}0 \text{ si }q=0\\ e^{(n+1)\ln \lvert q \rvert} \text{ si }q\neq 0\end{array}\right.$
        
        \begin{align*}
            \left\{ \begin{array}{ll}
                0 \\
                e^{(n+1)\ln \lvert q \rvert } 
            \end{array}\right. \xrightarrow[n\to \infty]{} \left\{ \begin{array}{ll}
                0 \\
                0 
            \end{array}\right.
        \end{align*}
        
        Ainsi, $\sum_{n\geqslant 0}q^{n}$ converge et $\sum_{n=0}^{+\infty}q^{n}=\frac{1}{1-q}$
        $$S_{n} = \frac{1-q^{n+1}}{1-q} \implies R_{n} = S_{n} - \sum_{n=0}^{+\infty}q^{n}= \frac{q^{n+1}}{1-q}$$
        
        \item Si $\lvert q \rvert=1$
        
        $$\forall n \in \N, \lvert q \rvert ^{n} = 1^{n}=1 \xrightarrow[n \to \infty]{}1$$
        
        Ainsi, $\lvert q \rvert^{n}$ ne converge pas vers 0 donc $(q^{n})_{n\geqslant 0}$ ne converge pas vers 0, donc la série est grossièrement divergente.
        
        \item Si $\lvert q \rvert>1$
        $\lvert q \rvert^{n}=\exp(n \ln \lvert q \rvert) \xrightarrow[n\to +\infty]{} +\infty$
        Donc $(q^{n})_{n\geqslant 0}$ ne converge pas vers $0$. Donc la série est grossièrement divergenteê
    \end{itemize}
\end{question_kholle}
\begin{question_kholle}[{Soit $\alpha \in \R$
    La série $\sum_{n\geqslant 1} \frac{1}{n^{\alpha}}$ converge $\iff \alpha >1$
    }]{Caractérisation de la convergence des séries de Riemann}
    \begin{itemize}[label=$\lozenge$]
        \item Supposons $\alpha < 0$ alors, $\frac{1}{n^{\alpha}}=n^{\lvert \alpha \rvert} \xrightarrow[n\to +\infty]{}+\infty$ donc la série est grossièrement divergente.
        \item Supposons $\alpha = 0$ alors $\frac{1}{n^{\alpha}}=1 \xrightarrow[n \to \infty]{} 1$ donc la série est grossièrement divergente.
        \item Supposons $\alpha > 0$
        Cherchons un équivalent de 
$$
        \frac{1}{(n+1)^{\beta}}-\frac{1}{n^{\beta}}
$$
        en fonction de $\beta \in \R^{*}$
        
        \begin{align*}
            \frac{1}{(n+1)^{\beta}} - \frac{1}{n^{\beta}} &= 
            \frac{1}{n^{\beta}}\left( \frac{1}{\left( 1+\frac{1}{n} \right)^{\beta}} - 1 \right) \\
            &= \frac{1}{n^{\beta}}\left[ \left( 1+\frac{1}{n} \right)^{-\beta} - 1\right]  \\
            &\underset{ \overset {\infty} {} } {\sim} \frac{1}{n^{\beta}} \times \left( -\frac{\beta}{n} \right) \\
            &\underset{ \overset {\infty} {} } {\sim} - \frac{\beta}{n^{\beta+1}}
        \end{align*}
        
        \begin{itemize}[label=$\star$]
            \item Pour $\alpha \in ]0, +\infty[ \setminus \left\{ 1 \right\}$
            Appliquons le calcul ci-dessus pour $\beta \leftarrow \alpha -1$ (autorisé car $\alpha \neq 1\implies \alpha-1 \neq 0$)
            $$\frac{1}{(n+1)^{\alpha-1}}-\frac{1}{n^{\alpha-1}} \underset{ \overset {\infty} {} } {\sim}
            -\frac{\alpha-1}{n^{\alpha}}$$
            De plus, $\left( -\frac{\alpha-1}{n^{\alpha}} \right)_{n\geqslant 1}$ est de signe constant donc, d'après le critère d'équivalence, $\sum_{n\geqslant 1} \frac{-(\alpha -1)}{n^{\alpha}}$ est de même nature que la série téléscopique $\sum_{n\geqslant 1}(\frac{1}{(n+1)^{\alpha-1}}-\frac{1}{n^{\alpha-1}} )$. Or, la série téléscopique est de même nature que $\left( \frac{1}{n^{\alpha-1}} \right)_{n\geqslant 1}$.
            
            Donc par transitivité, puisque $\sum_{n\geqslant 1} \frac{1}{n^{\alpha}}$ est de même nature que $\sum_{n\geqslant 1} \frac{-(\alpha-1)}{n^{\alpha}}$, la série de Riemann est de même nature que $\left( \frac{1}{n^{\alpha-1}} \right)_{n\geqslant 1}$
            Or, $\left( \frac{1}{n^{\alpha-1}} \right)_{n\geqslant 1}$ converge pour $\alpha>1$ et diverge pour $\alpha \in ]0, 1[$.
            
            \item Si $\alpha = 1$
            Appliquons la comparaison série intégrale pour $f \leftarrow (x \mapsto \frac{1}{x}) \left\{ \begin{array}{ll} \in \mathcal{C}^{0}([1, +\infty[, \R)\\ \text{décroissante sur }[1, +\infty[ \end{array}\right.$
            
            $$\forall n \in \N^{*}, \int_{1}^{n+1}  \frac{\mathrm{d}u}{u} \leqslant  \sum_{k=1}^{n} \frac{1}{k}$$
            Ainsi,
            $$\forall n \in \N^{*}, \underbrace{ \ln(n+1) }_{ \xrightarrow[n \to +\infty]{} +\infty } \leqslant \sum_{k=1}^{n} \frac{1}{k}$$
            Donc la série diverge.
        \end{itemize}
    \end{itemize}
\end{question_kholle}
\begin{question_kholle}{Comparaison série-intégrale}
    Soit $f:[n_{0}, +\infty[ \to \R$ une fonction continue et décroissante
    Soit $j \in [ \! [ n_{0}, +\infty [\![$ fixé quelconque.
    
    \begin{align*}
        \forall t \in [k, k+1], f(k+1) &\leqslant f(t) \leqslant f(k) \\
        \int_{k}^{k+1} f(k+1) \, \mathrm dt &\leqslant \int_{k}^{k+1} f(t) \, \mathrm dt \leqslant \int_{k}^{k+1}  f(k) \, \mathrm dt  \\
        f(k+1) & \leqslant \int_{k}^{k+1} f(t) \, \mathrm dt \leqslant f(k)
    \end{align*}
    
    Ainsi
    
    \begin{align*}
        \sum_{k=n_{0}}^{n}\int_{k}^{k+1}f(t)  \, \mathrm dt  & \leqslant \sum_{k=n_{0}}^{n}f(t) \\
        \int_{n_{0}}^{n+1} f(t) \, \mathrm dt & \leqslant \sum_{k=n_{0}}^{n}f(k)
    \end{align*}
    
    
    De même,
    
    \begin{align*}
        \forall k \in [ \! [ n_{0}+1 , +\infty[ \![, f(k) & \leqslant\int_{k-1}^{k} f(t) \, \mathrm dt  \\
        \sum_{k=n_{0}+1}^{n}f(k)  & \leqslant \sum_{k=n_{0}+1}^{n}\int_{k-1}^{k} f(t) \, \mathrm dt \\
        \sum_{k=n_{0}}^{n} f(k)&\leqslant f(n_{0})+ \int_{n_{0}}^{n} f(t) \, \mathrm dt 
    \end{align*}
    
    D'où l'encadrement.
\end{question_kholle}
\begin{question_kholle}[{Soit $n_{0} \in \N$ et $f:[n_{0}, +\infty[ \to \R$ une fonction continue, décroissante et minorée par $m \in \R$
    Alors la série de terme général
    $$\left( f(n)- \int_{n}^{n+1} f(u) \, \mathrm du  \right)_{n\geqslant n_{0}}$$
    est à termes positifs ou nuls et converge.}]{Pour $f$ continue sur $[n_0, +\infty[$, décroissante et minorée, $\sum_{n\geqslant n_{0}}\left( f(n)- \int_{n}^{n+1} f(u) \, \mathrm du  \right)$}
    
    
    Montrons que la suite $(S_{n})_{n\geqslant n_{0}}$ est majorée, et que la suite est à termes $\geqslant 0$
    La décroissance de $f$ donne l'encadrement suivant
    
    $$\forall n \in [ \! [ n_{0} , +\infty[\![,  f(n) - \int_{n}^{n+1} f(t) \, \mathrm dt \geqslant 0$$
    
    La comparaison série intégrale s'applique donc à $f$ qui est décroissante et continue et donne
    
    \begin{align*}
        \forall n \in [ \! [ n_{0}, +\infty [ \![ f(n+1)\leqslant \int_{n}^{n+1} f(t) \, \mathrm dt \leqslant f(n) &\implies -f(n+1) \geqslant -\int_{n}^{n+1} f(t) \, \mathrm dt \\
        &\implies f(n) - f(n+1) \geqslant f(n) - \int_{ n}^{n+1} f(t) \, \mathrm dt \geqslant 0
    \end{align*}
    
    En sommant sur $k \in [ \! [ n_{0}, n ] \!]$
    
    $$\sum_{k=n_{0}}^{n} (f(k) - f(k+1)) \geqslant  \sum_{k=n_{0}}^{n}\left( f(k) - \int_{k}^{k+1} f(t) \, \mathrm dt  \right) = S_{n}$$
    
    En reconnaissant un phénomène téléscopique
    
    $$S_{n} \leqslant f(n_{0})-f(n+1)\leqslant f(n_{0})-n$$
    
    Donc $(S_{n}) n\geqslant n_{0}$ est majorée, et croissante, elle converge donc.
\end{question_kholle}
\begin{question_kholle}[{Soit $(a_{n})_{n\geqslant n_{0}} \in \R^{[ \! [ n_{0}, +\infty [ \![}$ une suite réelle.
    Si
    $$
    \left\{ \begin{array}{ll}
        \forall n \in [ \! [ n_{0}, +\infty [\![, a_{n} \geqslant 0 \\
        (a_{n})_{n\geqslant n_{0}} \text{ est décroissante} \\
        \lim_{ n \to \infty } a_{n}=0
    \end{array}\right.
    $$
    alors $\sum_{n\geqslant n_{0}}(-1)^{n}a_{n}$}]{Théorème des séries alternées}
    
    \begin{itemize}[label=$\lozenge$]
        \item Traitons le cas $n_{0}\equiv 0 [2]$ il existe $p_{0} \in \N: n_{0}=2p_{0}$
        \begin{itemize}[label=$\star$]
            \item  Les suites $(S_{2p})_{p\geqslant p_{0}}$ et $(S_{2p+1})_{p\geqslant p_{0}}$ sont adjacentes:
            
            \begin{align*}
                \forall p \in [ \! [ p_{0}, +\infty[ \![, S_{2(p+1)} - S_{2p} &= S_{2p+2}- S_{2p} \\
                &= \sum_{k=2p_{0}}^{2p+2}(-1)^{k}a_{k} + \sum_{k=2p_{0}}^{2p}(-1)^{k}a_{k} \\
                &= -a_{2p+1}+a_{2p+2} \leqslant 0 \text{ car }a\downarrow
            \end{align*}
            
            \begin{align*}
                S_{2(p+1)+1} - S_{2p+1} &= S_{2p+3} - S_{2p+1} = (-1)^{2p+2}a_{2p+2}+(-1)^{2p+3}a_{2p+3} = a_{2p+2}- a_{2p+3} \geqslant 0 \text{ car }a\downarrow
            \end{align*}
            
            
            Donc $(S_{2p})$ est décroissante et $(S_{2p+1})$ est croissante.
            De plus
$$
            S_{2p+1} - S_{2p} = (-1)^{2p+2}a_{2p+1} = \underbrace{ -a_{2p+1} }_{ \xrightarrow[p \to \infty]{} 0 } \leqslant 0 \text{ car a positive}
$$
            Ainsi $(S_{2p})_{p\geqslant p_{0}}$ et $(S_{2p+1})_{p\geqslant p_{0}}$ sont adjacentes.
            \item Donc d'après le théorème des suites adjacentes, $(S_{2p})$ et $(S_{2p+1})$ convergent vers une même limite $\ell$, si bien que $(S_{n})$ converge vers $\ell$.
            
            \item De plus, les suites $(S_{2p})_{p\geqslant p_{0}}$ et $(S_{2p+1})_{p\geqslant p_{0}}$ étant adjacentes, pour $n \geqslant n_{0}$ posons $R_{n} = \ell -S_{n}$
            \begin{itemize}
                \item Si $n \equiv 0 [2]$, $\exists p \in [ \! [ p_{0}, +\infty [ \![:n=2p$ donc, puisque $(S_{2p})$ est décroissante et $(S_{2p+1})$ est croissante, on a 
$$
                S_{2p+1}\leqslant \ell \leqslant S_{2p} \implies S_{2p+1} - S_{2p} \leqslant \ell - S_{2p} \leqslant 0 \implies \lvert R_{2p} \rvert = \lvert \ell - S_{2p} \rvert \leqslant a_{2p+1}
$$
                
                \item Si $n \equiv 1 [2]$ $\exists p \in [ \! [ p_{0}, +\infty [ \![:n=2p+1$
$$
                S_{2p+1} \leqslant \ell \leqslant S_{2p+2} \implies 0 \leqslant \ell - S_{2p+1} \leqslant S_{2p+2} - S_{2p+1} = (-1)^{2p+2}a_{2p+2}= a_{2p+2}
$$
                donc $\lvert R_{2p+1} \rvert = \lvert \ell-S_{2p+1} \rvert \leqslant a_{2p+2}$
            \end{itemize}
        \end{itemize}
        Bonus, par croissance de $(S_{2p+1})$ qui converge vers $\ell$, $S_{2p+1} \leqslant \ell$  donc $a_{2p_{0}}-a_{2p_{0} +1}\leqslant\ell$
        Donc $\ell \geqslant 0$ qui est bien le signe du premier terme de la série $(-1)^{n_{0}}a_{n_{0}}$ car $n_{0}\equiv 0[2]$.
        
        \item Le cas $n_0 \equiv 1 [2]$ se traite de la même manière
    \end{itemize}
\end{question_kholle}
\begin{question_kholle}[{Soit $u \in \K^{[ \! [ n_{0}, +\infty [ \![}$
    Si la série $\sum_{n\geqslant n_{0}}u_{n}$ est absolument convergente, alors la série $\sum_{n\geqslant n_{0}}u_{n}$ est convergente.}]{L'absolue convergence implique la convergence}
    
    
    
    \begin{itemize}[label=$\lozenge$]
        \item Supposons que $u$ est le terme général réel d'une série absolument convergente.
        Posons, pour tout $n \in [ \! [ n_{0}, +\infty [ \![$, $u_{n}^{+}= \max(u_{n}, 0)$ et $u_{n}^{-}=- \min (u_{n}, 0)$
        Avec ces notations, $u_{n}^{+}- u_{n}^{-} = u_{n}$ et $u_{n}^{+}+u_{n}^{-} = \lvert u_{n} \rvert$.
$$
        \forall n \in [ \! [ n_{0}, +\infty [ \![, u_{n}^{+}\geqslant 0 \text{ et } u_{n}^{-}\geqslant 0
$$
        
$$
        \left. \begin{array}{ll}
            \forall n \geqslant n_{0}, 0\leqslant u_{n}^{+} \leqslant \lvert u_{n} \rvert = u_{n}^{+}+ u_{n}^{-} \\
            \sum_{n\geqslant n_{0}} u_{n} \text{ est ACV} \implies \sum_{n\geqslant n_{0}} \lvert u_{n} \rvert  \text{ CV} \\
            \forall n \geqslant n_{0}, u_{n}^{+}\geqslant 0 \text{ et } \lvert u_{n} \rvert \geqslant 0
        \end{array}\right\}\implies \sum_{n\geqslant n_{0}}u_{n} ^{+}\text{ converge}
$$
        
        On montre de même que $\sum_{n\geqslant n_{0}}u_{n}^{-}$ converge, donc, par structure vectorielle de l'ensemble des termes généraux de suites convergentes, $\sum_{ n\geqslant n_{0}}(u_{n}^{+} - u_{n}^{-})= \sum_{n\geqslant n_{0}}u_{n}$ converge
        
        \item Cas d'une série complexe,

        Posons, $\forall n\geqslant n_{0}, x_{n} = \mathrm{Re}(u_{n})$ et $y_{n} = \mathrm{Im}(u_{n})$
        Alors,
$$
        \left. \begin{array}{ll}
            \forall n \geqslant n_{0}, \lvert x_{n} \rvert \leqslant \lvert \mathrm{Re}(u_{n}) \rvert \leqslant \lvert u_{n} \rvert  \\
            \forall n\geqslant n_{0}, \lvert  x_{n} \rvert \geqslant 0 \text{ et } \lvert u_{n} \rvert \geqslant 0 \\
            \sum_{n\geqslant n_{0}}u_{n} \text{ ACV } \implies \sum_{n\geqslant n_{0}}\lvert u_{n} \rvert \text{ CV }
        \end{array}\right\}\implies \sum_{n\geqslant n_{0}}\lvert x_{n} \rvert \text{ converge}
$$
        Donc d'après le cas réel, $\sum_{n\geqslant n_{0}} x_{n}$ converge
        On montre de même que $\sum_{n\geqslant n_{0}}y_{n}$ converge
        Donc, par structure vectorielle, $\sum_{n\geqslant n_{0}}(x_{n}+ iy_{n})$ converge.
        Donc $u_{n}$ est le terme général d'une série convergente.
    \end{itemize}
\end{question_kholle}
\end{document}

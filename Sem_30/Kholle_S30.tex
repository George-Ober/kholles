\documentclass{article}
\usepackage{braket}

\date{22 juin 2024}
\usepackage[nb-sem=30, auteurs={Hugo Vangilluwen, George Ober}]{../kholles}

\begin{document}
\maketitle

\begin{question_kholle}[{
        \begin{equation}
          \forall (x, y) \in E^2, \
          \left| \braket{ x | y } \right| \leqslant \|x\|\|y\|
        \end{equation}
        Il y a égalité \ssi $x$ et $y$ sont liés.
      }]{Inégalité de Cauchy-Schwartz dans un espace préhilbertien réel, cas d'égalité}
  Soit $E$ un $\R$-espace vectoriel, et $\braket{ \cdot | \cdot  }$ un produit scalaire sur $E$.
  Soient $(x, y) \in E^{2}$
  \begin{enumerate}
    \item \begin{itemize}[label=$\star$]
            \item Si $y=0$, l'inégalité est une égalité et est évidente
            \item Sinon, posons
                  $$
                    P:\left|\begin{array}{ll} \R &\to \R \\ t &\mapsto \braket{ x+t.y | x+t.y } = t^{2} \|y\|^{2} + 2 t \braket{ x | y }  + \|x\|^{2} \end{array}\right.
                  $$
                  Puisque $\|y\|^{2} \neq 0$, P est un polynôme de degré $2$ à coefficientts réels et positif d'après le caractère positif du produit scalaire (on a donc $\forall t \in \R, P(t)\geqslant 0$)
                  Le discriminant de cette fonction polynômiale est $\Delta = 4 \braket{ x | y }^{2} - 4 \|x\|^{2}\|y\|^{2}$, qui est obligatoirement négatif ou nul puisque $P$ admet au mieux une racine double.
                  Donc $\braket{ x | y }^{2} - \|x\|^{2}\|y\|^{2} \leqslant 0$ donc en prenant la racine carrée $|\braket{ x | y }| \leqslant \|x\|\|y\|$.
          \end{itemize}
    \item \begin{itemize}[label=$\star$]
            \item Supposons que $(x, y)$ est liée, sans perte de généralité, supposons $y = \lambda.x$ alors
                  $$
                    \lvert \braket{ x | \lambda.x } \rvert  = \lvert  \lambda \rvert  \braket{ x | x }  = \lvert  \lambda \rvert  \|x\|^{2} = \| x \| \| \lambda.x\|
                  $$
                  Donc l'inégalité est une égalité.

            \item Réciproquement, supposons que $\lvert  \braket{ x | y } \rvert = \| x\| \| y\|$
                  \begin{itemize}
                    \item Si $y = 0$ alors $(x, y)$ est liée
                    \item Sinon, $\Delta = 4(\braket{ x | y }^{2} - \|x\|\|y\|) = 0$
                          $P$ est un polynôme de degré $2$ de discriminant nul : il admet une racine double $\lambda$
                          Ainsi
                          $$
                            P(\lambda) = 0 \implies \braket{ x+\lambda.y | x+\lambda.y } = 0
                          $$
                          Donc $x+\lambda .y = 0_{E}$ d'après le caractère défini du produit scalaire.
                  \end{itemize}
          \end{itemize}
  \end{enumerate}
\end{question_kholle}

\begin{question_kholle}[{
        L'application
        \begin{equation}
          \chi \left|\begin{array}{ll}
            E & \to E^{*}                                                      \\
            x & \mapsto \left(\begin{array}{ll} E & \to \R
             \\ y &\mapsto \braket{ x | y }\end{array}\right)\end{array}\right.
        \end{equation}
        est un isomorphisme d'espaces vectoriels. $\chi$ est appelé l'isomorphisme canonique entre un espace vectoriel euclidien et son espace dual.
      }]{Isomorphisme entre un espace euclidien et l'espace de ses formes linéaires (Théorème de représentation de Riesz)}
  \begin{itemize}[label=$\star$]
    \item $\chi$ est bien définie car, $\forall x \in E$, par linéarité du produit scalaire en sa seconde variable, $\chi(x): \left|\begin{array}{ll} E &\to \R \\ y &\mapsto \braket{ x | y }  \end{array}\right.$ est une forme linaire sur $E$.


    \item Soient $(x, x') \in E^{2}$ et $\lambda \in \R$ fixés quelconques


          \begin{align*}
            \forall y \in E,  \chi(x + \lambda .x')(y) & = \braket{ x+\lambda.x' | y }                        \\
                                                       & = \braket{ x | y } + \lambda \times\braket{ x' | y } \\
                                                       & = \chi(x)(y) + \lambda \times \chi(x')(y)            \\
                                                       & = (\chi(x) + \lambda . \chi(x') )(y)
          \end{align*}

          Donc $\chi(x+\lambda x') = \chi(x) + \lambda.\chi(x')$, donc $\chi$ est linéaire.

    \item Soit $x \in \ker \chi$ fixé quelconque.
          Alors $\chi (x) = 0_{E^{*}}$
          $$\forall y \in E, \braket{ x | y } = 0$$
          Donc $x \in E^{\perp} = \{ 0_{E} \}$ donc $x = 0_{E}$
          Donc $\chi$ est injective, or $E$ et $E^{*}$ sont de même dimension, donc $\chi$ est bijective.
          Donc $\chi$ est un isomorphisme.
  \end{itemize}
\end{question_kholle}

\begin{question_kholle}{Si $F$ est un sous-espace vectoriel de dimension finie d'un espace préhilbertien réel, $F^{\perp}$ est son supplémentaire orthogonal}


  Soient $(E, \braket{ \cdot | \cdot })$ un espace préhilbertien réel, et $F$ un sous espace vectoriel de dimension finie.
  Alors $F$ et $F^{\perp}$ sont supplémentaires orthogonaux, i.e.
  \begin{equation}
    E = F  \overset{\perp}{\oplus} F^{\perp}
  \end{equation}



  En notant $r = \dim F$, fixons une base orthonormale $(e_{1}, \dots, e_{r})$ de $F$, possible car $F$ est un espace euclidien (dimension finie et muni du produit scalaire induit par $E$).
  \begin{itemize}[label=$\lozenge$]
    \item \underline{Analyse}

          Soit $x \in E$ fixé quelconque, supposons que $\exists (x_{\sslash}, x_{\perp}) \in F \times F^{\perp} = x_{\sslash}+x_{\perp}$
          D'abord $$x_{\sslash} \in F \implies \exists (\lambda_{1}, \dots, \lambda_{r}) \in \R^{r}: x_{\sslash}= \sum_{i=1}^{r}\lambda_{i}.e_{i}$$
          Soit $j \in [ \! [ 1, r ] \!]$ fixé quelconque

          \begin{align*}
            \braket{ x | e_{j} } & =   \left\langle \sum_{i=1}^{r}\lambda_{i}.e_{i} + x_{\perp} \Bigg| e_{j}  \right\rangle                                                                                                                 \\
                                 & = \sum_{i=1}^{r}\lambda_{i}\times \underbrace{ \braket{ e_{i} | e_{j} } }_{ \delta_{ij} }  + \overbrace{ \braket{ \underbrace{ x_{\perp} }_{ \in F^{\perp} } | \underbrace{ e_{j} }_{ \in F } } }^{ =0 } \\
                                 & = \lambda_{j}
          \end{align*}

          Ainsi, $$\left\{ \begin{array}{ll}
              x_{\sslash} & = \sum_{i=1}^{r}\lambda_{i}.e_{i}= \sum_{i=1}^{r}\braket{ x | e_{i} } . e_{i} \\
              x_{\perp}   & = x - x_{\sslash}
            \end{array}\right.$$

    \item \underline{Synthèse}

          Posons donc
          $$\left\{ \begin{array}{ll}
              x_{\sslash} & = \sum_{i=1}^{r} \braket{ x | e_{i} } .e_{i} \\
              x_{\perp}   & = x - x_{\sslash}
            \end{array}\right. $$
          \begin{itemize}[label=$\star$]
            \item $(e_{1}, \dots, e_{r})$ est une base de $F$ donc $x_{\sslash} \in F$
            \item $x_{\sslash}+x_{\perp} = x_{\sslash}+ (x - x_{\sslash}) = x$
            \item Soit $j \in [ \! [ 1, r ] \!]$ fixé quelconque. Calculons $\braket{ x_{\perp} | e_{j} }$

                  \begin{align*}
                    \braket{ x_{\perp} | e_{j} } & = \braket{ x | e_{j} }  - \left\langle  \sum_{i=0}^{r}\braket{ x | e_{i} } .e_{i} \Bigg|  e_{j} \right\rangle                                                         \\
                                                 & = \left\langle x | e_{j} \right\rangle - \sum_{i=0}^{r} \left\langle x | e_{i} \right\rangle\underbrace{  \left\langle e_{i} | e_{j} \right\rangle  }_{ \delta_{ij} } \\
                                                 & = \left\langle x | e_{j} \right\rangle  - \left\langle x | e_{j} \right\rangle  = 0
                  \end{align*}


                  Donc $x_{\perp} \in \{e_{1}, \dots, e_{r}\}^{\perp}$
                  Donc $x_{\perp} \in \text{Vect}\{ e_{1}, \dots, e_{r} \}^{\perp} = F^{\perp}$

          \end{itemize}
  \end{itemize}
  Ainsi, $F$ et $F^{\perp}$ sont supplémentaires orthogonaux.

  De plus
  $$	\forall x \in E, x = \underbrace{ \sum_{i=1}^{r}\left\langle x | e_{i} \right\rangle .e_{i} }_{ \in F } + \underbrace{x - \sum_{i=1}^{r}\left\langle x | e_{i} \right\rangle .e_{i}  }_{ \in F^{\perp} }$$
  Donc $$p_{F}^{\perp}(x) = \sum_{i=1}^{r}\left\langle x | e_{i} \right\rangle .e_{i} $$

\end{question_kholle}

\begin{question_kholle}[{On utilisera le produit scalaire $$
          \left\langle P | Q \right\rangle  = \int_{0}^{1} P(u)Q(u) \, \mathrm du
        $$ }]{Orthonormalisation de la base canonique de $\R_2[X]$}
  Partons de la base canonique de $\R_{2}[X]$.
  \begin{itemize}[label=$\star$]
    \item $P_{1} =X^{0}$ est un vecteur unitaire avec ce produit scalaire

    \item Calcul du second vecteur
          $$
            P_{2}' = X - \left\langle X | 1 \right\rangle.1 = X - \left( \int_{0}^{1} u \, \mathrm du \right) .1 = X-\frac{1}{2}
          $$

          $$
            P_{2} = \frac{P_{2}'}{\|P_{2}'\|} = \frac{P_{2}'}{\sqrt{ \left\langle P_{2}' | P_{2}' \right\rangle  }}= \frac{P_{2}'}{\sqrt{ \int_{0}^{1} \left( u-\frac{1}{2} \right)^{2} \, \mathrm du }}
            = \frac{P_{2}'}{\sqrt{ \frac{1}{12} }}= \sqrt{ 12 }P_{2}'
          $$
          Ce qui donne
          $$
            P_{2}' = 2\sqrt{ 3 }X - \sqrt{ 3 }
          $$

    \item Enfin,

          \begin{align*}
            P_{3}' & = X^{2} - \left\langle X^{2} | 2\sqrt{ 3 }X-\sqrt{ 3 } \right\rangle.(2\sqrt{ 3 }X-\sqrt{ 3 })  - \left\langle X^{2} | 1 \right\rangle .1                            \\
                   & = X^{2} - \left( \int_{0}^{1} 2\sqrt{ 3 }u^{3} - \sqrt{ 3 }u^{2} \, \mathrm du  \right).(2\sqrt{ 3 }X-\sqrt{ 3 })-\left( \int_{0}^{1} u^{2} \, \mathrm du  \right).1 \\
                   & = X^{2} - \frac{\sqrt{ 3 }}{6}(2\sqrt{ 3 }X - \sqrt{ 3 }) - \frac{1}{3}                                                                                              \\ \\
                   & =X^{2} -X + \frac{1}{6}
          \end{align*}


          $$
            P_{3} = \frac{P'_{3}}{\|P_{3}'\|}= \frac{P'_{3}}{\sqrt{ \left\langle P_{3} | P_{3} \right\rangle  }}= \frac{P'_{3}}{\sqrt{ \int_{0}^{1} \left( u^{2}-u +\frac{1}{6} \right)^{2} \, \mathrm du }}
            = \frac{P_{3}'}{\sqrt{ \frac{1}{180} }} =6\sqrt{ 5 }P'_{3} = 6\sqrt{ 5 }\left( X^{2}-X + \frac{1}{6} \right)
          $$

          Donc une base orthonormée de $\R_{2}[X]$ muni de ce produit scalaire est
          $$
            \left( 1, \ 2\sqrt{ 3 }X-\sqrt{ 3 }, \ 6\sqrt{ 5 }\left( X^{2}-X+\frac{1}{6} \right) \right)
          $$
  \end{itemize}
\end{question_kholle}

\begin{question_kholle}[{Soit $(E, \left\langle \cdot | \cdot \right\rangle)$ un espace préhilbertien Réel.
        Soient $F$ un sous espace vectoriel de dimension finie de $E$, et $x \in E$.

        L'ensemble $\{ \|x-z\| \mid z \in F \}$ admet une borne inférieure appelée distance de $x$ à $F$ et notée $d(x, F)$, qui est un plus petit élément, atteinte uniquement pour pour $z = p_{F}^{\perp}(x)$
      }]{Distance d'un vecteur à un sous-espace vectoriel de dimension finie}

  $\{ \|x - z\| \mid z \in F \}$ est une partie de $\R$, non vide car elle contient $\|x\|$ pour $z \leftarrow 0_{F}$ d'éléments positifs ou nuls. Elle admet donc une borne inférieure

  $E$ est un espace euclidien, donc $E = F \overset{\perp}{\oplus} F^{\perp}$ donc $x$ se décompose selon ces supplémentaires orthogonaux
  $$x = \underbrace{ p_{F}^{\perp}(x) }_{ \in F } + \underbrace{ x - p_{F}^{\perp}(x) }_{ \in F^{\perp} }$$
  si bien que, pour tout $z \in F$

  \begin{align*}
    \|x - z\|^{2} & =  \|p_{F}^{\perp}(x) - z + x - p_{F}^{\perp}(x)\|^{2}                                                  \\
                  & = \|p_{F}^{\perp}(x) - z \|^{2} + \|x - p_{F}^{\perp}(x)\|^{2} \text{ d'après le théorème de Pythagore} \\
                  & \geqslant \|x - p_{F}^{\perp}(x)\|^{2}
  \end{align*}

  En prenant la racine carrée,
  $$\forall z \in F, \| x  - z\| \geqslant \|x - p_{F}^{\perp }(x)\|$$
  D'où $\| x - p_{F}^{\perp}(x)\|$ minore $\{ \| x - z\| \mid z \in F \}$ et donc sa borne inférieure.

  Or, en remonant le calcul précédent, il y a égalité pour $z = p_{F}^{\perp}(x)$ si bien que la borne inférieure est un plus petit élément, et vaut $d(x, F) = \|x-p_{F}^{\perp}(x)\|$

  De plus, si $z' \in F$ atteint ce plus petit élément on a

  \begin{align*}
    \| x- z'\|^{2}                & = \|p_{F}^{\perp}(x) - z' + x - p_{F}^{\perp}(x)\|^{2}          \\
    \| x - p_{F}^{\perp}(x)\|^{2} & = \|p_{F}^{\perp}(x) - z' \|^{2} + \|x - p_{F}^{\perp}(x)\|^{2} \\
    0                             & = \|p_{F}^{\perp}(x) - z' \|^{2}
  \end{align*}

  Si bien que $p_{F}^{\perp}(x) - z' = 0_{E}$ d'après le caractère défini du produit scalaire.
  Donc le plus petit élément $d(x, F) = \min\{ \|x-z\| \mid z \in F \}$ est uniquement atteint pour $z = p_{F}^{\perp}(x)$.
\end{question_kholle}

\begin{question_kholle}[{
  Soit $(E, \left\langle \cdot | \cdot \right\rangle)$ un espace vectoriel euclidien, $\mathcal{B}=(e_{1}, \dots, e_{n})$ une base orthonormée de $E$.
  Soit $u = \sum_{i=1}^{n}u_{i}.e_{i}$ un vecteur de $E$.
  Soient $(a_{1}, \dots, a_{n}) \in\R^{n} \setminus \{ 0_{\R^{n}} \}$, $\alpha \in \R$ et $H_{\alpha}$ l'hyperplan affine d'équation
  $$
    \sum_{i=1}^{n}a_{i}x_{i}=\alpha
  $$
  }]{Distance à un sous-espace affine}
  Posons $a = \sum_{i=1}^{n}a_{i}.e_{i}$ $H_{0}$ est un hyperplan vectoriel et, $H_{0} = a^{\perp}$ et $H_{0}^{\perp}= \text{Vect}\{ a \}$
  Introduisons $h_{\alpha} \in a^{\perp}$ tel que $H_{\alpha}= h_{\alpha}+H_{0}$ et souvenons nous que $h_{\alpha}=\frac{\alpha}{\| a\|^{2}}.a$

  Observons que l'égalité $H_{\alpha}=h_{\alpha}+H_{0}$ donne
  $$
    \{ \|u-z\| \mid z \in H_{\alpha} \} = \{ \|u-(h_{\alpha}+z')\| \mid z' \in H_{0} \}= \{ \|(u - h_{\alpha})-z'\| \mid z' \in H_{0}\}
  $$

  or, d'après la caractérisation de la distance à un sous-espace quelconque, on a
  \begin{itemize}[label=$\star$]
    \item L'ensemble $\{ \|(u-h_{\alpha}) -z'\| \mid z' \in H_{0} \}$ admet une borne inférieure donc $\{ \| u - z\| \mid z \in H_{\alpha} \}$ aussi qui vaut $\mathrm{d}(u-h_{\alpha}, H_{0})$, ce qui prouve que $\mathrm{d}(u, H_{\alpha})$ est bien définie

    \item $\inf \{  \| (u - h_{\alpha}) - z'\| \mid z' \in H_{0} \}$ est un plus petit élément atteint pour l'unique valeur $z' = p_{H_{0}}^{\perp}(u-h_{\alpha})=p_{H_{0}}^{\perp}(u)$ car $h_{\alpha} \in H_{0}^{\perp}= \ker p_{H_{0}}^{\perp}$, donc $\mathrm{d(u, H_{\alpha})}=\inf \{  \|u-z\| \mid z \in H_{\alpha} \}$ est un plus petit élément atteint pour l'unique valeur $z = h_{\alpha}+p_{H_{0}}^{\perp}(u-h_{\alpha})= h_{\alpha}+ p_{H_{0}}^{\perp}(u)$
          $$
            \mathrm{d}(u, H_{\alpha})= \| u - h_{\alpha}-p_{H_{0}}^{\perp}(u)\|
          $$
  \end{itemize}
  Or $u - p_{H_{0}}^{\perp}(u) = (\mathrm{Id}-p_{H_{0}}^{\perp})(u) = p_{H_{0}^{\perp}}^{\perp}(u) = \left\langle u | \frac{a}{\|a\|} \right\rangle. \frac{a}{\|a\|}$ car $H_{0}^{\perp} = \text{Vect}\{ a \}$ d'où, sachant aussi que $h_{\alpha}= \frac{\alpha}{\|a\|^{2}}.a$
  $$
    \mathrm{d}(u, H_{\alpha})= \|p_{H_{0}^{\perp}}^{\perp}(u)-h_{\alpha}\|= \left\| \left\langle a | \frac{a}{\|a\|} \right\rangle. \frac{a}{\|a\|}- \frac{\alpha}{\|a\|^{2}}.a \right\|
  $$
\end{question_kholle}

\begin{question_kholle}
  {Dénombrement des surjections de $\lient 1; n \rient$ dans $\lient 1; 2 \rient$ et dans $\lient 1; 3 \rient$}

  Soit $n \in \N^*$.

  Il y a ${\left| \lient 1; 2 \rient \right|} ^ {\left| \lient 1; n \rient \right|} = 2^n$ applications de $\lient 1; n \rient$ dans $\lient 1; 2 \rient$.
  Seules les applications constantes $\widetilde{1}$ et $\widetilde{2}$ ne sont pas surjectives.
  Il y a donc $2^n - 2$ surjections de $\lient 1; n \rient$ dans $\lient 1; 2 \rient$.

  Il y a ${\left| \lient 1; 3 \rient \right|} ^ {\left| \lient 1; n \rient \right|} = 3^n$ applications de $\lient 1; n \rient$ dans $\lient 1; 3 \rient$. Les applications non surjectives sont celles dont l'image n'est pas $\lient 1; 3 \rient$. C'est-à-dire, celles dont l'image est de cardinal 1 (les fonctions constantes $\widetilde{1}$, $\widetilde{2}$ et $\widetilde{3}$) et celles dont l'image est de cardinal 2. Ces dernières sont les surjections de $\lient 1; n \rient$ dans $\lient 1; 2 \rient$, $\{ 1; 3 \}$ et $\{ 2; 3 \}$. Comme ces trois ensembles ont la même taille, il y a $3 \times (2^n - 2)$ (voir résultat précédent) applications de $\lient 1; n \rient$ dans $\lient 1; 3 \rient$ dont l'image est de cardinal 2. Ainsi, le nombre de surjections de $\lient 1; n \rient$ dans $\lient 1; 3 \rient$ est $3^n - 3 - 3(2^n -2) = 3^n - 3 \times 2^n + 3$.
\end{question_kholle}

\begin{question_kholle}
  [Soient $E, F$ deux ensembles finis non vides et $f : E \rightarrow F$ telle que tout élément de $F$ possède le même nombre $k \in \N^*$ d'antécédents par $f$.
    Alors $\left|F\right| = \frac{\left|E\right|}{k}$
    \begin{quotation}
      \textquotedblleft Pour compter les moutons, il faut compter les pattes puis diviser par quatre. \textquotedblright
    \end{quotation}]
  {Lemme des bergers}

  Considérons la relation binaire définie sur $E$ par :
  \begin{equation*}
    \forall (x, y) \in E^2, x \sim y
    \iff f(x) = f(y)
  \end{equation*}
  Elle est réflexive, transitive et symétrique donc c'est bien une relation d'équivalence.
  Donc les classes d'équivalence réalise une partition de $E$.
  Nous avons $\displaystyle E = \bigsqcup_{C \in \nicefrac{E}{\sim}} C$ donc, en passant aux cardinaux, $\displaystyle \left|E\right| = \sum_{C \in \nicefrac{E}{\sim}} \left|C\right|$.

  Soit $x \in E$ \fq. Alors $\bar{x} = \left\{ y \in E \;|\; f(x) = f(y) \right\} = f^{-1}(f(\{x\}))$. Par hypothèse, tous les éléments de $F$ ont le même nombre $k$ d'antécédents, or $f({x})$ est un singleton d'élément de $F$ donc $\left|\bar{x}\right| = k$.
  Ainsi $\forall C \in \nicefrac{E}{\sim}, \left|C\right| = k$.

  Posons $\varphi \left|\begin{array}{ccc}
      \nicefrac{E}{\sim} & \mapsto     & F                        \\
      C                  & \rightarrow & f(x) \text{ où } x \in C
    \end{array}\right.$.
  $\varphi$ est bien défini car si $(x, y) \in E$ vérifie $\bar{x} = \bar{y}$ alors $f(x) = f(y)$ donc l'image par $\varphi$ ne dépend pas du représentant de classe choisi.
  $\varphi$ est surjective car soit $z \in F$, $f$ est surjective donc $\exists x_z \in E : f(x_z) = z$ et alors $\varphi(\bar{x_z}) = f(x_z) = z$.
  $\varphi$ est injective car soient $(C, C') \in \left( \nicefrac{E}{\sim} \right)^2, \varphi(C) = \varphi(C')$ alors $\exists (x, x') \in C \times C' : x \sim x'$, comme deux classes d'équivalence sont confondues ou disjointes, $C = C'$.
  Ainsi $\varphi$ est une bijection donc $\left|F\right| = \left|\nicefrac{E}{\sim}\right|$.

  Ainsi $\displaystyle \left|E\right| = \sum_{C \in \nicefrac{E}{\sim}} \left|C\right| = \sum_{C \in \nicefrac{E}{\sim}} k = \left| \nicefrac{E}{\sim} \right| k = \left|F\right| k$.
\end{question_kholle}
\end{document}

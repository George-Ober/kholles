\documentclass{article}

\date{2 juin 2024}
\usepackage[nb-sem=30, auteurs={Hugo Vangilluwen}]{../kholles}

\begin{document}
	\maketitle

	\setnbquestion{7}
	\begin{question_kholle}
		{Dénombrement des surjections de $\lient 1; n \rient$ dans $\lient 1; 2 \rient$ et dans $\lient 1; 3 \rient$}

		Soit $n \in \N^*$.

		Il y a ${\left| \lient 1; 2 \rient \right|} ^ {\left| \lient 1; n \rient \right|} = 2^n$ applications de $\lient 1; n \rient$ dans $\lient 1; 2 \rient$.
		Seules les applications constantes $\widetilde{1}$ et $\widetilde{2}$ ne sont pas surjectives.
		Il y a donc $2^n - 2$ surjections de $\lient 1; n \rient$ dans $\lient 1; 2 \rient$.

		Il y a ${\left| \lient 1; 3 \rient \right|} ^ {\left| \lient 1; n \rient \right|} = 3^n$ applications de $\lient 1; n \rient$ dans $\lient 1; 3 \rient$. Les applications non surjectives sont celles dont l'image n'est pas $\lient 1; 3 \rient$. C'est-à-dire, celles dont l'image est de cardinal 1 (les fonctions constantes $\widetilde{1}$, $\widetilde{2}$ et $\widetilde{3}$) et celles dont l'image est de cardinal 2. Ces dernières sont les surjections de $\lient 1; n \rient$ dans $\lient 1; 2 \rient$, $\{ 1; 3 \}$ et $\{ 2; 3 \}$. Comme ces trois ensembles ont la même taille, il y a $3 \times (2^n - 2)$ (voir résultat précédent) applications de $\lient 1; n \rient$ dans $\lient 1; 3 \rient$ dont l'image est de cardinal 2. Ainsi, le nombre de surjections de $\lient 1; n \rient$ dans $\lient 1; 3 \rient$ est $3^n - 3 - 3(2^n -2) = 3^n - 3 \times 2^n + 3$.
	\end{question_kholle}

	\begin{question_kholle}
		[Soient $E, F$ deux ensembles finis non vides et $f : E \rightarrow F$ telle que tout élément de $F$ possède le même nombre $k \in \N^*$ d'antécédents par $f$. \\
		Alors $\left|F\right| = \frac{\left|E\right|}{k}$
		\begin{quotation}
			\textquotedblleft Pour compter les moutons, il faut compter les pattes puis diviser par quatre. \textquotedblright
		\end{quotation}]
		{Lemme des bergers}

		Considérons la relation binaire définie sur $E$ par :
		\begin{equation*}
			\forall (x, y) \in E^2, x \sim y
			\iff f(x) = f(y)
		\end{equation*}
		Elle est réflexive, transitive et symétrique donc c'est bien une relation d'équivalence.
		Donc les classes d'équivalence réalise une partition de $E$.
		Nous avons $\displaystyle E = \coprod_{C \in \nicefrac{E}{\sim}} C$ donc, en passant aux cardinaux, $\displaystyle \left|E\right| = \sum_{C \in \nicefrac{E}{\sim}} \left|C\right|$.

		Soit $x \in E$ \fq. Alors $\bar{x} = \left\{ y \in E \;|\; f(x) = f(y) \right\} = f^{-1}(f(\{x\}))$. Par hypothèse, tous les éléments de $F$ ont le même nombre $k$ d'antécédents, or $f({x})$ est un singleton d'élément de $F$ donc $\left|\bar{x}\right| = k$.
		Ainsi $\forall C \in \nicefrac{E}{\sim}, \left|C\right| = k$.

		Posons $\varphi \left|\begin{array}{ccc}
			\nicefrac{E}{\sim} & \mapsto & F \\
			C & \rightarrow & f(x) \text{ où } x \in C
		\end{array}\right.$.
		$\varphi$ est bien défini car si $(x, y) \in E$ vérifie $\bar{x} = \bar{y}$ alors $f(x) = f(y)$ donc l'image par $\varphi$ ne dépend pas du représentant de classe choisi.
		$\varphi$ est surjective car soit $z \in F$, $f$ est surjective donc $\exists x_z \in E : f(x_z) = z$ et alors $\varphi(\bar{x_z}) = f(x_z) = z$.
		$\varphi$ est injective car soient $(C, C') \in \left( \nicefrac{E}{\sim} \right)^2, \varphi(C) = \varphi(C')$ alors $\exists (x, x') \in C \times C' : x \sim x'$, comme deux classes d'équivalence sont confondues ou disjointes, $C = C'$.
		Ainsi $\varphi$ est une bijection donc $\left|F\right| = \left|\nicefrac{E}{\sim}\right|$.

		Ainsi $\displaystyle \left|E\right| = \sum_{C \in \nicefrac{E}{\sim}} \left|C\right| = \sum_{C \in \nicefrac{E}{\sim}} k = \left| \nicefrac{E}{\sim} \right| k = \left|F\right| k$.
	\end{question_kholle}
\end{document}

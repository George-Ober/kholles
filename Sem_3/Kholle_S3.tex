\documentclass{article}

\date{17 avril 2024}
\usepackage[nb-sem=3, auteurs={George Ober}]{../kholles}


\begin{document}

\maketitle

\begin{question_kholle}
    [{Pour tout $(z_{1}, z_{2})\in \mathbb{C}^{2}$,
    
    (i) $\lvert z_{1}+z_{2} \rvert \leqslant \lvert z_{1} \rvert + \lvert z_{2} \rvert$
    
    (ii) $\bigg| \lvert z_{1} \rvert-\lvert z_{2} \rvert \bigg|\leqslant \lvert z_{1}-z_{2} \rvert$
    
    }]
    {Preuve de l'inégalité triangulaire et de l'inégalité montrant que le module est 1-lipschitzien + dessin et interprétation géométrique}
    Soient $(z_1,z_2) \in \C^2$ fixés quelconques.
    \begin{itemize}[label=$\lozenge$]
        \item 
        
        Si $z_{2} = 0$ l'inégalité est évidente
        Sinon, $z_{2} \neq 0$ alors $\lvert z_{1}+z_{2} \rvert \leqslant \lvert z_{1} \rvert + \lvert z_{2} \rvert \iff \left| 1+\frac{z_{1}}{z_{2}} \right|\leqslant 1 + \left\lvert  \frac{z_{1}}{z_{2}}  \right\rvert$.
        Posons $u = \frac{z_{1}}{z_{2}}$
        
        \begin{align*}
            \lvert 1+u \rvert ^{2} - (1+\lvert u \rvert )^{2}  & = (1+u)(\overline{1+u})-(1+2\lvert u \rvert +\lvert u \rvert ^{2}) \\
            & = (1+u)(1+ \overline u) - 1 - 2\lvert u \rvert  - \lvert u \rvert ^{2} \\
            &= u + \overline u -2 \lvert u \rvert  \\
            &= 2(\mathrm{Re}(u)-u) \leqslant 0
        \end{align*}
        
        \item Appliquons l'inégalité triangulaire
        $$
        \lvert  z_{1} \rvert  = \lvert z_{1} - z_{2} + z_{2} \rvert  \leqslant \lvert z_{1}-z_{2} \rvert +\lvert z_{2} \rvert  \implies \lvert z_{1} \rvert - \lvert z_{2} \rvert \leqslant \lvert z_{1}-z_{2} \rvert 
        $$
        Puisque $z_{1}$ et $z_{2}$ jouent de rôles symétriques on a aussi
        $$
        \lvert z_{2} \rvert - \lvert z_{1} \rvert \leqslant \lvert z_{2}-z_{1} \rvert =\lvert z_{1}-z_{2} \rvert 
        $$
        Donc 
        $$
        \bigg| \lvert z_{1} \rvert-\lvert z_{2} \rvert \bigg|\leqslant \lvert z_{1}-z_{2} \rvert
        $$
    \end{itemize}
\end{question_kholle}
\begin{question_kholle}{Caractérisation du cas d'égalité de l'inégalité triangulaire dans $\mathbb C$}
    \begin{itemize}[label=$\star$]
        \item ($\implies$) Supposons qu'il y ait égalité dans l'inégalité triangulaire
        \begin{itemize}[label=$\lozenge$]
            \item Si $z_{2} = 0$ alors $z_{1}$ et $z_{2}$ sont positivement liés
            \item Sinon $\lvert 1+u \rvert ^{2} - (1+\lvert u \rvert )^{2}  = 0$ donc $\mathrm{Re}(u) - \lvert u \rvert = 0$ Donc $u \in \mathbb{R}_{+}$ mais $z_{1} = uz_{2}$.
            Donc $z_{1}$ et $z_{2}$ sont positivement liés.
        \end{itemize}
        \item ($\impliedby$) Supposons que $z_{1}$ et $z_{2}$ sont positivement liés. Alors il existe $\lambda \in \mathbb{R}_{+}$ tel que $z_{1} = \lambda z_{2}$
        
        Si $z_{1} = \lambda z_{2}$
        $$
        \lvert z_{1}+z_{2} \rvert  = \lvert (\lambda+1) z_{2} \rvert  = \lvert \lambda + 1  \rvert \lvert z_{2} \rvert = (\lambda+1)\lvert z_{2} \rvert = \lambda \lvert z_{2} \rvert +\lvert z_{2} \rvert  = \lvert \lambda z_{2} \rvert +\lvert z_{2} \rvert = \lvert z_{1} \rvert + \lvert z_{2} \rvert 
        $$
        Donc l'inégalité est une égalité
        
        Si $z_{2} = \lambda z_{1}$, en échangeant les rôles joués par $z_{1}$ et $z_{2}$ on obtient que l'inégalité est une égalité.
    \end{itemize}
\end{question_kholle}
\begin{question_kholle}{Calcul de $\sum_{k=0}^{n}\cos(k\theta)$ pour tout $\theta \in \mathbb R$}
    Soit $\theta \in \mathbb{R}$ fixé quelconque, $n \in \mathbb{N}$ fixé quelconque.
    
    
    \begin{align*}
        C_{n}(\theta)=\sum_{k=0}^{n} \cos(k\theta) &= \sum_{k=0}^{n} \mathrm{Re}(e^{i k \theta}) \\
        &= \mathrm{Re} \left(\sum_{k=0}^{n} e^{i k \theta}\right) \\
        &= \mathrm{Re} \left( \sum_{k=0}^{n} (e^{i \theta})^{k} \right) \text{ par les formules de moivre} \\
    \end{align*}
    
    Ainsi, si $e^{i\theta} = 1 \iff \theta \equiv 0 [2\pi]$,
    $$
    C_{n}(\theta) = \mathrm{Re}\left( \sum_{k=0}^{n}(1)^{k} \right) = \mathrm{Re}(n+1) = n+1
    $$
    Sinon, 
    $$
    C_{n}(\theta) = \mathrm{Re}\left( \frac{1-(e^{i\theta})^{n+1}}{1-e^{i\theta}} \right)
    $$
    Simplifions donc ce quotient.
    
    \begin{align*}
        \frac{1-(e^{i\theta})^{n+1}}{1-e^{i\theta}} = \frac{1-e^{i\theta(n+1)}}{1-e^{i\theta}}
        &=\frac{e^{\frac{i\theta(n+1)}2{}}\left( e^{-\frac{i\theta(n+1)}2{}}-e^{\frac{i\theta(n+1)}2{}} \right)}{e^{i\frac{\theta}{2}}\left( e^{-i\frac{\theta}{2}}- e^{i \frac{\theta}{2}} \right)}\\
        &= e^{i \frac{\theta n}2}\left( \frac{-2i \sin\left( \frac{\theta (n+1)}{2} \right)}{-2i\sin \frac{\theta}{2}} \right) \\
        &= \frac{\sin\left( \frac{\theta(n+1)}{2} \right)}{\sin \frac{\theta}{2}}\left( \cos \frac{\theta n}{2} + i \sin \frac{\theta n}{2} \right) \tag{$\clubsuit$}\label{eqn:somme1}
    \end{align*}
    
    
    En prenant la partie réelle de ce résultat, on a
    $$
    C_{n}(\theta) = \mathrm{Re}\left[ \frac{\sin\left( \frac{\theta(n+1)}{2} \right)}{\sin \frac{\theta}{2}}\left( \cos \frac{\theta n}{2} + i \sin \frac{\theta n}{2} \right) \right] = \frac{\sin\left( \frac{\theta(n+1)}{2} \right)}{\sin \frac{\theta}{2}} \cos \frac{n\theta}{2}
    $$
    
    Donc
    $$
    C_{n}(\theta) = \left\{ \begin{array}{ll}
        n+1 & \text{ si } \theta \equiv 0 [2 \pi] \\
        \frac{\sin\left( \frac{\theta(n+1)}{2} \right)}{\sin \frac{\theta}{2}} \cos \frac{n\theta}{2}  & \text{ sinon}
    \end{array}\right.
    $$
\end{question_kholle}
\textbf{Remarque}
En prenant la partie imaginaire de \eqref{eqn:somme1}, on peut retrouver la somme $S_n(\theta)$:
$$
S_{n}(\theta)= \sum_{k=0}^n \sin(k \theta) = \left\{ \begin{array}{ll}
    0 & \text{ si } \theta \equiv 0 [2 \pi] \\
    \frac{\sin\left( \frac{\theta(n+1)}{2} \right)}{\sin \frac{\theta}{2}} \sin \frac{n\theta}{2}  & \text{ sinon}
\end{array}\right.
$$
\begin{question_kholle}
    [{Soient $n \in \mathbb{N}, (a_{0}, \dots, a_{n})\in\mathbb{C}^{n+1}$ et $z_{0} \in \mathbb{C}$ Posons pour tout $z \in \mathbb{C}, P(z) = \sum_{k=0}^{n}a_{k}z^{k}$
    
    (i) Si $P(z_{0}) = 0$, alors $\exists Q \in\mathbb{C}[z]:\forall z \in \mathbb{C}, P(z)=(z-z_{0})Q(z)$}]
    {Si $z_0$ est racine de la fonction polynômiale $P$, alors $P$ se factorise par $(z-z_0)$}
    Soit $z \in \mathbb{C}$ fixé quelconque,
    
    \begin{align*}
        P(z)  & = P(z) - P(z_{0}) \\
        & = \sum_{k=0}^{n}a_{k}z^{k} - \sum_{k=0}^{n}a_{k}z_{0}^{k}  \\
        & = \sum_{k=0}^{n}a_{k}(z^{k}-z_{0}^{k}) &\text{ nul pour }k = 0\\
        & = \sum_{k=1}^{n}\left( a_{k}(z-z_{0})\left( \sum _{j=0}^{k-1}z^{j}z_{0}^{k-1-j} \right) \right) \\
        & = (z-z_{0}) \sum_{k=1}^{n}a_{k}\left( \sum _{j=0}^{k-1}z^{j}z_{0}^{k-1-j} \right)
    \end{align*}
    
    Donc en posant $Q(z) = \sum_{k=1}^{n}a_{k}\left( \sum _{j=0}^{k-1}z^{j}z_{0}^{k-1-j} \right), \in \mathbb{C}[z]$, on a montré que $P$ se factorise.
\end{question_kholle}
\begin{question_kholle}
    [{Soient $n \in \mathbb{N}, (a_{0}, \dots, a_{n})\in\mathbb{C}^{n+1}$ et $z_{0} \in \mathbb{C}$ Posons pour tout $z \in \mathbb{C}, P(z) = \sum_{k=0}^{n}a_{k}z^{k}$
    
    (ii) Si $\exists p \in \mathbb{N}^{*}: \exists(z_{1},\dots ,z_{p})\in\mathbb{C}^{p}$ deux à deux distincts tels que $\forall k \in [ \! [ 1, p ] \!], P(z_{k}) = 0$ alors, $\exists Q \in \mathbb{C}[x]:\forall z \in \mathbb{C}, P(z) = Q(z) \times \prod_{k=1}^{p}(z-z_{k})$.
    }]
    {Si $z_1, \dots , z_n$ sont $n$ racines distinctes de la fonction polynômiale $P$ de degré $n$, alors $P(z)$ se factorise en ... }
    Considérons la propriété $\mathcal{P}(\cdot)$ définie pour tout $p \in \mathbb{N}^{*}$ par
    $$
    \mathcal{P}(p) : \forall P \in \mathbb{C}[z], (\exists (z_{1}, \dots, z_{p}) \in \mathbb{C}^{p}, \text{ 2 à 2 distincts }: \forall i \in [ \! [ 1, p ] \!], P(z_{i}) = 0)\\
    \implies \exists Q \in \mathbb{C}[z]: P(z) = Q(z) \prod_{i=1}^{p}(z-z_{i})
    $$
    \begin{itemize}[label=$\lozenge$]
        \item $\mathcal{P}(1)$ est vraie d'après la preuve précédente.
        \item Soit $p \in \mathbb{N}^{*}$ fixé quelconque tel que $\mathcal{P}(p)$ est vraie.
        Soit $P \in \mathbb{C}[z]$ fq tq $\exists (z_{1}, \dots, z_{p+1}) \in \mathbb{C} ^{p+1}$ deux à deux distincts tels que $\forall i \in [ \! [ 1, p + 1 ] \!], P(z_{i}) = 0$.
        Appliquons $\mathcal{P}(p)$ à $P \in \mathbb{C}[z]$ dont $(z_{1}, \dots, z_{p})$ sont les $p$ racines deux à deux distinctes.
        
        $$\exists Q_{1} \in \mathbb{C}[z]:\forall z \in \mathbb{C}, P(z) = Q_{1}(z)\prod_{i=1}^{p}(z-z_{i})$$
        Évaluons cette expression en $z_{p+1}$
        $$\underbrace{ P(z_{p+1}) }_{ =0 } = Q_{1}(z_{p+1}) \prod_{i=1}^{p}\underbrace{ (z_{p+1}-z_{i}) }_{ \neq 0 \text{ car distincts} }$$
        Donc $Q_{1}(z_{p+1}) = 0$, ce qui permet d'appliquer (i) pour $P \leftarrow Q_{1}$, $z_{0} \leftarrow z_{p+1}$.
        $$
        \exists Q \in \mathbb{C}[z]:\forall z \in \mathbb{C}, Q_{1}(z)=(z-z_{p+1})Q(z)
        $$
        Donc
        $$
        \forall z \in \mathbb{C}, P(z) = (z-z_{p+1})Q(z) \prod_{i=1}^{p}(z-z_{i}) = Q(z) \prod_{i=1}^{p+1}(z-z_{i})
        $$
        Donc $\mathcal{P}(p+1)$ est vraie.
    \end{itemize}
\end{question_kholle}
\begin{question_kholle}{Calculer le module et un argument de $z=1+ e^{i \theta}$ en fonction de $\theta \in [0, 2 \pi[$}
    Soit $\theta \in [0, 2\pi[$
    $$
    z = 1+ e^{i \theta} = e^{i\times0}+e^{i\theta} = e^{i \frac{\theta}{2}}\left( e^{-i\frac{\theta}{2}} + e^{i \frac{\theta}{2}} \right)= 2 \cos \frac{\theta}{2}e^{i\frac{\theta}{2}}
    $$
    Cette dernière notation est une notation exponentielle seulement si $2 \cos \frac{\theta}{2} \geqslant 0$.
    \begin{itemize}[label=$\star$]
        \item Si $\theta \in [0, \pi[$,
        $$\left\{ \begin{array}{ll}
            |z| = 2\cos \frac{\theta}{2} \\
            \frac{\theta}{2} \in \mathrm{Arg}  (z)
        \end{array}\right.$$
        \item Si $\theta = \pi$, $z = 0$ donc $\lvert z \rvert = 0$
        
        \item Si $\theta \in ]\pi, 2\pi[$,
        
        \begin{align*}
            z = 2 \cos \frac{\theta}{2}e^{i\frac{\theta}{2}} &= -2 \left\lvert  \cos \frac{\theta}{2}  \right\rvert e^{i\frac{\theta}{2}} \\
            & =-2 \left\lvert  \cos \frac{\theta}{2}  \right\rvert e^{i \left( \frac{\theta}{2} + \pi \right)}
        \end{align*}
        
        Donc 
        $$
        \left\{ \begin{array}{ll}
            |z| = -2 \lvert  \cos \frac{\theta}{2}  \rvert \\
            \frac{\theta}{2} + \pi \in \mathrm{Arg}  (z)
        \end{array}\right.
        $$
    \end{itemize}
\end{question_kholle}
\end{document}
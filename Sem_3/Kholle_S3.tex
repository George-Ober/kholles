\documentclass{article}

\date{17 avril 2024}
\usepackage[nb-sem=3, auteurs={George Ober}]{../kholles}


\begin{document}

\maketitle

\begin{question_kholle}{Preuve de l'inégalité triangulaire et de l'inégalité montrant que le module est 1-lipschitzien + dessin et interprétation géométrique}
    
\end{question_kholle}
\begin{question_kholle}{Caractérisation du cas d'égalité de l'inégalité triangulaire dans $\mathbb C$}
    
\end{question_kholle}
\begin{question_kholle}{Calcul de $\sum_{k=0}^{n}\cos(k\theta)$ pour tout $\theta \in \mathbb R$}
    
\end{question_kholle}
\begin{question_kholle}
    [{Soient $n \in \mathbb{N}, (a_{0}, \dots, a_{n})\in\mathbb{C}^{n+1}$ et $z_{0} \in \mathbb{C}$ Posons pour tout $z \in \mathbb{C}, P(z) = \sum_{k=0}^{n}a_{k}z^{k}$
    
    (i) Si $P(z_{0}) = 0$, alors $\exists Q \in\mathbb{C}[z]:\forall z \in \mathbb{C}, P(z)=(z-z_{0})Q(z)$}]
    {Si $z_0$ est racine de la fonction polynômiale $P$, alors $P$ se factorise par $(z-z_0)$}
    Soit $z \in \mathbb{C}$ fixé quelconque,
    
    \begin{align*}
        P(z)  & = P(z) - P(z_{0}) \\
        & = \sum_{k=0}^{n}a_{k}z^{k} - \sum_{k=0}^{n}a_{k}z_{0}^{k}  \\
        & = \sum_{k=0}^{n}a_{k}(z^{k}-z_{0}^{k}) &\text{ nul pour }k = 0\\
        & = \sum_{k=1}^{n}\left( a_{k}(z-z_{0})\left( \sum _{j=0}^{k-1}z^{j}z_{0}^{k-1-j} \right) \right) \\
        & = (z-z_{0}) \sum_{k=1}^{n}a_{k}\left( \sum _{j=0}^{k-1}z^{j}z_{0}^{k-1-j} \right)
    \end{align*}
    
    Donc en posant $Q(z) = \sum_{k=1}^{n}a_{k}\left( \sum _{j=0}^{k-1}z^{j}z_{0}^{k-1-j} \right), \in \mathbb{C}[z]$, on a montré que $P$ se factorise.
\end{question_kholle}
\begin{question_kholle}
    [{Soient $n \in \mathbb{N}, (a_{0}, \dots, a_{n})\in\mathbb{C}^{n+1}$ et $z_{0} \in \mathbb{C}$ Posons pour tout $z \in \mathbb{C}, P(z) = \sum_{k=0}^{n}a_{k}z^{k}$
    
    (ii) Si $\exists p \in \mathbb{N}^{*}: \exists(z_{1},\dots ,z_{p})\in\mathbb{C}^{p}$ deux à deux distincts tels que $\forall k \in [ \! [ 1, p ] \!], P(z_{k}) = 0$ alors, $\exists Q \in \mathbb{C}[x]:\forall z \in \mathbb{C}, P(z) = Q(z) \times \prod_{k=1}^{p}(z-z_{k})$.
    }]
    {Si $z_1, \dots , z_n$ sont $n$ racines distinctes de la fonction polynômiale $P$ de degré $n$, alors $P(z)$ se factorise en ... }
    Considérons la propriété $\mathcal{P}(\cdot)$ définie pour tout $p \in \mathbb{N}^{*}$ par
    $$
    \mathcal{P}(p) : \forall P \in \mathbb{C}[z], (\exists (z_{1}, \dots, z_{p}) \in \mathbb{C}^{p}, \text{ 2 à 2 distincts }: \forall i \in [ \! [ 1, p ] \!], P(z_{i}) = 0)\\
    \implies \exists Q \in \mathbb{C}[z]: P(z) = Q(z) \prod_{i=1}^{p}(z-z_{i})
    $$
    \begin{itemize}[label=$\lozenge$]
        \item $\mathcal{P}(1)$ est vraie d'après la preuve précédente.
        \item Soit $p \in \mathbb{N}^{*}$ fixé quelconque tel que $\mathcal{P}(p)$ est vraie.
        Soit $P \in \mathbb{C}[z]$ fq tq $\exists (z_{1}, \dots, z_{p+1}) \in \mathbb{C} ^{p+1}$ deux à deux distincts tels que $\forall i \in [ \! [ 1, p + 1 ] \!], P(z_{i}) = 0$.
        Appliquons $\mathcal{P}(p)$ à $P \in \mathbb{C}[z]$ dont $(z_{1}, \dots, z_{p})$ sont les $p$ racines deux à deux distinctes.
        
        $$\exists Q_{1} \in \mathbb{C}[z]:\forall z \in \mathbb{C}, P(z) = Q_{1}(z)\prod_{i=1}^{p}(z-z_{i})$$
        Évaluons cette expression en $z_{p+1}$
        $$\underbrace{ P(z_{p+1}) }_{ =0 } = Q_{1}(z_{p+1}) \prod_{i=1}^{p}\underbrace{ (z_{p+1}-z_{i}) }_{ \neq 0 \text{ car distincts} }$$
        Donc $Q_{1}(z_{p+1}) = 0$, ce qui permet d'appliquer (i) pour $P \leftarrow Q_{1}$, $z_{0} \leftarrow z_{p+1}$.
        $$
        \exists Q \in \mathbb{C}[z]:\forall z \in \mathbb{C}, Q_{1}(z)=(z-z_{p+1})Q(z)
        $$
        Donc
        $$
        \forall z \in \mathbb{C}, P(z) = (z-z_{p+1})Q(z) \prod_{i=1}^{p}(z-z_{i}) = Q(z) \prod_{i=1}^{p+1}(z-z_{i})
        $$
        Donc $\mathcal{P}(p+1)$ est vraie.
    \end{itemize}
\end{question_kholle}
\end{document}
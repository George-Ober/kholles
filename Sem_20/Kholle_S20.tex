\documentclass{article}

\date{10 Mars 2024}
\usepackage[nb-sem=20, auteurs={Hugo Vangilluwen, George Ober}]{../kholles}

\begin{document}
	\maketitle
	
	\begin{question_kholle}
		[{\begin{equation}
				\K[X] ^\times = \left\{ \lambda X^0, \lambda \in \K^* \right\}
		\end{equation}}]
		{Éléments inversibles de l'anneau $\K[X]$}
		
		Soit P un élément inversible de $\K[X]$.
		Alors $\exists Q \in \K[X] : P \cdot Q = Q \cdot P = 1_{\K[X]}$.
		En prenant les degrés des polynômes, $\text{deg } P + \text{deg } Q = 0$. \\
		Or $\text{deg } : \K[X] \rightarrow \N$ donc $\text{deg } P = \text{deg } Q = 0$.
		Donc $\exists \lambda \in \K^* : P = \lambda$. \\
		Ainsi $\K[X]^\times \subset \left\{ \lambda X^0, \lambda \in \K^* \right\}$.
		
		Soit $\lambda \in \K^*$. Considérons $P = \lambda$.
		Posons $Q = \lambda^{-1}$ (car \K est un corps). $P \cdot Q = \lambda \lambda^{-1}$ et $Q \cdot P = \lambda^{-1} \lambda$ donc $P$ est inversible. Ainsi $\left\{ \lambda X^0, \lambda \in \K^* \right\} \subset \K[X]^\times$.
	\end{question_kholle}
	
	\begin{question_kholle}
		[Le problème d'interpolation de Lagrange est, pour $n \in \N$ avec $a \in \K^{n+1}$ et $b \in \K^{n+1}$, l'ensemble des polynômes passant par tous les points de coordonnée $(a_i, b_i)$. C'est-à-dire l'ensemble des {$P \in \K[X]$} vérifiant :
		{\begin{equation}
			\forall i \in [\![0;n]\!], P(a_i) = b_i
		\end{equation}}
		Il existe une unique solution $P$ de degré $\leqslant n$ au problème d'interpolation de lagrange, et elle s'exprime de la manière suivante en posant
		\begin{equation}
			L_{i} = \prod_{\substack{j=0\\j \neq i}}^{n} \frac{X - a_{j}}{a_{i} - a_{j}}
		\end{equation}
		\begin{equation}
			P = \sum_{i=0}^{n}b_{i}L_{i}
		\end{equation}]
		{Théorème d'interpolation de lagrange}
		
		\begin{itemize}
			\item Unicité
			
			Supposons qu'il existe $(P, Q) \in \mathbb{K}_n[X]^{2}$ solutions du problème d'interpolation.
			
			Alors $\forall i \in [ \! [ 0, n ] \!], \tilde{P}(a_{i}) = \tilde{Q}(a_{i}) = b_{i}$ 
			
			Posons $H = P - Q$, alors, $\forall i \in [ \! [ 0, n ] \!], \tilde{H}(a_{i}) = \tilde{P}(a_{i}) - \tilde{Q}(a_{i}) = 0$.
			
			De plus, $\deg H = \deg(P-Q) \leqslant \max \left\{ \deg P, \deg Q \right\}$
			
			Donc $H$ est un polynôme de degré $\leqslant n$ avec $\lvert [ \! [ 0, n ] \!] \rvert = n+1$ racines.
			
			Donc $H$ est le polynôme nul.
			\item Existence
			Soit $i \in [ \! [ 0, n ] \!]$ fq
			Notons $L_{i}$ une solution de degré $\leqslant n$ au problème $Pb_{i}$ suivant:
			$$
			(Pb_i)
			\left\{ \begin{array}{ll}
				\tilde{P}(a_{0}) = 0 \\
				\vdots \\
				\tilde{P}(a_{i-1}) = 0 \\
				\tilde{P}(a_{i}) = 1 \\
				\tilde{P}(a_{n}) = 0 \\
				\vdots \\
				\tilde{P}(a_{n}) = 0
			\end{array} \right.
			$$
			On remarque que $(a_{0},\dots ,a_{i-1}, a_{i+1},\dots, a_{n})$ sont $n$ racines deux à deux distinctes de $L_{i}$. Or $L_{i}$ est de degré $\leqslant n$ et n'est pas le polynôme nul (car $\tilde{L_{i}}(a_{i}) = 0$) donc $(a_{0},\dots ,a_{i-1}, a_{i+1},\dots, a_{n})$ sont les \emph{seules} racines de $L_{i}$, toutes simples.
			
			Dès lors, 
			$$
			\exists c \in \mathbb{K}^{*} : L_{i} = c\prod_{\substack{j=0 \\ j\neq i}} ^{n}(X-a_{j})
			$$
			Pour trouver le $c$, remarquons que
			
			\begin{align*}
				\tilde{L_{i}}(a_{i}) = 1 &\iff c\prod_{\substack{j=0 \\ j\neq i}} ^{n}(a_{i}-a_{j}) = 1\\
				&\iff c = \prod_{\substack{j=0 \\ j\neq i}} ^{n}\left( \frac{1}{a_{i}-a_{j}} \right)
			\end{align*}
			
			Ainsi, s'il existe une solution au problème $Pb_{i}$ c'est nécéssairement
			$$L_{i} = \prod_{\substack{j=0 \\ j\neq i}} ^{n}\left( \frac{{X-a_{j}}}{a_{i}-a_{j}} \right)$$
			
			Réciproquement, cette solution est correcte puisque
			$$\forall k \in [ \! [ 0, n ] \!], k \neq i,  \tilde{L_{i}}(a_{k}) = \prod_{\substack{j=0 \\ j\neq i}} ^{n}\left( \frac{{\overbrace{ a_{k}-a_{j} }^{ =0 }}}{a_{i}-a_{j}} \right) = 0$$
			Et
			$$\tilde{L_{i}}(a_{i}) = \prod_{\substack{j=0 \\ j\neq i}} ^{n}\left( \frac{{a_{i}-a_{j}}}{a_{i}-a_{j}} \right) = \prod_{\substack{j=0 \\ j\neq i}} ^{n} 1 = 1$$
			
			Posons donc $P = \sum_{i=0} ^{n} b_{i}Li$.
			
			Alors, par construction,
			$$
			\forall k \in [ \! [ 0, n ] \!], \tilde{P}(a_{k}) = \sum_{i=0} ^{n}\left(  b_{i} \prod_{\substack{j=0 \\ j\neq i}} ^{n}\left( \frac{{a_{k}-a_{j}}}{a_{i}-a_{j}} \right) \right) = \sum_{i=0} ^{n}\left(  b_{i} \delta_{ki} \right) = b_{k} \delta_{kk} = b_{k}
			$$
			Nous avons donc construit une solution unique au problème d'interpolation de Lagrange
		\end{itemize}
	\end{question_kholle}
	
	\begin{question_kholle}
		[Les fonctions symétriques élémentaires $\displaystyle \left( \sigma_k \right)_{k \in [\![0;n]\!]}$ pour une famille $\displaystyle \left( x_k \right)_{k \in [\![1;n]\!]}$ sont définies par
		\begin{equation}
			\sigma_ k = \sum_{1 \leqslant i_1 < \ldots < i_k \leqslant n} \ \prod_{j=1}^{k} x_{i_j}
		\end{equation}]
		{Pour $P = (X-x_1)(X-x_2)(X-x_3)$, exprimer $x_1^3 + x_2^3 + x_3^3$ en fonction des fonctions symétriques élémentaires}
		
		Sous forme développée, $P = X^3 - (x1 + x_2 + x_3) X^2 + (x_1 x_2 + x_1 x_3 + x_2 x_3) X - x_1 x_2 x_3 = X^3 - \sigma_1 X^2 + \sigma_2 X - \sigma_3$. Comme $x_1, x_2, x_3$ sont racines de $P$, nous avons les trois égalité suivantes :
		\begin{equation*}
			\begin{aligned}
				0 = P(x_1) &= x_1^3 - \sigma_1 x_1^2 + \sigma_2 x_1 - \sigma_3 \\
				0 = P(x_1) &= x_2^3 - \sigma_1 x_2^2 + \sigma_2 x_2 - \sigma_3 \\
				0 = P(x_1) &= x_3^3 - \sigma_1 x_3^2 + \sigma_2 x_3 - \sigma_3
			\end{aligned}
		\end{equation*}
		En sommant ces trois équation,
		\begin{equation*}
			0 = x_1^3 + x_2^3 + x_3^3 - \sigma_1 (x_1^2 + x_2^2 + x_3^2) + \sigma_2 (x_1 + x_2 + x_3) - 3 \sigma_3
		\end{equation*}
		Cherchons la somme des carrés.
		\begin{equation*}
			\begin{aligned}
				(x_1 + x_2 + x_3)^2 &= x_1^2 +  x_2^2 + x_3^2 + 2 x_1 x_2 + 2 x_1 x_3 + 2 x_2 x_3 \\
				\implies x_1^2 +  x_2^2 + x_3^2 + x_1 x_2 &= \sigma_1^2 - 2 \sigma_2
			\end{aligned}
		\end{equation*}
		Ainsi \begin{equation*}
			x_1^3 + x_2^3 + x_3^3 = \sigma_1^3 - 3 \sigma_1 \sigma_2 + 3 \sigma_3
		\end{equation*}
	\end{question_kholle}
	
	\begin{question_kholle}
		[Les sommes de Newton $(S_k)_{k\in\Z^*}$ pour une famille $(x_k)_{k \in \N^*}$ sont définies par (sous réserve d'existence pour $k<0$) :
		\begin{equation}
			S_k = \sum_{i=1}^{n} x_i^k
		\end{equation}]
		{Expression de $S_2$, $S_{-1}$ et $S_{-2}$ à l'aide des fonctions élémentaires symétriques.}
		
		\begin{equation*}
			\begin{aligned}
				\sigma_1^2 &= \left( \sum_{i=1}^{n} x_i \right)^2 \\
				&= \underbrace{ \sum_{i=1}^{n} x_i^2 }_{S_2} + \ 2 \ \underbrace{ \sum_{1 \leqslant i < j \leqslant n} x_i x_j }_{\sigma_2} \\
				\implies S_2 &= \sigma_1^2 - 2 \sigma_2
			\end{aligned}
		\end{equation*}
		\begin{equation*}
			S_{-1} = \sum_{i=1}^{n} \frac{1}{x_i}
			= \frac{\displaystyle \sum_{i=1}^{n} \prod_{\substack{ j = 1 \\ j \neq i }}^{n} x_j }{\displaystyle \prod_{i=1}^{n} x_i }
			= \frac{\sigma_{n-1}}{\sigma_n}
		\end{equation*}
		\begin{equation*}
			\begin{aligned}
				S_{-2} &= \sum_{i=1}^{n} \frac{1}{x_i^2} \\
				&= \left( \sum_{i=1}^{n} \frac{1}{x_i} \right)^2 - 2 \sum_{1 \leqslant i < j \leqslant n} \frac{1}{x_i} \frac{1}{x_j} \\
				&= \frac{\sigma_{n-1}^2}{\sigma_n^2} - 2 \frac{\displaystyle \sum_{1 \leqslant i < j \leqslant n} \prod_{\substack{ k = 1 \\ k \notin \{i,j\} }} \frac{1}{x_j} }{\sigma_n} \\
				&= \frac{\sigma_{n-1}^2 - 2 \sigma_{n-2}\sigma_n}{\sigma_n^2}
			\end{aligned}
		\end{equation*}
	\end{question_kholle}
\end{document}
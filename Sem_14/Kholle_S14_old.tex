\documentclass{article}

\date{15 Janvier 2024}
\usepackage[nb-sem=14, auteurs={Kylian Boyet, Hugo Vangilluwen, Felix Rondeau}]{../kholles}

\begin{document}
\maketitle

\begin{question_kholle}
	[Soit $f \left| \begin{matrix}
				\R_+^* \rightarrow \R \\
				x \mapsto \frac{\ln x}{x}
			\end{matrix} \right. $. \\
		Exprimer $f^{(n)}$ pour tout $n \in \N$.]
	{Expression de dérivées successives}
	Soit $x\in \mathcal{D}_f$. \\
	Considérons le prédicat $P(\cdot)$ définit pour $n\in \N$ par :
	$$ P(n) \ : \ \text{\textquotedblleft} \ f^{(n)}(x) = \frac{(-1)^n n!}{x^{n+1}}\left[ \ln (x) - \sum_{k=1}^n \frac{1}{k} \right] \ " $$
	Initialisation : \\
	Pour $n = 0$, $$f^{(0)}(x) = f(x) = \frac{\ln (x)}{x} = \frac{(-1)^0 0!}{x^{0+1}}\left[ \ln (x) - \sum_{k=1}^0 \frac{1}{k} \right],$$ donc $P(0)$ est vrai. \\
	Hérédité :
	\\
	Soit $n\in \N$ tel que $P(n)$. On a,
	$$f^{(n+1)}(x) = (f^{(n)}(x))' = \left( \frac{(-1)^n n!}{x^{n+1}}\left[ \ln (x) - \sum_{k=1}^n \frac{1}{k} \right] \right)'$$
	par véracité de $P(n)$. Ainsi,
	$$\begin{array}{rcl}
			f^{(n+1)}(x) & = & \frac{(-1)^nn!x^n - (-1)^n(n+1)!x^n\left[ \ln (x) - \sum_{k=1}^n \frac{1}{k} \right]}{x^{2(n+1)}} \\ [1ex]
			             & = & \frac{(-1)^{n+1}(n+1)!\ln (x) - (-1)^{n+1}(n+1)!\sum_{k=1}^{n+1} \frac{1}{k} }{x^{n+2}}           \\ [1ex]
			             & = & \frac{(-1)^{n+1} (n+1)!}{x^{n+2}}\left[ \ln (x) - \sum_{k=1}^{n+1} \frac{1}{k} \right]
		\end{array}$$
	c'est l'expression recherchée, donc $P(n+1)$ est vrai. \\
	Par théorème de récurrence sur $\N$, $P(n)$ est vraie pour tout $n\in \N$.
\end{question_kholle}

\begin{question_kholle}
	[Soit $f : I \rightarrow f(I) \subset \R$ continue, strictement monotone sur $I$ et dérivable en $a \in I$.
		Si $f'(a) \neq 0$ alors $f$ est bijective, $f^{-1}$ est dérivable en $f(a)$ et $f^{-1}(f(a)) = \frac{1}{f'(a)}$.]
	{Dérivé d'une bijection réciproque}
	Soient de tels objets. \\
	Rappelons le lemme inattendu. Soit $g : J \rightarrow \R$ monotone (où $J$ est un intervalle). Nous avons l'équivalence suivante :
	\begin{equation*}
		f(J) \text{ est un intervalle } \iff f \text{ est continue sur } J
	\end{equation*}
	Par définition, $f$ est surjective. Comme elle est strictement monotone, $f$ est injective. Ainsi $f$ est bijective. \\
	D'après le lemme inattendu, $f(I)$ est un intervalle. Nous avons $f^{-1} : f(I) \rightarrow I$ avec $f(I)$ et $I$ des intervalles donc $f^{-1}$ est continue sur $f(I)$. \\
	Calculons la limite du taux d'accroissement de $f^{-1}$ en $f(a)$ :
	\begin{equation*}
		\forall x \in f(I), \tau_{f^{-1},f(a)} = \frac{f^{-1}(x) - f^{-1}(f(a))}{x - f(a)}
	\end{equation*}
	Posons $u = f^{-1}(x)$. D'où :
	\begin{equation*}
		\tau_{f^{-1},f(a)} = \frac{u - a}{f(u) - f(a)}
	\end{equation*}
	De plus, par continuité de $f^{-1}$, $u \arrowlim{x}{f(a)} f^{-1}(f(a)) = a$.
	Par dérivabilité en a et par continuité de $x \mapsto x^{-1}$ en $f(a) \neq 0$, $\frac{u - a}{f(u) - f(a)} \arrowlim{u}{a} \frac{1}{f('(a)}$. \\
	Ainsi, $f^{-1}$ est dérivable en $f(a)$ et $f^{-1}(f(a)) = \frac{1}{f'(a)}$.
\end{question_kholle}

\begin{question_kholle}
	[Soit $f : I \rightarrow \R$. Si $f$ admet un extremum local en $a \in \overset{\circ}{I}$ et si $f$ est dérivable en $a$ alors $f'(a) = 0$
		{\begin{figure}[!h]
					\centering
					\tikzmath{ integer \m; real \fm; \m = 3; \fm = 0.8; }
					\begin{tikzpicture}
						\draw[-stealth] (0,0) -- (6,0) node[anchor=west] {$x$};
						\draw[-stealth] (0,0) -- (0,3) node[anchor=south] {$y$};

						\draw[red] plot [smooth] coordinates {(0,2.5) (\m,\fm) (6,1.7)};
						\draw[blue, dashed] (\m,\fm) -- (\m,0) node[anchor=north] {minimum local};

						\draw[stealth-stealth, teal] (\m-1,\fm) -- (\m+1,\fm);
						\draw (\m,\fm) node[anchor=north east, teal] {$m$} node[anchor=south, teal] {$f'(m)=0$};
					\end{tikzpicture}
				\end{figure}}]
	{Dérivée d'un extremum local intérieur au domaine de définition}
	Soient de tels objets. \\
	$a \in \overset{\circ}{I} \implies \exists \eta_1 \in \R_+^* : [a-\eta_1;a+\eta_1] \subset I$ Fixons un tel $\eta_1$. \\
	Calculons le taux d'accroissement en $a$.
	\begin{equation*}
		\forall x \in [a-\eta_1;a+\eta_1], \tau_{f,a}(x) = \frac{f(x) - f(a)}{x - a}`
	\end{equation*}
	Or $f$ est dérivable en $a$ donc $\tau_{f,a}(x)$ admet une limite lorsque $x \rightarrow a$.
	Traitons le cas où $a$ est maximum local. Par définition :
	\begin{equation*}
		\exists \eta_2 \in \R_+^* : \forall x \in [a-\eta_2;a+\eta_2], f(x) \leqslant f(a)
	\end{equation*}
	Fixons un tel $\eta_2$. Soit $x \in [a-\eta_2;a+\eta_2] \setminus \{a\}$ fq. \\
	Alors $f(x) - f(a) \leqslant 0$. \\
	Si $x > a$, $x - a > 0$. Alors $\frac{f(x) - f(a)}{x - a} \leqslant 0$. Donc $\textlim{x}{a} \tau_{f,a}(x) \leqslant 0$. \\
	Sinon $x < a$, $x - a < 0$. Alors $\frac{f(x) - f(a)}{x - a} \geqslant 0$. Donc $\textlim{x}{a} \tau_{f,a}(x) \geqslant 0$. \\
	Ainsi $0 \leqslant \textlim{x}{a} \tau_{f,a}(x) \leqslant 0$. Donc $f'a) = 0$.
\end{question_kholle}

\begin{question_kholle}
	[Soient $(a,b)\in \R ^2$ tels que $a<b$. Soit $I$ le segment $a,b$.
	\\
	Soit $f \ : \ I \ \to \ \R $ continue sur ledit segment et dérivable sur l'ouvert associé.\\
	{\begin{enumerate}[label=($\roman*$)]
		\item Théroème de Rolle : \\
		      Si $f(a) = f(b)$, alors $\exists \ c \in \overset{\circ}{I}$ tel que $f'(c) =0$
		      \begin{figure}[!h]
			      \centering
			      \tikzmath{ integer \a \b \f \c \fc; \a = 1; \b = 9; \f = 2; \c = \a + 3; \fc = \f + 2; }
			      \begin{tikzpicture}
				      \draw[-stealth] (0,0) -- (10,0) node[anchor=west] {$x$};
				      \draw[-stealth] (0,0) -- (0,5) node[anchor=south] {$y$};

				      \coordinate (A) at (\a,\f);
				      \coordinate (B) at (\b,\f);
				      \coordinate (C) at (\c,\fc);

				      %\draw[red] (A) to[out angle=80, in angle=240, curve through={(C) (6,1.5)}] (B);
				      \draw [red] plot [smooth] coordinates {(A) (C) (6,1.5) (B)};
				      \draw[blue, thick] (A) -- (B);
				      \draw[blue, dashed] (A) -- (\a,0) node[anchor=north] {$a$};
				      \draw[blue, dashed] (B) -- (\b,0) node[anchor=north] {$b$};
				      \draw[blue, dashed] (A) -- (0,\f) node[anchor=east] {$f(a)=f(b)$};

				      \draw[stealth-stealth, teal] (\c-2,\fc) -- (\c+2,\fc);
				      \draw (C) node[anchor=north, teal] {$c$} node[anchor=south, teal] {$f'(c)=0$};
			      \end{tikzpicture}
			      \caption{Théorème de Rolle}
		      \end{figure}

		\item Formule des accroissements finis : \\
		      $\exists \ c \in \overset{\circ}{I} \ : \ f'(c) = \frac{f(b)-f(a)}{b-a}.$
		      \begin{figure}[!h]
			      \centering
			      \tikzmath{integer \a \fa \b \fb \c \fc; \a = 1; \fa = 1; let \b = 9; \fb = 5; \c = \a + 5; \fc = \fa + 1; }
			      \begin{tikzpicture}
				      \draw[-stealth] (0,0) -- (10,0) node[anchor=west] {$x$};
				      \draw[-stealth] (0,0) -- (0,6) node[anchor=south] {$y$};

				      \coordinate (A) at (\a,\fa);
				      \coordinate (B) at (\b,\fb);
				      \coordinate (C) at (\c,\fc);

				      %\draw[red] (A) to[out angle=80, in angle=240, curve through={(C) (6,1.5)}] (B);
				      \draw [red] plot [smooth] coordinates {(A) (3,4) (C) (7,4.5) (B)};
				      \draw[blue, thick] (A) -- (B);
				      \draw[blue, dashed] (A) -- (\a,0) node[anchor=north] {$a$};
				      \draw[blue, dashed] (A) -- (0,\fa) node[anchor=east] {$f(a)$};
				      \draw[blue, dashed] (B) -- (\b,0) node[anchor=north] {$b$};
				      \draw[blue, dashed] (B) -- (0,\fb) node[anchor=east] {$f(b)$};

				      \draw[stealth-stealth, teal] (\c-2,\fc-1) -- (\c+2,\fc+1);
				      \draw (C) node[anchor=south, teal] {$c$} node[anchor=north west, teal] {$f'(c)=\frac{f(b)-f(a)}{b-a}$};
			      \end{tikzpicture}
			      \caption{Formule des accroissements finis}
		      \end{figure}
	\end{enumerate}}
	]
	{Théorème de Rolle et formule des accroissements finis}
	Soient de tels objets. \\
	Prouvons $(i)$, donc supposons $f(a) = f(b)$. \\
	$f$ est continue sur $I$ donc par le théorème de Weierstraß, elle est bornée et atteint ses bornes sur ce segment :
	\\
	$$\exists \ (x_m, x_M)\in I^2 \ : \ (f(x_m) = \min f(I)) \wedge (f(x_M) = \max f(I))$$
	donc, si $(x_m, x_M)\in \{a,b\}^2$, alors,
	$$\forall x \in I, \ f(a)=f(x_m) \leq f(x) \leq f(x_M)=f(a)$$
	donc $\forall x \in I, f(x) = f(a)$ c'est-à-dire que $f$ est constante et donc tous les points intermédiaires à $I$ sont des $c$ valides.\\
	Sinon, $(x_m \notin \{a,b\}) \vee (x_M \notin \{a,b\}) $, quitte à prendre l'autre valeur, supposons que $x_M \notin \{a,b\}$, ainsi, $x_M \in \overset{\circ}{I}$ et $f(x_M)$ est un maximum global donc, $f$ étant dérivable sur $\overset{\circ}{I}$ elle est dérivable en $x_M$ donc $f'(x_M)=0$, on pose $c = x_M$, ce qui conclut. \\ \\
	Prouvons $(ii)$.\\
	Posons $d \ :\ I \ \to \ \R, \ x \ \mapsto \ f(x) - \left( \frac{f(b)-f(a)}{b-a}(x-a) +f(a) \right) $. $d$ est continue sur $I$ et dérivable sur $\overset{\circ}{I}$ comme combinaison linéaire de telles fonctions. On a $d(a) = 0$ et $d(b) = 0$ donc $d(a) = 0 = d(b)$. On peut alors appliquer le Théorème de Rolle pour $f \gets d, a \gets a $ et $ b \gets b$ : il existe $c\in \overset{\circ}{I}$ tel que $d'(c) = 0$, c'est le résultat.
\end{question_kholle}

\begin{question_kholle}
	[
	Soit $f\in \mathcal{C}^0(I, \R) \cap \mathcal{D}^1(\overset{\circ}{I}, \R)$ et $x_0 \in I$, posons {$X_- = ]-\infty;x_0]$} la demi-droite fermée en $x_0$ et vers $-\infty$, de même {$X_+ = [x_0;+\infty[$} la demi-droite fermée en $x_0$ et vers $+ \infty$. \\ \\
	{\begin{enumerate}[label=$(\roman*)$]
		\item \begin{itemize}[label=$\star$, leftmargin=0.4cm]
			      \item Si $\exists \ m \in \R \ : \ \forall x \in \overset{\circ}{I}, \ m\leq f'(x)$, alors, $$\forall x \in I \cap X_+, \ f(x_0) + m(x-x_0) \leq f(x)$$ et $$\forall x \in I \cap X_-, \ f(x) \leq f(x_0) + m(x-x_0)$$
			      \item Si $\exists \ M \in \R \ : \ \forall x \in \overset{\circ}{I}, \ f'(x) \leq M $, alors, $$\forall x \in I \cap X_+, \ f(x) \leq f(x_0) + M(x-x_0)$$ et $$\forall x \in I \cap X_-, \ f(x_0) + M(x-x_0) \leq f(x) $$
			      \item Si $\exists \ (m,M) \in \R^2 \ : \ \forall x \in \overset{\circ}{I}, \ m\leq f'(x) \leq M$, alors, $$\forall x \in I \cap X_+, \ f(x_0) + m(x-x_0) \leq f(x) \leq f(x_0) +M(x-x_0)$$ et $$\forall x \in I \cap X_-, \ f(x_0) + M(x-x_0) \leq f(x) \leq f(x_0) +m(x-x_0)$$
		      \end{itemize}
		\item Si $\exists M \in \R \ : \ \forall x \in \overset{\circ}{I}, \ |f'(x)|\leq M$, alors, $$\forall (x,y) \in I^2, \ |f(y) -f(x)| \leq M|y-x|$$
	\end{enumerate}}

	{\begin{figure}[!h]
		\centering
		\tikzmath{
			integer \x0,\fx0,\m,\M;
			\x0 = 5;
			\fx0 = 2.5;
			\m = 0.2;
			\Max = 0.7;
		}
		\begin{tikzpicture}
			\coordinate (A) at (1,\fx0-\m*\x0+\m*1);
			\coordinate (B) at (9,\fx0-\m*\x0+\m*9);
			\coordinate (C) at (1,\fx0-\Max*\x0+\Max*1);
			\coordinate (D) at (9,\fx0-\Max*\x0+\Max*9);
			\draw[ultra thick, black] (A) -- (B) node[anchor=north west] {$y = f(x_0) + m(x -x_0)$};
			\draw[ultra thick, black] (C) -- (D) node[anchor=south east] {$y = f(x_0) + M(x -x_0)$};
			\filldraw[magenta!30] (A) -- (B) -- (D) -- (C) -- cycle;

			\draw (\x0,\fx0) node[cross out, minimum size=8*\pgflinewidth, inner sep=0pt, outer sep=0pt, draw=blue, rotate=45, thick] {};
			\draw[blue, dashed] (\x0,\fx0) -- (0,\fx0) node[anchor=east] {$f(x_0)$};
			\draw[blue, dashed] (\x0,\fx0) -- (\x0,0) node[anchor=north] {$x_0$};

			\draw[-stealth] (0,0) -- (10,0) node[anchor=west] {$x$};
			\draw[-stealth] (0,0) -- (0,5) node[anchor=south] {$y$};
		\end{tikzpicture}
		\caption{Interprétation géométrique des accroissements finis}
	\end{figure}}
	]
	{Inégalité des accroissements finis}
	$(i)$ Soit $x\in I$ et posons $S$ le segment d'extrémités $x$ et $x_0$. \\
	$\star$ Si $x\neq x_0$, $f$ est continue sur $S$ et dérivable sur $\overset{\circ}{S}$, la formule des accroissements finis donne alors l'existence d'un $c$ appartenant à $\overset{\circ}{S}$ tel que $$f(x) - f(x_0) = (x-x_0)f'(c)$$ Si $x> x_0, \ x-x_0 > 0$, or $m \leq f'(c) \leq M$ donc $$m(x-x_0) \leq (x-x_0)f'(c) \leq M(x-x_0)$$ si bien que $$m(x-x_0) \leq f(x) - f(x_0) \leq M(x-x_0) $$ d'où $$f(x_0) + m(x-x_0) \leq f(x) \leq f(x_0) + M(x-x_0). $$ Si $x<x_0$, il suffit de retourner l'inégalité lors de la première multiplication et $(i)$ est prouvé.\\ \\
	$(ii)$ Soit $y \in I.$\\
	L'hypothèse $\forall x\in \overset{\circ}{I}, \ |f'(x)|\leq M$ équivaut à $\forall x \in \overset{\circ}{I}, \ -M \leq f'(x) \leq M$, donc on peut appliquer $(i)$ pour $x_0 \gets y, \ M \gets M$ et $m\gets -M$ : $$\forall x \in I\cap [y, +\infty [,  \ f(y) -M(x-y) \leq f(x) \leq f(y) + M(x-y)$$
					Or $x-y >0$ donc $|f(x) -f(y) | \leq M|x-y|$.
					Et $$\forall x \in I\cap ]-\infty, y ],  \ f(y) +M(x-y) \leq f(x) \leq f(y) - M(x-y)$$
	Or $x-y < 0$ donc $|f(x) -f(y) | \leq M|x-y|$. \\
	Par conséquent, $\forall (x,y)\in I^2, \ |f(y) -f(x)| \leq M|y-x|.$
\end{question_kholle}

\begin{question_kholle}
	[Soit $f\in \mathcal{C}^1(I,\R)$, $I$ le segment $a,b$. Alors $f$ est $||f'||_{\infty,I}$-lipschitzienne sur I.]
	{Caractère lipschitzien d'une fonction $\mathcal{C}^1$ sur un segment}
	Soient de tels objets. \\
	$\star$ $f\in \mathcal{C}^1(I, \R)$ donc $f\in \mathcal{C}^0(I, \R)$. \\
	$\star$ $f\in \mathcal{C}^1(I,\R)$ donc $f\in \mathcal{D}^1(\overset{\circ}{I}, \R)$.\\
	$\star$ $f\in \mathcal{C}^1(I,\R)$ donc $f'$ est continue sur $I$ donc le réel $||f'||_{\infty,I}$ est bien défini et $$\forall  x \in \overset{\circ}{I}, \ |f'(x)| \leq ||f'||_{\infty,I}.$$ Ces propriétés permettent d'appliquer le corollaire du TAF qui conclut que $f$ est $||f'||_{\infty,I}$-lipschitzienne.
\end{question_kholle}

\begin{question_kholle}
	[Soit $f\in \mathcal{F}(I, \R)$ et $a\in I$.\\
		\newline
		\textit{Lemme} : \\
		Si
		$\left\{ \begin{array}{cl}
				f \text{ est dérivable sur } I\backslash \{a\} \\
				f \text{ est continue en }a                    \\
				f'_{|I\backslash \{a\}} \text{ admet une limite $\ell \in \overline{\R}$ en }a
			\end{array}
			\right.$, alors $\lim_{x \to a} \frac{f(x) -f(a)}{x-a} = \ell$\\
		\newline
		\textit{Théorème} : \\
		Si
		$\left\{ \begin{array}{cl}
				f \text{ est dérivable sur } I\backslash \{a\} \\
				f \text{ est continue en }a                    \\
				f'_{|I\backslash \{a\}} \text{ admet une limite \textbf{finie} $\ell \in \R$ en }a
			\end{array}
			\right.$, alors $\left\{ \begin{array}{cl}
				f \text{ est dérivable en } a \\
				f'(a) = \ell \ (\textbf{donc } f' \textbf{ est continue en } a)
			\end{array}
			\right.$ ]
	{Théorème du prolongement de la propriété de la dérivabilité}
	Prouvons le lemme pour $\ell\in \R$, c'est le cas qui nous intéresse. \\
	Soient de tels objets. Soit $\varepsilon\in \R_+^*$. Appliquons la définition de $\lim_{\substack{x\to a \\ x\neq a}}  f'_{|I\backslash \{a\}}(x) =\ell $ pour $\varepsilon \gets \varepsilon$ :
	$$\exists \ \eta \in \R_+^* \ : \ \forall x\in I\backslash \{a\}, \ |x-a| \leq \eta \ \implies \ | f'_{|I\backslash \{a\}}(x) - \ell |\leq \varepsilon.$$ Fixons un tel $\eta$.\\
	Soit $x\in I\backslash \{a\}$ tel que $|x-a|\leq \eta$.\\ La fonction $f$ est continue sur $I$ donc $f$ est continue sur le segment d'extrémités $a$ et $x$ qui est par ailleurs inclus dans $I$ par convexité d'un intervalle.\\ La fonction $f$ est dérivable sur $I$ donc $f$ est dérivable sur l'intervalle ouvert $a$, $x$ qui est aussi inclus dans $\overset{\circ}{I}$ par convexité.\\ L'égalité des accroissements finis s'applique à $f$ sur l'intervalle $a$ et $x$ : $$\exists \ c_x \in ]a,x[ \cup ]x,a[ \ : \ \frac{f(x) -f(a)}{x-a} = f'(c_x)$$
	Or $|c_x-a| \leq |x-a| \leq \eta $ donc ladite définition de la limite s'applique pour $x\gets c_x$ : $|f'(c_x) - \ell|\leq \varepsilon $ si bien que $$|\frac{f(x)-f(a)}{x-a}- \ell |\leq \varepsilon.$$ D'où le lemme. \\
	\newline
	Prouvons alors le théorème.\\
	Sous ces hypothèses, le lemme s'applique donc $\lim_{x\to a} \frac{f(x) -f(a)}{x-a} = \ell$, or $\ell \in \R$, donc le taux d'accroissement de $f$ en $a$ admet une limite finie en $a$ ce qui prouve la dérivabilité de $f$ en $a$ et $f'(a) = \ell$. Ce qui suffit.
\end{question_kholle}

\begin{question_kholle}
	[Posons $\zeta \left| \begin{matrix}
				\R \longrightarrow \R \\
				x \mapsto \left\{ \begin{array}{cc}
					                  0                 & \text{si } x \leqslant 0 \\
					                  \e^{-\frac{1}{x}} & \text{si } x > 0
				                  \end{array} \right.
			\end{matrix} \right.$.
		Montrons que $\zeta \in \Cont{\infty}{\R}{\R}$.
			{\begin{figure}[!h]
					\centering
					\begin{tikzpicture}
						\begin{axis}[
								axis lines = center,
								xlabel = $x$,
								ylabel = {$f(x)$},
								width=15cm,
								height=5cm,
								ymax=1.5
							]
							\addplot[
								domain=0.01:20,
								samples=200,
								color=red,
							]
							{exp(-1/x)};
							\addplot[
								domain=-10:0,
								samples=100,
								color=red,
							]
							{0};
							\addlegendentry{$\zeta$}

							\addplot[
								domain=0:20,
								samples=100,
								color=blue,
								dashed
							]
							{1};
						\end{axis}
					\end{tikzpicture}
				\end{figure}}]
	{La fonction $\zeta$ (pas celle-là une autre) est de classe \Cont{\infty}{}{} sur \R}
	~ \\

	\begin{itemize}[label=$\star$]
		\item $\zeta_{]-\infty;0[}$ est constante donc $\zeta \in \Cont{\infty}{]-\infty;0[}{}$.
		\item $x \mapsto - \frac{1}{x} \in \mathcal{C}^\infty(]0;+\infty[,]-\infty;0[)$ et $\exp \in \Cont{\infty}{]-\infty;0[}{}$ donc, par stabilité de \Cont{\infty}{}{} par composition, $\zeta \in \Cont{\infty}{]0;+\infty[}{}$.
	\end{itemize}

	Considérons le prédicat $\mathcal{P}(\cdot)$ défini pour tout $n \in \N$ :
	\begin{equation}
		\mathcal{P} : \text{\textquotedblleft} \ \exists P_n \in \R[x] : \forall x \in \R^*, \ \zeta^{(n)} = \left\{ \begin{array}{lc}
			0                                       & \text{si } x < 0 \\
			\frac{P_n(x)}{x^{2n}} \e^{-\frac{1}{x}} & \text{si } x > 0
		\end{array} \right. \ \text{\textquotedblright}
	\end{equation}
	\begin{itemize}[label=$\star$]
		\item $\mathcal{P}(0)$ est vrai par définition de $\zeta$ en posant $P_0(x) = 1$
		\item Soit $n \in \N^*$ \fq \ tel que $\mathcal{P}$ est vrai.
		      D'une part, $\forall x \in ]-\infty;0[, \zeta^{(n)}(x) = 0$ donc
		      \begin{equation*}
			      \forall x \in ]-\infty;0[, \zeta^{(n+1)}(x) = 0
		      \end{equation*}
		      D'autre part, $\forall x \in ]0;+\infty[, \zeta^{(n)}(x) = \frac{P_n(x)}{x^{2n}} \e^{-\frac{1}{x}}$ ce qui est un produit de trois expressions dérivables. D'où :
		      \begin{equation*}
			      \begin{aligned}
				      \forall x \in ]-\infty;0[,
				      \zeta^{(n+1)}(x)
				       & = \left( P_n'(x) \frac{1}{x^{2n}} + P_n(x) \frac{-2n}{x^{2n+1}} + \frac{P_n(x)}{x^{2n}} \frac{1}{x^2} \right) \e^{-\frac{1}{x}} \\
				       & = \frac{x^2 P_n'(x) - 2nxP_n(x) + P_n(x)}{x^{2(n+1)}} \e^{-\frac{1}{x}}
			      \end{aligned}
		      \end{equation*}
		      Si bien qu'en posant $P_{n+1}(x) = x^2 P_n'(x) - 2 n x P_n(x) + P_n(x) \in \R[x]$, on obtient :
		      \begin{equation*}
			      \forall x \in ]0;+\infty[, \zeta^{(n+1)}(x) = \frac{P_{n+1}(x)}{x^{2(n+1)}} \e^{-\frac{1}{x}}
		      \end{equation*}
		      Par conséquent, $\mathcal{P}(x)$ est vrai.

		      Appliquons maintenant le théorème de prolongement du caractère \Cont{\infty}{}{}.
		      \begin{itemize}[label=$\star$]
			      \item Nous avons montré que $\zeta \in  \Cont{\infty}{\R \!\setminus\! \{0\}}{}$.
			      \item Calculons les limites à gauche et à droite de 0. Soit $k \in \N$ \fq.
			            \begin{itemize}[label=$\star\star$]
				            \item $\zeta^{(k)}$ est nulle sur $]-\infty;0[$, $\zeta^{(k)} \arrowlim{x}{0^-} 0$.
				            \item De plus, $\exists P_n \in \R[x] : \ \forall x \in ]0;+\infty[, \zeta^{(k)}(x) = \frac{P_k(x)}{x^{2k}} \e^{-\frac{1}{x}}$. Posons $u = \frac{1}{x}$, ainsi $\zeta^{(k)}(x) = u^{2k} P_k(\frac{1}{u}) \e^{-\frac{1}{x}}$ et $u \arrowlim{x}{0^+} +\infty$. \\
				                  Le théorème des croissances comparées donne $u^{2k} P_k(\frac{1}{u}) \e^{-u} \arrowlim{u}{+\infty} 0$ donc $\zeta^{(k)}(x) \arrowlim{x}{0^+} 0$.
			            \end{itemize}
		      \end{itemize}
		      Donc $\zeta \in \Cont{\infty}{\R}{\R}$.
	\end{itemize}
\end{question_kholle}

\end{document}

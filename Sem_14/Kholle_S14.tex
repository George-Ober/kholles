\documentclass{article}

\date{12 janvier 2024}
\usepackage[nb-sem=14, auteurs={Felix Rondeau}]{../kholles}

\begin{document}
\maketitle

\begin{question_kholle}{Calcul de la signature d’une transposition par dénombrement de ses inversion}
	Soient $n, i_{1}, i_{2}$ trois entiers tels que $1\leq i_{1}<i_{2}\leq n$. Observons que
	\[
		\tau_{i_{1}, i_{2}} = \left(\begin{array}{ccccccccccc}
				1 & \cdots & i_{1}-1 & i_{1} & i_{1}+1 & \cdots & i_{2}-1 & i_{2} & i_{2}+1 & \cdots & n \\
				1 & \cdots & i_{1}-1 & i_{2} & i_{1}+1 & \cdots & i_{2}-1 & i_{1} & i_{2}+1 & \cdots & n
			\end{array}\right)
	\]
	si bien que la liste des inversions de $\tau_{i_{1}, i_{2}}$ (couple $(i,j)$ tel que $i<j$ et $\tau_{i_{1}, i_{2}}(i)>\tau_{i_{1}, i_{2}}(j)$) est
	\[
		I(\tau_{i_{1}, i_{2}})=\{\underbrace{(i_{1}, i_{1}+1),(i_{1}, i_{1}+2), \dots, (i_{1}, i_{2})}_{\text{$i_{2}-i_{1}$ inversions}}, \underbrace{(i_{1}+1, i_{2}), (i_{1}+2, i_{2}), \dots, (i_{2}-1, i_{2})}_{\text{$i_{2}-i_{1}-1$ inversions}}\}
	\]
	Ainsi, $|I(\tau_{i_{1}, i_{2}})| = 2(i_{2}-i_{1})-1 \equiv 1[2]$ donc $\varepsilon(\tau_{i_{1}, i_{2}}) = (-1)^{|I(\tau_{i_{1}, i_{2}})|}=-1$, d’où toute transposition est impaire (de signature -1).
\end{question_kholle}

\begin{question_kholle}{Calculs des cardinaux du groupe symétrique $\mathcal{S}_{n}$ et du groupe alterné $\mathcal{A}_{n}$.}
	\hfill
	\begin{description}
		\item[\textit{Cardinal de $\mathcal{S}_{n}$.}] La recherche de toutes les permutations possibles de $\{1,2,\dots,n\}$ par un arbre de dénombrement montre que l’on dispose de \begin{itemize}
			      \item $n$ choix pour l’image de 1,
			      \item $(n-1)$ choix pour l’image de 2,
			      \item ...
			      \item 1 choix pour l’image de $n$,
		      \end{itemize} ce qui permet de dénombrer, par le principe des choix successifs, exactement $n(n-1)\cdots 1= n!$ permutations deux à deux distinctes.

		\item [\textit{Cardinal de $\mathcal{A}_{n}$.}] Fixons $\tau=(1,2)$. Considérons l’application
		      \[
			      \Phi:\applic{\mathcal{A}_{n}}{\mathcal{S}_{n}\setminus \mathcal{A}_{n}}{\sigma}{\tau\circ \sigma}
		      \]
		      \begin{itemize}
			      \item \textit{$\Phi$ est bien définie.}\\
			            Soit $\sigma\in \mathcal{A}_{n}$ fixée quelconque. Par propriété de morphisme de la signature,
			            \[
				            \varepsilon(\tau\circ \sigma) = \varepsilon(\tau)\times \varepsilon(\sigma) = (-1)\times (+1)= -1
			            \]
			            donc $\tau\circ \sigma\notin \mathcal{A}_{n}$. Par conséquent, $\Phi(\sigma)\in \mathcal{S}_{n}\setminus \mathcal{A}_{n}$.
			      \item \textit{$\Phi$ est bijective.} Soit $\gamma\in \mathcal{S}_{n}\setminus \mathcal{A}_{n}$ fixé quelconque. Résolvons l’éq. d’inconnue $\sigma\in \mathcal{A}_{n}$ :
			            \begin{align*}
				            \Phi(\sigma)=\gamma & \iff  \tau\circ \sigma=\gamma                                                                                                                               \\
				                                & \iff \tau\circ \tau \circ \sigma = \tau\circ \gamma                                                                                                         \\
				                                & \iff  \sigma = \tau\circ \gamma \in\mathcal{A}_{n} \quad\text{car $\varepsilon(\tau\circ \gamma)=\varepsilon(\tau)\times \varepsilon(\gamma) = (-1)^{2}=1$}
			            \end{align*}
			            d’où la bijectivité.
		      \end{itemize}
		      Ainsi, puisque $\Phi$ est une bijection,
		      \[
			      |\mathcal{A}_{n}|=|\mathcal{S}_{n}\setminus \mathcal{A}_{n}| = |\mathcal{S}_{n}| - |\mathcal{A}_{n}|
		      \]
		      d’où
		      \[
			      |\mathcal{A}_{n}| = \frac{|\mathcal{A}_{n}}{2} = \frac{n!}{2}
		      \]

	\end{description}

\end{question_kholle}

\begin{question_kholle}{Les transpositions engendrent les groupes symétriques}
	Soit $n\geq 3$. Considérons la propriété $\prop(\cdot)$ définie pour tout $k\in\iset{1,n}$ par
	\begin{multline*}
		\prop(k): \text{<<~toute permutation de $\mathcal{S}_{n}$ qui fixe au moins les éléments de l’ensemble} \\ \text{$\{1,\dots, n\}\setminus \{1,\dots, k\}$ s’écrit comme un produit de transpositions~>>}
	\end{multline*}
	\begin{itemize}
		\item Soit $\sigma\in\Sn$ une permutation qui fixe au moins tous les éléments de $\{2,\dots, n\}$. Alors, $\sigma$ étant une bijection de $\{1,\dots,n\}$ dans $\{1,\dots, n\}$, $\sigma=\mathrm{Id}$ donc $\sigma=\tau_{1,2}\circ \tau_{1,2}$. Par conséquent, $\prop(1)$ est vraie.
		\item Soit $k\in\iset{1,n-1}$ fixé quelconque tel que $\prop(k)$ est vraie.\\
		      Soit $\sigma\in\Sn$ une permutation qui fixe au moins tous les éléments de $\{1,\dots,n\setminus \{1,\dots,k+1\}\}$ (on a bien $k+1\leq n$ car $k\leq n-1$).
		      \begin{itemize}
			      \item Si $\sigma(k+1)=k+1$, alors $\sigma$ fixe les éléments de l’ensemble $\{1,\dots,n\}\setminus \{1,\dots,k\}$ or $\prop(k)$ est vraie donc $\sigma$ s’écrit comme un produit de transpositions.
			      \item Si $\sigma(k+1)\neq k+1$, puisque $\forall i\in\{k+2,\dots, n\}, \sigma(i)=i, \sigma(k+1)<k+1$.\\
			            Considérons la permutations $\sigma_{k} = \tau_{k+1,\sigma(k+1)}\circ \sigma$. Alors, $\forall i\in\{k+2,\dots,n\},\sigma(i) = i \implies \sigma_{k}(i)=i$ et de plus, $\sigma_{k}(k+1)=\tau_{k+1,\sigma(k+1)}(\sigma(k+1)) = k+1$. Par conséquent, $\sigma_{k}$ fixe les éléments de l’ensemble $\{1,\dots,n\}\setminus\{1,\dots,k\}$ or $\prop(k)$ est vraie dons $\exists p\in\N^{*}, \exists (\tau_{1},\dots, \tau_{p})$ une famille de transpositions telles que
			            \[
				            \tau_{k+1,\sigma(k+1)}\circ \sigma = \sigma_{k} = \tau_{1}\circ \cdots \circ \tau_{p}
			            \]
			            si bien qu’en composant par $\tau_{k+1,\sigma(k+1)}$ à gauche,
			            \[
				            \sigma=\tau_{k+1,\sigma(k+1)}\circ \tau_{1}\circ\cdots\circ \tau_{p}
			            \]
			            Ainsi, $\prop(k+1)$ est vraie.
		      \end{itemize}
	\end{itemize}


\end{question_kholle}

\begin{question_kholle}{Montrer que si $E$ est un ensemble fini et $f:E\longrightarrow F$, alors $f(E)$ est un ensemble fini et $|f(E)|\leq |E|$.}
	\hfill\\
	\textbf{Résultat préliminaire: si $A$ est un ensemble non vide et $f:A\longrightarrow B$ est surjective, il existe $g:B\longrightarrow A$ injective telle que $f\circ g=\mathrm{id}_{B}$.}\\
	$A$ est fini et non vide donc on peut numéroter ses éléments: $A=\{a_{1},\dots,a_{n}\}$. Posons
	\[
		g:\applic{B}{A}{b}{a_{k(b)}} \quad\text{où $k(b)=\min\{i\in\iset{1,|A|} \mid f(a_{i})=b\}$}
	\]
	\begin{itemize}
		\item Soit $b\in B$ fixé quelconque.
		      \[
			      f\circ g(b)=f(a_{k(b)}) = b \quad\text{car } k(b)\in\{i\in\iset{1,|A|}\mid f(a_{i})=b\}
		      \]
		      donc $f\circ g=\mathrm{id}_{B}$.
		\item $\mathrm{id}_{B}$ est bijective donc $f\circ g$ est injective donc $g$ est injective.
	\end{itemize}

	\vspace{2em}

	\noindent\textbf{Preuve du théorème.}
	Soit $E$ un ensemble fini, $F$ un ensemble et $f$ une fonction de $E$ dans $F$.\\
	La correstriction $f^{|f(E)}$ est surjective de $E$ dans $f(E)$ or $E$ est fini donc le lemme précédent s’applique et permer d’affirmer qu’il existe une application $g:f(E)\longrightarrow E$ injective telle que $f^{|f(E)}\circ g = \mathrm{id}_{f(E)}$.\\
	$f(E)$ s’injecte donc dans l’ensemble fini $E$ d’ou la finitude de $f(E)$.\\
	De plus, $f(E)$ est en bijection avec $g(f(E))$ donc $|f(E)|=|g(f(E))|\leq |E|$ car $g(f(E))$ est une partie de $E$.

\end{question_kholle}

\begin{question_kholle}{Dénombrer les surjections de $\iset{1,n}$ dans $\iset{1,2}$ puis de $\iset{1,n}$ dans $\iset{1,3}$}

\end{question_kholle}

\begin{question_kholle}
	[Soient $E, F$ deux ensembles finis non vides et $f : E \rightarrow F$ telle que tout élément de $F$ possède le même nombre $k \in \N^*$ d'antécédents par $f$.
		Alors $\left|F\right| = \frac{\left|E\right|}{k}$
		\begin{quotation}
			\textquotedblleft Pour compter les moutons, il faut compter les pattes puis diviser par quatre. \textquotedblright
		\end{quotation}]
	{Lemme des bergers et anagrammes de BARBARA}

	Considérons la relation binaire définie sur $E$ par :
	\begin{equation*}
		\forall (x, y) \in E^2, x \sim y
		\iff f(x) = f(y)
	\end{equation*}
	Elle est réflexive, transitive et symétrique donc c'est bien une relation d'équivalence.\\
	Soit $x \in E$ \fq. Alors
	\[
		\bar{x} = \left\{ y \in E \;|\; f(x) = f(y) \right\} = f^{-1}(\{f(x)\})
	\]
	or $f$ est surjective donc il y a autant de classes d’équivalences que d’éléments dans $F$, et ces classes sont les éléments de l’ensemble
	\[
		\{f^{-1}(\{t\}) \mid t\in F\}
	\]
	De plus, $~$ étant une relation d’équivalence, ces classes forment une partition de $E$ donc
	\[
		|E| = \sum_{t\in F}\left|f^{-1}(\{t\})\right| = k|F|
	\]
	\begin{figure}[H]
		\centering
		\begin{tikzpicture}[ele/.style={ellipse,draw,inner sep=-2pt}]

			\node (E1) at (0, 2) {$e_1$};
			\node (E2) at (0, 1) {$e_2$};
			\node (E3) at (0, 0) {$e_3$};
			\node (E4) at (0, -1) {$e_4$};
			\node (E5) at (0, -2) {$e_5$};
			\node (E6) at (0, -3) {$e_6$};
			\node (E7) at (0, -4) {$e_7$};
			\node (E8) at (0, -5) {$e_8$};

			\node (F1) at (3, 1.5) {$f_1$};
			\node (F2) at (3, -0.5) {$f_2$};
			\node (F3) at (3, -2.5) {$f_3$};
			\node (F4) at (3, -4.5) {$f_4$};


			\node (E) at (0, -6) {$E$};
			\node (F) at (3, -5.5) {$F$};

			\node (f) at (1.5, -5.5) {$f$};

			\node[draw,fit=(E) (E1) (E2) (E3) (E4) (E5) (E6) (E7) (E8), minimum width=2cm, inner sep=10pt] {} ;
			\node[draw,fit=(F) (F1) (F2) (F3) (F4),minimum width=2cm] {} ;

			\node[ele, draw,fit=(E1) (E2),minimum width=1cm] {};
			\node[ele, draw,fit=(E3) (E4),minimum width=1cm] {};
			\node[ele, draw,fit=(E5) (E6),minimum width=1cm] {};
			\node[ele, draw,fit=(E7) (E8),minimum width=1cm] {};



			\draw[->] (E1) -- (F1);
			\draw[->] (E2) -- (F1);

			\draw[->] (E3) -- (F2);
			\draw[->] (E4) -- (F2);

			\draw[->] (E5) -- (F3);
			\draw[->] (E6) -- (F3);

			\draw[->] (E7) -- (F4);
			\draw[->] (E8) -- (F4);

		\end{tikzpicture}
		\caption{Représentation schématique du lemme des bergers. Les classes d'équivalence de $\sim$ sont les ovales qui contiennent des éléments qui ont la même image par $f$. Le lemme s'applique ici car tous les éléments de $F$ ont le même nombre d'antécédents par $f$.}
	\end{figure}

	\textbf{Application aux anagrammes: exemple du mot BARBARA.} Les lettres du mot BARBARA étiquetées en $\mathrm{B_{1}A_{1}R_{1}B_{2}A_{2}R_{2}A_{3}}$, on peur construire $7!$ mots différents avec. Chacun de ces mots peut être obtenu de plusieures façon: en échangeant l’ordre des même lettres au sein de celui-ci. Comme il y a $2!$ façons d’échanger les lettres $\mathrm{B_{1}}$ et $\mathrm{B_{2}}$, autant d’échanger les lettres $\mathrm{R_{1}}$ et $\mathrm{R_{2}}$ et $3!$ façons d’échanger les lettres $\mathrm{A_{1}}$, $\mathrm{A_{2}}$ et $\mathrm{A_{3}}$, un même mot est formé $2!2!3!$ fois. On peut alors appliquer le lemme des bergers pour trouver un nombre total d’anagrammes de $\frac{7!}{2!2!3!}$.
\end{question_kholle}

\begin{question_kholle}
	[Soit $p \in \N^*$. Un $p$-partage de $E$ est un $p$-liste $(A_1, \ldots, A_p) \in \mathcal{P}(E)^p$ de parties de $E$ (éventuellement vide), deux à deux disjointes qui recouvrent $E$, i.e.
		\begin{equation*}
			\forall (i, j) \in \lient 1 ; p \rient,
			i \neq j \implies A_i \cap A_j = \emptyset
			\qquad \text{et} \qquad
			\bigcup_{i=1}^{p} A_i = E
		\end{equation*}
		Soient $(n_1, \ldots n_p) \in \N^p$ \tqs $n = n_1 + \cdots + n_p$. Un $p$-partage de $E$ de type $(n_{1},\dots,n_{p})$ est un $p$-partage $(A_{1},\dots,A_{p})$ de $E$ tel que
		\begin{equation*}
			\forall (i, j) \in \iset{1,p},|A_i| = n_i
		\end{equation*}
		Le nombre de $p$-partage de type $(n_1, \ldots, n_p)$ est :
		\begin{equation}
			\frac{n!}{\displaystyle \prod_{i=1}^{p} n_i !}
		\end{equation}
	]
	{$p$-partage d'un ensemble $E$ et leur dénombrement. Anagrammes de MISSISSIPPI.}

	Considérons les $p$-partages de type $(n_1, \ldots, n_p)$ et appliquons le principe des choix successifs :
	\begin{equation*}
		\left(
		\underbrace{A_1}_{\binom{n}{n_1} \text{ choix}},
		\underbrace{A_2}_{\binom{n}{n_2} \text{ choix}},
		\underbrace{A_3}_{\binom{n}{n_3} \text{ choix}},
		\ldots,
		\underbrace{A_p}_{\binom{n}{n_p} \text{ choix}}
		\right)
	\end{equation*}
	donc il y a
	\begin{equation*}
		\frac{ n! }{n_1! \cancel{(n-n_1)!} }
		\frac{ \bcancel{(n-n_1)!} }{n_2! \cancel{(n-n_1-n_2)!} }
		\frac{ \bcancel{(n-n_1-n_2)!} }{n_2! \cancel{(n-n_1-n_2-n_3)!} }
		\ldots
		\frac{ \bcancel{(n-(n_1+\ldots+n_{p-1})!} }{n_p! \underbrace{(n_1+\ldots+n_p)!}_{=0!} }
	\end{equation*}
	Donc, au total, il y a $\displaystyle\frac{n!}{n_1! n_2! \ldots n_p!}$ $p$-partages.\\

	\vspace{1em}\noindent\textbf{Application aux anagrammes: exemple du mot MISSISSIPPI.} Ce mot comporte 11 lettres (1 M, 4 I, 4 S et 2 P). L’ensemble des anagrammes est en bijection avec l’ensemble des p-partages de $\iset{1,11}$ du type $(1,4,4,2)$. Par exemple, l’anagramme \textit{MISSSSIIIPP} correspond au p-partage $(\{1\}, \{2,7,8,9\}, \{3,4,5,6\}, {10,11})$. Par conséquent leur nombre est le nombre de p-partages d’un tel type, soit $\displaystyle\frac{11!}{1!4!4!2!}$.
\end{question_kholle}

\begin{question_kholle}[La formule de Van der Monde est la suivante: pour tout $(n,p)\in\N^{2}$,
		\[
			\forall k\in\N,\;\sum_{i=0}^{k}\binom{n}{i}\binom{p}{k-i} = \binom{n+p}{k}
		\]]{Énoncé et démonstration combinatoire de la formule de Van der Monde.}
	\begin{itemize}
		\item Si $k>n+p$. On a
		      \[
			      \binom{n+p}{k}=0 \quad\text{par définition}
		      \]
		      et
		      \[
			      \sum_{i=0}^{k}\binom{n}{i}\binom{p}{k-i} = \sum_{i=0}^{n}\binom{n}{i}\binom{p}{k-i} + \sum_{i=n+1}^{k}\binom{n}{i}\binom{p}{k-i} = 0+0 = 0
		      \]
		\item  Sinon, $k\leq n+p$. Considérons $E$ un ensemble de cardinal $n+p$ et $A$ une partie de cette ensemble, de cardinal $n$.\\
		      Dénombrons les parties de $E$ de cardinal $k$ en fonction du cardinal de leur intersection avec $A$:
		      \begin{align*}
			      \mathcal{P}_{k}(E)=\bigsqcup_{i=0}^{n}\{B\in \mathcal{P}_{k}(E) \mid |B\cap A| = i\}
			       & = \bigsqcup_{i=0}^{n}\{C\sqcup D\in \mathcal{P}_{k}(E) \mid C\in \mathcal{P}_{i}(A), D\in \mathcal{P}_{k-i}(\overline{A})\}
		      \end{align*}
		      or l’application
		      \[
			      \applic{\mathcal{P}_{i}(A)\times \mathcal{P}_{k-i}(\overline{A})}{\{C\sqcup D \mid C\in \mathcal{P}_{i}(A), D\in\mathcal{P}_{k-i}(\overline{A})\}}{(C,D)}{C\sqcup D}
		      \]
		      est bijective donc
		      \begin{align*}
			      |\{B\in\mathcal{P}_{k}(E) \mid |B\cap A| = i\}| & = |\mathcal{P}_{i}(A) \times \mathcal{P}_{k-i}(\overline{A})|   \\
			                                                      & = |\mathcal{P}_{i}(A)| \times |\mathcal{P}_{k-i}(\overline{A})| \\
			                                                      & = \binom{n}{i} \times \binom{p}{k-i}
		      \end{align*}
		      d’où
		      \begin{align*}
			      |\mathcal{P}_{k}(E)| & = \sum_{i=0}^{n}\left|\left\{B\in \mathcal{P}_{k}(E)\mid |B\cap A|=i\right\}\right|   \\
			                           & = \sum_{i=0}^{n}\binom{n}{i}\binom{p}{k-i} = \sum_{i=0}^{n}\binom{n}{i}\binom{p}{k-i}
		      \end{align*}
	\end{itemize}

\end{question_kholle}



\end{document}

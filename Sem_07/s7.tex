\documentclass{article}

\usepackage[french]{babel}
\usepackage{amsmath}
%\usepackage{stmaryrd}
\usepackage{amssymb}
\usepackage[T1]{fontenc}
\usepackage[a4paper,top=2cm,bottom=2cm,left=3cm,right=3cm,marginparwidth=1.75cm]{geometry}
\usepackage{graphicx}
\usepackage[colorlinks=true, allcolors=blue]{hyperref}
\usepackage{amsthm}
\usepackage{amsfonts}
\usepackage{hyperref}
\usepackage{tikz}
\usepackage{pgfplots}
\pgfplotsset{compat=1.15}
\usepackage{mathrsfs}
\usetikzlibrary{shapes.geometric}
\usetikzlibrary{arrows.meta,arrows}
\hypersetup{
    colorlinks=true,
    linkcolor=blue,
    filecolor=magenta,      
    urlcolor=cyan,
    pdftitle={Khôlles Maths 6-16},
    pdfpagemode=FullScreen,
    }

\title{Khôlles : Maths}
\author{Kylian Boyet, George Ober}

\begin{document}
\maketitle
\tableofcontents

\newpage

\section{Calcul de $\int_0^{2\pi}e^{imt}dt$ en fonction de $m\in \mathbb{Z}$. En déduire qu'une fonction polynomiale nulle sur un cercle centré en l'origine a tous ses coefficients nuls.}
\begin{proof}
  Soit $m\in \mathbb{Z}$ et $n\in \mathbb{N}^*$, on pose :
  $$
    I_m = \int_0^{2\pi}e^{imt}dt.
  $$
  Bien sûr, $I_0=2\pi$ et pour tout $m\neq 0$, $I_m=0$. \\
  Soit alors le polynôme $P \in \mathbb{C}_n [z]$, de coefficients les $a_k\in \mathbb{R}, \text{ pour }  k\in [\![ 0,n ]\!]$ , s'annulant sur le cercle $\Omega (O,r)$, centré en l'origine $(O)$, et de rayon $r\in \mathbb{R}^*$. De fait, soit $t\in \mathbb{R}$ et $s\in \mathbb{Z}$, on a :

  \begin{eqnarray*}
    P\left( re^{it}\right)  & = &  0 \\
    P \left( re^{it}\right)  e^{-st}  & = &  0 \\
    \int_0^{2\pi} P\left( re^{it}\right)e^{-st}dt  & = &  0 \\
    \sum_{k\in [\![0,n]\!]} a_k r^k\int_0^{2\pi} e^{ikt}e^{-st}dt  & = &  0 \\
    \sum_{k\in [\![0,n]\!]} a_k r^k\int_0^{2\pi} e^{i(k-s)t}dt  & = &  0.
  \end{eqnarray*}
  Mais pour tout polynôme :
  \[
    \sum_{k\in [\![ 0,n ]\!]} a_k r^k\int_0^{2\pi} e^{i(k-s)t}dt  =  \left\{ \begin{array}{cl}
      2\pi a_s r^s & \text{si } \ s \in [\![ 0,n ]\!]                     \\
      0            & \text{si } \ s\in \mathbb{Z}\backslash [\![ 0,n ]\!]
    \end{array} \right.
  \]
  En particulier, en combinant les deux :
  \[
    \forall s \in  [\![ 0,n ]\!], \quad 2\pi a_s r^s = 0 \implies a_s =0,
  \]
  car $2\pi r^s \neq 0$, d'où $P=\widetilde{0}$ sur tout $\mathbb{C}$.
\end{proof}

%\newpage

\section{Technique de l'intégration par parties.}
\begin{proof}
  Soient $u,v\in \mathcal{D}^1 (I,J)$, deux fonctions définies, continues et dérivables sur $I$ à valeurs dans $J$, deux intervalles non-triviaux de $\mathbb{R}$. Soient $a,b\in \mathbb{R}$, $a<b$ et $I=[a,b]$. De manière générale :
  \[
    \int_{[a,b]}(uv)'(x)dx = [(uv)(x)]_a^b.
  \]
  Mais aussi :
  \[
    \int_{[a,b]}(uv)'(x)dx = \int_{[a,b]}(u'v)(x) + (v'u)(x)dx = \int_{[a,b]}(u'v)(x)dx + \int_{[a,b]}(v'u)(x)dx.
  \]
  Donc finalement :
  \[
    \int_{[a,b]}(u'v)(x)dx + \int_{[a,b]}(v'u)(x)dx = [(uv)(x)]_a^b.
  \]
  Ce qui s'écrit aussi,
  \[
    \int_{[a,b]}(u'v)(x)dx = [(uv)(x)]_a^b -\int_{[a,b]}(v'u)(x)dx.
  \]
\end{proof}

\section{Technique du changement de variable.}
\begin{proof}
  Soient $(f',\varphi') \in \mathcal{C}^0 (I,\mathbb{R}) \times \mathcal{C}^0 (J,I) $, deux fonctions définies et continues sur $I$ et $J$ à valeurs dans $\mathbb{R}$ et $I$. $I,J$ deux intervalles non-triviaux de $\mathbb{R}$. Soient $a,b\in \mathbb{R}$, $a<b$ et $J=[a,b]$. D'après le théorème fondamental de l'analyse, $f'$ et $\varphi'$ admettent des primitives sur $I$ et $J$, on note $f$ une primitive de $f'$ et $\varphi$ une primitive de $\varphi'$, ainsi il est possible de calculer de la manière suivante (les notations ont un sens) :
  \[
    \int_{\varphi ([a,b])} f'(t)dt = [f(t)]_{\varphi ([a,b])}.
  \]
  On suppose $\varphi(b) > \varphi(a)$, quitte à les inverser, ainsi :
  \[
    \int_{[\varphi(a),\varphi(b)]} f'(t)dt = f(\varphi(b))-f(\varphi(a)) = f\circ\varphi (b) - f\circ\varphi (a) = [f\circ\varphi (t)]_{[a,b]} = \int_{[a,b]} (f'\circ\varphi)(t)\varphi'(t)dt
  \]
\end{proof}

\section{Montrer que, pour $f$ $T$-périodique sur $\mathbb{R}$, pour tout $a\in\mathbb{R}$, $\int_a^{a+T}f(t)dt = \int_0^Tf(t)dt$}.
\begin{proof}
  \text{plus tard}
\end{proof}

\section{Résolution de $a_1y'+a_0y=b$ sur $J$ par équivalence avec la méthoded du facteur intégrant sous l'hypothèse $\forall t \in I, a_1(t)\neq 0$. On précisera les hypothèses analytiques sur $a_0, a_1$ et $b$.}
\end{document}
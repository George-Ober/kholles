\documentclass{article}

\date{10 novembre 2023}
\usepackage[nb-sem=7, auteurs={Kylian Boyet, George Ober, Elijah Gaillard, Hugo Vangilluwen (relecture)}]{../kholles}

\begin{document}

\maketitle

\begin{question_kholle}{Résolution de l'ED  $\forall t \in J, y' + a_{0}(t) = b(t)$ par équivalences avec la méthode du facteur intégral.}
	
	Pour cette preuve, il est nécessaire de supposer $a_{0}$ et $b$ continus.
	Ainsi, on note $A$ la primitive de $a_{0}$ définie sur J. 
  
  	$$\begin{array}{ccl}
  	f : J \to \mathbb{C} \text{ est sol. de } y' + a_{0} = b \text{ sur J} 
  	& \iff & \left\{ \begin{array}{l}
  		f \in \mathcal{D}^1 (J, \mathbb{K}) \\
  		f' + a_{0} = b \text{ sur J }
  	\end{array}  \right.                                                                                \\\\
  	& \iff & \left\{ \begin{array}{l}
  		f \in \mathcal{D}^1 (J, \mathbb{K}) \\
  		f'  e^A + \underset{(= A')}{a_{0}} f  e^A = b  e^A \text{ sur J } (e^A \text{ est le facteur intégrant})
  	\end{array}\right.        
  \end{array} $$
  
  Notons B une primitive de $be^A$ definie sur J (car $b$ et $e^A$ sont continues)
  
  $$\begin{array}{ccl}  	
   & \iff &  \left\{ \begin{array}{l}
  			f \in \mathcal{D}^1 (J, \mathbb{K}) \\
  			(fe^A)' = B' \text{ sur J}
  	\end{array} \right. \\\\
  	 & \iff &  \left\{ \begin{array}{l}
  		f \in \mathcal{D}^1 (J, \mathbb{K}) \\
  		(fe^A - B)' = 0 \text{ sur J}
  	\end{array} \right. \\\\
   & \iff &  \left\{ \begin{array}{l}
  	f \in \mathcal{D}^1 (J, \mathbb{K}) \\
  	\exists \lambda \in \mathbb{K} : fe^A - B = \lambda
  \end{array} \right. \\\\
  & \iff & \exists \lambda \in \mathbb{K} : f = \lambda e^{-A} + B e^{-A} \\\\
  & \iff & f \in \left\lbrace \lambda e^{-A} + B e^{-A} \right\rbrace 
  \end{array} $$
  
  Ainsi $\mathcal{S} = \underset{\text{sol. particluière}}{Be^{-A}} + \underset{\text{droite vectorielle des sol. de l'EDLH}}{\left\lbrace \lambda e^{-A} | \lambda \in \mathbb{K}\right\rbrace }$
 
\end{question_kholle}

\begin{question_kholle}{Théorème de résolution des EDLH d'ordre 2 à coefficients constants complexes.}

\end{question_kholle}

\begin{question_kholle}{Caractérisation des fonctions exponentielles et de la fonction nulle par la propriété de dérivabilité en 0 et celle de morphisme de (\R, +) dans (\C, $\times$).}
  
\end{question_kholle}

\begin{question_kholle}{Preuve de l’expression des solutions réelles des EDL homogènes d’ordre 2 à coefficients constants réels dans le cas $\Delta < 0$ (en admettant la connaissance de l’expression des solutions à valeurs complexes des EDLH2 à coeff. constants).}
  Notons $\Sol_{H, \C}$ et $\Sol_{H, \R}$ les ensembles des solutions complexes et réelles de l'équation différentielle, puisque nous nous plaçons dans le cas $\Delta < 0$ et $\alpha \pm i \beta$ les deux racines complexes conjuguées.
  $$
  \Sol_{H, \C} =
  \left\{
  \begin{array}{l}
  	\R \to \C \\
  	t \mapsto \lambda e^{(\alpha + i \beta) t}  + \mu e^{(\alpha - i \beta)t}
  \end{array}
  \middle\vert  (\lambda, \mu) \in \C ^2 \right\}
  $$
  
  Montrons que $\forall f \in \Sol_{H ,\C}, \Re(f) \in  \Sol_{H ,\R}$\\
  Soit $f \in \Sol_{H ,\C}$ fq.
  $$f \in \mathcal D^2(\R, \C) \implies \Re(f) \in \mathcal D^2(\R, \R)$$
  Et, de plus, par morphisme additif de \Re
  $$
  a_2\Re(f)'' + a_1\Re(f)' + a_0\Re(f) = \Re( a_2 f'' + a_1 f' + a_0 f) = 0
  $$
  D'où, avec $f:t \mapsto e^{(\alpha + i \beta)t}$; $\Re(f(t)) = \Re(e^{(\alpha + i \beta)t}) = e^{\alpha t } \cos (\beta t)$. Qui appartient donc à $\Sol_{H, \R}$\\
  En suivant le même raisonnement pour $\Im(f)$, $(t \mapsto e^\alpha \sin(\beta t)) \in \Sol_{H, \R}$
  
  
  Ainsi, par combinaison linéaire (qui se base sur le principe de superposition),
  $$
  \left\{
  \begin{array}{l}
  	\R \to \R \\
  	t \mapsto \lambda e^{\alpha t } \cos (\beta t)   + \mu e^{\alpha t } \sin (\beta t)
  \end{array}
  \middle\vert  (\lambda, \mu) \in \R ^2 \right\}
  \subset \Sol_{H ,\R}
  $$
  
  Réciproquement, soit $ f \in \Sol_{H ,\R}$ fq. Puisque $\R \subset \C$,  $ f \in \Sol_{H ,\C}$.
  
  $$
  \exists (a, b) \in \C^2 : f \left| \begin{array}{l}
  	\R \to \C \\
  	t \mapsto a e^{(\alpha + i \beta) t}  + b e^{(\alpha - i \beta)t}
  \end{array}\right.$$
  
  Or, puisque toutes les valeurs de $f$ sont réelles, en notant $(a_r, a_i, b_r, b_i)$ les parties réelles et imaginaires respectives de $a$ et $b$.
  \begin{align*}
  	\forall t \in \R, f(t) & = \Re(f(t))                                                                                                     \\
  	& = \Re(a e^{(\alpha + i \beta) t}  + b e^{(\alpha - i \beta)t})                                                  \\
  	& = \Re((a_r + i a_i) e^{(\alpha + i \beta) t}  + (b_r + i b_i) e^{(\alpha - i \beta)t})                          \\
  	& = a_r \cos(\beta t)e^\alpha - a_i\sin(\beta t)e^\alpha + b_r \cos(\beta t)e^\alpha + b_i \sin(\beta t) e^\alpha \\
  	& = (a_r + b_r) \cos(\beta t) e^\alpha + (b_i - a_i) \sin(\beta t) e^\alpha
  \end{align*}
  Ainsi,
  $$f\in \left\{
  \begin{array}{l}
  	\R \to \R \\
  	t \mapsto \lambda e^{\alpha t } \cos (\beta t)   + \mu e^{\alpha t } \sin (\beta t)
  \end{array}
  \middle\vert  (\lambda, \mu) \in \R ^2 \right\}
  $$
  Ce qui conclut la preuve par double inclusion.

\end{question_kholle}

 \pagebreak
 
 \begin{question_kholle}[
 	Considérons le problème de Cauchy suivant :
 	$$\left\{ \begin{array}{l}
 		a_{2}y''+a_{1}y'+a_{0}y = b \text{ sur } J \\
 		y(t_{0}) = \alpha_{0}                      \\
 		y'(t_{0}) = \alpha_{1}
 	\end{array} \right. \text{ où } (\alpha_{0}, \alpha_{1}) \in \mathbb{K}^{2}, t_{0} \in J, (a_{0}, a_{1}, a_{2}) \in \mathbb{K}^{2} \times \mathbb{K}^{*}, b \in \mathcal{F}(J, \mathbb{K})$$
 	Si $b$ est continu sur $J$, alors ce problème de Cauchy admet une unique solution définie sur $J$.
 	]
 	{Existence et unicité d'une solution au problème de Cauchy pour les EDL d'ordre 2 à coefficients constants et second membre continu sur $I$ (cas complexe puis cas réel).}
 	
 	\textbf{Cas 1. } $\mathbb{K} = \mathbb{C}$ \\
 	Nous savons que sous l'hyphothèse de continuité de $b$ sur $J$, les solutions de (EDL2) définies sur $J$ constituent le plan affine $S$ :
 	$$S = \left\{ \lambda f_{1} + \mu f_{2} + s | (\lambda, \mu) \in \mathbb{C}^{2} \right\}$$
 	où $s$ est une solution particulière de (EDL2), $(f_{1}, f_{2})$ sont deux solutions de (EDLH2) qui engendrent $S_{h}$. On a : \\
 	
 	$$\begin{array}{ccl}
 		f : J \to \mathbb{C} \text{ est sol. du pb de Cauchy }
 		& \iff & \left\{ \begin{array}{l}
 			f \text{ sol de (EDL2) sur } J \\
 			f(t_{0}) = \alpha_{0}          \\
 			f'(t_{0}) = \alpha_{1}
 		\end{array}  \right.                                                                                \\\\
 		& \iff & \left\{ \begin{array}{l}
 			f \in S               \\
 			f(t_{0}) = \alpha_{0} \\
 			f'(t_{0}) = \alpha_{1}
 		\end{array}\right.                                                                                         \\\\
 		& \iff & \exists (\lambda, \mu) \in \mathbb{C}^{2}: \left\{ \begin{array}{l}
 			f = \lambda f_{1} + \mu f_{2} + s                                  \\
 			\lambda f_{1}(t_{0}) + \mu f_{2}(t_{0}) + s(t_{0}) = \alpha_{0}    \\
 			\lambda f'_{1}(t_{0}) + \mu f'_{2}(t_{0}) + s'(t_{0}) = \alpha_{1} \\
 		\end{array} \right. \\\\
 		& \iff & \exists (\lambda, \mu) \in \mathbb{C}^{2}: \left\{ \begin{array}{l}
 			f = \lambda f_{1} + \mu f_{2} + s                                  \\
 			\lambda f_{1}(t_{0}) + \mu f_{2}(t_{0}) = \alpha_{0} - s(t_{0})    \\
 			\lambda f'_{1}(t_{0}) + \mu f'_{2}(t_{0}) = \alpha_{1} - s'(t_{0}) \\
 		\end{array} \right. \\\\
 	\end{array} $$
 	On en déduit donc que $(\lambda, \mu)$ doit être solution d'un système linéaire $(2,2)$. On a une unique solution si et seulement si les déterminant de ce système est nul. \\
 	Explicitons alors le déterminant de ce système, que l'on notera $D$.
 	$$D = \left|
 	\begin{array}{cc}
 		f_{1}(t_{0})  & f_{2}(t_{0})  \\
 		f'_{1}(t_{0}) & f'_{2}(t_{0}) \\
 	\end{array}
 	\right| = f_{1}(t_{0}) \cdot f'_{2}(t_{0}) - f_{2}(t_{0}) \cdot f'_{1}(t_{0}) $$
 	Notons $\Delta$ le discriminant de l'équation caractéristique de (EDL2) ($a_{2}r^{2} + a_{1}r^{1} + a_{0} = 0$). On distingue alors deux cas selon la nullité ou non de $\Delta$. Traitons d'abord le cas $\Delta \neq 0$. On peut choisir :
 	$$ f_{1}(t_{0}) = e^{r_{1}t_{0}} \text{ et } f_{2}(t_{0}) = e^{r_{2}t_{0}}$$
 	$$ f'_{1}(t_{0}) = r_{1}e^{r_{1}t_{0}} \text{ et } f'_{2}(t_{0}) = r_{2}e^{r_{2}t_{0}}$$
 	Donc (en sachant que $\Delta \neq 0 \Rightarrow r_{1} \neq r_{2}$):
 	$$ D = e^{r_{1}t_{0}} \cdot r_{2}e^{r_{2}t_{0}} - r_{1}e^{r_{1}t_{0}} \cdot e^{r_{2}t_{0}} = (r_{2} - r_{1}) \cdot e^{r_{1}t_{0} + r_{2}t_{0}} \neq 0$$
 	
 	Dans le deuxième cas, on a $\Delta = 0$ ; on peut alors prendre :
 	$$ f_{1}(t_{0}) = e^{r_{0}t_{0}} \text{ et } f_{2}(t_{0}) = t_{0}e^{r_{0}t_{0}}$$
 	Ainsi :
 	$$ D = e^{r_{0}t_{0}} \left(r_{0}t_{0}e^{r_{0}t_{0}} + e^{r_{0}t_{0}} \right) - r_{0}e^{r_{0}t_{0}} \times t_{0}e^{r_{0}t_{0}} = e^{2r_{0}t_{0}} \neq 0$$
 	On remarque alors que, dans les deux cas, $D \neq 0$, donc le système $(2, 2)$ étudié admet une unique solution, donc il existe un unique couple $(\lambda, \mu)$ le vérifiant d'où l'unicité et existence d'une solution au problème de Cauchy.
 	\newline\newline
 	
 	\textbf{Cas 2. } $\mathbb{K} = \mathbb{R}$ \\
 	$(a_{0}, a_{1}, a_{2}) \in \mathbb{R}^{2} \times \mathbb{R}^{*},(\alpha_{0}, \alpha_{1}) \in \mathbb{R}^{2}, b \in C^{0}(J, \mathbb{R})$
 	\newline
 	\textbf{Existence :} Puisque $\mathbb{R} \subset \mathbb{C}$, le problème de Cauchy admet, dans $\mathbb{R}$, une solution à valeurs complexes $g$. Posons $f = \Re(g)$ et montrons que $f$ est une solution réelle du problème de Cauchy. \\
 	\begin{itemize}
 		\item[$\star$] $g \in \mathcal{D}^{2}(J, \mathbb{C}) \text{ donc } f \in \mathcal{D}^{2}(J, \mathbb{R})$
 		\item[$\star$] $g$ vérifie $a_{2}g'' + a_{1}g' + a_{0}g = b$ sur $J$ donc en prenant $\Re(\cdot)$ :
 		$$\begin{array}{ccl}
 			\Re(a_{2}g'' + a_{1}g' + a_{0}g = b) = \Re(b)
 			& \iff & a_{2}\Re(g'') + a_{1}\Re(g') + a_{0}\Re(g) = b \\\\
 			& \iff & a_{2}f'' + a_{1}f' + a_{0}f = b \text{ sur } J
 		\end{array}$$
 		\item[$\star$] $f(t_{0}) = \Re(g(t_{0})) = \Re(\alpha_{0}) = \alpha_{0}$
 		\item[$\star$] $f'(t_{0}) = \Re(g(t_{0}))' = \Re(g'(t_{0})) = \Re(\alpha_{1}) = \alpha_{1}$
 	\end{itemize}
 	Donc $f$ est une solution réelle définie sur $J$ au problème de Cauchy.
 	\newline
 	
 	\textbf{Unicité : }Soient $f_{1}$ et $f_{2}$ deux fonctions à valeurs réelles solutions du problème de Cauchy ci-dessus fixées quelconques : puisque $\mathbb{R} \subset \mathbb{C}$, $f_{1}$ et $f_{2}$ sont des fonctions à valeurs dans $\mathbb{C}$ solutions du même problème de Cauchy; or il y a unicité de la solution au problème de Cauchy dans les fonctions à valeurs complexes, donc $f_{1} = f_{2}$ dans $\mathcal{F}(J, \mathbb{C})$, donc $f_{1} = f_{2}$ dans $\mathcal{F}(J, \mathbb{R})$.
 \end{question_kholle}
 
\end{document}



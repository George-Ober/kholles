\documentclass{article}

\date{20 Janvier 2024}
\usepackage[nb-sem=15, auteurs={Hugo Vangilluwen, Ober George, Felix Rondeau}]{../kholles}

\begin{document}
\maketitle

\begin{question_kholle}
	[Soient $g$ une fonction définie sur $\mathcal{D}_g \subset \R$ et $f$ une fonction définie sur $\mathcal{D}_f \subset \R$ telle que $f(\mathcal{D}_f) \subset \mathcal{D}_g$.
		Si $\left. \begin{array}{cc}
				g \text{ admet une limite } \ell \in \overline{\R} \text{ en } b \in \overline{\mathcal{D}_g} \\
				f \text{ admet } b \text{ comme limite en } a \in \overline{\mathcal{D}_f}
			\end{array} \right\}$
		alors $g \circ f$ admet $\ell$ comme limite en $a$.]
	{Théorème de composition des limites}

	Traitons le cas où $\ell \in \R$, $a \in \R$ et $b \in \R$. \\
	Soit $\varepsilon \in \R_+^*$ fixé quelconque. \\
	Appliquons la définition de la convergence de $g(y)$ vers $\ell$ en $b$ pour cet $\varepsilon$ :
	\begin{equation*}
		\exists \eta_g \in \R_+^* : \forall y \in \mathcal{D}_g, | y - b | \leqslant \eta_g \implies | g(y) - \ell | \leqslant \varepsilon
	\end{equation*}
	Appliquons la définition de la convergence de $f(x)$ vers $b$ en $a$ pour cet $\eta_g$ :
	\begin{equation*}
		\exists \eta_f \in \R_+^* : \forall x \in \mathcal{D}_f, | x - a | \leqslant \eta_f \implies | f(x) - b | \leqslant \eta_g
	\end{equation*}
	Soit $x \in \mathcal{D}_{f}$ fixé quelconque tel que $ | x - a | \leqslant \eta_{f} $.\\
	Alors, $ | f(x) - b | \leqslant \eta_g $ d'où $ | g(f(x)) - \ell | \leqslant \varepsilon $. Ce qui est exactement la définition de la convergence de $g\circ f$ vers $\ell$ en $a$.
\end{question_kholle}

\begin{question_kholle}[{
				Soit $f : \mathcal{D}_{f}\to \R$, $a \in \overline{\mathcal{D}_{f}}$ et $\ell \in \overline{\R}$
				$$
					f \text{ admet } \ell \text{ pour limite en } a \iff \left\{ \begin{array}{ll}
						\text{pour toute suite } u \in \mathcal{D}_{f}^{\N}, \\
						\text{si } u \text{ tend vers } a, \text{ alors } f(u) \text{ tend vers } \ell
					\end{array}\right.
				$$
			}]{Caractérisation séquentielle de la limite}

	\hfill\\
	\begin{itemize}[label=$\star$]
		\item Supposons que $f$ admet $\ell$ pour limite en $a$. Traitons le cas $a \in \R$ et $\ell \in \R$.
		      Soit $u \in \mathcal{D}_{f}^{\N}$ fixée quelconque telle que $u$ tend vers $a$.
		      Soit $\varepsilon>0$ fixé quelconque. Appliquons la définition de la limite de $f$ en $a$ pour $\varepsilon$~:

		      $$
			      \exists \eta >0 : \forall x \in \mathcal{D}_{f}, \lvert x - a \rvert \leqslant \eta \implies \lvert f(x) - \ell \rvert  \leqslant \varepsilon
		      $$
		      Fixons un tel $\eta$ et appliquons la définition de la convergence de $u$ pour $\varepsilon \leftarrow \eta$~:
		      $$
			      \exists N \in \N : \forall n \geqslant N, \lvert u_{n} - a \rvert  \leqslant \eta
		      $$
		      Fixons un tel $N$.
		      Soit $n \in \N$ tel que $n \geqslant N$.
		      On a alors
		      $$
			      \lvert u_{n} - a \rvert  \leqslant \eta \implies \lvert f(u_{n}) - \ell \rvert \leqslant \varepsilon
		      $$
		      Ce qui montre la convergence de $(f(u_{n}))_{n \in \N}$ vers $\ell$.

		\item Réciproquement, raisonnons par contraposée et montrons l'implication suivante
		      $$
			      \mathrm{non}(f \text{ admet }\ell \text{ pour limite en }a) \implies \underbrace{ \mathrm{non} \left\{ \begin{array}{ll}
					      \text{pour toute suite } u \in \mathcal{D}_{f}^{\N}, \\
					      \text{si } u \text{ tend vers } a, \text{ alors } f(u) \text{ tend vers } \ell
				      \end{array}\right. }_{ \exists u \in \mathcal{D}_{f}^{\N} : u \text{ tend vers }a \text{ et } f(u) \text{ ne tend pas vers }\ell }
		      $$

		      Supposons donc

		      \begin{align}
			      \mathrm{non}(f \text{ admet }\ell \text{ pour limite en }a) & \iff \mathrm{non}(\forall \varepsilon>0, \exists \eta >0 : \forall x \in \mathcal{D_{f}}, \lvert x -a\rvert \leqslant \eta \implies \lvert f(x) - \ell \rvert\leqslant \varepsilon  ) \\
			                                                                  & \iff \exists \varepsilon>0 : \forall \eta >0, \exists x \in \mathcal{D}_{f} : \lvert x-a \rvert \leqslant \eta \text{ et } \lvert f(x) - \ell \rvert  > \varepsilon \label{S13:Q2:1}
		      \end{align}


		      Fixons donc un tel $\varepsilon$ et construisons une suite $u \in \mathcal{D}_{f}^{\N}$ telle que $u$ tend vers $a$ et $f(u)$ ne tend pas vers $\ell$.

		      Soit $n \in \N$ fixé quelconque. Appliquons l'hypothèse \eqref{S13:Q2:1} pour $\eta \leftarrow \frac{1}{2^{n}}$~:
		      $$
			      \exists x_{n} \in \mathcal{D}_{f} : \lvert x_{n} - a \rvert  \leqslant \frac{1}{2^{n}} \text{ et } \lvert f(x_{n}) - \ell \rvert  > \varepsilon
		      $$
		      En relâchant le caractère fixé de $n$, on a constuit une suite $(x_{n})_{n \in \N} \in \mathcal{D}_{f}^{\N}$ telle que
		      $$
			      \forall n \in \N, \lvert x_{n} - a \rvert \leqslant \underbrace{ \frac{1}{2^{n}} }_{ \xrightarrow[n \to \infty]{} 0}
		      $$
		      Donc $(x_{n})_{n \in \N}$ tend vers $a$.

		      La suite vérifie aussi
		      $$
			      \forall n \in \N, \lvert f(x_{n}) - \ell \rvert > \varepsilon
		      $$
		      ce qui montre que $(f(x_{n}))_{n \in \N}$ ne converge pas vers $\ell$ car

		      \begin{align*}
			      \text{non}((f(x_{n}))_{n \in \N} \text{ converge vers } \ell) & \iff \mathrm{non}(\forall \varepsilon_{1}>0, \exists N \in \N : \forall n \in \N, n \geqslant N \implies \lvert f(x_{n}) - \ell \rvert \leqslant \varepsilon_{1}) \\
			                                                                    & \iff \exists \varepsilon_{1}>0: \forall N \in \N, \exists n  \in \N : n\geqslant N \text{ et } \lvert f(x_{n}) - \ell \rvert  > \varepsilon_{1}
		      \end{align*}

		      Ce qui est immédiat en posant $\varepsilon_{1} = \varepsilon$ et pour n'importe quel $N$ en posant $n = N$.
	\end{itemize}
\end{question_kholle}

\begin{question_kholle}[]{Deux stratégies pour prouver qu'une fonction n'admet pas de limite en un point}
	\hfill\\
	\begin{itemize}[label=$\bullet$]
		\item Soit en exhibant une suite $(y_n)_{n \in \N} \in \mathcal D_f^{\N}$ qui tend vers $a$ et telle que $(f(y_n))_{n \in \N}$ n'admet pas de limite en $a$.

		      Par exemple $f(x) = \cos \frac 1 x$ n'a pas de limite en zéro : observons que la suite $y = \left( \frac{1}{n\pi} \right)_{n \in \N^{*}}$ converge vers $0$ tandis que la suite $f(y)=((-1)^{n})_{n \in \N^{*}}$ diverge.

		\item Soit en exhibant deux suites $y, z$ qui tendent vers $a$ et telles que $(f(y_{n}))_{n \in \N}$ et $(f(z_{n}))_{n \in \N}$ admettent deux limites différentes.

		      Par exemple, pour montrer que $f(x) = \sin x$ n'admet pas de limite en $+\infty$, il suffit d'observer que les suites $y = (n \pi)_{n \in \N}$ et $z = (2n\pi + \frac{\pi}{2})_{n \in \N}$ tendent vers $+\infty$ et ont respectivement pour suites images $\tilde{0}$ et $\tilde{1}$ qui convergent respectivement vers $0$ et $1$.
	\end{itemize}

\end{question_kholle}

\begin{question_kholle}[{
				Soient $f$ et $g$ définies sur $\mathcal{D}$ et $a \in \overline{\mathcal{D}}$. Si
				\begin{itemize}[label=$\bullet$]
					\item $f \leqslant g$ sur un voisinnage de $a$
					\item $f$ et $g$ admettent une limite finie en $a$
				\end{itemize}
				alors $\displaystyle \lim_{ x \to a }f(x) \leqslant \lim_{ x \to a }g(x)$
			}]{Passage à la limite dans une inégalité}
	\hfill\\
	\begin{itemize}[label=$\star$]
		\item \textbf{En utilisant la caractérisation séquentielle de la limite}

		      Traitons le cas $a = +\infty$.

		      Posons $\ell_{f} \in\R$ et $\ell_{g} \in \R$ les limites finies respectives de $f$ et $g$.

		      Par définition de $a \in \overline{\mathcal{D}}$, $\exists (a_{n})_{n \in \N} \in \mathcal{D}^{\N} : \lim_{ n \to \infty }a_{n} = a = +\infty$.

		      Par définition de voisinnage de $a$ en $+\infty$, $\exists A \in \R : \forall x \in \mathcal{D}, x\geqslant A \implies f(x) \leqslant g(x)$.

		      Fixons un tel $A$ et appliquons la définition de la divergence de $(a_{n})_{n \in \N}$ vers $+\infty$ pour $A$~:
		      $$
			      \exists N \in \N : \forall n \geqslant N, a_{n} \geqslant A
		      $$
		      Fixons un tel $N$. On a alors
		      $$
			      \forall n \geqslant N, a_{n} \geqslant A \implies f(a_{n}) \leqslant g(a_{n})
		      $$
		      Donc par passage à la limite dans l'inégalité pour les suites
		      $$
			      \lim_{ n \to \infty } f(a_{n}) \leqslant \lim_{ n \to \infty } g(a_{n})
		      $$
		      donc par caractérisation séquentielle de la limite
		      $$
			      \lim_{ x \to a=+\infty}  f(x) \leqslant \lim_{ x \to a  = +\infty} g(x)
		      $$

		\item \textbf{En utilisant le caractère local de la limite}
		      Tout d'abord $f \leqslant g$ sur un voisinnage de $a$, donc $g - f \geqslant 0$ sur un voisinnage de $a$ donc $g-f = \lvert g-f \rvert$ sur un voisinnage de $a$.
		      Or,
		      $$
			      \left. \begin{array}{ll}
				      \lim_{ x \to a }g(x) = \ell_{g} \\
				      \lim_{ x \to a } f(x) = \ell_{f}
			      \end{array}\right\}
			      \implies \lim_{ x \to a } \lvert g(x) - f(x) \rvert  = \lvert \ell_{g} - \ell _{f} \rvert
		      $$
		      Donc, avec le caractère local de la limite, puisque $g-f$ et $|g-f|$ coïncident sur un voisinnage de $a$,
		      $$
			      \lim_{ x \to a } g(x) - f(x) = \lim_{ x \to a } \lvert g(x) - f(x) \rvert  = \lvert \ell_{g} - \ell_{f} \rvert
		      $$
		      Or,  on a aussi $\lim_{ x \to a }g(x) - f(x) = \ell_{g} - \ell_{f}$.
		      Donc
		      $$
			      \ell_{g} - \ell_{f} = \lvert \ell_{g} - \ell_{f} \rvert \geqslant 0 \implies \ell_{g} \geqslant \ell_{f}
		      $$
	\end{itemize}
\end{question_kholle}

\begin{question_kholle}[{
	Soit $f$ une fonction croissante définie sur $]a, b[$ avec $(a, b) \in \overline{\R}^{2}, a<b$.
			\begin{itemize}[label=$\bullet$]
				\item Si $f$ est majorée, alors $f$ admet une limite finie en $b$ qui vaut $\lim_{ x \to b }f(x) = \sup f(]a, b[)$.
				\item Si $f$ n'est pas majorée, alors $f$ tend vers $+\infty$ en $b$.
			\end{itemize}
			}]{Limite de fonctions monotones sur un segment.}
	\hfill\\
	\begin{itemize}[label=$\star$]
		\item Supposons que $f$ est majorée sur $]a, b[$. L'ensemble $f(]a, b[)$ est une partie de $\R$, non vide et majorée, donc admet une borne supérieure $S \in \R$.
						      Montrons que $\lim_{ x \to b }f(x) = S$.

						      Soit $\varepsilon>0$ fixé quelconque. On veut construire un $\eta>0$ tel que $\forall x \in ]b-\eta, b[$, $\lvert f(x) - S \rvert \leqslant \varepsilon$.
						      D'après la caractérisation de la borne supérieure par les epsilon appliquée pour $\varepsilon$, $$\exists y_{\varepsilon} \in f(]a, b[) : S - \varepsilon < y_{\varepsilon} \leqslant \varepsilon$$
					      Or, $y_{\varepsilon} \in f(]a, b[) \implies \exists x_{\varepsilon}\in ]a, b[ : y_{\varepsilon} = f(x_{\varepsilon})$.\\
						      Posons $\eta = b - x_{\varepsilon} > 0$ et vérifions qu'il convient.
						      Soit $x \in ]b - \eta, b[$ fixé quelconque.
		      on a $$
			      b-\eta < x \implies b- (b - x_{\varepsilon})< x \implies x_{\varepsilon} < x \implies \underbrace{ f(x_{\varepsilon}) }_{ y_{\varepsilon} } \leqslant f(x)
		      $$
		      De plus, $f(x) \leqslant S$ par définition de la borne supérieure, donc
		      $$
			      S - \varepsilon < y_{\varepsilon} \leqslant f(x) \leqslant S
		      $$
		      Donc $\lvert f(x) - S \rvert \leqslant \varepsilon$ ce qui prouve la convergence.

		\item Supposons que $f$ n'est pas majorée sur $]a, b[$.
						      On veut montrer que $f$ tend vers $+\infty$, autrement dit que
					      $$
					      \forall A \in \R, \exists \eta >0 : \forall x \in ]b- \eta , b[, f(x)\geqslant A
					      $$
					      Soit $A \in \R$ fixé quelconque.
					      $f$ n'est pas majorée, donc $\exists x_{0} \in ]a, b[ : f(x_{0}) \geqslant A$.\\
						      Posons $\eta = b - x_{0} > 0$.
						      Soit $x \in ]b-\eta, b[$ fixé quelconque.
		      $$
			      b-\eta< x \implies b-(b-x_{0})< x \implies x_{0}<x \implies f(x_{0}) \leqslant f(x)
		      $$
		      Donc $f(x) \geqslant f(x_{0})\geqslant A$, donc $\forall x \in ]b-\eta, b[, f(x) \geqslant A$.
		      Ainsi $f$ tend vers $+\infty$ en $b$.
	\end{itemize}
\end{question_kholle}

\begin{question_kholle}
	[{Soit une fonction continue $f : [a;b] \rightarrow \R$ avec $(a,b) \in \R^2$ et $a < b$. \\
	Si $f(a)f(b) \leqslant 0$ alors $\exists c \in [a;b] : f(c) = 0$.}]
	{Théorème des valeurs intermédiaires}

	La démonstration repose sur la technique de la dichotomie.

	\begin{figure}[!h]
		\centering
		\tikzmath{ \labTVI = 12; } % la longueur du segment [a;b] dans la démonstration du TVI
		\begin{tikzpicture}
			\draw (0,0) node[anchor=north] {a} -- (\labTVI,0) node[anchor=north] {b};
			\foreach \x in {0,...,\labTVI} {
					\draw (\x,0.1) -- (\x,-0.1);
				};
			\draw[purple] (0,0) to[out angle=80, in angle=240, curve through={(\labTVI/5,2) (\labTVI/3,-1) (\labTVI*3/5,0.5)}] (\labTVI,0);

			\draw[teal] (\labTVI/2,0.1) -- (\labTVI/2,-0.1);
			\draw (\labTVI/2,0) node[teal, anchor=north] {$b_1$};
			\draw[teal] (\labTVI/4,0.1) -- (\labTVI/4,-0.1);
			\draw (\labTVI/4,0) node[teal, anchor=north] {$a_2$};
			\draw[teal] (\labTVI*3/8,0.1) -- (\labTVI*3/8,-0.1);
			\draw (\labTVI*3/8,0) node[teal, anchor=north] {$b_3$};
			\draw[teal] (\labTVI*5/16,0.1) -- (\labTVI*5/16,-0.1);
			\draw (\labTVI*5/16,0) node[teal, anchor=north] {$b_4$};
		\end{tikzpicture}
	\end{figure}


	\noindent Soient $a,b,f$ de tels objets. Procédons à la construction des suites $(a_n)_{n\in\N}, (b_n)_{n\in\N}, (c_n)_{n\in\N}$.\\
	Posons $a_0 = a$, $b_0 = b$ et $c_0 = \frac{a+b}{2}$ (le milieu du segment $[a;b]$).\\
	Nous avons, par hypothèse $f(a_0)f(b_0) \leqslant 0$.
	Soit $n \in \N$ fixé quelconque.\\
	Supposons les trois suites construites au rang $n$ telles que $f(a_n)f(b_n) \leqslant 0$ et $c_n = \frac{a_n+b_n}{2}$.
	\begin{itemize}
		\item Si $f(a_n)f(c_n) \leqslant 0$, posons $$\left\{ \begin{array}{lcl}
				      a_{n+1} & = & a_n                       \\
				      b_{n+1} & = & c_n                       \\
				      c_{n+1} & = & \frac{a_{n+1}+b_{n+1}}{2}
			      \end{array} \right.$$
		\item Sinon $f(a_n)f(c_n) > 0$. Comme $f(a_n)f(b_n) \leqslant 0$, on a en multipliant par $f(a_{n})f(b_{n})$ $$f(a_n)^2 f(b_n) f(c_n) \leqslant 0 \quad \text{donc}\quad f(b_n)f(c_n) \leqslant 0$$
		      Posons $$\left\{ \begin{array}{lcl}
				      a_{n+1} & = & c_n                       \\
				      b_{n+1} & = & b_n                       \\
				      c_{n+1} & = & \frac{a_{n+1}+b_{n+1}}{2}
			      \end{array} \right.$$
	\end{itemize}
	Ainsi, nous avons bien construits $a_{n+1}, b_{n+1}, c_{n+1}$ telles que $f(a_{n+1})f(b_{n+1}) \leqslant 0$ et ${ c_{n+1} = \frac{a_{n+1}+b_{n+1}}{2} }$.
	Par récurrence immédiate, $(a_n)_{n\in\N}$ est croissante, $(b_n)_{n\in\N}$ est décroissante et ${ \forall n \in \N, b_n - a_n = \frac{b-a}{2^n} }$ d'où $b_n - a_n \arrowlim{n}{+\infty} 0$.
	Les suites $a$ et $b$ sont donc adjacentes.\\
	D'après le théorème des suites adjacentes, elles convergent vers la même limite. Notons la $c$.\\
	D'après le bonus de ce même théorème, $\forall n \in \N, a_n \leqslant c \leqslant b_n$ donc pour $n = 0$, $a \leqslant c \leqslant b$. Ainsi, $c \in [a;b]$.\\
	Par ailleurs, $\forall n \in \N, f(a_n)f(b_n) \leqslant 0$. Par continuité de $f$ sur $[a;b]$ donc en $c$, $f(a_n) \arrowlim{n}{+\infty} f(c)$ et $f(b_n) \arrowlim{n}{+\infty} f(c)$. Ainsi, par passage à limite dans l'inégalité,
	\begin{equation*}
		f(c) \times f(c) \leqslant 0
	\end{equation*}
	Or $f(c)^2 \geqslant 0$, d'où $f(c)^2 = 0$. Ainsi,
	\begin{equation*}
		f(c) = 0
	\end{equation*}
	Donc $c$ est un point fixe.

\end{question_kholle}

\begin{question_kholle}
	[{L'image d'un segment par une fonction continue sur ce segment est un segment : soient $(a, b) \in \R^2$ tels que $a < b$ et $f: [a, b] \to \R$. Si $f \in \mathcal{C}^0([a, b], \R)$ alors $\exists (x_{1}, x_{2}) \in \R^2 : f([a, b]) = [f(x_{1}), f(x_{2})]$.}]
	{Théorème de Weierstraß}
	\hfill\\
	\begin{itemize}
		\item \emph{Étape 1} Montrons que $f([a, b])$ est majoré.

		      Par l'absurde, supposons que $f([a, b])$ n'est pas majoré

		      Alors \begin{equation}\label{eq:1}
			      \forall A \in \R, \exists x \in [a, b] : f(x) > A
		      \end{equation}

		      Soit $n \in \N$ fixé quelconque.
		      Appliquons (\ref{eq:1}) pour $A \leftarrow n$:
		      $\exists x \in [a, b] : f(x) > n$, et fixons un tel $x$ que l'on note $x_{n}$
		      Nous venons de créer la suite $(x_{n})_{n \in \N} \in [a, b]^{\N}$ qui vérifie:


		      $$
			      \left.
			      \begin{array}{ll}
				      \forall n \in \N, f(x_{n}) \geqslant n \\
				      \lim_{ n \to \infty } n =  +\infty
			      \end{array}
			      \right\} \underbrace{ \implies }_{ \text{théorème de divergence par minoration} } f(x_{n}) \xrightarrow[n \to +\infty]{} + \infty
		      $$


		      $(x_{n})_{n \in \N}$ est bornée (à valeurs dans $[a, b]$) donc, selon le théorème de Bolzanno-Weierstraß:
		      $$
			      \exists \ell \in \R : \exists \varphi : \N \to \N : \text{strict. croissante tel que } (x_{\varphi(n)})_{n \in \N} \text{ tend vers } \ell
		      $$
		      Donc, en passant à la limite : $\forall n \in \N, a \leqslant x_{\varphi(n)} \leqslant b \implies a \leqslant \ell \leqslant b \implies \ell \in [a, b]$

		      Par continuité de $f$ sur $[a, b]$, donc en $\ell$, $(f(x_{\varphi(n)}))_{n \in \N}$ converge vers $f(\ell)$.

		      Or $$
			      \left\{ \begin{array}{ll}
				      (f(x_{\varphi(n)}))_{n \in \N} \text{ est une sous suite de } (f(x_{n}))_{n \in \N} \\
				      f(x_{n}) \xrightarrow[n \to + \infty]{} + \infty
			      \end{array}\right.$$

		      donc $(f(x_{\varphi(n)}))_{n \in \N}$, tend vers $+ \infty$, ce qui est absurde, donc $f$ est majorée.

		      On fait de même pour la minoration.

		\item  \emph{Étape 2:} Montrons que $f([a, b])$ admet un pge et un ppe.

		      Montrons donc que $f([a, b])$ admet une borne sup, qui, puisque c'est une valeur atteinte, deviendra un max.

		      $$
			      f([a, b]) \text{ est } \left\{ \begin{array}{ll}
				      \text{ une partie de } \R           \\
				      \text{ non vide car contient } f(a) \\
				      \text{majorée d'après l'étape 1}
			      \end{array}\right.
		      $$

		      $f([a, b])$ admet donc une borne supérieure $\sigma$.

		      Appliquons la caractérisation séquentielle de la borne supérieure:
		      $$
			      \exists (y_{n})_{n \in \N}, \in f([a, b])^{\N} : (y_{n}) \text{ converge vers } \sigma
		      $$
		      $$
			      \forall n \in \N, y_{n} \in f([a, b]) \implies \exists x_{n} \in [a, b] : f(x_{n} ) = y_{n}
		      $$
		      Fixons un tel $x_{n}$ pour tout $y_{n}$.
		      On a donc construit $(x_{n})_{n \in \N} \in [a, b]^{\N} : f(x_{n}) \xrightarrow[n \to +\infty]{} \sigma$

		      De plus, $(x_{n})$ est bornée (à valeurs dans $[a, b]$) donc, selon le théorème de Bolzanno-Weierstraß:
		      $$
			      \exists \ell \in \R : \exists \varphi : \N \to \N : \text{strict. croissante tel que } (x_{\varphi(n)})_{n \in \N} \text{ tend vers } \ell
		      $$
		      Donc, en passant à la limite : $\forall n \in \N, a \leqslant x_{\varphi(n)} \leqslant b \implies a \leqslant \ell \leqslant b \implies \ell \in [a, b]$


		      Par continuité de $f$ sur $[a, b]$, donc en $\ell$, $(f(x_{\varphi(n)}))_{n \in \N}$ converge vers $f(\ell)$.

		      Or,
		      $$
			      \left\{ \begin{array}{ll}
				      (f(x_{\varphi(n)}))_{n \in \N} \text{ est une sous suite de } (f(x_{n}))_{n \in \N} \\
				      f(x_{n}) \xrightarrow[n \to + \infty]{} \sigma
			      \end{array}\right.$$

		      Par unicité de la limite, $\sigma = f(\ell)$.

		      On montre de même qu'il existe $\ell' \in [a, b]: f(\ell') = \inf f([a, b])$

		      Ainsi, $f(\ell) = \max f([a, b])$ et $f(\ell') = \min f([a, b])$



		\item \emph{Étape 3:}
		      Montrons que $f([a, b]) = [f(\ell'), f(\ell)]$.

		      Par la construction précédente, $\forall y \in f([a, b]), y \in [f(\ell'), f(\ell)]$.

		      Ainsi, $f([a, b]) \subset [f(\ell'), f(\ell)]$.

		      Réciproquement, l'image par la fonction continue $f$ du segment $[a, b]$ qui est un intervalle est un intervalle:

		      $$
			      \left.
			      \begin{array}{ll}
				      f([a,b]) \text{ est un intevalle} \\
				      f(\ell) \in f([a, b])             \\
				      f(\ell') \in f([a, b])
			      \end{array}
			      \right\}
			      \implies
			      [f(\ell'), f(\ell)] \subset f([a,b])
		      $$

		      D'où $[f(\ell'), f(\ell)] = f([a,b])$

	\end{itemize}
\end{question_kholle}

\begin{question_kholle}
	[Soit $f$ une fonction continue et strictement monotone définie sur un intervalle $I$. Alors \begin{propositions}
			\item $f$ est une bijection de $I$ dans $f(I)$
			\item $f^{-1}$ est une fonction strictement monotone et continue sur $f(I)$.
		\end{propositions}]
	{Théorème de la bijection}
	\hfill\\
	\begin{itemize}[label=$\vartriangleright$]
		\item \textbf{Résultat préliminaire.} Soit $f$ une fonction monotone définie sur un intervalle $I\subset\R$.
		      Supposons que $f$ est croissante (il suffit d’appliquer ce résultat à $-f$ pour prouver l’autre cas). Soit $x_{0}\in I$ fixé quelconque.\\
		      \begin{itemize}[label=$\star$]
			      \item
			            Supposons que $x_{0}$ est un point intérieur à $I$. Alors, $\exists\eta\in\R_{+}^{*}: [x_{0}-\eta,x_{0}+\eta]\subset I$.\\
			            La fonction $f$ est croissante sur $[x_{0}-\eta,x_{0}[$ et majorée par $f(x_{0})$ donc $f$ admet une limite finie à gauche $\ell_{g}$ en $x_{0}$.\\
			            De même, $f$ étant croissante sur $]x_{0}, x_{0}+\eta]$, elle admet une limite finie à droite $\ell_{d}$ en $x_{0}$.\\
			            De plus,
			            \begin{equation*}
				            f(x_{0}-\eta)\leq \ell_{g} \leq f(x_{0})\leq \ell_{d} \leq f(x_{0}+\eta)
			            \end{equation*}
			            Supposons que $\ell_{g}<f(x_{0})$. Montrons alors que $y_{0}=\frac{\ell_{g}+f(x_{0})}{2}$ ne possède aucun antécédent par $f$ ce qui contredit le fait que $f(I)$ est un intervalle car
			            \begin{equation*}
				            \left(f(x_{0}-\eta), f(x_{0})\right)\in f(I)^{2} \implies  \left[f(x_{0}-\eta), f(x_{0})\right]\subset f(I)
			            \end{equation*}
			            en effet,
			            \begin{itemize}
				            \item si $x\in I$ vérifie $x<x_{0}$, alors $f(x)\leq \sup f(I\cap \left]-\infty,x_{0}\right[)=\ell_{g}<y_{0}$
				            \item si $x\in I$ vérifie $x\geq x_{0}$, alors $f(x)\geq f(x_{0})>y_{0}$
			            \end{itemize}
			            par conséquent, $y_{0}\notin f(I)$.\\
			            Ainsi, $\ell_{g}=f(x_{0})$ et on montre de même que $f(x_{0})=\ell_{d}$ si bien que nous pouvons conclure que $f$ est continue en $x_{0}$.\\
			      \item Supposons à présent que $x_{0}$ est un bord de $I$. Il suffit d’adapter la preuve ci-dessus en ne considérant que l’intervalle contenant $I$ à choisir entre $[x_{0},+\infty[$ et $]-\infty,x_{0}]$.
		      \end{itemize}

		\item \textbf{Preuve du théorème.} Soit de tels objets. La preuve de la surjectivité est triviale car on se limite à $f(I)$. Celle de l’injectifité vient de la stricte monotonie de $f$. Montrons donc le second point.
		      \begin{itemize}
			      \item $f$ est continue sur l’intervalle $I$ donc $J=f(I)$ est un intervalle.
			      \item $f$ est bijective et monotone donc $f^{-1}$ est monotone, et de plus, $f^{-1}(I)=J$ est un intervalle donc (résultat précédent), $f^{-1}$ est continue sur $J$.
		      \end{itemize}
	\end{itemize}
\end{question_kholle}
\end{document}

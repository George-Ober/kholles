\documentclass{article}

\date{3 Mars 2024}
\usepackage[nb-sem=19, auteurs={Hugo Vangilluwen, George Ober, Kylian Boyet}]{../kholles}
 \begin{document}
	\maketitle
\begin{question_kholle}
		[Pour une matrice $A \in \mathcal{M}_{(n,p)}(\K)$, la matrice transposée est définit :
		{\begin{equation*}
			\forall (k, l) \in [\![1,p]\!] \!\times\! [\![1,n]\!], \ \left[A^T\right]_{kl} = A_{lk}
		\end{equation*}}
		Formellement, la transposition est une application de $\mathcal{M}_{(n,p)}(\K)$ dans $\mathcal{M}_{(p,n)}(\K)$.]
		{$\left(A \times B\right)^T = B^T \times A^T$}
		
		Soit $(A, B) \in \mathcal{M}_{(n,p)}(\K) \times \mathcal{M}_{(p,q)}(\K)$. \\
		$\left(A \times B\right)^T \in \mathcal{M}_{(q,n)}(\K)$. Soit $(i, j) \in [\![1,q]\!] \!\times\! [\![1,n]\!]$.
		\begin{equation*}
			\begin{aligned}
				\left[ \left(A \times B\right)^T \right]_{i,j}
				&= \left[A \times B\right]_{j,i} \\
				&= \sum_{k=1}^{p} A_{j,k} \times_\K B_{k,i} \\
				&= \sum_{k=1}^{p} B_{k,i} \times_\K A_{j,k} \\
				&= \sum_{k=1}^{p} \left[B^T\right]_{i,k} \times_\K \left[A^T\right]_{k,j} \\
				&= \left(B^T\right) \times \left(A^T\right)
			\end{aligned}
		\end{equation*}
	\end{question_kholle}
	
	\begin{question_kholle}
		[Le symbole de Kronecker est définit de la manière suivante :
		\begin{equation*}
			\forall (x, y) \in \R^2, \delta_{xy} = \left\{ \begin{matrix}
				0 \text{ si } x \neq y \\
				1 \text{ si } x = y
			\end{matrix} \right.
		\end{equation*}
		La matrice $E^{i,j} \in \mathcal{M}(n, p)(\K)$ avec {$(i, j) \in [\![ 1, n ]\!] \!\times\! [\![ 1, p ]\!]$} ne possède que des coefficients nuls sauf le coefficient de la $i^{e}$ ligne et $j^{e}$ colonne qui vaut 1. Formellement :
		{\begin{equation*}
			\forall (r, s) \in [\![ 1, n ]\!] \times [\![ 1, p ]\!], \
			\left[E^{i,j}\right]_{rs} = \delta_{ir} \delta_{js}
		\end{equation*}}]
		{Calculer $E^{i,j} \times E^{k,l}$ en fonction de $i$, $j$, $k$, $l$ et des symboles de Kronecker}
				
		Calculons $E^{i,j}(n,p) \times E^{k,l}(p,q)$.
	 
		Soient $(r, s) \in [ \! [ 1, n] \!] \times [ \! [ 1, q ] \!]$ fq
		
		\begin{align*}
			\left[ E^{i,j} \times E^{k,l} \right] _{rs}
			& = \sum_{t = 1}^{n}E^{i,j}_{r,t} E^{k,l}_{t,s} \\
		 	& =\sum_{t = 1}^{n} \delta_{ir} \delta_{jt} \delta_{kt} \delta_{ls} \\
		 	& = \delta_{jk} \delta_{ir} \delta_{ls} \\
		 	& = \delta_{jk} \left[ E^{i,l} \right] _{rs}
		\end{align*}
		
		Donc $E^{i,j} \times E^{k,l} = \delta_{jk} E^{i,l}$.
		
		
		Ainsi, pour le calcul de $(E^{i,j})^{2}$, $q \leftarrow n$, $k \leftarrow i$, $l \leftarrow j$.
		
		\begin{align*}
			(E^{i,j})^{2} = \delta_{ji} E^{i,j} = \left\{ 
			\begin{array}{ll}
			  E^{i,j} \text{ si } i = j  \\ 
			  0_{n,p} \text{ si } i \neq j 
			\end{array}
			\right.
		\end{align*}
	
	\end{question_kholle}
	
	\begin{question_kholle}
		{Les matrices triangulaires supérieures forment un sous-anneau de $\mathcal{M}_n(\K)$}
		
		$\mathcal{T}_n^+(\K) \subset \mathcal(M)_n(\K)$ et $(\mathcal{M}_n(\K), +, \times)$ est un anneau. \\
		$\mathcal{T}_n^+(\K) \neq \emptyset$ car $I_n \in \mathcal{T}_n^+(\K)$ ($I_n$ est le neutre multiplicatif de $\mathcal{M}_n(\K)$). \\
		Soient $(A, B) \in \mathcal{T}_n^+(\K)^2$ . \\
		Soient $(i, j) \in [\![1,n]\!]^2$ $\text{ tels que }$ $i > j$.
		\begin{equation*}
			(A - B)_{i,j}
			= \underbrace{A_{i,j}}_{=0 \text{ car } A \in \mathcal{T}_n^+(\K)} - \underbrace{B_{i,j}}_{=0 \text{ car } B \in \mathcal{T}_n^+(\K)}
			= 0
		\end{equation*}
		Donc, $A - B \in \mathcal{T}_n^+(\K)$.
		
		\begin{equation*}
			\begin{aligned}
				(A \times B)_{i,j}
				&= \sum_{k=1}^{n} A_{i,k} \times_\K B_{k,j} \\
				&= \sum_{k=1}^{j} \underbrace{A_{i,k}}_{=0 \text{ car } i > j \geqslant k \text{ et } A \in \mathcal{T}_n^+(\K)} \times_\K B_{k,j}
				+ \sum_{k=j+1}^{n} A_{i,k} \times_\K \underbrace{B_{k,j}}_{=0 \text{ car } k > j \text{ et } B \in \mathcal{T}_n^+(\K)} \\
				&= 0
			\end{aligned}
		\end{equation*}
		Donc, $A \times B \in \mathcal{T}_n^+(\K)$.
	\end{question_kholle}
 
 \begin{question_kholle}
    []
    {Si $A$ est une matrice d'ordre $n$ et $\lambda$ un scalaire non nul d'un corps, alors la transposée de $A$ et $\lambda A$ sont inversibles aussi.}
    Soient $A,\lambda \in \mathcal{GL}_n(\mathbb{K})\times \mathbb{K}^*$, avec $\mathbb{K}$ un corps. \\
    Par définition, il existe $B\in \mathcal{GL}_n(\mathbb{K})$ tel que $AB=BA=I_n$. Ainsi : 
    \[
    (AB)^T = I_n^T \ \iff \ B^TA^T = I_n,
    \]
    donc $A^T$ admet un inverse à gauche, $B^T$, donc un inverse tout court et donc $A^T$ est inversible (on notera que $A^T$ reste dans les matrices d'ordre $n$). De même, 
    \[
    \lambda AB = \lambda I_n \ \iff \ (\lambda A)B = \lambda I_n \ \iff \ (\lambda A) \left(\frac{1}{\lambda}B \right) = I_n, 
    \]
    car les scalaires commutent avec toutes les matrices. Ainsi, $\lambda A$ admet un inverse à droite, donc un inverse tout court, donc est inversible, d'inverse $\frac{1}{\lambda}B$. Concluant la preuve.    
 \end{question_kholle}

 \begin{question_kholle}
    []
    {Si $N$ est une matrice d'ordre $n$ nilpotente, alors $I_n + \lambda N$ est inversible pour tout $\lambda$, scalaire d'un corps.}
    Soient $N$ une matrice d'ordre $n$ à coefficient dans $\mathbb{K}$, un corps, nilpotente, d'indice de nilpotence $k$ (un entier naturel donc) et $\lambda \in \mathbb{K}$. Calculons : 
    \[
    I_n^{2k+1} + (\lambda N)^{2k+1} = I_n^{2k+1} - (- \lambda N)^{2k+1} = (I_n + \lambda N)\sum_{i=0}^{2k}(-\lambda N)^i =  (I_n + \lambda N)\sum_{i=0}^{k-1}(-\lambda N)^i,
    \]
    car $\lambda N$ commute avec $I_n$, or le membre de gauche est égal à $I_n$ car $2k+1 > k$, donc $I_n + \lambda N$ est inversible à droite, donc inversible tout court, d'inverse $\sum_{i=0}^{k-1}(-\lambda N)^i$. Ce qui conclut la preuve.
 \end{question_kholle}
\end{document}
\documentclass{article}

\date{15 février 2025}
\usepackage[nb-sem=19, auteurs={Hugo Vangilluwen, George Ober, Kylian Boyet, Félix Rondeau}]{../kholles}
\begin{document}
\maketitle

\begin{question_kholle}
	[Une famille est liée si et seulement si l'un de ses vecteurs est une combinaison linéaires d'autres vecteurs de la famille.
		\begin{equation}
			(x_i)_{i \in I} \text{ est liée}
			\iff \exists i_0 \in I : \exists (\lambda_i)_{i \in I\setminus\{i_0\}} \in \K^{\left( I \setminus \{i_0\} \right)} :
			x_{i_0} = \sum_{\substack{i \in I \\ i \neq i_0}} \lambda_i \ldotp x_i
		\end{equation}]
	{Caractérisation d'une famille liée}

	Supposons que $(x_i)_{i \in I}$ est liée.
	Par définition,
	\[\displaystyle \exists (\mu_i) \K^{(I)} :
		\left\{ \begin{array}{ccc}
			\sum_{i \in I} \mu_i x_i & = 0_E                 \\
			(\mu_i)_{i \in I}        & \neq (0_\K)_{i \in I}
		\end{array} \right.\]
	Donc $\exists i_0 \in I : \mu_{i_0} \neq 0_\K$. Fixons un tel $i_0$. \\
	\[
		\mu_{i_0} x_{i_0} + \sum_{i \in I \setminus \{i_0\}} \mu_i x_i = 0_E
	\]
	Or $\mu_{i_0} \neq 0$, donc
	\[
		\displaystyle x_{i_0} = \sum_{i \in I \setminus \{i_0\}} \left( \mu_{i_0}^{-1} \times (-\mu_i) \right) \ldotp x_i
	\]
	En posant $\lambda_i = \mu_{i_0}^{-1} \times (-\mu_i)$, on obtient
	\[
		x_{i_0} = \displaystyle \sum_{i \in I \setminus \{i_0\}} \lambda_i \ldotp x_i
	\]
	Supposons maintenant qu’il existe $i_0 \in I$ et $(\lambda_i)_{i \in I\setminus\{i_0\}} \in \K^{\left( I \setminus \{i_0\} \right)}$ tels que
	\[
		x_{i_0} = \sum_{\substack{i \in I \\ i \neq i_0}} \lambda_i \ldotp x_i
	\]
	Alors $-x_{i_0} + \sum_{\substack{i \in I \\ i \neq i_0}} \lambda_i \ldotp x_i = 0_E$.
	Posons $\mu_{i_0} = - 1_\K$ et $\forall i \in I \!\setminus\! \{i_0\}, \mu_i = \lambda_i$.
	Ainsi, $(\mu_i)_{i \in I} \in \K^{(I)}$ et $\sum_{i \in I} \mu_i \ldotp x_i = 0_\K$. Or $\mu_{i_0} \neq 0_\K$ donc $(\mu_i)_{i \in I} \neq (0_\K)_{i \in I}$. \\
	Ainsi, $(\mu_i)_{i \in I}$ est liée.
\end{question_kholle}

\begin{question_kholle}[{
				Soient $(E,+, \cdot)$ un \ev et $A$ une partie de $E$. Notons $\mathcal{A}$ l’ensemble des sous-espaces vectoriels de $E$ contenant $A$. En ordonnant (partiellement) $\mathcal{P}(E)$ par l’inclusion, l’ensemble $\mathcal{A}$ admet un plus petit élément, noté $\Vect_{E}(A)$, et appelé \textbf{le sous-espace vectoriel engendré par $A$ dans $E$}.}]{Définition et existence du sous-espace engendré par une partie d’un espace vectoriel.}
	Posons $\mathcal{F}=\{$F$ \text{ sous-espace vectoriel de } E \mid A\subset F\}$ et $\displaystyle V=\bigcap_{F\in \mathcal{F}}F$ (qui est bien défini car $\mathcal{F}\neq \emptyset$ puisque $E\in\mathcal F$). Montrons que $V=\min \mathcal{F}$ pour l’inclusion.
	\begin{itemize}
		\item $V\in \mathcal{F}$ car $V$ est une intersection de sous-espaces vectoriels de $E$ donc est un sous-espace vectoriel de $E$, et
		      \[
			      \forall F\in \mathcal{F}, A\subset F \implies  A\subset \bigcap_{F\in \mathcal{F}}F \implies A\subset V
		      \]
		\item Soit $F_{0}\in \mathcal{F}$ fixé quelconque.
		      \[
			      V=\bigcap_{F\in \mathcal{F}} F = F \cap \left(\bigcap_{F\in \mathcal{F}\setminus F_{0}}\right)\subset F_{0}
		      \]
		      donc $V$ est plus petit que tous les autres éléments de $\mathcal{F}$ pour l’inclusion.
	\end{itemize}
\end{question_kholle}

\begin{question_kholle}[{Le sous-espace vectoriel engendré dane $E$ par la famille $(a_{i})_{i\in I}$ est
	\[
		\Vect_{E}\{a_{i} \mid i\in I\} = \left\{\sum_{i\in I}\lambda_{i}a_{i} \;\middle |\; (\lambda_{i})_{i\in I}\in\K^{(I)}\right\}
	\]
	On note cet ensemble $W$ pour la démonstration.}]{Description explicite du sous-espace vectoriel engendré par une partie.}
	\hfill\\
	\begin{itemize}[label=$\vartriangleright$]
		\item \textit{$W$ est un sous-espace vectoriel de $E$.}
		      \begin{itemize}[label=$\star$]
			      \item $W\subset E$ et $E$ est un \ev.
			      \item $W\neq \emptyset$ car $0_{E}\in W$ (pour $(\lambda_{i})_{i\in I}\leftarrow (O_{\K})_{i\in I}$).
			      \item Soient $(x,y)\in W^{2}$ et $\lambda\in\K$ fixé quelconque. Il existe $(\lambda_{i})$ et $(\mu_{i})$ deux familles presque nulles de \K telles que
			            \[
				            x= \sum_{i\in I}\lambda_{i} a_{i} \quad\text{et}\quad y=\sum_{i\in I}\mu_{i} a_{i}
			            \]
			            Ainsi,
			            \[
				            x+y = \sum_{i\in I}(\lambda \times \lambda_{i} + \mu_{i})\cdot a_{i}
			            \]

		      \end{itemize}
	\end{itemize}
\end{question_kholle}

\begin{question_kholle}{Stabilité de la liberté d’une famille par adjonction d’un vecteur n’appartenant pas au sous-espace qu’elle engendre.}
\end{question_kholle}

\begin{question_kholle}{Équivalence $\mathcal{F}$ base, tout vecteur se décompose de manière unique dans $\mathcal{F}$, $\mathcal{F}$ générarice minimale et $\mathcal{F}$ libre maximale.}
\end{question_kholle}

\begin{question_kholle}
	[Soit $f \in \mathcal{L}_\K(E, F)$.
		\begin{equation}
			\begin{aligned}
				\ker f & = \left\{ x \in E \;|\; f(x) = 0_F \right\} = f^{-1} (\{0_F\}) \\
				\Im f  & = \left\{ y \in F \;|\; \exists x \in E : f(x) = y \right\}
			\end{aligned}
		\end{equation}
		Nous démontrerons le résultat plus général suivant :
		\begin{propositions}
			\item $f(E')$ est un \sev de $F$.
			\item $f^{-1}(F')$ est un \sev de $E$.
		\end{propositions}
	]
	{Le noyau et l'image d'une application linéaire sont des \sevs}

	$(i)$ $0_E \in E'$ et $f(0_E) = 0_F$ donc $0_F \in f(E')$ d'où $f(E') \neq \emptyset$ \\
	Soit $(\alpha, \beta, y, y') \in \K^2 \times f(E')^2$ \fqs. \\
	Par définition, $\exists (x, x') \in E'^2 : f(x) = y \wedge f(x') = y$.
	\begin{equation*}
		\begin{aligned}
			\alpha y + \beta y'
			 & = \alpha f(x) + \beta f(x')                                                                     \\
			 & = f( \alpha x + \beta x' ) \quad \text{ car } f \in \mathcal{L}_\K(E, F)                        \\
			 & \in f(E') \quad \text{ car } \alpha x + \beta x' \in E' \text{ puisque } E' \text{ est un \sev}
		\end{aligned}
	\end{equation*}
	Donc $f(E')$ est un \sev.

	$(ii)$ $0_F \in F'$ et $f(0_E) = 0_F$ donc $0_E \in f^{-1}(F')$ d'où $f(F') \neq \emptyset$ \\
	Soit $(\alpha, \beta, x, x') \in \K^2 \times f^{-1}(F')^2$ \fqs. \\
	Par définition, $\exists (y, y') \in F'^2 : f(x) = y \wedge f(x') = y$. \\
	Or $F'$ est \sev donc $\alpha y + \beta y' \in F'$. $f \in \mathcal{L}_\K(E, F)$ d'où $f(\alpha x + \beta x') = \alpha y + \beta y'$. Donc $\alpha x + \beta x' \in f^{-1}(F')$. \\
	Ainsi, $f^{-1}(F')$ est un \sev.

	En appliquant pour $E' = E$ et $F' = \{0_F\}$, nous obtenons que $\ker f$ et $\Im f$ sont des \sevs.
\end{question_kholle}

\end{document}

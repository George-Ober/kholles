\documentclass{article}

\date{15 février 2025}
\usepackage[nb-sem=19, auteurs={Hugo Vangilluwen, George Ober, Kylian Boyet, Félix Rondeau}]{../kholles}
\begin{document}
\maketitle

\begin{question_kholle}
	[Une famille est liée si et seulement si l'un de ses vecteurs est une combinaison linéaires d'autres vecteurs de la famille.
		\begin{equation}
			(x_i)_{i \in I} \text{ est liée}
			\iff \exists i_0 \in I : \exists (\lambda_i)_{i \in I\setminus\{i_0\}} \in \K^{\left( I \setminus \{i_0\} \right)} :
			x_{i_0} = \sum_{\substack{i \in I \\ i \neq i_0}} \lambda_i \ldotp x_i
		\end{equation}]
	{Caractérisation d'une famille liée}

	Supposons que $(x_i)_{i \in I}$ est liée.
	Par définition,
	\[
		\displaystyle \exists (\mu_i) \K^{(I)} :
		\begin{cases}
			\sum_{i\in I}\mu_{i} x_{i} & =0_{E}                 \\
			(\mu_{i})_{i\in I}         & \neq (0_{\K})_{i\in I}
		\end{cases}
	\]
	Donc $\exists i_0 \in I : \mu_{i_0} \neq 0_\K$. Fixons un tel $i_0$. \\
	\[
		\mu_{i_0} x_{i_0} + \sum_{i \in I \setminus \{i_0\}} \mu_i x_i = 0_E
	\]
	Or $\mu_{i_0} \neq 0$, donc
	\[
		\displaystyle x_{i_0} = \sum_{i \in I \setminus \{i_0\}} \left( \mu_{i_0}^{-1} \times (-\mu_i) \right) \ldotp x_i
	\]
	En posant $\lambda_i = \mu_{i_0}^{-1} \times (-\mu_i)$, on obtient
	\[
		x_{i_0} = \displaystyle \sum_{i \in I \setminus \{i_0\}} \lambda_i \ldotp x_i
	\]
	Supposons maintenant qu’il existe $i_0 \in I$ et $(\lambda_i)_{i \in I\setminus\{i_0\}} \in \K^{\left( I \setminus \{i_0\} \right)}$ tels que
	\[
		x_{i_0} = \sum_{\substack{i \in I \\ i \neq i_0}} \lambda_i \ldotp x_i
	\]
	Alors $-x_{i_0} + \sum_{\substack{i \in I \\ i \neq i_0}} \lambda_i \ldotp x_i = 0_E$.
	Posons $\mu_{i_0} = - 1_\K$ et $\forall i \in I \!\setminus\! \{i_0\}, \mu_i = \lambda_i$.
	Ainsi, $(\mu_i)_{i \in I} \in \K^{(I)}$ et $\sum_{i \in I} \mu_i \ldotp x_i = 0_\K$. Or $\mu_{i_0} \neq 0_\K$ donc $(\mu_i)_{i \in I} \neq (0_\K)_{i \in I}$.
	Ainsi, $(\mu_i)_{i \in I}$ est liée.
\end{question_kholle}

\begin{question_kholle}[{
				Soient $(E,+, \cdot)$ un \ev et $A$ une partie de $E$. Notons $\mathcal{A}$ l’ensemble des sous-espaces vectoriels de $E$ contenant $A$. En ordonnant (partiellement) $\mathcal{P}(E)$ par l’inclusion, l’ensemble $\mathcal{A}$ admet un plus petit élément, noté $\Vect_{E}(A)$, et appelé \textbf{le sous-espace vectoriel engendré par $A$ dans $E$}.}]{Définition et existence du sous-espace engendré par une partie d’un espace vectoriel.}
	Posons $\mathcal{F}=\{$F$ \text{ sous-espace vectoriel de } E \mid A\subset F\}$ et $\displaystyle V=\bigcap_{F\in \mathcal{F}}F$ (qui est bien défini car $\mathcal{F}\neq \emptyset$ puisque $E\in\mathcal F$). Montrons que $V=\min \mathcal{F}$ pour l’inclusion.
	\begin{itemize}
		\item $V\in \mathcal{F}$ car $V$ est une intersection de sous-espaces vectoriels de $E$ donc est un sous-espace vectoriel de $E$, et
		      \[
			      \forall F\in \mathcal{F}, A\subset F \implies  A\subset \bigcap_{F\in \mathcal{F}}F \implies A\subset V
		      \]
		\item Soit $F_{0}\in \mathcal{F}$ fixé quelconque.
		      \[
			      V=\bigcap_{F\in \mathcal{F}} F = F \cap \left(\bigcap_{F\in \mathcal{F}\setminus F_{0}}\right)\subset F_{0}
		      \]
		      donc $V$ est plus petit que tous les autres éléments de $\mathcal{F}$ pour l’inclusion.
	\end{itemize}
\end{question_kholle}

\begin{question_kholle}[{Le sous-espace vectoriel engendré dane $E$ par la famille $(a_{i})_{i\in I}$ est
	\[
		\Vect_{E}\{a_{i} \mid i\in I\} = \left\{\sum_{i\in I}\lambda_{i}a_{i} \;\middle |\; (\lambda_{i})_{i\in I}\in\K^{(I)}\right\}
	\]
	On note cet ensemble $W$ pour la démonstration.}]{Description explicite du sous-espace vectoriel engendré par une partie.}
	\hfill\\
	\begin{itemize}[label=$\vartriangleright$]
		\item \textit{$W$ est un sous-espace vectoriel de $E$.}
		      \begin{itemize}[label=$\star$]
			      \item $W\subset E$ et $E$ est un \ev.
			      \item $W\neq \emptyset$ car $0_{E}\in W$ (pour $(\lambda_{i})_{i\in I}\leftarrow (O_{\K})_{i\in I}$).
			      \item Soient $(x,y)\in W^{2}$ et $\lambda\in\K$ fixé quelconque. Il existe $(\lambda_{i})$ et $(\mu_{i})$ deux familles presque nulles de \K telles que
			            \[
				            x= \sum_{i\in I}\lambda_{i} a_{i} \quad\text{et}\quad y=\sum_{i\in I}\mu_{i} a_{i}
			            \]
			            Ainsi,
			            \[
				            \lambda x+y = \sum_{i\in I}\underbrace{(\lambda \times \lambda_{i} + \mu_{i})}_{\in\K}\cdot a_{i} \in W
			            \]
		      \end{itemize}
		\item Montrons que $\left\{a_{i} \mid i\in I\right\}\in W$~: Soit $i_{0}\in I$ fixé quelconque. Pour $(\lambda_{i})\leftarrow (\delta_{i_{0}, i})$ (autorisé car $\mathrm{supp}\;(\delta_{i_{0}, i}) = \{i_{0}\}$ est fini),
		      \[
			      a_{i_{0}}=\sum_{i\in I}\delta_{i_{0}, i}a_{i}\in W
		      \]

		\item Montrons que $W\subset \Vect\{a_{i} \mid i\in I\}$~: Soit $x\in W$ fixé quelconque. Il existe une famille $(\lambda_{i})$ presque nulle de scalaires telle que
		      \[
			      x=\sum_{i\in I}\lambda_{i} a_{i} \in \Vect \{a_{i} \mid i\in I\}
		      \]
		      car $x$ est une combinaison linéaire de vecteurs de $\left\{a_{i} \mid i\in I\right\}$ donc de vecteurs de $\Vect \left\{a_{i} \mid i\in ei\right\}$, donc $\Vect \left\{a_{i} \mid i\in I\right\}$ est un sous-espace vectoriel de $E$.
	\end{itemize}
	Ainsi, les deux premiers points donnant l’inclusion directe, i.e.
	\[
		\Vect \{a_{i} \mid i\in I\}\subset W
	\]
	et le troisième l’inclusion réciproque, on a bien l’égalité recherchée.
\end{question_kholle}

\begin{question_kholle}[{
				Si $\mathcal{F}$ est une famille libre de vecteurs de $E$ et $x\in E\setminus \Vect \mathcal{F}$, alors la famille $\mathcal{F}\cup\{x\}$ est libre.}]{Stabilité de la liberté d’une famille par adjonction d’un vecteur n’appartenant pas au sous-espace qu’elle engendre.}
	Soit $x_{\triangle}\in E\setminus \Vect \{x_{i} \mid i\in I\}$ fixé quelconque. Soient $(\lambda_{i})_{i\in I\cup\{\triangle\}}\in\K^{(I\cup \{\triangle\})}$ fixée quelconque telle que
	\begin{equation}\tag{$*$}
		\sum_{i\in I\cup\{\triangle\}}\lambda_{i} x_{i} = 0_{E}
	\end{equation}
	Supposons $\lambda_{\triangle}\neq 0_{\K}$. Alors, $(*)$ donne
	\[
		\lambda_{\triangle}x_{\triangle} = \sum_{i\in I}-\lambda_{i}x_{i} \implies x_{\triangle}=\sum_{i\in I}-\frac{\lambda_{i}}{\lambda_{\triangle}}x_{i}
	\]
	donc $x_{\triangle}\in\Vect\{x_{i} \mid i\in I\}$ ce qui contredit le choix de $x_{\triangle}$. Par conséquent, $\lambda_{\triangle}=0_{\K}\quad (1)$ donc $(*)$ devient
	\[
		\sum_{i\in I}\lambda_{i} x_{i} = 0_{E}
	\]
	or $(x_{i})_{i\in I}$ est libre donc
	\begin{equation}\tag{$2$}
		\forall i\in I, \lambda_{i} = 0_{\K}
	\end{equation}
	Les relations $(1)$ et $(2)$ donnent $\forall i\in I\cup \{\triangle\}, \lambda_{i} = 0_{\K}$ donc la famille $(x_{i})_{i\in I\cup \{\triangle\}}$ est libre.
\end{question_kholle}

\begin{question_kholle}[{
				Si $E$ est un $\K$-espace vectoriel et $\mathcal{F}$ une famille de vecteurs de $E$, les propositions suivantes sont équivalentes~:
				\begin{enumerate}[label=$(\roman*)$, labelindent=0pt, leftmargin=!]
					\item $\mathcal{F}$ est une base de $E$
					\item Tout vecteur de $E$ s’écrit d’une manière unique comme une combinaison linéaire des vecteurs de $\mathcal F$
					\item $\mathcal{F}$ est une famille génératrice minimale (au sens de l’inclusion)
					\item $\mathcal{F}$ est une famille libre maximale (au sens de l’inclusion)
				\end{enumerate}
			}]{Équivalence $\mathcal{F}$ base, tout vecteur se décompose de manière unique dans $\mathcal{F}$, $\mathcal{F}$ générarice minimale et $\mathcal{F}$ libre maximale.}
	Notons $(a_{i})_{i\in I}$ la famille $\mathcal{F}$.
	\begin{description}
		\item[$\bigl((i) \implies (ii)\bigr)$] Supposons que $\mathcal{F}$ est une base de $E$. Soit $x\in E$ fixé quelconque.
		      \begin{itemize}
			      \item $\mathcal{F}$ est une base donc $\mathcal{F}$ est génératrice donc $x$ s’écrit comme une combinaison linéaire de vecteurs de $\mathcal{F}$
			      \item $\mathcal{F}$ est une base donc $\mathcal{F}$ est libre donc $x$ s’écrit de manière unique comme combinaison linéaire de vecteurs de $\mathcal{F}$
		      \end{itemize}

		\item[$\bigl((ii) \implies (iii)\bigr)$] Supposons que tout vecteur de $E$ s’écrit d’une manière unique comme une combinaison linéaire des vecteurs de $\mathcal{F}$. Alors $E\subset \mathcal{F}$ or $\Vect \mathcal{F}\subset E$ donc $\Vect \mathcal{F} = E$ donc $\mathcal F$ est génératrice.\\[4pt]
		      \quad \textit{Rappel~:} $\mathcal{F}$ est génératrice minimale signifie qu’aucune sous-famille stricte de $\mathcal{F}$ n’est génératrice. Il suffit donc de montrer qu’une sous-famille $\mathcal{F}'$ de $\mathcal{F}$ quelconque n’est pas génératrice.\\[4pt]
		      Soit $\mathcal{F'}$ une sous-famille stricte de $\mathcal{F}$; supposons-la génératrice. Par définition, il existe un élément $a$ de $\mathcal{F}$ n’appartenant pas oà $\mathcal{F'}$. De plus, $\mathcal{F'}$ étant génératrice, cet élément s’écrit comme combinaison linéaire de vecteurs de $\mathcal{F'}$ (combinaison linéaire des vecteurs de $\mathcal{F}$ avec le coefficient devant $a$ nul). Or $a=1 \cdot a$ ce qui contredit l’unicité de lécriture de $a$ comme combinaison linéaire de vecteurs de $\mathcal{F}$. Ainsi, $\mathcal{F}$ est génératrice minimale.

		\item[$\bigl((iii) \implies (iv)\bigr)$] Supposons que $\mathcal{F}$ est une famille génératrice minimale. Représentons $\mathcal{F}$ par $(x_{i})_{i\in I}$.
		      \begin{itemize}
			      \item Supposons $\mathcal{F}$ liée. Alors un des vecteur doté $x_{i_{0}}$ s’écrit comme combinaison linéaire des autres vecteurs de la famille~: $x_{i_{0}} \in \left\{x_{i} \mid i\in I\setminus \{i_{0}\}\right\}$. Or $\mathcal{F}$ est génératrice donc $(x_{i})_{i\in I\setminus \{i_{0}\}}$ l’est également. Ainsi $(x_{i})_{i\in ei\setminus \{i_{0}\}}$ est une sous-famille stricte de $\mathcal{F}$ qui est génératrice, ce qui contredit le caractère générateur minimal de $\mathcal{F}$, donc $\mathcal{F}$ est libre.
			      \item \textit{Rappel~:} $\mathcal{F}$ est libre maximale signifie que toute famille ayant $\mathcal{F}$ comme sous-famille stricte est liée, ou encore qu’il n’existe pas de famille libre contenant strictement $\mathcal{F}$\\[4pt]
			            Soit $\mathcal{F'}$ une famille libre admettant $\mathcal{F}$ comme sous-famille stricte. Notons $a$ un élément de $\mathcal{F}$ n’appartenant pas oà $\mathcal{F'}$.\\
			            $\mathcal{F}$ est génératrice donc $a\in\Vect\mathcal{F}$ donc $a$ s’écrit comme combinaison linéaire des autres vecteurs de $\mathcal{F'}$. Or $a\in \mathcal{F'}$ donc $\mathcal{F'}$ est liée d’où une contradiction. Ainsi $\mathcal{F}$ n’admet aucune famille libre la contenant strictement, donc $\mathcal{F}$ est libre maximale.
		      \end{itemize}

		\item[$\bigl((iv) \implies (i)\bigr)$] Supposons que $\mathcal{F}$ est une famille libre maximale.
		      \begin{itemize}
			      \item $\mathcal{F}$ est libre
			      \item Supposons que $\mathcal{F}$ n’est pas génératrice. Alors $\Vect\mathcal{F} \varsubsetneq E$ donc $\exists a\in E: a\notin \Vect \mathcal{F}$, or $\mathcal{F}$ est libre donc $(\mathcal{F}, a)$ (adjonction du vecteur $a$ oà la famille $\mathcal{F}$) est libre, ce qui contredit le fait que $\mathcal{F}$ est libre maximale (car $(\mathcal{F}, a)$ est libre et admet $\mathcal{F}$ comme sous-famille stricte). Par conséquent, $\mathcal{F}$ est génératrice, et ainsi, $\mathcal{F}$ est une base de $E$.
		      \end{itemize}


	\end{description}
\end{question_kholle}

\begin{question_kholle}
	[{
				Soit $f \in \mathcal{L}_\K(E, F)$.
				\begin{equation*}
					\begin{aligned}
						\ker f & = \left\{ x \in E \;|\; f(x) = 0_F \right\} = f^{-1} (\{0_F\}) \\
						\Im f  & = \left\{ y \in F \;|\; \exists x \in E : f(x) = y \right\}
					\end{aligned}
				\end{equation*}
				Nous démontrerons le résultat plus général suivant~:
				\begin{enumerate}[label=$(\roman*)$, labelindent=0pt, leftmargin=!]
					\item $f(E')$ est un \sev de $F$.
					\item $f^{-1}(F')$ est un \sev de $E$.
				\end{enumerate}
			}]
	{Le noyau et l'image d'une application linéaire sont des \sevs}
	\hfill\\
	\begin{enumerate}[label=$(\roman*)$, labelindent=0pt, leftmargin=!]
		\item $0_E \in E'$ et $f(0_E) = 0_F$ donc $0_F \in f(E')$ d'où $f(E') \neq \emptyset$ \\
		      Soit $(\alpha, \beta, y, y') \in \K^2 \times f(E')^2$ \fqs. \\
		      Par définition, il existe $(x, x') \in E'^2$ tels que $f(x) = y$ et $f(x') = y$.
		      \begin{equation*}
			      \begin{aligned}
				      \alpha y + \beta y'
				       & = \alpha f(x) + \beta f(x')                                                                     \\
				       & = f( \alpha x + \beta x' ) \quad \text{ car } f \in \mathcal{L}_\K(E, F)                        \\
				       & \in f(E') \quad \text{ car } \alpha x + \beta x' \in E' \text{ puisque } E' \text{ est un \sev}
			      \end{aligned}
		      \end{equation*}
		      Donc $f(E')$ est un \sev.

		\item $0_F \in F'$ et $f(0_E) = 0_F$ donc $0_E \in f^{-1}(F')$ d'où $f(F') \neq \emptyset$ \\
		      Soit $(\alpha, \beta, x, x') \in \K^2 \times f^{-1}(F')^2$ \fqs. \\
		      Par définition, il existe $(y, y') \in F'^2$ tels que $f(x) = y$ et $f(x') = y$. \\
		      Or $F'$ est \sev donc $\alpha y + \beta y' \in F'$. $f \in \mathcal{L}_\K(E, F)$ d'où $f(\alpha x + \beta x') = \alpha y + \beta y'$. Donc $\alpha x + \beta x' \in f^{-1}(F')$. \\
		      Ainsi, $f^{-1}(F')$ est un \sev.

		      En appliquant ce résultat pour $E' = E$ et $F' = \{0_F\}$, nous obtenons que $\ker f$ et $\Im f$ sont des \sevs.
	\end{enumerate}
\end{question_kholle}

\end{document}

\documentclass{article}

\date{02 Mars 2024}
\usepackage[nb-sem=19, auteurs={Hugo Vangilluwen, Ober George}]{../kholles}

\begin{document}
\maketitle
	
\begin{question_kholle}
	[]
	{Calculer $E^{i,j} \times E^{k,l}$ en fonction de $i$, $j$, $k$, $l$ et des Symboles de Kronecker}
			
	Calculons $E^{i,j}(n,p) \times E^{k,l}(p,q)$.
 
Soient $(r, s) \in [ \! [ 1, n] \!] \times [ \! [ 1, q ] \!]$ fq

\begin{align*}
\left[ E^{i,j} \times E^{k,l} \right] _{rs}  & = \sum_{t = 1}^{n}E^{i,j}_{r,t} E^{k,l}_{t,s} \\
 & =\sum_{t = 1}^{n} \delta_{ir} \delta_{jt} \delta_{kt} \delta_{ls} \\
 & = \delta_{jk} \delta_{ir} \delta_{ls}
\end{align*}


Ainsi, pour le calcul de $(E^{i,j})^{2}$, $q \leftarrow n$, $p \leftarrow n$, $k \leftarrow i$, $l \leftarrow j$.

Soient $(r, s) \in [ \! [ 1, n] \!]^{2}$ fq




\begin{align*}


\left[  (E^{i,j})^{2} \right]_{rs} = \delta_{ir} \delta_{js} \delta_{ij} = \left\{ 
\begin{array}{ll}
  \left[   E^{i,i} \right]_{rs} \text{ si } i = j  \\ 
  \left[ 0_{n,n} \right] _{rs} \text{ si } i \neq j 
\end{array}
\right.


\end{align*}

\end{question_kholle}
	
\end{document}
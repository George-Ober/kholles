\documentclass{article}

\date{03 Mars 2024}
\usepackage[nb-sem=19, auteurs={Hugo Vangilluwen, Ober George}]{../kholles}

\begin{document}
	\maketitle
	
	\begin{question_kholle}
		[Le symbole de Kronecker est définit de la manière suivante :
		\begin{equation}
			\forall (x, y) \in \R^2, \delta_{xy} = \left\{ \begin{matrix}
				0 \text{ si } x \neq y \\
				1 \text{ si } x = y
			\end{matrix} \right.
		\end{equation}
		La matrice $E^{i,j} \in \mathcal{M}(n, p)(\K)$ avec {$(i, j) \in [\![ 1, n ]\!] \!\times\! [\![ 1, p ]\!]$} ne possède que des coefficients nuls sauf le coefficient de la $i^{ème}$ ligne et $j^{ème}$ colonne qui vaut 1. Formellement :
		{\begin{equation}
			\forall (r, s) \in [\![ 1, n ]\!] \times [\![ 1, p ]\!], \
			\left[E^{i,j}\right]_{rs} = \delta_{ir} \delta_{js}
		\end{equation}}]
		{Calculer $E^{i,j} \times E^{k,l}$ en fonction de $i$, $j$, $k$, $l$ et des symboles de Kronecker}
				
		Calculons $E^{i,j}(n,p) \times E^{k,l}(p,q)$.
	 
		Soient $(r, s) \in [ \! [ 1, n] \!] \times [ \! [ 1, q ] \!]$ fq
		
		\begin{align*}
			\left[ E^{i,j} \times E^{k,l} \right] _{rs}
			& = \sum_{t = 1}^{n}E^{i,j}_{r,t} E^{k,l}_{t,s} \\
		 	& =\sum_{t = 1}^{n} \delta_{ir} \delta_{jt} \delta_{kt} \delta_{ls} \\
		 	& = \delta_{jk} \delta_{ir} \delta_{ls} \\
		 	& = \delta_{jk} \left[ E^{i,l} \right] _{rs}
		\end{align*}
		
		Donc $E^{i,j} \times E^{k,l} = \delta_{jk} E^{i,l}$.
		
		
		Ainsi, pour le calcul de $(E^{i,j})^{2}$, $q \leftarrow n$, $k \leftarrow i$, $l \leftarrow j$.
		
		\begin{align*}
			(E^{i,j})^{2} = \delta_{ji} E^{i,j} = \left\{ 
			\begin{array}{ll}
			  E^{i,j} \text{ si } i = j  \\ 
			  0_{n,p} \text{ si } i \neq j 
			\end{array}
			\right.
		\end{align*}
	
	\end{question_kholle}
	
\end{document}
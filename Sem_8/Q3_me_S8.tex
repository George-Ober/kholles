\documentclass{article}

\usepackage[french]{babel}
\usepackage{amsmath}
\usepackage{stmaryrd}
\usepackage{amssymb}
\usepackage[T1]{fontenc}
\usepackage[a4paper,top=2cm,bottom=2cm,left=3cm,right=3cm,marginparwidth=1.75cm]{geometry}
\usepackage{graphicx}
\usepackage[colorlinks=true, allcolors=blue]{hyperref}
\usepackage{amsthm}
\usepackage{amsfonts}
\usepackage{hyperref}
\usepackage{tikz}
\usepackage{pgfplots}
\pgfplotsset{compat=1.15}
\usepackage{mathrsfs}
\usetikzlibrary{shapes.geometric}
\usetikzlibrary{arrows.meta,arrows}
\hypersetup{
    colorlinks=true,
    linkcolor=blue,
    filecolor=magenta,      
    urlcolor=cyan,
    pdftitle={Khôlles Maths 6-16},
    pdfpagemode=FullScreen,
    }

\title{Khôlles : Maths}
\author{Kylian Boyet, George Ober, Hugo Vangilluwen, Jérémie Menard}


\begin{document}

\maketitle

\section{Soient $(a,b)\in \mathbb{C}^2$, $f$ et $g$ les  solutions, définies sur $\mathbb{R}$ à valeurs
dans $\mathbb{C}$, des problèmes de Cauchy suivants :}


\[
    \left\{ \begin{array}{cl}
        y'' +ay'+by = 0 \\
        y(3) = 1\\
        y'(3) = 0
        \end{array} \right.        
	\quad \text{et} \quad
    \left\{ \begin{array}{cl}
        y'' +ay'+by = 0 \\
        y(3) = 0\\
        y'(3) = 1
        \end{array} \right.
\]

Comment s'exprime la solution définie sur $\mathbb{R}$ de $\left\{ \begin{array}{cl}
    y'' +ay'+by = 0 \\
    y(3) = \alpha \\
    y'(3) = \beta
    \end{array} \right. $ pour $(\alpha, \beta)\in \mathbb{R}^2$ fixés ? 

Peut-on affirmer que le plan vectoriel des solutions définies sur $\mathbb{R}$ à valeurs dans 
$\mathbb{C}$ de $y'' + ay' + by = 0$ est $\{ \lambda \cdot f + \mu \cdot g  | 
(\lambda, \mu)\in \mathbb{C}^2\}$

\begin{proof}
    La solution s'exprime simplement comme combinaison linéaire de f et g, plus précisément, la 
    combinaison linéaire en $\alpha$ et $\beta$. En effet, soient de tels scalaires, et soient $f$ et 
    $g$ de telles solutions, on a : 
    \[
        (\alpha \cdot f + \beta \cdot g)'' + a (\alpha \cdot f + \beta \cdot g)' + b (\alpha \cdot f + 
        \beta \cdot g) = 0 \text{, par définition des espaces vectoriels.}
    \]
    Et de même, $(\alpha \cdot f + \beta \cdot g)'(3) = \alpha \cdot f'(3) + \beta \cdot g'(3) = \alpha$,
    et $(\alpha \cdot f + \beta \cdot g)''(3) = \alpha \cdot f''(3) + \beta \cdot g''(3) = \beta$.
    \newline
    Ce qui suffit par unicité des solutions ( de la donc) d'un problème de Cauchy dans le cadre du 
    théorème du cours.
    \newline
    Pour ce qui est du plan vectoriel des solutions, noté $\Omega$, notons aussi $\Phi$ l'ensemble proposé.
    L'inclusion $\Phi \subset \Omega$ est triviale par propriété de linéarité des espaces vectoriels.
    Finalement, pour $\Omega \subset \Phi$, soit $\omega \in \Omega$, forcément, $\omega$ vérifie 
    l'$EDL_2$, mais aussi des conditions de Cauchy bien que celles-ci soient non-spécifiées, ainsi
    posons $\omega'(3) = \delta$ et $\omega''(3) = \theta$, donc en particulier, $ \omega = 
    \delta \cdot f + \theta \cdot g$, d'où l'égalité par double inclusion.
\end{proof}

\end{document}
\documentclass{article}

\date{22 Avril 2024}
\usepackage[nb-sem=24, auteurs={Hugo Vangilluwen, George Ober}]{../kholles}

\begin{document}
	\maketitle
	
	Pour cette semaine, \K désigne un corps commutatif, $E$ et $F$ des \K\!\!-espaces vectoriels, $E'$ et $F'$ des sous-espaces vectoriels respectivement de $E$ et de $F$.
	
	Nous rappelons que $\dim \{0_E\} = 0$ et que $\{0_E\} = \Vect \emptyset$.
	
	\begin{question_kholle}
		[Pour tout \sev de $E$, il existe un \sev complémentaire.]
		{Existence d'un supplémentaire en base finie}
		
		~\newline
		\textit{Théorème de la base incomplète} (admis ici mais démontré dans le cours) : pour toute famille libre de E, nous pouvons y adjoindre une partie d'une famille quelconque génératrice de $E$ (généralement une base, la base canonique si elle a un sens) pour en faire une base de $E$. \\
		
		Posons $n = \dim E$ et $p = \dim E'$. Ainsi, il existe $(e_1, \ldots, e_p)$ base de $E'$. \\
		Appliquons le théorème de la base incomplète pour cette famille. Il existe $(e_{p+1}, \ldots, e_n)$ $n-p$ vecteurs de E \tq $(e_1, \ldots, e_n)$ est un base de $E$. \\
		Posons $E'' = \Vect \{ e_{p+1}, \ldots, e_n \}$ et vérifions qu'il est complémentaire à $E'$.
		
		\begin{itemize}[label=$*$]
			\item Par définition de \Vect\!\!, $E''$ est un \sev.
			\item Trivialement, $E' + E'' = E$.
			\item $\{0_E\} \subset E' \cap E''$ car $E'$ et $E''$ sont deux \sevs.
			\item Soit $x \in E' \cap E''$. \\
			$X \in E' \implies \exists (\lambda_1, \ldots, \lambda_p) \in \K^p : x = \sum_{i=1}^{p} \lambda_i e_i$ \\
			$X \in E'' \implies \exists (\lambda_{p+1}, \ldots, \lambda_n) \in \K^{n-p} : x = \sum_{i=p+1}^{n} \lambda_i e_i$ \\
			Par différence, $\sum_{i=1}^{p} \lambda_i e_i + \sum_{i=p+1}^{n} \left(-\lambda_i\right) e_i = 0_E$. \\
			Or $\famille{e}{\lient 1 ; n \rient}$ est une base de $E$ donc $\forall i \in \lient 1 ; p \rient, \lambda_i = 0_\K$. \\
			donc $x = 0_E$.
			Ainsi, $E' \cap E'' \subset \{0_E\}$.
		\end{itemize}
	\end{question_kholle}
	
	\begin{question_kholle}
		[$\mathcal{L}_\K(E,F)$ est dimension finie et
		\begin{equation}
			\dim \mathcal{L}_\K(E,F) = \dim E \times \dim F
		\end{equation}]
		{Dimension de $\mathcal{L}_\K(E,F)$}
		
		Notons $n = \dim E$ et $\famille{e}{\lient 1 ; n \rient}$ une base de $E$. Considérons
		\begin{equation*}
			\varphi
			\left| \begin{matrix}
				\mathcal{L}_\K(E,F) &\rightarrow& F^n \\
				f &\mapsto& \underbrace{\left(f(e_i))\right)_{1 \leqslant i \leqslant n}}_{\text{évaluation de f en la base choisie}}
			\end{matrix} \right.
		\end{equation*}
		
		\begin{itemize}[label=$*$]
			\item $\varphi$ est linéaire.
			\item $\varphi$ est bijective d'après le théorème de création des applications linéaires qui établit que pour toute famille de $n$ vecteurs de $F$, il existe une unique application linéaire de $E$ dans $F$ envoyant la base $\famille{e}{\lient 1 ; n \rient}$ sur cette famille.
		\end{itemize}
		
		Ainsi, $\mathcal{L}_\K(E,F)$ et $F^n$ sont isomorphes. $F^n$ est de dimension finie, ce qui conclut.
	\end{question_kholle}
	
	\begin{question_kholle}
		[Supposons $E$ de dimension finie. \\
		Soient $E_1$ et $E_2$ deux \sevs. Alors $E_1 + E_2$ est de dimension finie et
		\begin{equation}
			\dim E_1 + E_2 = \dim E_1 + \dim E_2 - \dim E_1 \cap E_2
		\end{equation}]
		{Formule de Grassman}
		
		Commençons par prouver une version simplifier de la somme directe. Supposons que $E_1$ et $E_2$ sont en somme directe. \\
		Fixons $\mathcal{B}_1$ et $\mathcal{B}_2$ deux bases de $E_1$ et $E_2$.
		
		Alors $\left(\mathcal{B}_1, \mathcal{B}_2\right)$ engendre $E_1 + E_2$. Or $\left(\mathcal{B}_1, \mathcal{B}_2\right)$ est finie donc $E_1 + E_2$ est de dimension finie. \\
		Posons $n = \dim E_1$ et $p = \dim E_2$. Notons $\famille{e}{\lient 1 ; n \rient}$ la base $\mathcal{B}_1$ et $\famille{f}{\lient 1 ; n \rient}$ la base $\mathcal{B}_2$. \\
		Soient $\lambda_1, \ldots, \lambda_n, \mu_1, \ldots, \mu_p) \in \K^{n+p}$ \fqs \fqs \ $\displaystyle \sum_{i=1}^{n} \lambda_i e_i + \sum_{i=1}^{p} \mu_i f_i = 0_E$. \\
		Alors $\sum_{i=1}^{n} \lambda_i e_i = \sum_{i=1}^{p} (-\mu_i) f_i$.
		Or $\sum_{i=1}^{n} \lambda_i e_i \in E_1$ et $\sum_{i=1}^{n} (-\mu_i) e_i \in E_2$ donc $\sum_{i=1}^{n} \lambda_i e_i \in E_1 \cap E_2 = \{0_E\}$.
		Donc $\lambda = \tilde{0}$. De même, $\mu = \tilde{0}$. \\
		Donc $\left(\mathcal{B}_1, \mathcal{B}_2\right)$ est libre.
		
		Ainsi, $\left(\mathcal{B}_1, \mathcal{B}_2\right)$ est une base de $E_1 \oplus E_2$.
		Donc $ \dim E_1 \oplus E_2 = |(\mathcal{B}_1, \mathcal{B}_2)| = |\mathcal{B}_1| + |\mathcal{B}_2| = \dim E_1 + \dim E_2$. \\
		
		Enlevons l'hypothèse que $E_1$ et $E_2$ sont en somme directe. \\
		$E_1 \cap E_2$ est un \sev de $E_2$ et $E_2$ est un \K-espace vectoriel de dimension finie donc il existe $E_2'$ \sev de $E_2$ \tq $E_2 = (E1 \cap E_2) \oplus E_2'$. \\
		Montrons que $E_1 + E_2 = E_1 \oplus E_2'$
		\begin{itemize}[label=$*$]
			\item $E_1$ et $E_2'$ sont en somme directe. \\
			\begin{equation*}
				\begin{aligned}
					E_1 \cap E_2' &= E_1 \cap \left( E_2' \cap E_2 \right) \text{ car } E_2' \subset E_2 \\
					&= \left( E_1 \cap E_2 \right) \cap E_2' \text{ car } \cap \text{ est associative et commutative} \\
					&= {0_E} \text{ car $E_1$ et $E_2$ sont en somme directe et $E_2'$ sev}
				\end{aligned}
			\end{equation*}
			\item $E_1 + E_2 \subset E_1 + E_2'$ \\
			Soit $x \in E_1 + E_2$.
			Alors $\exists (x_1, x_2) \in E_1 \times E_2 : x = x_1 + x_2$. \\
			Or $x_2 \in E_2 = \left( E_1 \cap E_2 \right) \oplus E_2'$ donc $\exists (x_{21}, x_2') \times E_2' : x_2 = x_{21} + x_2'$. \\
			D'où $x = \underbrace{x_1 + x_{21}}_{\in E_1} + \underbrace{x_2'}_{\in E_2} \in E_1 + E_2$.
			\item Trivialement, $E_1 + E_2' \subset E_1 + E_2$ (car $E_2' \subset E_2$). 
		\end{itemize}
		
		Ainsi, $E_1$ et $E_2'$ (car sev) étant de dimension finie, $\dim E_1 \oplus E_2' = \dim E_1 + \dim E_2'$. \\
		De plus, $\dim E_2 = \dim (E_1 \cap E_2) \oplus E_2' = \dim E_1 \cap E_2 + \dim E_2'$. \\
		Donc $\dim E_1 + E_2 = \dim E_1 + \dim E_2 - \dim E_1 \cap E_2$.
	\end{question_kholle}

	\begin{question_kholle}
		[Soit $f \in \mathcal{L}_{\mathbb{K}}(E, F)$.
		\begin{propositions}
			\item Si $E$ est de dimension finie 
			\begin{equation}
				f \text{ injective } \iff \mathrm{rg} f = \dim E
			\end{equation}
			\item Si $F$ est de dimension finie
			\begin{equation}
				f \text{ surjective } \iff \mathrm{rg} f = \dim F
			\end{equation}
			\item Si $E$ et $F$ sont de même dimension finie $$f \text{ bijective } \iff f \text{ injective } \iff f \text{ sujective }$$
			C'est \textit{l'accident de la dimension finie} !
		\end{propositions}]
		{Caractérisation injectivité/bijectivité/surjectivité par le rang}
		
		~
		\begin{propositions}
			\item Supposons $E$ de dimension finie, fixons $(e_{1}, \dots, e_{n})$ une base de $E$ (avec $n = \dim E$)
			Supposons $f$ injective :
			$$
			\mathrm{rg} f = \dim \mathrm{Im} f = \dim \text{Vect} \left\{ f(e_{1}) \dots f(e_{n}) \right\}  
			$$
			
			Donc $(f(e_{1}), \dots f(e_{n}))$ est génératrice.
			$(f(e_{1}), \dots f(e_{n}))$ est de plus libre car $f$ est injective.
			Donc c'est une base, donc
			$$
			\dim \text{Vect} \left\{ f(e_{1}) \dots f(e_{n}) \right\}  =n = \dim E
			$$
			donc $\mathrm{rg} f = \dim E$.
			Réciproquement, supposons que $\mathrm{rg} f = \dim E = n$.
			Alors $$n = \mathrm{rg} f = \dim \text{Vect} \left\{ f(e_{1}),\dots,f(e_{n}) \right\}$$
			Donc $(f(e_{1}), \dots f(e_{n}))$ est génératrice de cardinal $n$, égal à la dimension du sous-espace vectoriel engendré. C'est donc une base du sous-espace vectoriel engendré.
			Donc $(f(e_{1}),\dots , f(e_{n}))$ est libre, donc $f$ est injective.
			
			\item Supposons $F$ de dimension finie
			$$
			f \text{ surjective } \iff \mathrm{Im} f = F \iff \dim \mathrm{Im} f = \dim F
			$$
			
			\item Supposons $E$ et $F$ de même dimension finie
			$$
			f \text{ injective } \iff \mathrm{rg} f = \dim E \iff \mathrm{rg} f = \dim F \iff f \text{ surjective}
			$$
			D'où la bijectivité.
		\end{propositions}
	\end{question_kholle}
	
	\begin{question_kholle}
		[Soit $G$ un \K-espace vectoriel et $(u,v) \in \mathcal{L}_\K(E, F) \times \mathcal{L}_\K(F, G)$. Si $E$ et $F$ sont de dimension finie alors
		\begin{equation}
			\rg u = \rg v \circ u + \dim \ker v \cap \Im u
		\end{equation}]
		{Rang d'une composition d'applications linéaires}
		
		Considérons que $E$ et $F$ sont de dimension finie. Soient de tels objets. \\
		Appliquons le théorème du rang à $v_{|\Im u}$ ce qui est autorisé puisque $v_{|\Im u}$ est une application linéaire et $\Im u$ est un \K-ev de dimension finie (car sev de $F$).
		\begin{equation*}
			\dim \Im u = \rg v_{|\Im u} + \dim \ker v_{|\Im u}
		\end{equation*}
		$\ker v_{|\Im u} = \left\{ y \in \Im u \;|\; v(y) = 0_G \right\} = \left\{ y \in \Im u \;|\; y \in \ker v \right\} = \Im u \cap \ker v$ \\
		$\Im v_{|\Im u} = v(Im u) = \Im v \circ u$ (cette égalité est vraie pour deux fonctions de $E$ dans $F$ et de $F$ dans $G$ quelconques, pas forcément linéaires.) \\
		Donc \begin{equation*}
			\rg f = \rg v \circ u + \dim \ker v \cap \Im u
		\end{equation*}
	\end{question_kholle}

\end{document}

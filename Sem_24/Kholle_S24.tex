\documentclass{article}

\date{22 Avril 2024}
\usepackage[nb-sem=24, auteurs={Hugo Vangilluwen, George Ober}]{../kholles}

\begin{document}
	\maketitle
	
	Pour cette semaine, \K désigne un corps commutatif, $E$ et $F$ des \K\!\!-espaces vectoriels, $E'$ et $F'$ des sous-espaces vectoriels respectivement de $E$ et de $F$.
	
	Nous rappelons que $\dim \{0_E\} = 0$ et que $\{0_E\} = \Vect \emptyset$.
	
	\begin{question_kholle}
		[Pour tout \sev de $E$, il existe un \sev complémentaire.]
		{Existence d'un supplémentaire en dimension finie}
		
		~\newline
		\textit{Théorème de la base incomplète} (admis ici mais démontré dans le cours) : pour toute famille libre de E, nous pouvons y adjoindre une partie d'une famille quelconque génératrice de $E$ (généralement une base, la base canonique si elle a un sens) pour en faire une base de $E$. \\
		
		Posons $n = \dim E$ et $p = \dim E'$. Ainsi, il existe $(e_1, \ldots, e_p)$ base de $E'$. \\
		Appliquons le théorème de la base incomplète pour cette famille. Il existe $(e_{p+1}, \ldots, e_n)$ $n-p$ vecteurs de E \tq $(e_1, \ldots, e_n)$ est un base de $E$. \\
		Posons $E'' = \Vect \{ e_{p+1}, \ldots, e_n \}$ et vérifions qu'il est complémentaire à $E'$.
		
		\begin{itemize}[label=$*$]
			\item Par définition de \Vect\!\!, $E''$ est un \sev.
			\item Trivialement, $E' + E'' = E$.
			\item $\{0_E\} \subset E' \cap E''$ car $E'$ et $E''$ sont deux \sevs.
			\item Soit $x \in E' \cap E''$. \\
			$X \in E' \implies \exists (\lambda_1, \ldots, \lambda_p) \in \K^p : x = \sum_{i=1}^{p} \lambda_i e_i$ \\
			$X \in E'' \implies \exists (\lambda_{p+1}, \ldots, \lambda_n) \in \K^{n-p} : x = \sum_{i=p+1}^{n} \lambda_i e_i$ \\
			Par différence, $\sum_{i=1}^{p} \lambda_i e_i + \sum_{i=p+1}^{n} \left(-\lambda_i\right) e_i = 0_E$. \\
			Or $\famille{e}{\lient 1 ; n \rient}$ est une base de $E$ donc $\forall i \in \lient 1 ; p \rient, \lambda_i = 0_\K$. \\
			donc $x = 0_E$.
			Ainsi, $E' \cap E'' \subset \{0_E\}$.
		\end{itemize}
	\end{question_kholle}
	
	\begin{question_kholle}
		[$\mathcal{L}_\K(E,F)$ est dimension finie et
		\begin{equation}
			\dim \mathcal{L}_\K(E,F) = \dim E \times \dim F
		\end{equation}]
		{Dimension de $\mathcal{L}_\K(E,F)$}
		
		Notons $n = \dim E$ et $\famille{e}{\lient 1 ; n \rient}$ une base de $E$. Considérons
		\begin{equation*}
			\varphi
			\left| \begin{matrix}
				\mathcal{L}_\K(E,F) &\rightarrow& F^n \\
				f &\mapsto& \underbrace{\left(f(e_i))\right)_{1 \leqslant i \leqslant n}}_{\text{évaluation de f en la base choisie}}
			\end{matrix} \right.
		\end{equation*}
		
		\begin{itemize}[label=$*$]
			\item $\varphi$ est linéaire.
			\item $\varphi$ est bijective d'après le théorème de création des applications linéaires qui établit que pour toute famille de $n$ vecteurs de $F$, il existe une unique application linéaire de $E$ dans $F$ envoyant la base $\famille{e}{\lient 1 ; n \rient}$ sur cette famille.
		\end{itemize}
		
		Ainsi, $\mathcal{L}_\K(E,F)$ et $F^n$ sont isomorphes. $F^n$ est de dimension finie, ce qui conclut.
	\end{question_kholle}
	
	\begin{question_kholle}
		[Supposons $E$ de dimension finie. \\
		Soient $E_1$ et $E_2$ deux \sevs. Alors $E_1 + E_2$ est de dimension finie et
		\begin{equation}
			\dim E_1 + E_2 = \dim E_1 + \dim E_2 - \dim E_1 \cap E_2
		\end{equation}]
		{Formule de Grassman}
		
		Commençons par prouver une version simplifier de la somme directe. Supposons que $E_1$ et $E_2$ sont en somme directe. \\
		Fixons $\mathcal{B}_1$ et $\mathcal{B}_2$ deux bases de $E_1$ et $E_2$.
		
		Alors $\left(\mathcal{B}_1, \mathcal{B}_2\right)$ engendre $E_1 + E_2$. Or $\left(\mathcal{B}_1, \mathcal{B}_2\right)$ est finie donc $E_1 + E_2$ est de dimension finie. \\
		Posons $n = \dim E_1$ et $p = \dim E_2$. Notons $\famille{e}{\lient 1 ; n \rient}$ la base $\mathcal{B}_1$ et $\famille{f}{\lient 1 ; n \rient}$ la base $\mathcal{B}_2$. \\
		Soient $\lambda_1, \ldots, \lambda_n, \mu_1, \ldots, \mu_p) \in \K^{n+p}$ \fqs \fqs \ $\displaystyle \sum_{i=1}^{n} \lambda_i e_i + \sum_{i=1}^{p} \mu_i f_i = 0_E$. \\
		Alors $\sum_{i=1}^{n} \lambda_i e_i = \sum_{i=1}^{p} (-\mu_i) f_i$.
		Or $\sum_{i=1}^{n} \lambda_i e_i \in E_1$ et $\sum_{i=1}^{n} (-\mu_i) e_i \in E_2$ donc $\sum_{i=1}^{n} \lambda_i e_i \in E_1 \cap E_2 = \{0_E\}$.
		Donc $\lambda = \tilde{0}$. De même, $\mu = \tilde{0}$. \\
		Donc $\left(\mathcal{B}_1, \mathcal{B}_2\right)$ est libre.
		
		Ainsi, $\left(\mathcal{B}_1, \mathcal{B}_2\right)$ est une base de $E_1 \oplus E_2$.
		Donc $ \dim E_1 \oplus E_2 = |(\mathcal{B}_1, \mathcal{B}_2)| = |\mathcal{B}_1| + |\mathcal{B}_2| = \dim E_1 + \dim E_2$. \\
		
		Enlevons l'hypothèse que $E_1$ et $E_2$ sont en somme directe. \\
		$E_1 \cap E_2$ est un \sev de $E_2$ et $E_2$ est un \K-espace vectoriel de dimension finie donc il existe $E_2'$ \sev de $E_2$ \tq $E_2 = (E1 \cap E_2) \oplus E_2'$. \\
		Montrons que $E_1 + E_2 = E_1 \oplus E_2'$
		\begin{itemize}[label=$*$]
			\item $E_1$ et $E_2'$ sont en somme directe. \\
			\begin{equation*}
				\begin{aligned}
					E_1 \cap E_2' &= E_1 \cap \left( E_2' \cap E_2 \right) \text{ car } E_2' \subset E_2 \\
					&= \left( E_1 \cap E_2 \right) \cap E_2' \text{ car } \cap \text{ est associative et commutative} \\
					&= {0_E} \text{ car $E_1$ et $E_2$ sont en somme directe et $E_2'$ sev}
				\end{aligned}
			\end{equation*}
			\item $E_1 + E_2 \subset E_1 + E_2'$ \\
			Soit $x \in E_1 + E_2$.
			Alors $\exists (x_1, x_2) \in E_1 \times E_2 : x = x_1 + x_2$. \\
			Or $x_2 \in E_2 = \left( E_1 \cap E_2 \right) \oplus E_2'$ donc $\exists (x_{21}, x_2') \times E_2' : x_2 = x_{21} + x_2'$. \\
			D'où $x = \underbrace{x_1 + x_{21}}_{\in E_1} + \underbrace{x_2'}_{\in E_2} \in E_1 + E_2$.
			\item Trivialement, $E_1 + E_2' \subset E_1 + E_2$ (car $E_2' \subset E_2$). 
		\end{itemize}
		
		Ainsi, $E_1$ et $E_2'$ (car sev) étant de dimension finie, $\dim E_1 \oplus E_2' = \dim E_1 + \dim E_2'$. \\
		De plus, $\dim E_2 = \dim (E_1 \cap E_2) \oplus E_2' = \dim E_1 \cap E_2 + \dim E_2'$. \\
		Donc $\dim E_1 + E_2 = \dim E_1 + \dim E_2 - \dim E_1 \cap E_2$.
	\end{question_kholle}

	\begin{question_kholle}
		[Soit $f \in \mathcal{L}_{\mathbb{K}}(E, F)$.
		\begin{propositions}
			\item Si $E$ est de dimension finie 
			\begin{equation}
				f \text{ injective } \iff \mathrm{rg} f = \dim E
			\end{equation}
			\item Si $F$ est de dimension finie
			\begin{equation}
				f \text{ surjective } \iff \mathrm{rg} f = \dim F
			\end{equation}
			\item Si $E$ et $F$ sont de même dimension finie $$f \text{ bijective } \iff f \text{ injective } \iff f \text{ sujective }$$
			C'est \textit{l'accident de la dimension finie} !
		\end{propositions}]
		{Caractérisation injectivité/bijectivité/surjectivité par le rang}
		
		~
		\begin{propositions}
			\item Supposons $E$ de dimension finie, fixons $(e_{1}, \dots, e_{n})$ une base de $E$ (avec $n = \dim E$)
			Supposons $f$ injective :
			$$
			\mathrm{rg} f = \dim \mathrm{Im} f = \dim \text{Vect} \left\{ f(e_{1}) \dots f(e_{n}) \right\}  
			$$
			
			Donc $(f(e_{1}), \dots f(e_{n}))$ est génératrice.
			$(f(e_{1}), \dots f(e_{n}))$ est de plus libre car $f$ est injective.
			Donc c'est une base, donc
			$$
			\dim \text{Vect} \left\{ f(e_{1}) \dots f(e_{n}) \right\}  =n = \dim E
			$$
			donc $\mathrm{rg} f = \dim E$.
			Réciproquement, supposons que $\mathrm{rg} f = \dim E = n$.
			Alors $$n = \mathrm{rg} f = \dim \text{Vect} \left\{ f(e_{1}),\dots,f(e_{n}) \right\}$$
			Donc $(f(e_{1}), \dots f(e_{n}))$ est génératrice de cardinal $n$, égal à la dimension du sous-espace vectoriel engendré. C'est donc une base du sous-espace vectoriel engendré.
			Donc $(f(e_{1}),\dots , f(e_{n}))$ est libre, donc $f$ est injective.
			
			\item Supposons $F$ de dimension finie
			$$
			f \text{ surjective } \iff \mathrm{Im} f = F \iff \dim \mathrm{Im} f = \dim F
			$$
			
			\item Supposons $E$ et $F$ de même dimension finie
			$$
			f \text{ injective } \iff \mathrm{rg} f = \dim E \iff \mathrm{rg} f = \dim F \iff f \text{ surjective}
			$$
			D'où la bijectivité.
		\end{propositions}
	\end{question_kholle}
	
	\begin{question_kholle}
		[Si $E$ est de dimension finie alors pour toute $f \in \mathcal{L}_\K(E, F)$ application linéaire,
		\begin{equation}
			\dim E = \rg f + \dim \ker f
		\end{equation}]
		{Théorème du rang}
		
		Démontrons d'abord le lemme suivant.
		Soient $f \in \mathcal{L}_\K(E, F)$ et $H$ un supplémentaire de $\ker f$ dans $E$.
		Alors $f_{|H}^{|\Im f}$ est un isomorphisme de $H$ sur $\Im f$. \\
		Notons $\hat{f}$ un telle restriction et corestriction. Cette application est bien définie (car $f(H) \subset \Im f$) et $\hat{f} \in \mathcal{L}_\K(H, \Im f)$. \\
		$\ker \hat{f} = \{ x \in H \;|\; \hat{f}(x) = 0_E \} = \{ x \in H | x \in \ker f \} = H \cap \ker f = \{0_E\}$  car $H$ et $\ker f$ sont complémentaire.
		Donc $\hat{f}$ est injective. \\
		Soit $y \in \Im f$. D'où $\exists x \in E: y = f(x)$. \\
		Décomposons $x$ dans $E = H \oplus \ker f$, $\exists (x_H, x_k) \in H \times \ker f : x = x_H + x_k$. \\
		Ainsi, $y = f(x) = f(x_H) + f(x_k)= f(x_H)$ car $x_k \in \ker f$.
		Donc $y$ admet un antécédent par $\hat{f}$ (qui est $x_H$). \\
		Donc $\hat{f}$ est surjective. \\
		Donc $f_{|H}^{|\Im f}$ est un isomorphisme de $H$ sur $\Im f$.
		
		Supposons maintenant que $E$ est de dimension finie.
		Soit $f \in \mathcal{L}_\K(E, F)$. \\
		D'après le théorème d'existence d'un supplémentaire en dimension finie, $\ker f$, étant un \sev de $E$, admet un supplémentaire $H$ c'est-à-dire $E = H  \oplus \ker f$. \\
		En prenant la dimension sur cette égalité, $\dim E = \dim \ker f + \dim H$.
		D'après le lemme précédent, $\dim H = \dim \Im f = \rg f$. \\
		D'où $\dim E = \rg f + \dim \ker f$.
	\end{question_kholle}
	
	\begin{question_kholle}
		[Soit $G$ un \K-espace vectoriel et $(u,v) \in \mathcal{L}_\K(E, F) \times \mathcal{L}_\K(F, G)$. Si $E$ et $F$ sont de dimension finie alors
		\begin{equation}
			\rg u = \rg v \circ u + \dim \ker v \cap \Im u
		\end{equation}]
		{Rang d'une composition d'applications linéaires}
		
		Considérons que $E$ et $F$ sont de dimension finie. Soient de tels objets. \\
		Appliquons le théorème du rang à $v_{|\Im u}$ ce qui est autorisé puisque $v_{|\Im u}$ est une application linéaire et $\Im u$ est un \K-ev de dimension finie (car sev de $F$).
		\begin{equation*}
			\dim \Im u = \rg v_{|\Im u} + \dim \ker v_{|\Im u}
		\end{equation*}
		$\ker v_{|\Im u} = \left\{ y \in \Im u \;|\; v(y) = 0_G \right\} = \left\{ y \in \Im u \;|\; y \in \ker v \right\} = \Im u \cap \ker v$ \\
		$\Im v_{|\Im u} = v(Im u) = \Im v \circ u$ (cette égalité est vraie pour deux fonctions de $E$ dans $F$ et de $F$ dans $G$ quelconques, pas forcément linéaires.) \\
		Donc \begin{equation*}
			\rg f = \rg v \circ u + \dim \ker v \cap \Im u
		\end{equation*}
	\end{question_kholle}
	
	\begin{question_kholle}
		[Soit $H$ un \sev de $E$.
		Les conditions suivantes sont équivalentes :
		\begin{propositions}
			\item $H$ est un hyperplan de $E$ :
			$\exists \varphi \in E^* : H = \ker \varphi$
			\item $H$ admet une droite vectorielle comme supplémentaire :
			$\exists a \in E \setminus \{0_E\} : H \oplus \Vect{\{a\}} = E$
		\end{propositions}]
		{Caractérisation des hyperplans}
		
		$(i) \implies (ii)$ Supposons que $H$ est un hyperplan de $E$. \\
		Appliquons la définition de l'hyperplan, $\exists \varphi \in E^* : H = \ker \varphi$. \\
		Par l'absurde, supposons que $E \setminus H = \emptyset$. Or $H \subset E$ donc $E = H$. Donc $\varphi = 0_{E^*}$ ce qui est une contradiction.

		Ainsi fixons $a \in E \setminus H$ quelconque.
		Montrons que $E = H \oplus \Vect{\{a\}}$.
		\begin{itemize}[label=$\star$]
			\item Trivialement, $\{0_E\} \subset H \cap \Vect{\{a\}}$.
			Soit $x \in H \cap \Vect{\{a\}}$. \\
			$x \in \Vect{\{a\}}$ donc $\exists \lambda \in  \K : x = \lambda$. De plus, $x \in H = \ker \varphi$ donc $0_\K = \varphi(x) = \lambda \varphi(a)$. \\
			Si $\lambda \neq 0_\K$, alors $a \in \ker \varphi$ ce qui est impossible car $a \notin H$. \\
			Donc $\lambda = 0_\K$, d'où $x = 0_E$. \\
			Ainsi, $H \cap \Vect{\{a\}} = \{0_E\}$. $H$ et $\Vect{\{a\}}$ sont en somme directe.
			\item Trivialement, $H + \Vect{\{a\}} \subset E$.
			Soit $x \in E$ \fq. \\
			$a \notin H$ donc $\varphi(a) \neq 0_\K$. $\varphi(a)$ est ainsi symétrisable par la multiplication dans \K d'où :
			\begin{equation*}
				\varphi(x)
				= \frac{\varphi(x)}{\varphi(a)} \cdot \varphi(a)
				= \varphi\left( \frac{\varphi(x)}{\varphi(a)} \times a \right)
			\end{equation*}
			Donc $x - \frac{\varphi(x)}{\varphi(a)} \!\cdot\! a \in H$. D'où
			\begin{equation*}
				x =
				\underbrace{x - \frac{\varphi(x)}{\varphi(a)} \!\cdot\! a}_{\in H}
				+ \underbrace{\frac{\varphi(x)}{\varphi(a)} \!\cdot\! a}_{\in \Vect{\{a\}}}
			\end{equation*}
			Ainsi, $E = H + \Vect{\{a\}}$.
		\end{itemize}
		
		$(ii) \implies (i)$ Supposons maintenant que $H$ soit un sous-espace vectoriel tel que $\exists a \in E \setminus \{0_E\} : E = H \oplus \Vect{\{a\}}$. \\
		Posons $\displaystyle \varphi : \begin{matrix}
			E &=& H \oplus \Vect{\{a\}} &\rightarrow& \K \\
			x &=& h_x + \lambda_x \cdot a &\mapsto& \lambda_x
		\end{matrix}$.
		Montrons que $\varphi$ est une forme linéaire non triviale dont $H$ est le noyau.
		\begin{itemize}
			\item $\varphi$ est bien définie car $h_x$ et $\lambda_x$ sont uniques.
			\item $\varphi$ est linéaire.
			\item $\varphi$ est à valeur dans le corps de base \K donc $\varphi$ est un forme linéaire.
			\item $\varphi \neq 0_{E^*}$ car $\varphi(a) = 1_\K \neq 0_\K$.
			\item Soit $x \in E$ \fq. Alors $\exists (h_x, \lambda_x) \in H \times \K : x = h_x + \lambda_x \cdot a$.
			\begin{equation*}
				x \in \ker \varphi
				\iff \varphi(x) = 0_\K
				\iff \lambda_x = 0_\K
				\iff x \in H
			\end{equation*}
			donc $\ker \varphi = H$.
		\end{itemize}
		Donc $H$ est un hyperplan de $E$.
		\bigbreak
		
		Si $E$ est de dimension finie, alors les deux conditions sont équivalentes à
		\begin{enumerate}[label=$(\roman*)$, leftmargin=1.5cm]
			\setcounter{enumi}{2}
			\item $H$ est de codimension 1 c'est-à-dire de dimension $n - 1$.
		\end{enumerate}
		$(ii) \implies (iii)$ Il faut prendre la dimension de l'égalité $H \oplus \Vect{\{a\}}$. \\
		$(iii) \implies (ii)$ Supposons que $\dim H = n - 1$. \\
		Comme $E$ est de dimension finie, $H$ admet un supplémentaire $I$ dans $E$ : $H \oplus I = E$.
		En prenant la dimension, $\dim I = 1$. Donc $I$ est une droite vectorielle.
		D'où $\exists a \in E : I = \Vect{\{a\}}$. \\
		$a \notin H$ car sinon $I \subset H$ ce qui contredit $I \cap H = \{0_E\}$ ($I$ et $H$ sont en somme directe).
	\end{question_kholle}
	
	\begin{question_kholle}
		[\textit{Lemme fondamental dans l'étude des formes linéaires} \ \
		Soit $\varphi \in E^* \setminus \{0_{E^*}\}$. \\
		Tout vecteur de $E$ n'appartenant pas au noyau de $\varphi$ engendre une droite qui est supplémentaire au noyau de $\varphi$ dans $E$.
		\begin{equation}
			\forall a \in E \setminus \ker \varphi,
			E = \ker \varphi \oplus \Vect{\{a\}}
		\end{equation}
		
		Deux formes linéaires non nulles $\varphi$ et $\psi$ ont le même noyau si est seulement si elles sont proportionnelles ce qui revient à dire que la famille $\left(\varphi,\psi\right)$ est liée.
		\begin{equation}
			\forall \left(\varphi,\psi\right) \in \left( E^* \setminus \{0_{E^*}\} \right) ^2,
			\ker \varphi = \ker \psi \iff \exists \lambda \in \K^* : \varphi = \lambda \cdot \psi
		\end{equation}]
		{Proportionnalité des formes linéaires ayant le même noyau}
		
		Commençons par prouver le lemme.
		Soit $a \in E \setminus \ker \varphi$. \\
		Soit $x \in E$ \fq.
		Exhibons la décomposition unique de $x$ dans $\ker \varphi + \Vect{\{a\}}$.
		
		\underline{Analyse} Supposons qu'il existe $(x_k, \lambda) \in \ker \varphi \times \K$ \tq $x = x_k + \lambda a$. \\
		Puisque $x_k \in \ker \varphi$, $\varphi(x) = \lambda \cdot \varphi(a)$. Or $\varphi(a) \neq 0_\K$ (car $a \notin \ker \varphi$) donc $\varphi(a)$ est inversible dans \K. \\
		D'où $\lambda = \nicefrac{\varphi(x)}{\varphi(a)}$ et $x_k = x - \nicefrac{\varphi(x)}{\varphi(a)}$. \\
		Ainsi, sous réserve d'existence, $\lambda$ et $x_k$ sont unique.
		
		\underline{Synthèse} Posons $\displaystyle \left\{ \begin{matrix}
			\lambda &=& \frac{\varphi(x)}{\varphi(a)} \\
			x_k &=& x - \frac{\varphi(x)}{\varphi(a)}
		\end{matrix} \right.$
		\begin{itemize}[label=$*$]
			\item $x = x_k + \lambda \cdot a$
			\item $\lambda \in \K$ donc $\lambda \cdot a \in \Vect{\{a\}}$
			\item $\varphi(x_k) = \varphi(x) - \varphi\left( \frac{\varphi(x)}{\varphi(a)} a \right) = \varphi(x) - \frac{\varphi(x)}{\varphi(a)} \varphi(a) = 0_\K$ donc $x_k \in \ker \varphi$
		\end{itemize}
		Ainsi $E = \ker \varphi \oplus \Vect{\{a\}}$.
		\newline \newline
		
		Soient $\left(\varphi,\psi\right) \in \left( E^* \setminus \{0_{E^*}\} \right) ^2$ \fqs.
		
		$\implies$ Supposons que $\ker \varphi = \ker \psi$. \\
		$\varphi \neq 0_{E^*}$ donc $\ker \varphi \neq E$ donc $\exists a \in E : a \notin \ker \varphi$. Appliquons la lemme ci-dessus :
		\begin{equation*}
			\begin{array}{ccccccccc}
				& & E &=& \substack{\ker \varphi \\ \shortparallel \\ \ker \psi} &\oplus& \Vect{\{a\}} & \rightarrow & \K \\
				& \varphi : & x &=& \left( x - \frac{\varphi(x)}{\varphi(a)} \cdot a \right) &+& \frac{\varphi(x)}{\varphi(a)} \cdot a & \mapsto & \varphi(x) \\
				& \psi : & x &=& \left( x - \frac{\varphi(x)}{\varphi(a)} \cdot a \right) &+& \frac{\varphi(x)}{\varphi(a)} \cdot a & \mapsto & \psi(x)
			\end{array}
		\end{equation*}
		Or $\left( x - \frac{\varphi(x)}{\varphi(a)} \cdot a \right) \in \ker \psi$ donc $\psi(x) = \frac{\psi(a)}{\varphi(a)} \varphi(x)$. \\
		Ainsi, $\psi = \frac{\psi(a)}{\varphi(a)} \varphi$. Donc $\varphi$ et $\psi$ sont proportionnelles.
		
		$\impliedby$ Supposons que $\varphi$ et $\psi$ sont proportionnelles. Alors $\exists \lambda \in \K^* : \varphi = \lambda \psi$. \\
		$\varphi = \lambda \psi \implies \ker \psi \subset \ker \varphi$ et
		$\psi = \lambda^{-1} \varphi \implies \ker \varphi \subset \ker \psi$. Ce qui donne l'égalité.
	\end{question_kholle}
	
	\begin{question_kholle}
		[{Soient $E$ un $\mathbb{K}$-espace vectoriel, et $\varphi \in E^{*}$ une forme linéaire \emph{non nulle}. Soit $F$ un sous-espace vectoriel de $E$ de dimension finie $p \in \mathbb{N}$, alors
		\begin{equation}
			\dim_{\mathbb{K}}F \cap \ker \varphi =
			\left\{ \begin{array}{ll}
				p  & \text{ si } F \subset \ker \varphi \\
				p-1  & \text{ sinon}
			\end{array}\right.
		\end{equation}
		En particulier, on a toujours $\dim_{\mathbb{K}}F \cap \ker F \geqslant p-1$}]
		{Intersection d'hyperplans}
		% {Lemme sur les formes linéaires non nulles} ???
		
		Si $F \subset \ker\varphi$, $F \cap \ker \varphi = F$ donc $\dim F \cap \ker \varphi = p$
		
		Sinon, il existe $a \in F$ tel que $a \not\in \ker \varphi$. Ainsi,
		$$
		\text{Vect}\left\{ a \right\}  \oplus \ker \varphi = E
		$$
		Montrons alors que $F = \text{Vect} \left\{ a \right\} \oplus (F \cap \ker \varphi)$.
		$$\text{Vect} \left\{ a \right\} \cap (F \cap \ker \varphi) = \underbrace{ \text{Vect} \left\{ a \right\} \cap F }_{ =\text{Vect}\left\{ a \right\}  } \cap \ker \varphi = \text{Vect}\left\{ a \right\}  \cap \ker \varphi = \left\{ 0_{E} \right\} $$ car les deux espaces sont supplémentaires donc en somme directe.
		
		Donc $\text{Vect} \left\{ a \right\} \oplus (F \cap \ker \varphi)$.
		
		Par double inclusion, montrons que $\text{Vect} \left\{ a \right\} + (F \cap \ker \varphi) = F$
		
		Pour l'inclusion directe, remarquons que $a \in F$ donc $\text{Vect}\left\{ a \right\} \subset F$ or $F \cap \ker \varphi \subset F$ donc leur somme est bien incluse $\text{Vect} \left\{ a \right\} + (F \cap \ker \varphi) \subset F$
		Réciproquement, soit $x \in F$ fq.
		Puisque $\text{Vect} \left\{ a \right\} \oplus \ker \varphi = E$
		$$
		\exists (\lambda, x_{K}) \in \mathbb{K} \times \ker \varphi : x = \lambda .a+x_{K}
		$$
		De plus, $x_{K} = x - \lambda.a \in F$ car $(a, x) \in F^{2}$ donc
		$$
		x = \underbrace{ \lambda . a }_{ \in \text{Vect} \left\{ a \right\}  } + \underbrace{ x_{K} }_{ \in F \cap \ker \varphi } \in \text{Vect} \left\{ a \right\} + (F \cap \ker \varphi)
		$$
		D'où l'inclusion réciproque.

		Donc $F = \text{Vect} \left\{ a \right\} \oplus (F \cap \ker \varphi)$
		en passant à la dimension :
		$$
		\underbrace{ \dim F }_{= p } = \underbrace{ \dim \text{Vect} \left\{ a \right\} }_{ =1 } + \dim (F \cap \ker \varphi)
		$$
		Donc $\dim (F \cap \ker \varphi) = p - 1$.
		
		
		
		Appliquons ce lemme pour la démonstration de la propriété suivante \\
		\fbox{\begin{minipage}{\textwidth}
		Soit $E$ un $\mathbb{K}$ espace vectoriel de dimension $n \in \mathbb{N}^{*}$
		
		Soient $m \in \mathbb{N}^{*}$ et $(H_{i})_{n \in [ \! [ 1,m ] \!]}$, $m$ hyperplans de $E$.
		
		Alors $$\dim_{\mathbb{K}} \bigcap_{i=1}^{m}H_{i} \geqslant n-m$$
		\end{minipage}}
		
		Considérons la propriété $\mathcal{P}(\cdot)$ définie pour tout $m \in \mathbb{N}^{*}$ par.
		$$
		\mathcal{P}(m) : « \text{ pour tous } H_{1},\dots, H_{m} \text{ hyperplans de } E, \dim_{\mathbb{K}} \bigcap_{i=1}^{m}H_{i} \geqslant n-m »
		$$
		Soit $H_{1}$ un hyperplan de $E$ fixé quelconque. D'après la caractérisation des hyperplans en dimension finie, 
		$$
		\dim_{\mathbb{K}} \bigcap_{i=1}^{1}H_{i} = \dim_{\mathbb{K}}H_{1}= n-1 \geqslant n-1
		$$
		Donc $\mathcal{P}(1)$ est vraie.
		
		Soit $m \in \mathbb{N}^{*}$ fixé quelconque tel que $\mathcal{P}(m)$ est vraie.
		Soient $H_{1},\dots,H_{m}$ et $H_{m+1}$ $m+1$ hyperplans de $E$.
		D'après la définition d'un hyperplan, il existe $\varphi \in E ^{*}$ non nulle telle que $H_{m+1} = \ker \varphi$.
		
		Appliquons donc le lemme précédent pour $F \leftarrow \bigcap_{i=1}^{m}H_{i}$ (autorisé car c'est un sous espace de l'espace $E$, qui est de dimension finie, donc ses sous espaces les sont aussi) et $\varphi \leftarrow \varphi$ (autorisé car c'est une forme linéaire non nulle) :
		
		$$
		\dim_{\mathbb{K}} \underbrace{ \left( \bigcap_{i=1}^{m}H_{i} \right) \cap \ker \varphi  }_{ =\left( \bigcap_{i=1}^{m}H_{i}  \right)\cap H_{m+1} }\geqslant \dim_{\mathbb{K}} \left( \bigcap_{i=1}^{m}H_{i} \right) - 1 \underbrace{ \geqslant n - m - 1 }_{ \text{ en appliquant }\mathcal{P(m)} \text{ pour } H_{1},\dots,H_{m} }
		$$
		
		Donc par associativité de l'intersection, $\dim_{\mathbb{K}} \bigcap_{i=1}^{m+1}H_{i} \geqslant n - (m+1)$
		
		Donc $\mathcal{P}(m+1)$ est vraie.
	\end{question_kholle}

\end{document}

\documentclass{article}

\date{22 Avril 2024}
\usepackage[nb-sem=24, auteurs={Hugo Vangilluwen, George Ober}]{../kholles}

\begin{document}
	\maketitle

	
%Question 4
	\begin{question_kholle}[{Soient $E, F$ deux $\mathbb{K}$ espaces vectoriels, $f \in \mathcal{L}_{\mathbb{K}}(E, F)$\\
1. Si $E$ est de dimension finie $$f \text{ injective } \iff \mathrm{rg} f = \dim E$$
2. Si $F$ est de dimension finie$$f \text{ surjective } \iff \mathrm{rg} f = \dim F$$
3. Si $E$ et $F$ sont de même dimension finie $$f \text{ bijective } \iff f \text{ injective } \iff f \text{ sujective }$$}]
		{Caractérisation injectivité/bijectivité/surjectivité par le rang}

(i)
Supposons $E$ de dimension finie, fixons $(e_{1}, \dots, e_{n})$ une base de $E$ (avec $n = \dim E$)
Supposons $f$ injective :
$$
\mathrm{rg} f = \dim \mathrm{Im} f = \dim \text{Vect} \left\{ f(e_{1}) \dots f(e_{n}) \right\}  
$$

Donc $(f(e_{1}), \dots f(e_{n}))$ est génératrice.
$(f(e_{1}), \dots f(e_{n}))$ est de plus libre car $f$ est injective.
Donc c'est une base, donc
$$
\dim \text{Vect} \left\{ f(e_{1}) \dots f(e_{n}) \right\}  =n = \dim E
$$
donc $\mathrm{rg} f = \dim E$.
Réciproquement, supposons que $\mathrm{rg} f = \dim E = n$.
Alors $$n = \mathrm{rg} f = \dim \text{Vect} \left\{ f(e_{1}),\dots,f(e_{n}) \right\}$$
Donc $(f(e_{1}), \dots f(e_{n}))$ est génératrice de cardinal $n$, égal à la dimension du sous-espace vectoriel engendré. C'est donc une base du sous-espace vectoriel engendré.
Donc $(f(e_{1}),\dots , f(e_{n}))$ est libre, donc $f$ est injective.

(ii)
Supposons $F$ de dimension finie
$$
f \text{ surjective } \iff \mathrm{Im} f = F \iff \dim \mathrm{Im} f = \dim F
$$

(iii)
Supposons $E$ et $F$ de même dimension finie
$$
f \text{ injective } \iff \mathrm{rg} f = \dim E \iff \mathrm{rg} f = \dim F \iff f \text{ surjective}
$$
D'où la bijectivité.
	\end{question_kholle}

\end{document}
